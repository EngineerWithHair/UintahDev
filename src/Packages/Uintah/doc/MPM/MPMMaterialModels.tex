\documentclass[twoside,11pt]{article}
\usepackage[pdftex]{graphicx}
%\usepackage{times}
\usepackage{palatino}
\usepackage{epsf}
\usepackage{epsfig}
\usepackage{fleqn}
\usepackage{boxedminipage}
\usepackage{amsfonts, amssymb, amsmath}
\usepackage{stmaryrd}
\usepackage{cancel}
%\usepackage[noplots]{maple2e}
\usepackage[usenames]{color}
\definecolor{red}{rgb}{1,0,0}
\definecolor{green}{rgb}{0,1,0}
\definecolor{blue}{rgb}{0,0,1}
\definecolor{Red}{rgb}{0.8666,0.03137,0.02352}
\definecolor{Blue}{rgb}{0.00784,0.67059,0.91764}
\definecolor{Darkgreen}{rgb}{0,0.68235,0}
\definecolor{Green}{rgb}{0,0.8,0}
\definecolor{Bl}{rgb}{0,0.2,0.91764}
\definecolor{Royalblue}{rgb}{0,0.2,0.91764}
\definecolor{Brickred}{rgb}{0.644541,0.164065,0.164065}
\definecolor{Brown}{rgb}{0.6,0.4,0.4}
\definecolor{Orange}{rgb}{1,0.647059,0}
\definecolor{Indigo}{rgb}{0.746105,0,0.996109}
\definecolor{Violet}{rgb}{0.308598,0.183597,0.308598}
\definecolor{Lightgrey}{rgb}{0.762951,0.762951,0.762951}
\definecolor{Darkgrey}{rgb}{0.503548,0.503548,0.503548}
\definecolor{Pink}{rgb}{1,0.6,0.6}
\definecolor{MyLightMagenta}{cmyk}{0.1,0.8,0,0.1}
\definecolor{MyDarkBlue}{rgb}{0,0.08,0.45}

\raggedright
%\setlength{\parsep}{0pt}
%\setlength{\itemsep}{0pt}
\setlength{\parskip}{8pt}
%\setlength{\parsep}{-8pt}
%\setlength{\itemsep}{-8pt}
%\setlength{\parindent}{0pt}
\thispagestyle{empty}
%\pagestyle{plain}
\pagestyle{myheadings}

% The page layout plan.

\setlength{\textheight}{9true in}
\setlength{\textwidth}{7.0true in}
\setlength{\topmargin}{-0.3true in}
\setlength{\headheight}{0.1true in}
\setlength{\headsep}{0.2true in}
\setlength{\footskip}{0.2true in}
\setlength{\topskip}{12pt}
\setlength{\oddsidemargin}{-0.3true in}
\setlength{\evensidemargin}{-0.3true in}

\newcommand{\handout}[3]{
        \begin{center}
         % \copyright 
          Biswajit Banerjee \hspace{4.2in} University of Utah\\
          \vspace{10pt}
          {\Large\bf Waves in Composites and Metamaterials}\\
          \vspace{6pt}
          (Instructor: Prof. G. W. Milton)
        \end{center}
        \vspace{8pt}\noindent
        %\begin{center}
        {\underline{\makebox[7.0in]{\large\bf\noindent
                \makebox[1.5in][l]{#1~~~} \hfill {~~~#2~~~} \hfill
                \makebox[1.5in][r]{~~~#3}}}}
        %\end{center}
        }

\newcommand{\heading}[1]{
        \begin{center}{\large\bf{#1}}\end{center}}

\newcommand{\subheading}[1]{
        \begin{center}{\normalsize\bf{#1}}\end{center}}

\newcommand{\subsubheading}[1]{
        \begin{center}{\small\bf{#1}}\end{center}}

\newcommand{\Heading}[1]{
        \vspace{12pt}\begin{center}{\Large\bf{#1}}\end{center}}

\newcommand{\Subheading}[1]{
        \vspace{8pt}\begin{center}{\large\bf{#1}}\end{center}}

\newcommand{\Red}{\color{Brickred}}
\newcommand{\Blue}{\color{Royalblue}}
\newcommand{\Green}{\color{Darkgreen}}
\newcommand{\Violet}{\color{Violet}}
\newcommand{\Jump}[1]{\ensuremath{\llbracket#1\rrbracket}}
\newcommand{\Blimitx}[1]{\ensuremath{\left[#1\right]_{x_a}^{x_b}}}
\newcommand{\Deriv}[2]{\ensuremath{\cfrac{d#1}{d#2}}}
\newcommand{\MDeriv}[2]{\ensuremath{\cfrac{D#1}{D#2}}}
\newcommand{\DDeriv}[2]{\ensuremath{\cfrac{d^2#1}{d#2^2}}}
\newcommand{\DDDeriv}[2]{\ensuremath{\cfrac{d^3#1}{d#2^3}}}
\newcommand{\DDDDeriv}[2]{\ensuremath{\cfrac{d^4#1}{d#2^4}}}
\newcommand{\Intx}{\ensuremath{\int_{x_a}^{x_b}}}
\newcommand{\IntX}{\ensuremath{\int_{X_a}^{X_b}}}
\newcommand{\Intiso}{\ensuremath{\int_{-1}^{1}}}
\newcommand{\IntOmegaA}{\ensuremath{\int_{\Omega_0}}}
\newcommand{\IntOmega}{\ensuremath{\int_{\Omega}}}
\newcommand{\IntDOmega}{\ensuremath{\int_{\partial\Omega}}}
\newcommand{\IntOmegap}{\ensuremath{\int_{\Omega'}}}
\newcommand{\Norm}[2]{\ensuremath{\left\lVert#1\right\rVert_{#2}}}
\newcommand{\Var}[1]{\ensuremath{\delta #1}}
\newcommand{\DelT}{\ensuremath{\Delta t}}
\newcommand{\CalD}{\ensuremath{\mathcal{D}}}
\newcommand{\BCalD}{\ensuremath{\boldsymbol{\CalD}}}
\newcommand{\CalF}{\ensuremath{\mathcal{F}}}
\newcommand{\CalL}{\ensuremath{\mathcal{L}}}
\newcommand{\CalM}{\ensuremath{\mathcal{M}}}
\newcommand{\BCalM}{\ensuremath{\boldsymbol{\CalM}}}
\newcommand{\CalN}{\ensuremath{\mathcal{N}}}
\newcommand{\CalP}{\ensuremath{\mathcal{P}}}
\newcommand{\CalS}{\ensuremath{\mathcal{S}}}
\newcommand{\BCalS}{\ensuremath{\boldsymbol{\CalS}}}
\newcommand{\CalT}{\ensuremath{\mathcal{T}}}
\newcommand{\CalV}{\ensuremath{\mathcal{V}}}
\newcommand{\CalW}{\ensuremath{\mathcal{W}}}
\newcommand{\CalX}{\ensuremath{\mathcal{X}}}
\newcommand{\Comp}[2]{\ensuremath{#1 \circ #2}}
\newcommand{\Map}[3]{\ensuremath{#1 : #2 \rightarrow #3}}
\newcommand{\MapTo}[3]{\ensuremath{#1 : #2 \mapsto #3}}
\newcommand{\Real}[1]{\ensuremath{\mathbb{R}^{#1}}}
\newcommand{\Ve}{\ensuremath{\varepsilon}}
\newcommand{\BHat}[1]{\ensuremath{\widehat{\boldsymbol{#1}}}}
\newcommand{\BTx}{\ensuremath{\tilde{\boldsymbol{x}}}}
\newcommand{\Beh}{\ensuremath{\hat{\boldsymbol{e}}}}
\newcommand{\BHex}{\ensuremath{\hat{\boldsymbol{e}}_1}}
\newcommand{\BHey}{\ensuremath{\hat{\boldsymbol{e}}_2}}
\newcommand{\BHez}{\ensuremath{\hat{\boldsymbol{e}}_3}}
\newcommand{\BHn}[1]{\ensuremath{\hat{\boldsymbol{n}}_{#1}}}
\newcommand{\BHe}[1]{\ensuremath{\hat{\boldsymbol{e}}_{#1}}}
\newcommand{\BHg}[1]{\ensuremath{\hat{\boldsymbol{g}}_{#1}}}
\newcommand{\BHG}[1]{\ensuremath{\hat{\boldsymbol{G}}_{#1}}}
\newcommand{\Hn}{\ensuremath{\hat{\boldsymbol{n}}}}
\newcommand{\Mba}{\ensuremath{\mathbf{a}}}
\newcommand{\Mbatilde}{\ensuremath{\widetilde{\mathbf{a}}}}
\newcommand{\Mbb}{\ensuremath{\mathbf{b}}}
\newcommand{\Mbd}{\ensuremath{\mathbf{d}}}
\newcommand{\Mbf}{\ensuremath{\mathbf{f}}}
\newcommand{\Mbn}{\ensuremath{\mathbf{n}}}
\newcommand{\Mbntilde}{\ensuremath{\widetilde{\mathbf{n}}}}
\newcommand{\Mbr}{\ensuremath{\mathbf{r}}}
\newcommand{\Mbu}{\ensuremath{\mathbf{u}}}
\newcommand{\Mbv}{\ensuremath{\mathbf{v}}}
\newcommand{\Mbx}{\ensuremath{\mathbf{x}}}
\newcommand{\MbA}{\ensuremath{\mathbf{A}}}
\newcommand{\MbB}{\ensuremath{\mathbf{B}}}
\newcommand{\MbC}{\ensuremath{\mathbf{C}}}
\newcommand{\MbD}{\ensuremath{\mathbf{D}}}
\newcommand{\MbH}{\ensuremath{\mathbf{H}}}
\newcommand{\MbHbar}{\ensuremath{\mathbf{\overline{H}}}}
\newcommand{\MbI}{\ensuremath{\mathbf{I}}}
\newcommand{\MbK}{\ensuremath{\mathbf{K}}}
\newcommand{\MbKbar}{\ensuremath{\overline{\mathbf{K}}}}
\newcommand{\MbKtilde}{\ensuremath{\widetilde{\mathbf{K}}}}
\newcommand{\MbM}{\ensuremath{\mathbf{M}}}
\newcommand{\MbN}{\ensuremath{\mathbf{N}}}
\newcommand{\MbP}{\ensuremath{\mathbf{P}}}
\newcommand{\MbPbar}{\ensuremath{\overline{\mathbf{P}}}}
\newcommand{\MbR}{\ensuremath{\mathbf{R}}}
\newcommand{\MbT}{\ensuremath{\mathbf{T}}}
\newcommand{\MbV}{\ensuremath{\mathbf{V}}}
\newcommand{\MbX}{\ensuremath{\mathbf{X}}}
\newcommand{\MbSig}{\ensuremath{\boldsymbol{\sigma}}}
\newcommand{\Mbone}{\ensuremath{\mathbf{1}}}
\newcommand{\Mbzero}{\ensuremath{\mathbf{0}}}
\newcommand{\Mb}{\ensuremath{\left[\mathsf{b}\right]}}
\newcommand{\Mu}{\ensuremath{\left[\mathsf{u}\right]}}
\newcommand{\Mv}{\ensuremath{\left[\mathsf{v}\right]}}
\newcommand{\Mw}{\ensuremath{\left[\mathsf{w}\right]}}
\newcommand{\Mx}{\ensuremath{\left[\mathsf{x}\right]}}
\newcommand{\MA}{\ensuremath{\left[\mathsf{A}\right]}}
\newcommand{\MC}{\ensuremath{\left[\mathsf{C}\right]}}
\newcommand{\MD}{\ensuremath{\left[\mathsf{D}\right]}}
\newcommand{\MH}{\ensuremath{\left[\mathsf{H}\right]}}
\newcommand{\MI}{\ensuremath{\left[\mathsf{I}\right]}}
\newcommand{\ML}{\ensuremath{\left[\mathsf{L}\right]}}
\newcommand{\MM}{\ensuremath{\left[\mathsf{M}\right]}}
\newcommand{\MN}{\ensuremath{\left[\mathsf{N}\right]}}
\newcommand{\MP}{\ensuremath{\left[\mathsf{P}\right]}}
\newcommand{\MR}{\ensuremath{\left[\mathsf{R}\right]}}
\newcommand{\MT}{\ensuremath{\left[\mathsf{T}\right]}}
\newcommand{\MV}{\ensuremath{\left[\mathsf{V}\right]}}
\newcommand{\Mone}{\ensuremath{\left[\mathsf{1}\right]}}
\newcommand{\Mzero}{\ensuremath{\left[\mathsf{0}\right]}}
\newcommand{\SfA}{\ensuremath{\boldsymbol{\mathsf{A}}}}
\newcommand{\SfC}{\ensuremath{\boldsymbol{\mathsf{C}}}}
\newcommand{\SfD}{\ensuremath{\boldsymbol{\mathsf{D}}}}
\newcommand{\SfI}{\ensuremath{\boldsymbol{\mathsf{I}}}}
\newcommand{\SfL}{\ensuremath{\boldsymbol{\mathsf{L}}}}
\newcommand{\SfS}{\ensuremath{\boldsymbol{\mathsf{S}}}}
\newcommand{\SfT}{\ensuremath{\boldsymbol{\mathsf{T}}}}
\newcommand{\Msig}{\ensuremath{\left[\boldsymbol{\sigma}\right]}}
\newcommand{\Meps}{\ensuremath{\left[\boldsymbol{\varepsilon}\right]}}
\newcommand{\Ex}{\ensuremath{\boldsymbol{e}_1}}
\newcommand{\Ey}{\ensuremath{\boldsymbol{e}_2}}
\newcommand{\Ez}{\ensuremath{\boldsymbol{e}_3}}
\newcommand{\Exp}{\ensuremath{\boldsymbol{e}^{'}_1}}
\newcommand{\Eyp}{\ensuremath{\boldsymbol{e}^{'}_2}}
\newcommand{\Ezp}{\ensuremath{\boldsymbol{e}^{'}_3}}
\newcommand{\Epsxx}{\ensuremath{\varepsilon_{11}}}
\newcommand{\Epsyy}{\ensuremath{\varepsilon_{22}}}
\newcommand{\Epszz}{\ensuremath{\varepsilon_{33}}}
\newcommand{\Epsyz}{\ensuremath{\varepsilon_{23}}}
\newcommand{\Epszx}{\ensuremath{\varepsilon_{31}}}
\newcommand{\Epsxy}{\ensuremath{\varepsilon_{12}}}
\newcommand{\Sigxx}{\ensuremath{\sigma_{11}}}
\newcommand{\Sigyy}{\ensuremath{\sigma_{22}}}
\newcommand{\Sigzz}{\ensuremath{\sigma_{33}}}
\newcommand{\Sigyz}{\ensuremath{\sigma_{23}}}
\newcommand{\Sigzx}{\ensuremath{\sigma_{31}}}
\newcommand{\Sigxy}{\ensuremath{\sigma_{12}}}
\newcommand{\Eps}[1]{\ensuremath{\varepsilon_{#1}}}
\newcommand{\Sig}[1]{\ensuremath{\sigma_{#1}}}
\newcommand{\X}{\ensuremath{X_1}}
\newcommand{\Y}{\ensuremath{X_2}}
\newcommand{\Z}{\ensuremath{X_3}}
\newcommand{\Balpha}{\ensuremath{\boldsymbol{\alpha}}}
\newcommand{\Balphahat}{\ensuremath{\widehat{\boldsymbol{\alpha}}}}
\newcommand{\Bbeta}{\ensuremath{\boldsymbol{\beta}}}
\newcommand{\Bchi}{\ensuremath{\boldsymbol{\chi}}}
\newcommand{\Beta}{\ensuremath{\boldsymbol{\eta}}}
\newcommand{\Bveps}{\ensuremath{\boldsymbol{\varepsilon}}}
\newcommand{\BGamma}{\ensuremath{\boldsymbol{\mathit{\Gamma}}}}
\newcommand{\BGammahat}{\ensuremath{\boldsymbol{\mathit{\widehat{\Gamma}}}}}
%\newcommand{\BGammahat}{\ensuremath{\widehat{\BGamma}}}
\newcommand{\Bkappa}{\ensuremath{\boldsymbol{\kappa}}}
\newcommand{\Bbeps}{\ensuremath{\bar{\boldsymbol{\varepsilon}}}}
\newcommand{\Bnabla}{\ensuremath{\boldsymbol{\nabla}}}
\newcommand{\Bomega}{\ensuremath{\boldsymbol{\omega}}}
\newcommand{\Bsig}{\ensuremath{\boldsymbol{\sigma}}}
\newcommand{\Btau}{\ensuremath{\boldsymbol{\tau}}}
\newcommand{\Bpi}{\ensuremath{\boldsymbol{\pi}}}
\newcommand{\Brho}{\ensuremath{\boldsymbol{\rho}}}
\newcommand{\Bvarphi}{\ensuremath{\boldsymbol{\varphi}}}
\newcommand{\Blambda}{\ensuremath{\boldsymbol{\lambda}}}
\newcommand{\Btheta}{\ensuremath{\boldsymbol{\theta}}}
\newcommand{\Bmu}{\ensuremath{\boldsymbol{\mu}}}
\newcommand{\Bxi}{\ensuremath{\boldsymbol{\xi}}}
\newcommand{\BPi}{\ensuremath{\boldsymbol{\Pi}}}
\newcommand{\Bone}{\ensuremath{\boldsymbol{\mathit{1}}}}
\newcommand{\Bonev}{\ensuremath{\boldsymbol{1}}}
\newcommand{\Bzero}{\ensuremath{\boldsymbol{0}}}
\newcommand{\BzeroT}{\ensuremath{\boldsymbol{\mathit{0}}}}
\newcommand{\Ba}{\ensuremath{\mathbf{a}}}
\newcommand{\Bb}{\ensuremath{\mathbf{b}}}
\newcommand{\BbT}{\ensuremath{\boldsymbol{b}}}
\newcommand{\Bc}{\ensuremath{\mathbf{c}}}
\newcommand{\Bd}{\ensuremath{\boldsymbol{d}}}
\newcommand{\Bdv}{\ensuremath{\mathbf{d}}}
\newcommand{\Be}{\ensuremath{\mathbf{e}}}
\newcommand{\Bf}{\ensuremath{\mathbf{f}}}
\newcommand{\Bg}{\ensuremath{\mathbf{g}}}
\newcommand{\Bh}{\ensuremath{\boldsymbol{h}}}
\newcommand{\Bhv}{\ensuremath{\mathbf{h}}}
\newcommand{\Bk}{\ensuremath{\mathbf{k}}}
\newcommand{\Bl}{\ensuremath{\mathbf{l}}}
\newcommand{\Bm}{\ensuremath{\mathbf{m}}}
\newcommand{\Bn}{\ensuremath{\mathbf{n}}}
\newcommand{\Bo}{\ensuremath{\mathbf{o}}}
\newcommand{\Bp}{\ensuremath{\mathbf{p}}}
\newcommand{\Bq}{\ensuremath{\mathbf{q}}}
\newcommand{\Br}{\ensuremath{\boldsymbol{r}}}
\newcommand{\Brv}{\ensuremath{\mathbf{r}}}
\newcommand{\Bs}{\ensuremath{\boldsymbol{s}}}
\newcommand{\Bsv}{\ensuremath{\mathbf{s}}}
\newcommand{\Bt}{\ensuremath{\mathbf{t}}}
\newcommand{\Bu}{\ensuremath{\mathbf{u}}}
\newcommand{\Bv}{\ensuremath{\mathbf{v}}}
\newcommand{\Bw}{\ensuremath{\mathbf{w}}}
\newcommand{\Bx}{\ensuremath{\mathbf{x}}}
\newcommand{\By}{\ensuremath{\mathbf{y}}}
\newcommand{\BA}{\ensuremath{\boldsymbol{A}}}
\newcommand{\BB}{\ensuremath{\boldsymbol{B}}}
\newcommand{\BC}{\ensuremath{\boldsymbol{C}}}
\newcommand{\BD}{\ensuremath{\boldsymbol{D}}}
\newcommand{\BE}{\ensuremath{\boldsymbol{E}}}
\newcommand{\BF}{\ensuremath{\boldsymbol{F}}}
\newcommand{\BG}{\ensuremath{\boldsymbol{G}}}
\newcommand{\BH}{\ensuremath{\boldsymbol{H}}}
\newcommand{\BI}{\ensuremath{\boldsymbol{I}}}
\newcommand{\BJ}{\ensuremath{\boldsymbol{J}}}
\newcommand{\BK}{\ensuremath{\boldsymbol{K}}}
\newcommand{\BL}{\ensuremath{\boldsymbol{L}}}
\newcommand{\BM}{\ensuremath{\boldsymbol{M}}}
\newcommand{\BNv}{\ensuremath{\mathbf{N}}}
\newcommand{\BN}{\ensuremath{\boldsymbol{N}}}
\newcommand{\BP}{\ensuremath{\boldsymbol{P}}}
\newcommand{\BQ}{\ensuremath{\boldsymbol{Q}}}
\newcommand{\BR}{\ensuremath{\boldsymbol{R}}}
\newcommand{\BS}{\ensuremath{\boldsymbol{S}}}
\newcommand{\BT}{\ensuremath{\boldsymbol{T}}}
\newcommand{\BTv}{\ensuremath{\mathbf{T}}}
\newcommand{\BU}{\ensuremath{\boldsymbol{U}}}
\newcommand{\BW}{\ensuremath{\boldsymbol{W}}}
\newcommand{\BX}{\ensuremath{\mathbf{X}}}
\newcommand{\BXT}{\ensuremath{\boldsymbol{X}}}
\newcommand{\BY}{\ensuremath{\boldsymbol{Y}}}
\newcommand{\Dev}{\ensuremath{\text{dev}}}
\newcommand{\Trial}{\ensuremath{\text{trial}}}
\newcommand{\Tint}{\ensuremath{\text{int}}}
\newcommand{\Text}{\ensuremath{\text{ext}}}
\newcommand{\Tkin}{\ensuremath{\text{kin}}}
\newcommand{\Tbody}{\ensuremath{\text{body}}}
\newcommand{\Tmin}{\ensuremath{\text{min}}}
\newcommand{\Tmax}{\ensuremath{\text{max}}}
\newcommand{\Tor}{\ensuremath{\text{or}}}
\newcommand{\Tr}{\ensuremath{\text{tr}}}
\newcommand{\Half}{\ensuremath{\frac{1}{2}}}
\newcommand{\SThr}{\ensuremath{\sqrt{3}}}
\newcommand{\STT}{\ensuremath{\frac{\sqrt{3}}{2}}}
\newcommand{\Third}{\ensuremath{\frac{1}{3}}}
\newcommand{\Inner}[2]{\ensuremath{\langle#1,~#2\rangle}}
\newcommand{\Bcross}[2]{\ensuremath{#1\boldsymbol{\times}#2}}
\newcommand{\Bdot}[2]{\ensuremath{#1\cdot#2}}
\newcommand{\Dyad}[2]{\ensuremath{#1\boldsymbol{\otimes}#2}}
\newcommand{\Grad}[1]{\ensuremath{\Bnabla #1}}
\newcommand{\Gradp}[1]{\ensuremath{\Bnabla' #1}}
\newcommand{\Grads}[1]{\ensuremath{\Bnabla_s #1}}
\newcommand{\Grady}[1]{\ensuremath{\Bnabla_y #1}}
\newcommand{\Lap}[1]{\ensuremath{\nabla^2 #1}}
\newcommand{\Biharm}[1]{\ensuremath{\nabla^4 #1}}
\newcommand{\Div}[1]{\ensuremath{\Bdot{\Bnabla}{#1}}}
\newcommand{\Divp}[1]{\ensuremath{\Bdot{\Bnabla'}{#1}}}
\newcommand{\Divy}[1]{\ensuremath{\Bdot{\Bnabla_y}{#1}}}
\newcommand{\Curl}[1]{\ensuremath{\Bcross{\Bnabla}{#1}}}
\newcommand{\Curlp}[1]{\ensuremath{\Bcross{\Bnabla'}{#1}}}
\newcommand{\Curls}[1]{\ensuremath{\Bcross{\Bnabla_s}{#1}}}
\newcommand{\Curly}[1]{\ensuremath{\Bcross{\Bnabla_y}{#1}}}
\newcommand{\Gradu}{\ensuremath{\Grad{\Bu}}}
\newcommand{\Divu}{\ensuremath{\Div{\Bu}}}
\newcommand{\Curlu}{\ensuremath{\Curl{\Bu}}}
\newcommand{\Gradv}{\ensuremath{\Grad{\Bv}}}
\newcommand{\Divv}{\ensuremath{\Div{\Bv}}}
\newcommand{\Curlv}{\ensuremath{\Curl{\Bv}}}
\newcommand{\Dualn}{\ensuremath{\Bdual{\Bn}{\Bn}}}
\newcommand{\Over}[1]{\ensuremath{\frac{1}{#1}}}
\newcommand{\Diff}[2]{\ensuremath{\frac{d #1}{d #2}}}
\newcommand{\Partial}[2]{\ensuremath{\frac{\displaystyle\partial #1}{\displaystyle\partial #2}}}
\newcommand{\PPartial}[2]{\ensuremath{\frac{\partial^2 #1}{\partial #2^2}}}
\newcommand{\PPartialA}[3]{\ensuremath{\frac{\partial^2 #1}{\partial #2\partial#3}}}
\newcommand{\FPartial}[2]{\ensuremath{\frac{\partial^4 #1}{\partial #2^4}}}
\newcommand{\FPartialA}[3]{\ensuremath{\frac{\partial^4 #1}{\partial #2^2
         \partial #3^2}}}
\newcommand{\DotMbT}{\ensuremath{\dot{\MbT}}}
\newcommand{\TildeMbT}{\ensuremath{\widetilde{\MbT}}}
\newcommand{\BarT}{\ensuremath{\overline{T}}}
\newcommand{\Barq}{\ensuremath{\overline{q}}}
\newcommand{\Domega}{\ensuremath{\partial{\Omega}}}
\newcommand{\Av}[1]{\ensuremath{\left\langle#1\right\rangle}}
\newcommand{\AvSig}{\ensuremath{\langle\Bsig\rangle}}
\newcommand{\AvTau}{\ensuremath{\langle\Btau\rangle}}
\newcommand{\AvP}{\ensuremath{\langle\BP\rangle}}
\newcommand{\AvEps}{\ensuremath{\langle\Beps\rangle}}
\newcommand{\AvEpsdot}{\ensuremath{\langle\dot{\Beps}\rangle}}
\newcommand{\AvDisp}{\ensuremath{\langle\Bu\rangle}}
\newcommand{\AvF}{\ensuremath{\langle\BF\rangle}}
\newcommand{\AvFdot}{\ensuremath{\langle\dot{\BF}\rangle}}
\newcommand{\Avl}{\ensuremath{\overline{\Bl}}}
\newcommand{\AvSigBar}{\ensuremath{\overline{\Bsig}}}
\newcommand{\AvTauBar}{\ensuremath{\overline{\Btau}}}
\newcommand{\AvOmega}{\ensuremath{\langle\Bomega\rangle}}
\newcommand{\AvGradu}{\ensuremath{\langle\Gradu\rangle}}
\newcommand{\AvGradudot}{\ensuremath{\langle\Grad{\dot{\Bu}}\rangle}}
\newcommand{\AvGradv}{\ensuremath{\langle\Gradv\rangle}}
\newcommand{\AvPower}{\ensuremath{\langle\Bsig:\Gradv\rangle}}
\newcommand{\AvPowerInf}{\ensuremath{\langle\Bsig:\dot{\Beps}\rangle}}
\newcommand{\AvWorkInf}{\ensuremath{\langle\Bsig:\Beps\rangle}}
\newcommand{\AvPowerPF}{\ensuremath{\langle\BP^T:\dot{\BF}\rangle}}
\newcommand{\DA}{\ensuremath{\text{dA}}}
\newcommand{\DAvec}{\ensuremath{\text{d}\mathbf{A}}}
\newcommand{\Da}{\ensuremath{\text{da}}}
\newcommand{\Davec}{\ensuremath{\text{d}\mathbf{a}}}
\newcommand{\DV}{\ensuremath{\text{dV}}}
\newcommand{\BCe}{\ensuremath{\mathcal{E}}}
\newcommand{\GradX}[1]{\ensuremath{\Bnabla_0~#1}}
\newcommand{\DivX}[1]{\ensuremath{\Bdot{\Bnabla_0}{#1}}}
\newcommand{\Bxdot}{\ensuremath{\dot{\Bx}}}
\newcommand{\BFdot}{\ensuremath{\dot{\BF}}}
\newcommand{\BAv}{\ensuremath{\mathbf{A}}}
\newcommand{\BBv}{\ensuremath{\mathbf{B}}}
\newcommand{\BDv}{\ensuremath{\mathbf{D}}}
\newcommand{\BEv}{\ensuremath{\mathbf{E}}}
\newcommand{\BFv}{\ensuremath{\mathbf{F}}}
\newcommand{\BHv}{\ensuremath{\mathbf{H}}}
\newcommand{\BJv}{\ensuremath{\mathbf{J}}}
\newcommand{\BMv}{\ensuremath{\mathbf{M}}}
\newcommand{\BPv}{\ensuremath{\mathbf{P}}}
\newcommand{\BRv}{\ensuremath{\mathbf{R}}}
\newcommand{\BVv}{\ensuremath{\mathbf{V}}}
\newcommand{\Bdelta}{\ensuremath{\boldsymbol{\delta}}}
\newcommand{\Beps}{\ensuremath{\boldsymbol{\epsilon}}}
\newcommand{\BVeps}{\ensuremath{\boldsymbol{\varepsilon}}}
\newcommand{\BDtildev}{\ensuremath{\mathbf{\widetilde{D}}}}
\newcommand{\IntInfT}{\ensuremath{\int_{-\infty}^t}}
\newcommand{\IntInfInf}{\ensuremath{\int_{-\infty}^{\infty}}}
\newcommand{\IntInfZero}{\ensuremath{\int_{-\infty}^{0}}}
\newcommand{\IntZeroInf}{\ensuremath{\int_{0}^{\infty}}}
\newcommand{\IIntInfInf}{\ensuremath{\int_{-\infty}^{\infty}\int_{-\infty}^{\infty}}}
\newcommand{\IIIntInfInf}{\ensuremath{\int_{-\infty}^{\infty}\int_{-\infty}^{\infty}\int_{-\infty}^{\infty}}}
\newcommand{\Dtau}{\ensuremath{\text{d}\tau}}
\newcommand{\domega}{\ensuremath{\text{d}\omega}}
\newcommand{\dOmega}{\ensuremath{\text{d}\Omega}}
\newcommand{\dGamma}{\ensuremath{\text{d}\Gamma}}
\newcommand{\dzeta}{\ensuremath{\text{d}\zeta}}
\newcommand{\Ds}{\ensuremath{\text{d}s}}
\newcommand{\Dt}{\ensuremath{\text{d}t}}
\newcommand{\Dx}{\ensuremath{\text{d}\Bx}}
\newcommand{\dr}{\ensuremath{\text{d}r}}
\newcommand{\dx}{\ensuremath{\text{d}x}}
\newcommand{\dy}{\ensuremath{\text{d}y}}
\newcommand{\dz}{\ensuremath{\text{d}z}}
\newcommand{\dk}{\ensuremath{\text{d}k}}
\newcommand{\dBx}{\ensuremath{\text{d}\Bx}}
\newcommand{\dBk}{\ensuremath{\text{d}\Bk}}
\newcommand{\dBr}{\ensuremath{\text{d}\Br}}
\newcommand{\BKbar}{\ensuremath{\boldsymbol{\bar{K}}}}
\newcommand{\Bahat}{\ensuremath{\widehat{\Ba}}}
\newcommand{\Bbhat}{\ensuremath{\widehat{\Bb}}}
\newcommand{\BAhat}{\ensuremath{\widehat{\BA}}}
\newcommand{\BBhat}{\ensuremath{\widehat{\BB}}}
\newcommand{\BBhatv}{\ensuremath{\mathbf{\widehat{B}}}}
\newcommand{\BDhatv}{\ensuremath{\mathbf{\widehat{D}}}}
\newcommand{\BEhatv}{\ensuremath{\mathbf{\widehat{E}}}}
\newcommand{\BFhatv}{\ensuremath{\mathbf{\widehat{F}}}}
\newcommand{\BHhatv}{\ensuremath{\mathbf{\widehat{H}}}}
\newcommand{\BPhatv}{\ensuremath{\mathbf{\widehat{P}}}}
\newcommand{\BVhatv}{\ensuremath{\mathbf{\widehat{V}}}}
\newcommand{\Rea}{\ensuremath{\text{Re}}}
\newcommand{\Img}{\ensuremath{\text{Im}}}
\newcommand{\Teff}{\ensuremath{\text{eff}}}
\newcommand{\Tand}{\ensuremath{\text{and}}}
\newcommand{\CalE}{\ensuremath{\mathcal{E}}}
\newcommand{\CalH}{\ensuremath{\mathcal{H}}}
\newcommand{\CalJ}{\ensuremath{\mathcal{J}}}
\newcommand{\CalU}{\ensuremath{\mathcal{U}}}
\newcommand{\Dhat}{\ensuremath{\widehat{D}}}
\newcommand{\Ehat}{\ensuremath{\widehat{E}}}
\newcommand{\Fhat}{\ensuremath{\widehat{F}}}
\newcommand{\Phat}{\ensuremath{\widehat{P}}}
\newcommand{\Uhat}{\ensuremath{\widehat{U}}}
\newcommand{\Vhat}{\ensuremath{\widehat{V}}}
\newcommand{\fhat}{\ensuremath{\widehat{f}}}
\newcommand{\ghat}{\ensuremath{\widehat{g}}}
\newcommand{\phat}{\ensuremath{\widehat{p}}}
\newcommand{\uhat}{\ensuremath{\widehat{u}}}
\newcommand{\vhat}{\ensuremath{\widehat{v}}}
\newcommand{\xhat}{\ensuremath{\widehat{x}}}
\newcommand{\yhat}{\ensuremath{\widehat{y}}}
\newcommand{\Beq}{\begin{equation}}
\newcommand{\Eeq}{\end{equation}}
\newcommand{\Bal}{\begin{aligned}}
\newcommand{\Eal}{\end{aligned}}
\newcommand{\Tbar}{\ensuremath{\text{bar}}}
\newcommand{\Tball}{\ensuremath{\text{ball}}}
\newcommand{\Buhat}{\ensuremath{\widehat{\Bu}}}
\newcommand{\BHhat}{\ensuremath{\widehat{\BH}}}
\newcommand{\Bsighat}{\ensuremath{\widehat{\Bsig}}}
\newcommand{\Bepshat}{\ensuremath{\widehat{\Beps}}}
\newcommand{\sighat}{\ensuremath{\widehat{\sigma}}}
\newcommand{\epshat}{\ensuremath{\widehat{\epsilon}}}
\newcommand{\rhohat}{\ensuremath{\widehat{\rho}}}
\newcommand{\phihat}{\ensuremath{\widehat{\varphi}}}
\newcommand{\kappahat}{\ensuremath{\widehat{\kappa}}}
\newcommand{\SfK}{\ensuremath{\boldsymbol{\mathsf{K}}}}
\newcommand{\SfZero}{\ensuremath{\boldsymbol{\mathsf{0}}}}
\newcommand{\Gradbar}[1]{\ensuremath{\overline{\Bnabla} #1}}
\newcommand{\Divbar}[1]{\ensuremath{\Bdot{\overline{\Bnabla}}{#1}}}
\newcommand{\ktilde}{\ensuremath{\widetilde{k}}}
\newcommand{\Etilde}{\ensuremath{\widetilde{E}}}
\newcommand{\Htilde}{\ensuremath{\widetilde{H}}}
\newcommand{\Rtilde}{\ensuremath{\widetilde{R}}}
\newcommand{\Ttilde}{\ensuremath{\widetilde{T}}}
\newcommand{\TAi}{\ensuremath{\text{Ai}}}
\newcommand{\TBi}{\ensuremath{\text{Bi}}}
\newcommand{\sgn}{\ensuremath{\text{sgn}}}
\newcommand{\DelTwo}{\ensuremath{\Delta/2}}
\newcommand{\Bepseff}{\ensuremath{\Beps_\Teff}}
\newcommand{\Bmueff}{\ensuremath{\Bmu_\Teff}}
\newcommand{\epseff}{\ensuremath{\epsilon_\Teff}}
\newcommand{\mueff}{\ensuremath{\mu_\Teff}}
\newcommand{\BAeff}{\ensuremath{\BA_\Teff}}
\newcommand{\Conj}[1]{\ensuremath{\overline{#1}}}

\title{Material Models and the Uintah Computational Framework}

\begin{document}
\maketitle
\tableofcontents

\section{Introduction}
This document deals with some material models for solids that
have been implemented in the Uintah Computational Framework 
for use with the Material Point Method.  The approach taken
has been to separate the stress-strain relations from the 
numerical stress update algorithms as far as possible.  Two
stress update algorithms (hypoelastic-plastic and 
hyperelastic-plastic) are discussed and the manner in which 
plasticity flow rules, damage models and equations of state
fit into the stress update algorithms are shown.
  
\section{Stress Update Algorithms}
The hypoelastic-plastic stress update is based on an additive
decomposition of the rate of deformation tensor into elastic
and plastic parts while the hyperelastic-plastic stress update
is based on a multiplicative decomposition of the elastic and 
plastic deformation gradients.  Incompressibility is assumed
for plastic deformations.  The volumetric response is therefore
determined either by a bulk modulus and the trace of the rate
of deformation tensor or using an equation of state.  The 
deviatoric response is determined either by an elastic constitutive 
equation or using a plastic flow rule in combination with a 
yield condition. 
  
The material models that can be varied in these stress update approaches
are (not all are applicable to both hypo- and hyperelastic formulations
nor is the list exhaustive):
\begin{enumerate}
    \item The elasticity model, for example, 
      \begin{itemize}
        \item Isotropic linear elastic model. 
        \item Anisotropic linear elastic models.
        \item Isotropic nonlinear elastic models.
        \item Anisotropic nonlinear elastic models.
      \end{itemize}
    \item Isotropic hardening or Kinematic hardening using a 
          back stress evolution rule, for example,
      \begin{itemize}
        \item Ziegler evolution rule   .
      \end{itemize}
    \item The flow rule and hardening/softening law, for example,
      \begin{itemize} 
        \item Perfect plasticity/power law hardening plasticity.
        \item Johnson-Cook plasticity .
        \item Mechanical Threshold Stress (MTS) plasticity .
        \item Anand plasticity .
      \end{itemize} 
    \item The yield condition, for example,
      \begin{itemize} 
        \item von Mises yield condition.
        \item Drucker-Prager yield condition.
        \item Mohr-Coulomb yield condition.
      \end{itemize} 
    \item A continuum or nonlocal damage model with damage evolution
          given by, for example, 
      \begin{itemize}
        \item Johnson-Cook damage model.
        \item Gurson-Needleman-Tvergaard model.
        \item Sandia damage model.
      \end{itemize}
    \item An equation of state to determine the pressure (or 
          volumetric response), for example, 
      \begin{itemize}
        \item Mie-Gruneisen equation of state.
      \end{itemize}
\end{enumerate}
  
The currently implemented stress update algorithms in the Uintah
Computational Framework do not allow for arbitrary elasticity models,
kinematic hardening, arbitrary yield conditions and continuum or 
nonlocal damage (however the a damage parameter is updated and 
used in the erosion algorithm).  The models that can be varied
are the flow rule models, damage variable evolution models and the
equation of state models.

Note that there are no checks in the Unitah Computational Framework
to prevent users from mixing and matching inappropriate models.

\subsection{Hypoelastic-plastic Stress Update}
This section describes the current implementation of the hypoelastic-
plastic model.

The elastic response is assumed to be isotropic.  The material
constants that are taken as input for the elastic response are the
bulk and shear modulus.  The flow rule is determined from the input
and the appropriate plasticity model is created using the 
\verb+PlasticityModelFactory+ class.  The damage evolution rule
is determined from the input and a damage model is created using
the \verb+DamageModelFactory+ class.  The equation of state 
that is used to determine the pressure is also determined from the
input.  The equation of state model is created using the 
\verb+MPMEquationOfStateFactory+ class.

The evolution variables specific to the hypoelastic-plastic model
are the left stretch ($\mathbf{V}$) and the rotation ($\mathbf{R}$).
In addition, a damage evolution variable ($D$) is stored at each time 
step (this need not be the case and will be transfered to the 
damage models in the future).  The left stretch and rotation are 
updated incrementally at each
time step (instead of performing a polar decomposition) and the 
rotation tensor is used to rotate the Cauchy stress and rate of deformation
to the material coordinates at each time step (instead of using a 
objective stress rate formulation).  The stress update formula implemented
is one based on the approach taken by  Zocher et al. (2000)~\cite{Zocher00}.  

Any evolution variables for the plasticity model, damage model or the
equation of state are specified in the class that encapsulates the 
particular model.  

The flow stress is calculated from the plasticity model using a 
function call of the form
\begin{verbatim}
    double flowStress = d_plasticity->computeFlowStress(tensorEta, tensorS, 
                                                        pTemperature[idx],
                                                        delT, d_tol, matl, idx);
\end{verbatim}
A number of plasticity models can be evaluated using the inputs in the
\verb+computeFlowStress+ call.  The variable \verb+d_plasticity+ is
polymorphic and can represent any of the plasticity models that can be
created by the plasticity model factory.  The plastic evolution variables
are updated using a polymorphic function along the lines of
\verb+computeFlowStress+.

The equation of state is used to calculate the hydrostatic stress using
a function call of the form
\begin{verbatim}
    Matrix3 tensorHy = d_eos->computePressure(matl, bulk, shear, 
                                              tensorF_new, tensorD, 
                                              tensorP, pTemperature[idx], 
                                              rho_cur, delT);
\end{verbatim}

Similarly, the damage model is called using a function of the type
\begin{verbatim}
    double damage = d_damage->computeScalarDamage(tensorEta, tensorS, 
                                                  pTemperature[idx],
                                                  delT, matl, d_tol, 
                                                  pDamage[idx]);
\end{verbatim}

Therefore, the plasticity, damage and equation of state models are 
easily be inserted into any other type of stress update algorithm 
without any change being needed in them as can be seen in the 
hyperelastic-plastic stress update algorithm discussed below.

\subsection{Hyperelastic-plastic Stress Update}
The stress update used for the hyperelastic-plastic material is
a return mapping plasticity algorithm based on a multiplicative 
decomposition of the deformation gradient and the intermediate
configuration concept.  The algorithms has been taken from
Simo and Hughes (1998)~\cite{Simo1998}.

The variables that are evolved locally in the Hyperelastic-plastic 
stress update algorithm are the deviatoric part of the left Cauchy-Green
tensor and a scalar damage variable.  

The plasticity, damage and equation of state models are initialized
and called in exactly the same way as in the hypoelastic-plastic 
model.  Therefore, the stress update algorithm and the plasticity,
damage and equation of state models are encapsulated from each other
and no modifications of the latter are required.

\section{Plasticity Model Framework}
Currently implemented plasticity models are isotropic hardening, 
Johnson-Cook and MTS.  Addition of a new model requires the 
following steps :

\begin{enumerate}
    \item Creation of a new class that encapsulates the plasticity 
    model.  The template for this class can be copied from the
    existing plasticity models.  The data that is unique to 
    the new model are specified in the form of 
    \begin{itemize}
      \item A structure containing the constants for the plasticity
            model.
      \item Particle variables that specify the variables that 
            evolve in the plasticity model.
    \end{itemize}
    \item The implementation of the plasticity model involves the
    following steps.
    \begin{itemize}
      \item Reading the input file for the model constants in the
            constructor.
      \item Adding the variables that evolve in the plasticity model
            appropriately to the task graph.
      \item Adding the appropriate flow stress calculation method.
    \end{itemize}
    \item The \verb+PlasticityModelFactory+ is then modified so that
          it recognizes the added plasticity model.
\end{enumerate}

\section{Damage Models}
Only the Johnson-Cook damage evolution rule has been added to the 
DamageModelFactory so far.  The damage model framework is designed 
to be similar to the plasticity model framework.  New models can
be added using the approach described in the previous section.

\section{Equation of State Models}
The elastic-plastic stress update assumes that the volumetric part of the 
Cauchy stress can be calculated using an equation of state.  There are 
three equations of state that are implemented in Uintah.  These are
\begin{enumerate}
    \item A default hypoelastic equation of state.
    \item A neo-Hookean equation of state.    
    \item A Mie-Gruneisen type equation of state.
\end{enumerate}

\subsection{Default hypoelastic equation of state}
In this case we assume that the stress rate is given by
\Beq
    \dot{\Bsig} = \lambda~\Tr(\Bd^e)~\Bone + 2~\mu~\Bd^e
\Eeq
where $\Bsig$ is the Cauchy stress, $\Bd^e$ is the elastic part of
the rate of deformation, and $\lambda, \mu$ are constants.

If $\Beta^e$ is the deviatoric part of $\Bd^e$ then we can write
\Beq
    \dot{\Bsig} = \left(\lambda + \frac{2}{3}~\mu\right)~\Tr(\Bd^e)~\Bone + 
        2~\mu~\Beta^e = 3~\kappa~\Tr(\Bd^e)~\Bone + 2~\mu~\Beta^e ~.
\Eeq
If we split $\Bsig$ into a volumetric and a deviatoric part, i.e.,
$\Bsig = p~\Bone + \Bs$ and take the time derivative to get
$\dot{\Bsig} = \dot{p}~\Bone + \dot{\Bs}$ then
\Beq
    \dot{p} = 3~\kappa~\Tr(\Bd^e) ~.
\Eeq
In addition we assume that $\Bd = \Bd^e + \Bd^p$.  If we also assume that 
the plastic volume change is negligible, we can then write that
\Beq
    \dot{p} = 3~\kappa~\Tr(\Bd) ~.
\Eeq
This is the equation that is used to calculate the pressure $p$ in the 
default hypoelastic equation of state, i.e.,
\Beq
    \boxed{
    p_{n+1} = p_n + 3~\kappa~\Tr(\Bd_{n+1})~\Delta t ~.
    }
\Eeq
To get the derivative of $p$ with respect to $J$, where $J = \det(\BF)$,
we note that
\Beq
    \dot{p} = \Partial{p}{J}~\dot{J} = \Partial{p}{J}~J~\Tr(\Bd) ~.
\Eeq
Therefore,
\Beq
    \boxed{
    \Partial{p}{J} = \cfrac{3~\kappa}{J} ~.
    }
\Eeq

\subsection{Mie-Gruneisen equation of state}
The pressure ($p$) is calculated using a Mie-Gr{\"u}neisen equation of state 
of the form (\cite{Wilkins99,Zocher00})
\begin{equation} \label{eq:EOSMG}
  \boxed{
  p_{n+1} =  - \frac{\rho_0~C_0^2~(1 - J^e_{n+1})
           [1 - \Gamma_0 (1 - J^e_{n+1})]}
           {[1 - S_{\alpha}(1 - J^e_{n+1})]^2} - 2~\Gamma_0~e_{n+1} 
  ~;~~~ J^e := \det{\BF^e} 
  }
\end{equation}
where $C_0$ is the bulk speed of sound, $\rho_0$ is the initial mass density,
$2~\Gamma_0$ is the Gr{\"u}neisen's gamma at the reference state,
$S_{\alpha} = dU_s/dU_p$ is a linear Hugoniot slope coefficient,
$U_s$ is the shock wave velocity, $U_p$ is the particle velocity, and
$e$ is the internal energy density (per unit reference volume), $\BF^e$ is
the elastic part of the deformation gradient.  For isochoric plasticity,
\begin{equation*}
  J^e = J = \det(\BF) = \cfrac{\rho_0}{\rho} ~.
\end{equation*}

Also,
\Beq
  \boxed{
  \Partial{p}{J^e} = 
  \cfrac{\rho_0~C_0^2~[1 + (S_{\alpha}-2~\Gamma_0)~(1-J^e)]}
        {[1-S_{\alpha}~(1-J^e)]^3} - 2~\Gamma_0~\Partial{e}{J^e}.
  }
\Eeq
We neglect the $\Partial{e}{J^e}$ term in our calculations.

  \section{Example Input Files}
  \subsection{Hypoelastic-plastic Stress Update}
  An example of the portion of an input file that specifies a copper body
  with a hypoelastic stress update, Johnson-Cook plasticity model,
  Johnson-Cook Damage Model and Mie-Gruneisen Equation of State is shown 
  below.
  \begin{verbatim}
  <material>

    <include href="inputs/MPM/MaterialData/MaterialConstAnnCopper.xml"/>
    <constitutive_model type="hypoelastic_plastic">
      <tolerance>5.0e-10</tolerance>
      <include href="inputs/MPM/MaterialData/IsotropicElasticAnnCopper.xml"/>
      <include href="inputs/MPM/MaterialData/JohnsonCookPlasticAnnCopper.xml"/>
      <include href="inputs/MPM/MaterialData/JohnsonCookDamageAnnCopper.xml"/>
      <include href="inputs/MPM/MaterialData/MieGruneisenEOSAnnCopper.xml"/>
    </constitutive_model>

    <burn type = "null" />
    <velocity_field>1</velocity_field>

    <geom_object>
      <cylinder label = "Cylinder">
        <bottom>[0.0,0.0,0.0]</bottom>
        <top>[0.0,2.54e-2,0.0]</top>
        <radius>0.762e-2</radius>
      </cylinder>
      <res>[3,3,3]</res>
      <velocity>[0.0,-208.0,0.0]</velocity>
      <temperature>294</temperature>
    </geom_object>

  </material>
  \end{verbatim}

  The general material constants for copper are in the file 
  \verb+MaterialConstAnnCopper.xml+.  The contents are shown below
  \begin{verbatim}
  <?xml version='1.0' encoding='ISO-8859-1' ?>
  <Uintah_Include>
    <density>8930.0</density>
    <toughness>10.e6</toughness>
    <thermal_conductivity>1.0</thermal_conductivity>
    <specific_heat>383</specific_heat>
    <room_temp>294.0</room_temp>
    <melt_temp>1356.0</melt_temp>
  </Uintah_Include>
  \end{verbatim}

  The elastic properties are in the file \verb+IsotropicElasticAnnCopper.xml+.
  The contents of this file are shown below.
  \begin{verbatim}
  <?xml version='1.0' encoding='ISO-8859-1' ?>
  <Uintah_Include>
    <shear_modulus>45.45e9</shear_modulus>
    <bulk_modulus>136.35e9</bulk_modulus>
  </Uintah_Include>
  \end{verbatim}
  
  The constants for the Johnson-Cook plasticity model are in the file
  \verb+JohnsonCookPlasticAnnCopper.xml+.  The contents of this file are
  shown below.
  \begin{verbatim}
  <?xml version='1.0' encoding='ISO-8859-1' ?>
  <Uintah_Include>
    <plasticity_model type="johnson_cook">
      <A>89.6e6</A>
      <B>292.0e6</B>
      <C>0.025</C>
      <n>0.31</n>
      <m>1.09</m>
    </plasticity_model>
  </Uintah_Include>
  \end{verbatim}

  The constants for the Johnson-Cook damage model are in the file
  \verb+JohnsonCookDamageAnnCopper.xml+.  The contents of this file are
  shown below.
  \begin{verbatim}
  <?xml version='1.0' encoding='ISO-8859-1' ?>
  <Uintah_Include>
    <damage_model type="johnson_cook">
      <D1>0.54</D1>
      <D2>4.89</D2>
      <D3>-3.03</D3>
      <D4>0.014</D4>
      <D5>1.12</D5>
    </damage_model>
  </Uintah_Include>
  \end{verbatim}

  The constants for the Mie-Gruneisen model (as implemented in the 
  Uintah Computational Framework) are in the file
  \verb+MieGruneisenEOSAnnCopper.xml+.  The contents of this file are
  shown below.
  \begin{verbatim}
  <?xml version='1.0' encoding='ISO-8859-1' ?>
  <Uintah_Include>
    <equation_of_state type="mie_gruneisen">
      <C_0>3940</C_0>
      <Gamma_0>2.02</Gamma_0>
      <S_alpha>1.489</S_alpha>
    </equation_of_state>
  </Uintah_Include>
  \end{verbatim}

  As can be seen from the input file, any other plasticity model, damage
  model and equation of state can be used to replace the Johnson-Cook
  and Mie-Gruneisen models without any extra effort (provided the models
  have been implemented and the data exists).

  \subsection{Hyperelastic-plastic Stress Update}
  An example of the portion of an input file that specifies a copper body
  with a hyperelastic stress update, Mechanical Threshold Stress plasticity model,
  Johnson-Cook Damage Model and Mie-Gruneisen Equation of State is shown 
  below.
  \begin{verbatim}
  <material>

    <include href="inputs/MPM/MaterialData/MaterialConstAnnCopper.xml"/>
    <constitutive_model type="hyperelastic_plastic">
      <tolerance>5.0e-10</tolerance>
      <include href="inputs/MPM/MaterialData/IsotropicElasticAnnCopper.xml"/>
      <include href="inputs/MPM/MaterialData/MTSPlasticAnnCopper.xml"/>
      <include href="inputs/MPM/MaterialData/JohnsonCookDamageAnnCopper.xml"/>
      <include href="inputs/MPM/MaterialData/MieGruneisenEOSAnnCopper.xml"/>
    </constitutive_model>

    <burn type = "null" />
    <velocity_field>1</velocity_field>

    <geom_object>
      <cylinder label = "Cylinder">
        <bottom>[0.0,0.0,0.0]</bottom>
        <top>[0.0,2.54e-2,0.0]</top>
        <radius>0.762e-2</radius>
      </cylinder>
      <res>[3,3,3]</res>
      <velocity>[0.0,-208.0,0.0]</velocity>
      <temperature>294</temperature>
    </geom_object>

  </material>
  \end{verbatim}

  The data files are the same as those for the hypoelastic-plastic case, 
  except for the plasticity model.  The constants for the MTS plasticity model
  are specified in the file \verb+MTSPlasticAnnCopper.xml+.  The contents
  of this file are show below.
  \begin{verbatim}
  <?xml version='1.0' encoding='ISO-8859-1' ?>
  <Uintah_Include>
    <plasticity_model type="mts_model">
      <s_a>40.0e6</s_a>
      <koverbcubed>0.823e6</koverbcubed>
      <edot0>1.0e7</edot0>
      <g0>1.6</g0>
      <q>1.0</q>
      <p>0.666667</p>
      <alpha>2</alpha>
      <edot_s0>1.0e7</edot_s0>
      <A>0.2625</A>
      <s_s0>770.0e6</s_s0>
      <a0>2390.0e6</a0>
      <a1>12.0e6</a1>
      <a2>1.696e6</a2>
      <b1>45.78e9</b1>
      <b2>3.0e9</b2>
      <b3>180</b3>
      <mu_0>47.7e9</mu_0>
    </plasticity_model>
  </Uintah_Include>
  \end{verbatim}

  The material data can easily be taken from a material database or specified
  for a new material in an input file kept at a centralized location.

  \bibliographystyle{unsrt}
  \bibliography{Bibliography}

\end{document}
