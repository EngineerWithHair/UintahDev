\subsection{MatlabFieldsReader}
\index{MatlabFieldsReader}
\subsubsection*{Use}
Module description


	This module takes a specially prepared MATLAB file and converts it into a SCIRun Field object. Currently only files from matlab version 5
	and higher are supported by this module. The GUI of the module lets the user choose one MATLAB file and subsequently displays all the matrices
	inside the file. Currently the implementation only shows those MATLAB matrices for which a suitable converter exists, the other ones are not
	shown. Hence, if your MATLAB matrix does not show up in the selection window the module does not know how to translate the that data set 
	into a SCIRun Field object. 
	Since, a MATLAB file can contain multiple MATLAB matrices, the module is equipped with six output ports. Depending on how the user configures the 
	module, each output port can represent one of the MATLAB matrices in the file. Hence, multiple geometries/fields can be stored in the same MATLAB
	file. 
	



 Follow the next two steps in order to setup the GUI properly: 


\begin{enumerate}

\item 
		Choose the MATLAB file that contains the geometry/field data. You can either use the BROWSE button to select a file or enter the filename in the 
		filename entry on top of the GUI and press the OPEN button to list the contents of the file in the listbox in the center of the GUI.
		
\item 
		Next the MATLAB matrices in the file need to be connected to an output port. In order to do this click on one of the PORT buttons and then
		select the MATLAB matrix you want to load on that outport. Note that for convenience the first suitable matrix in the file is automatically 
		connected to port 1, when a MATLAB file is opened. Hence, if your file only contains one field, you can skip this step 
		as the matrix will be selected automatically. 
		
\end{enumerate}
 Preparing MATLAB files for SCIRun Fields 


	One crucial step for the conversion from MATLAB to SCIRun is a proper preparation of the data files. Since a SCIRun field object is a complex
	entity, a simple numeric dense matrix is difficult to translate. Hence, it is required that you build a STRUCTURED MATRIX in matlab with some
	of the fields listed below. Based on the fields that are supplied the data is converted in one of the many types of geometries available in SCIRun.
	The module will try to match the data you supplied with the closest Field object it finds. 
	




	The following sections describe the fields of the structure matrix can be defined and are recognized by the module.
	

Unstructured Meshes 
\begin{description}

\item[ FIELDNAME .node ]  
			This field is required for unstructured meshes and defines the position of the nodes within the mesh. 
			This matrix should be a dense 3 by M matrix, where M is the number of nodes.
			
\item[ FIELDNAME .edge ]  
			This field is required for curve meshes and defines the line elements in the mesh. This matrix should be
			a dense 2 by N matrix, where N is the number of line segments. The numbers in this mesh refer to the node
			positions in the NODE matrix. By default it is assumed that the numbering of nodes starts at one. However if
			one of the indices in this EDGE matrix is zero, a zero base is assumed. 
			
\item[ FIELDNAME .face ]  
			This field is required for surfaces meshes and defines the surface elements in the mesh. This matrix should be
			a dense 3 by N matrix for triangulated meshes and a 4 by N matrix for quadsurf meshes. Here N is the number of 
			surface elements. The numbers in this mesh refer to the node
			positions in the NODE matrix. By default it is assumed that the numbering of nodes starts at one. However if
			one of the indices in this FACE matrix is zero, a zero base is assumed.
			
\item[ FIELDNAME .cell ]  
			This field is required for volume meshes and defines the volume elements in the mesh. This matrix should be
			a dense 4 by N matrix for tetrahedral meshes, or a 6 by N matrix for prism shaped volume elements, or a 8 by N matrix
			for hexahedral elements. Here N is the number of volume elements. The numbers in this mesh refer to the node
			positions in the NODE matrix. By default it is assumed that the numbering of nodes starts at one. However if
			one of the indices in this CELL matrix is zero, a zero base is assumed.
			
\end{description}
Structured Meshes
\begin{description}

\item[ FIELDNAMES x , y , AND z]  
			The fields X, Y, and Z form the description of a structured mesh. These fields are 1D, 2D, or 3D matrices defining
			the structured line, surface, or volume data. The connectivity of these meshes is defined by the position of the 
			matrix, neighboring elements are connected. In this definition matrix X defines the x cartesian co-ordinate of each node,
			matrix Y the y cartesian co-ordinate and matrix Z the z cartesian co-ordinate. This kind of definition is compatible
			with MATLAB functions such as ndmesh() and sphere(). 
			
\item[ FIELDNAME xyz ] 
			This field is an alternative for the three fields X, Y, and Z. This fields combines all three matrices in one matrix with
			the first dimension now being the cartensian co-ordinates (x, y, and z). Hence this matrix has to be 2D, 3D, or 4D, depending
			on the data contained in the mesh.
			
\end{description}
Field Data
\begin{description}

\item[ FIELDNAME .scalarfield OR .field OR .data ] 
			A matrix specifying scalar data for each node/element in the mesh. Each subsequent element in this vector is added to the
			next node/element in the field. Use the field FIELDLOCATION to specify where the data should be located. 
			
\item[ FIELDNAME .vectorfield ] 
			A matrix specifying vector data for each node/element in the mesh. This field should be a 3 by N matrix, where the first 
			dimension specifies the vector. Each subsequent column in this matrix is added to the
			next node/element in the field. Use the field FIELDLOCATION to specify where the data should be located. 
			
\item[ FIELDNAME .tensorfield ] 
			A matrix specifying tensor data for each node/element in the mesh. This field should be a 6 by N matrix for a compressed
			tensor representation [xx,yy,zz,xy,xz,yz], or a 9 by N matrix for a n uncompressed tensor representation
			[xx,xy,xz,yx,yy,yz,zx,zy,zz]. Each subsequent column in this matrix is added to the
			next node/element in the field. Use the field FIELDLOCATION to specify where the data should be located. 
			
\item[ FIELDNAME .fieldlocation ] 
			The location of the data. This field is a string describing where the data should be located.
			The default field location is assumed to be the nodes, meaning each node has a scalar, vector, or tensor value. 
			In case the data is at the nodes, this field does not to be specified.
			This field is a string with the following options: "node", "edge", "face", or "cell".
			
\end{description}


{\slshape Note: 
	The module will try to reconstruct data, for instance if a matrix is transposed, it will detect this and read the data properly.
	Most of the fields mentioned are optional and are not necessary. Only choose those fields from the list that are needed to 
	describe your data. Currently not every field type supported by SCIRun is implemented in this module. Hopefully future versions
	will support more data types and have even less restrictive converters.
	}

 Example 1: preparing MATLAB file 

 The following lines of MATLAB code demonstrate how to structure a matrix for the use in SCIRun: 



Assuming that the nodes are specified in nodematrix and the connectivity of these nodes is specified in facematrix 



 \textgreater{}\textgreater{} geom.node = nodematrix 



 \textgreater{}\textgreater{} geom.face = facematrix 



 \textgreater{}\textgreater{} save mymesh.mat geom 



 Opening the file with the MatlabFieldsReader module will show that there is one data matrix called "geom" whose contents
	is a TRISURFMESH with no data on any of the node points 



{\slshape Note: 
	In case MATLAB is not available to structure the data, use the  module to read a MATLAB matrix data directly and
	use the  module to construct a Field out of the Nrrd object. 
	}

 Example 2: Creating a structured mesh 

 The following lines of MATLAB code demonstrate the creation of a matlab file with the matlab logo on a structured mesh: 



 \textgreater{}\textgreater{} [X,Y,DATA] = peaks(100); 



 \textgreater{}\textgreater{} field.x = X; 



 \textgreater{}\textgreater{} field.y = Y; 



 \textgreater{}\textgreater{} field.z = DATA; 



 \textgreater{}\textgreater{} field.scalarfield = DATA; 



 \textgreater{}\textgreater{} save myfield field 



 This will create a surface mesh in the shape of the peaks logo and uses the z value as its data values. Be sure to specify
	all the three cartensian coordinates, omitting one will result in the module not to recognise the mesh and it will not display
	the object in its selection box.

 Exampe 3: Creating an unstructured mesh 

 The following lines of MATLAB code demonstrate the creation of a matlab fiel with an unstructured tesselated surface: 



 \textgreater{}\textgreater{} [X,Y,Z] = ndmesh(1:10,1:10,0); 



 \textgreater{}\textgreater{} field.node = [X(:)'; Y(:)'; Z(:)']; 



 \textgreater{}\textgreater{} field.face = delauney(X,Y); 



 \textgreater{}\textgreater{} field.scalarfield = X(:).\textasciicircum{}2; 



 \textgreater{}\textgreater{} save myfield2 field 

See Also 

,
	,
	,
	,
	

\subsubsection*{Outputs}

\begin{centering}
\begin{tabular}{|p{6cm}|p{6cm}|} \hline
{\emph{Name:} Field1}&{\emph{Type:} Port}\\ \hline
\multicolumn{2}{|p{12cm}|}{

 SCIRun Field imported from the MATLAB file using port 1. 

}\\ \hline
\multicolumn{2}{|p{12cm}|}{\emph{Downstream Module(s):} }\\ \hline
\end{tabular} \\
\vspace{0.25cm}
\begin{tabular}{|p{6cm}|p{6cm}|} \hline
{\emph{Name:} Field2}&{\emph{Type:} Port}\\ \hline
\multicolumn{2}{|p{12cm}|}{

 SCIRun Field imported from the MATLAB file using port 2. 

}\\ \hline
\multicolumn{2}{|p{12cm}|}{\emph{Downstream Module(s):} }\\ \hline
\end{tabular} \\
\vspace{0.25cm}
\begin{tabular}{|p{6cm}|p{6cm}|} \hline
{\emph{Name:} Fiekd3}&{\emph{Type:} Port}\\ \hline
\multicolumn{2}{|p{12cm}|}{

 SCIRun Field imported from the MATLAB file using port 3. 

}\\ \hline
\multicolumn{2}{|p{12cm}|}{\emph{Downstream Module(s):} }\\ \hline
\end{tabular} \\
\vspace{0.25cm}
\begin{tabular}{|p{6cm}|p{6cm}|} \hline
{\emph{Name:} Field4}&{\emph{Type:} Port}\\ \hline
\multicolumn{2}{|p{12cm}|}{

 SCIRun Field imported from the MATLAB file using port 4. 

}\\ \hline
\multicolumn{2}{|p{12cm}|}{\emph{Downstream Module(s):} }\\ \hline
\end{tabular} \\
\vspace{0.25cm}
\begin{tabular}{|p{6cm}|p{6cm}|} \hline
{\emph{Name:} Field5}&{\emph{Type:} Port}\\ \hline
\multicolumn{2}{|p{12cm}|}{

 SCIRun Field imported from the MATLAB file using port 5. 

}\\ \hline
\multicolumn{2}{|p{12cm}|}{\emph{Downstream Module(s):} }\\ \hline
\end{tabular} \\
\vspace{0.25cm}
\begin{tabular}{|p{6cm}|p{6cm}|} \hline
{\emph{Name:} Field6}&{\emph{Type:} Port}\\ \hline
\multicolumn{2}{|p{12cm}|}{

 SCIRun Field imported from the MATLAB file using port 6. 

}\\ \hline
\multicolumn{2}{|p{12cm}|}{\emph{Downstream Module(s):} }\\ \hline
\end{tabular} \\
\vspace{0.25cm}
\end{centering}
\subsubsection*{Gui}


Each \acronym{GUI} widget is described next.  See also Figure~.
\begin{centering}
\begin{tabular}{|p{4cm}|p{4cm}|p{4cm}|} \hline
{\emph{Label:}  .mat file }&{\emph{Widget:}  Entry }&{\emph{Data Type:} } \\ \hline
\multicolumn{3}{|p{12cm}|}{ The name of the MATLAB file } \\ \hline
\end{tabular} \\
\vspace{0.25cm}
\begin{tabular}{|p{4cm}|p{4cm}|p{4cm}|} \hline
{\emph{Label:}  Open }&{\emph{Widget:}  Button }&{\emph{Data Type:} } \\ \hline
\multicolumn{3}{|p{12cm}|}{ Opens the file specified in the MATLAB file entrybox } \\ \hline
\end{tabular} \\
\vspace{0.25cm}
\begin{tabular}{|p{4cm}|p{4cm}|p{4cm}|} \hline
{\emph{Label:}  Browse }&{\emph{Widget:}  Button }&{\emph{Data Type:} } \\ \hline
\multicolumn{3}{|p{12cm}|}{ Open a filesectionbox and choose a MATLAB file } \\ \hline
\end{tabular} \\
\vspace{0.25cm}
\begin{tabular}{|p{4cm}|p{4cm}|p{4cm}|} \hline
{\emph{Label:}  matlab matrix selector }&{\emph{Widget:}  Listbox }&{\emph{Data Type:} } \\ \hline
\multicolumn{3}{|p{12cm}|}{ Displays the matrices that are stored in the MATLAB file and an information 
	              string describing the data in the matrix} \\ \hline
\end{tabular} \\
\vspace{0.25cm}
\begin{tabular}{|p{4cm}|p{4cm}|p{4cm}|} \hline
{\emph{Label:}  Port N }&{\emph{Widget:}  Button }&{\emph{Data Type:} } \\ \hline
\multicolumn{3}{|p{12cm}|}{ Button for selecting which output port is used to export the data. The button highlighted
	white is currently selected one. Selecting a matrix from the list below will put that matrix
	on the port whose label is highlighted in white. } \\ \hline
\end{tabular} \\
\vspace{0.25cm}
\end{centering}
\subsubsection*{Credits}
 Jeroen Stinstra 

