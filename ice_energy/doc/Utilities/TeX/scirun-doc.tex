% -*-latex-*-
%
%  For more information, please see: http://software.sci.utah.edu
% 
%  The MIT License
% 
%  Copyright (c) 2004 Scientific Computing and Imaging Institute,
%  University of Utah.
% 
%  License for the specific language governing rights and limitations under
%  Permission is hereby granted, free of charge, to any person obtaining a
%  copy of this software and associated documentation files (the "Software"),
%  to deal in the Software without restriction, including without limitation
%  the rights to use, copy, modify, merge, publish, distribute, sublicense,
%  and/or sell copies of the Software, and to permit persons to whom the
%  Software is furnished to do so, subject to the following conditions:
% 
%  The above copyright notice and this permission notice shall be included
%  in all copies or substantial portions of the Software.
% 
%  THE SOFTWARE IS PROVIDED "AS IS", WITHOUT WARRANTY OF ANY KIND, EXPRESS
%  OR IMPLIED, INCLUDING BUT NOT LIMITED TO THE WARRANTIES OF MERCHANTABILITY,
%  FITNESS FOR A PARTICULAR PURPOSE AND NONINFRINGEMENT. IN NO EVENT SHALL
%  THE AUTHORS OR COPYRIGHT HOLDERS BE LIABLE FOR ANY CLAIM, DAMAGES OR OTHER
%  LIABILITY, WHETHER IN AN ACTION OF CONTRACT, TORT OR OTHERWISE, ARISING
%  FROM, OUT OF OR IN CONNECTION WITH THE SOFTWARE OR THE USE OR OTHER
%  DEALINGS IN THE SOFTWARE.
%


%
% Latex package to be used by scirun documents.  Markup commands exist
% in two files.  This one and scirun-doc.tex.  This file contains
% commands that are processed by both latex and by the builtin
% capabilities of latex2html.  Scirun-doc.sty contains commands that are
% either completely ignored by latex2html or are processed by functions in
% the latex2html extension file scirun-doc.perl.
%
% Documents should include the
% following latex commands:
%
% \usepackage{scirun-doc}
% % -*-latex-*-
%
%  The contents of this file are subject to the University of Utah Public
%  License (the "License"); you may not use this file except in compliance
%  with the License.
%
%  Software distributed under the License is distributed on an "AS IS"
%  basis, WITHOUT WARRANTY OF ANY KIND, either express or implied. See the
%  License for the specific language governing rights and limitations under
%  the License.
%
%  The Original Source Code is SCIRun, released March 12, 2001.
%
%  The Original Source Code was developed by the University of Utah.
%  Portions created by UNIVERSITY are Copyright (C) 2001, 1994
%  University of Utah. All Rights Reserved.
%

%
% Latex package to be used by scirun documents.  Markup commands exist
% in two files.  This one and scirun-doc.tex.  This file contains
% commands that are processed by both latex and by the builtin
% capabilities of latex2html.  Scirun-doc.sty contains commands that are
% either completely ignored by latex2html or are processed by functions in
% the latex2html extension file scirun-doc.perl.
%
% Documents should include the
% following latex commands:
%
% \usepackage{scirun-doc}
% % -*-latex-*-
%
%  The contents of this file are subject to the University of Utah Public
%  License (the "License"); you may not use this file except in compliance
%  with the License.
%
%  Software distributed under the License is distributed on an "AS IS"
%  basis, WITHOUT WARRANTY OF ANY KIND, either express or implied. See the
%  License for the specific language governing rights and limitations under
%  the License.
%
%  The Original Source Code is SCIRun, released March 12, 2001.
%
%  The Original Source Code was developed by the University of Utah.
%  Portions created by UNIVERSITY are Copyright (C) 2001, 1994
%  University of Utah. All Rights Reserved.
%

%
% Latex package to be used by scirun documents.  Markup commands exist
% in two files.  This one and scirun-doc.tex.  This file contains
% commands that are processed by both latex and by the builtin
% capabilities of latex2html.  Scirun-doc.sty contains commands that are
% either completely ignored by latex2html or are processed by functions in
% the latex2html extension file scirun-doc.perl.
%
% Documents should include the
% following latex commands:
%
% \usepackage{scirun-doc}
% % -*-latex-*-
%
%  The contents of this file are subject to the University of Utah Public
%  License (the "License"); you may not use this file except in compliance
%  with the License.
%
%  Software distributed under the License is distributed on an "AS IS"
%  basis, WITHOUT WARRANTY OF ANY KIND, either express or implied. See the
%  License for the specific language governing rights and limitations under
%  the License.
%
%  The Original Source Code is SCIRun, released March 12, 2001.
%
%  The Original Source Code was developed by the University of Utah.
%  Portions created by UNIVERSITY are Copyright (C) 2001, 1994
%  University of Utah. All Rights Reserved.
%

%
% Latex package to be used by scirun documents.  Markup commands exist
% in two files.  This one and scirun-doc.tex.  This file contains
% commands that are processed by both latex and by the builtin
% capabilities of latex2html.  Scirun-doc.sty contains commands that are
% either completely ignored by latex2html or are processed by functions in
% the latex2html extension file scirun-doc.perl.
%
% Documents should include the
% following latex commands:
%
% \usepackage{scirun-doc}
% \input{scirun-doc.tex}
%

% (Mostly) Short cuts  ==============================
\newcommand{\SCI}{{\em SCI}}
\newcommand{\sci}{\SCI}
\newcommand{\scii}{SCI Institute}
\newcommand{\BIOPSE}{\textbf{BioPSE}}
\newcommand{\biopse}{\BIOPSE}
\newcommand{\SR}{\textbf{SCIRun}}
\newcommand{\sr}{\SR}
\newcommand{\eg}{{\em e.g.,}}
\newcommand{\ie}{{\em i.e.,}}
\newcommand{\etc}{{\em etc.}}
\newcommand{\etal}{{\em et al.}}
\newcommand{\degrees}{{$^{\circ}$}}
\newcommand{\splitline}{\begin{center}\rule{\columnwidth}{.7mm}\end{center}}
\newcommand{\X}[1]{#1\index{#1}}
\newcommand{\srig}{\sr{} Installation Guide}
\newcommand{\srug}{\sr{} User's Guide}
\newcommand{\srdg}{\sr{} Developer's Guide}

% Literal ~ character
\newcommand{\ltilde}{\textasciitilde}

% Encloses its argument between angle brackets.
\newcommand{\ab}[1]{\textless{}#1\textgreater}

% Inserts a left angle bracket.
\newcommand{\la}{\textless}

% Inserts a right angle bracket.
\newcommand{\ra}{\textgreater}


% Mark up commands =================================

% Url
%\newcommand{\url}[1]{#1}

% ip address
\newcommand{\ipaddr}[1]{#1}
\newcommand{\localhost}{\ipaddr{127.0.0.1}}

% Predefined acronyms
\newcommand{\gui}{\acronym{GUI}}
\newcommand{\tcl}{\acronym{TCL}}
\newcommand{\xml}{\acronym{XML}}
\newcommand{\pse}{\acronym{PSE}}

% Markup the first time use of term that may be unfamiliar to the
% reader. 
\newmucmd{dfn}{emph}

% In the next command #1 is the term and #2 is the shortcut or acronym that
% will be used in the rest of the document. 
\newcommand{\dfna}[2]{\emph{#1} (\acronym{#2})}

% Markup a file name.
\newmucmd{filename}{texttt}

% Directory name markup.
\newmucmd{directory}{texttt}

% Markup text which is the name of a command.
\newmucmd{command}{texttt} 

% Command option.
\newmucmd{option}{texttt}

% Markup text typed at the keyboard.
\newmucmd{keyboard}{texttt} 

% Parameterized text - marks up text that is to be
% substituted for by the reader.
\newmucmd{ptext}{textit}        % Obsolete-use replaceable instead.
\newmucmd{replaceable}{textit}

% Markup text the user might see on his screen.
\newmucmd{screen}{texttt}

% Markup the name of a GUI menu.
\newmucmd{guimenu}{textbf}

% Markup the name of a GUI menu.
\newcommand{\menu}[1]{\guimenu{#1}}

% Markup a gui menu item name.
\newmucmd{guimenuitem}{textbf}

% Markup a menu item name.
\newcommand{\menuitem}[1]{\guimenuitem{#1}}

% Markup name of a ui button.
\newmucmd{guibutton}{textbf}

% Markup name of a ui button.
\newcommand{\button}[1]{\guibutton{#1}}

% Markup name of a gui text item.
\newmucmd{guitext}{textit}

% Markup name of a gui label.
\newmucmd{guilabel}{textbf}

% GUI variable
\newmucmd{guivar}{texttt}

% Scirun port
\newmucmd{srport}{texttt}

% Sockets port
\newcommand{\port}[1]{\texttt{#1}}

% Socket
\newcommand{\socket}[2]{\texttt{#1:#2}}

% Variable
\newcommand{\variable}[1]{\texttt{#1}}

% Data type
\newcommand{\datatype}[1]{\texttt{#1}}

% Function
\newcommand{\function}[1]{\texttt{#1}}

% Markup an inline code fragment.
\newcommand{\icode}[1]{\texttt{#1}}

% Markup a module name.
\newcommand{\module}[1]{\texttt{#1}}

% Markup a sr package name.
\newcommand{\package}[1]{\texttt{#1}}

% Markup a sr category name.
\newcommand{\category}[1]{\texttt{#1}}

% Env variable markup.
\newmucmd{envvar}{texttt}

% Markup for the title of a book or article or whatever.
\newcommand{\etitle}[1]{\texttt{#1}}

% Markup name of a latex section.  First arg is section name.  Second arg
% is section's label.  When processed with latex, the section
% name is emphasized.  When processed with latex2html section name is made
% into a link to section.
\newcommand{\secname}[2]{\latexhtml{\emph{#1}}{\htmlref{#1}{#2}}}

% Section ref command.  Use of this command
% in place of \ref to reference sections (subsections etc.) will
% create a more web friendly version of your document.
% The command's first argument is the link text (used only on the html
% page) and the second argument is the section's label
\newcommand{\secref}[2]{\hyperref[ref]{\emph{#1}}{Section~}{}{#2}}
\newcommand{\chref}[2]{\hyperref[ref]{\emph{#1}}{Chapter~}{}{#2}}

% A latex command
\newcommand{\latexcommand}[1]{\texttt{\textbackslash#1}}

% A latex package
\newcommand{\latexpackage}[1]{\textit{#1}}

% Use these in place of missing content.
\newcommand{\missing}[1]{\emph{#1 - Coming Soon.}}

% Use this to make note of incomplete content.
\newcommand{\incomplete}{\emph{More Comming Soon.}}

% Mark up a mail address
\newmucmd{mailto}{texttt}

% Mark up an xml attribute
\newmucmd{xmlattrname}{texttt}

% Sci Urls ========================================= 

% www style urls.
\newcommand{\scisoftware}{http://software.sci.utah.edu}
\newcommand{\scisoftwareurl}{\scisoftware}
\newcommand{\sciurl}{http://www.sci.utah.edu}
\newcommand{\scidocurl}{\scisoftware{}/doc}
\newcommand{\scidocurlplus}[1]{\scisoftware{}/doc/#1}
\newcommand{\bugsurl}{\scisoftware{}/bugzilla}
\newcommand{\scisoftwarearchiveurl}{\scisoftware{}/archive\_entry.html}

%

% (Mostly) Short cuts  ==============================
\newcommand{\SCI}{{\em SCI}}
\newcommand{\sci}{\SCI}
\newcommand{\scii}{SCI Institute}
\newcommand{\BIOPSE}{\textbf{BioPSE}}
\newcommand{\biopse}{\BIOPSE}
\newcommand{\SR}{\textbf{SCIRun}}
\newcommand{\sr}{\SR}
\newcommand{\eg}{{\em e.g.,}}
\newcommand{\ie}{{\em i.e.,}}
\newcommand{\etc}{{\em etc.}}
\newcommand{\etal}{{\em et al.}}
\newcommand{\degrees}{{$^{\circ}$}}
\newcommand{\splitline}{\begin{center}\rule{\columnwidth}{.7mm}\end{center}}
\newcommand{\X}[1]{#1\index{#1}}
\newcommand{\srig}{\sr{} Installation Guide}
\newcommand{\srug}{\sr{} User's Guide}
\newcommand{\srdg}{\sr{} Developer's Guide}

% Literal ~ character
\newcommand{\ltilde}{\textasciitilde}

% Encloses its argument between angle brackets.
\newcommand{\ab}[1]{\textless{}#1\textgreater}

% Inserts a left angle bracket.
\newcommand{\la}{\textless}

% Inserts a right angle bracket.
\newcommand{\ra}{\textgreater}


% Mark up commands =================================

% Url
%\newcommand{\url}[1]{#1}

% ip address
\newcommand{\ipaddr}[1]{#1}
\newcommand{\localhost}{\ipaddr{127.0.0.1}}

% Predefined acronyms
\newcommand{\gui}{\acronym{GUI}}
\newcommand{\tcl}{\acronym{TCL}}
\newcommand{\xml}{\acronym{XML}}
\newcommand{\pse}{\acronym{PSE}}

% Markup the first time use of term that may be unfamiliar to the
% reader. 
\newmucmd{dfn}{emph}

% In the next command #1 is the term and #2 is the shortcut or acronym that
% will be used in the rest of the document. 
\newcommand{\dfna}[2]{\emph{#1} (\acronym{#2})}

% Markup a file name.
\newmucmd{filename}{texttt}

% Directory name markup.
\newmucmd{directory}{texttt}

% Markup text which is the name of a command.
\newmucmd{command}{texttt} 

% Command option.
\newmucmd{option}{texttt}

% Markup text typed at the keyboard.
\newmucmd{keyboard}{texttt} 

% Parameterized text - marks up text that is to be
% substituted for by the reader.
\newmucmd{ptext}{textit}        % Obsolete-use replaceable instead.
\newmucmd{replaceable}{textit}

% Markup text the user might see on his screen.
\newmucmd{screen}{texttt}

% Markup the name of a GUI menu.
\newmucmd{guimenu}{textbf}

% Markup the name of a GUI menu.
\newcommand{\menu}[1]{\guimenu{#1}}

% Markup a gui menu item name.
\newmucmd{guimenuitem}{textbf}

% Markup a menu item name.
\newcommand{\menuitem}[1]{\guimenuitem{#1}}

% Markup name of a ui button.
\newmucmd{guibutton}{textbf}

% Markup name of a ui button.
\newcommand{\button}[1]{\guibutton{#1}}

% Markup name of a gui text item.
\newmucmd{guitext}{textit}

% Markup name of a gui label.
\newmucmd{guilabel}{textbf}

% GUI variable
\newmucmd{guivar}{texttt}

% Scirun port
\newmucmd{srport}{texttt}

% Sockets port
\newcommand{\port}[1]{\texttt{#1}}

% Socket
\newcommand{\socket}[2]{\texttt{#1:#2}}

% Variable
\newcommand{\variable}[1]{\texttt{#1}}

% Data type
\newcommand{\datatype}[1]{\texttt{#1}}

% Function
\newcommand{\function}[1]{\texttt{#1}}

% Markup an inline code fragment.
\newcommand{\icode}[1]{\texttt{#1}}

% Markup a module name.
\newcommand{\module}[1]{\texttt{#1}}

% Markup a sr package name.
\newcommand{\package}[1]{\texttt{#1}}

% Markup a sr category name.
\newcommand{\category}[1]{\texttt{#1}}

% Env variable markup.
\newmucmd{envvar}{texttt}

% Markup for the title of a book or article or whatever.
\newcommand{\etitle}[1]{\texttt{#1}}

% Markup name of a latex section.  First arg is section name.  Second arg
% is section's label.  When processed with latex, the section
% name is emphasized.  When processed with latex2html section name is made
% into a link to section.
\newcommand{\secname}[2]{\latexhtml{\emph{#1}}{\htmlref{#1}{#2}}}

% Section ref command.  Use of this command
% in place of \ref to reference sections (subsections etc.) will
% create a more web friendly version of your document.
% The command's first argument is the link text (used only on the html
% page) and the second argument is the section's label
\newcommand{\secref}[2]{\hyperref[ref]{\emph{#1}}{Section~}{}{#2}}
\newcommand{\chref}[2]{\hyperref[ref]{\emph{#1}}{Chapter~}{}{#2}}

% A latex command
\newcommand{\latexcommand}[1]{\texttt{\textbackslash#1}}

% A latex package
\newcommand{\latexpackage}[1]{\textit{#1}}

% Use these in place of missing content.
\newcommand{\missing}[1]{\emph{#1 - Coming Soon.}}

% Use this to make note of incomplete content.
\newcommand{\incomplete}{\emph{More Comming Soon.}}

% Mark up a mail address
\newmucmd{mailto}{texttt}

% Mark up an xml attribute
\newmucmd{xmlattrname}{texttt}

% Sci Urls ========================================= 

% www style urls.
\newcommand{\scisoftware}{http://software.sci.utah.edu}
\newcommand{\scisoftwareurl}{\scisoftware}
\newcommand{\sciurl}{http://www.sci.utah.edu}
\newcommand{\scidocurl}{\scisoftware{}/doc}
\newcommand{\scidocurlplus}[1]{\scisoftware{}/doc/#1}
\newcommand{\bugsurl}{\scisoftware{}/bugzilla}
\newcommand{\scisoftwarearchiveurl}{\scisoftware{}/archive\_entry.html}

%

% (Mostly) Short cuts  ==============================
\newcommand{\SCI}{{\em SCI}}
\newcommand{\sci}{\SCI}
\newcommand{\scii}{SCI Institute}
\newcommand{\BIOPSE}{\textbf{BioPSE}}
\newcommand{\biopse}{\BIOPSE}
\newcommand{\SR}{\textbf{SCIRun}}
\newcommand{\sr}{\SR}
\newcommand{\eg}{{\em e.g.,}}
\newcommand{\ie}{{\em i.e.,}}
\newcommand{\etc}{{\em etc.}}
\newcommand{\etal}{{\em et al.}}
\newcommand{\degrees}{{$^{\circ}$}}
\newcommand{\splitline}{\begin{center}\rule{\columnwidth}{.7mm}\end{center}}
\newcommand{\X}[1]{#1\index{#1}}
\newcommand{\srig}{\sr{} Installation Guide}
\newcommand{\srug}{\sr{} User's Guide}
\newcommand{\srdg}{\sr{} Developer's Guide}

% Literal ~ character
\newcommand{\ltilde}{\textasciitilde}

% Encloses its argument between angle brackets.
\newcommand{\ab}[1]{\textless{}#1\textgreater}

% Inserts a left angle bracket.
\newcommand{\la}{\textless}

% Inserts a right angle bracket.
\newcommand{\ra}{\textgreater}


% Mark up commands =================================

% Url
%\newcommand{\url}[1]{#1}

% ip address
\newcommand{\ipaddr}[1]{#1}
\newcommand{\localhost}{\ipaddr{127.0.0.1}}

% Predefined acronyms
\newcommand{\gui}{\acronym{GUI}}
\newcommand{\tcl}{\acronym{TCL}}
\newcommand{\xml}{\acronym{XML}}
\newcommand{\pse}{\acronym{PSE}}

% Markup the first time use of term that may be unfamiliar to the
% reader. 
\newmucmd{dfn}{emph}

% In the next command #1 is the term and #2 is the shortcut or acronym that
% will be used in the rest of the document. 
\newcommand{\dfna}[2]{\emph{#1} (\acronym{#2})}

% Markup a file name.
\newmucmd{filename}{texttt}

% Directory name markup.
\newmucmd{directory}{texttt}

% Markup text which is the name of a command.
\newmucmd{command}{texttt} 

% Command option.
\newmucmd{option}{texttt}

% Markup text typed at the keyboard.
\newmucmd{keyboard}{texttt} 

% Parameterized text - marks up text that is to be
% substituted for by the reader.
\newmucmd{ptext}{textit}        % Obsolete-use replaceable instead.
\newmucmd{replaceable}{textit}

% Markup text the user might see on his screen.
\newmucmd{screen}{texttt}

% Markup the name of a GUI menu.
\newmucmd{guimenu}{textbf}

% Markup the name of a GUI menu.
\newcommand{\menu}[1]{\guimenu{#1}}

% Markup a gui menu item name.
\newmucmd{guimenuitem}{textbf}

% Markup a menu item name.
\newcommand{\menuitem}[1]{\guimenuitem{#1}}

% Markup name of a ui button.
\newmucmd{guibutton}{textbf}

% Markup name of a ui button.
\newcommand{\button}[1]{\guibutton{#1}}

% Markup name of a gui text item.
\newmucmd{guitext}{textit}

% Markup name of a gui label.
\newmucmd{guilabel}{textbf}

% GUI variable
\newmucmd{guivar}{texttt}

% Scirun port
\newmucmd{srport}{texttt}

% Sockets port
\newcommand{\port}[1]{\texttt{#1}}

% Socket
\newcommand{\socket}[2]{\texttt{#1:#2}}

% Variable
\newcommand{\variable}[1]{\texttt{#1}}

% Data type
\newcommand{\datatype}[1]{\texttt{#1}}

% Function
\newcommand{\function}[1]{\texttt{#1}}

% Markup an inline code fragment.
\newcommand{\icode}[1]{\texttt{#1}}

% Markup a module name.
\newcommand{\module}[1]{\texttt{#1}}

% Markup a sr package name.
\newcommand{\package}[1]{\texttt{#1}}

% Markup a sr category name.
\newcommand{\category}[1]{\texttt{#1}}

% Env variable markup.
\newmucmd{envvar}{texttt}

% Markup for the title of a book or article or whatever.
\newcommand{\etitle}[1]{\texttt{#1}}

% Markup name of a latex section.  First arg is section name.  Second arg
% is section's label.  When processed with latex, the section
% name is emphasized.  When processed with latex2html section name is made
% into a link to section.
\newcommand{\secname}[2]{\latexhtml{\emph{#1}}{\htmlref{#1}{#2}}}

% Section ref command.  Use of this command
% in place of \ref to reference sections (subsections etc.) will
% create a more web friendly version of your document.
% The command's first argument is the link text (used only on the html
% page) and the second argument is the section's label
\newcommand{\secref}[2]{\hyperref[ref]{\emph{#1}}{Section~}{}{#2}}
\newcommand{\chref}[2]{\hyperref[ref]{\emph{#1}}{Chapter~}{}{#2}}

% A latex command
\newcommand{\latexcommand}[1]{\texttt{\textbackslash#1}}

% A latex package
\newcommand{\latexpackage}[1]{\textit{#1}}

% Use these in place of missing content.
\newcommand{\missing}[1]{\emph{#1 - Coming Soon.}}

% Use this to make note of incomplete content.
\newcommand{\incomplete}{\emph{More Comming Soon.}}

% Mark up a mail address
\newmucmd{mailto}{texttt}

% Mark up an xml attribute
\newmucmd{xmlattrname}{texttt}

% Sci Urls ========================================= 

% www style urls.
\newcommand{\scisoftware}{http://software.sci.utah.edu}
\newcommand{\scisoftwareurl}{\scisoftware}
\newcommand{\sciurl}{http://www.sci.utah.edu}
\newcommand{\scidocurl}{\scisoftware{}/doc}
\newcommand{\scidocurlplus}[1]{\scisoftware{}/doc/#1}
\newcommand{\bugsurl}{\scisoftware{}/bugzilla}
\newcommand{\scisoftwarearchiveurl}{\scisoftware{}/archive\_entry.html}

%

% (Mostly) Short cuts  ==============================
\newcommand{\SCI}{{\em SCI}}
\newcommand{\sci}{\SCI}
\newcommand{\scii}{SCI Institute}
\newcommand{\BIOPSE}{\textbf{BioPSE}}
\newcommand{\biopse}{\BIOPSE}
\newcommand{\SR}{\textbf{SCIRun}}
\newcommand{\sr}{\SR}
\newcommand{\eg}{{\em e.g.,}}
\newcommand{\ie}{{\em i.e.,}}
\newcommand{\etc}{{\em etc.}}
\newcommand{\etal}{{\em et al.}}
\newcommand{\degrees}{{$^{\circ}$}}
\newcommand{\splitline}{\begin{center}\rule{\columnwidth}{.7mm}\end{center}}
\newcommand{\X}[1]{#1\index{#1}}
\newcommand{\srig}{\sr{} Installation Guide}
\newcommand{\srug}{\sr{} User's Guide}
\newcommand{\srdg}{\sr{} Developer's Guide}

% Literal ~ character
\newcommand{\ltilde}{\textasciitilde}

% Encloses its argument between angle brackets.
\newcommand{\ab}[1]{\textless{}#1\textgreater}

% Inserts a left angle bracket.
\newcommand{\la}{\textless}

% Inserts a right angle bracket.
\newcommand{\ra}{\textgreater}


% Mark up commands =================================

% Url
%\newcommand{\url}[1]{#1}

% ip address
\newcommand{\ipaddr}[1]{#1}
\newcommand{\localhost}{\ipaddr{127.0.0.1}}

% Predefined acronyms
\newcommand{\gui}{\acronym{GUI}}
\newcommand{\tcl}{\acronym{TCL}}
\newcommand{\xml}{\acronym{XML}}
\newcommand{\pse}{\acronym{PSE}}

% Markup the first time use of term that may be unfamiliar to the
% reader. 
\newmucmd{dfn}{emph}

% In the next command #1 is the term and #2 is the shortcut or acronym that
% will be used in the rest of the document. 
\newcommand{\dfna}[2]{\emph{#1} (\acronym{#2})}

% Markup a file name.
\newmucmd{filename}{texttt}

% Directory name markup.
\newmucmd{directory}{texttt}

% Markup text which is the name of a command.
\newmucmd{command}{texttt} 

% Command option.
\newmucmd{option}{texttt}

% Markup text typed at the keyboard.
\newmucmd{keyboard}{texttt} 

% Parameterized text - marks up text that is to be
% substituted for by the reader.
\newmucmd{ptext}{textit}        % Obsolete-use replaceable instead.
\newmucmd{replaceable}{textit}

% Markup text the user might see on his screen.
\newmucmd{screen}{texttt}

% Markup the name of a GUI menu.
\newmucmd{guimenu}{textbf}

% Markup the name of a GUI menu.
\newcommand{\menu}[1]{\guimenu{#1}}

% Markup a gui menu item name.
\newmucmd{guimenuitem}{textbf}

% Markup a menu item name.
\newcommand{\menuitem}[1]{\guimenuitem{#1}}

% Markup name of a ui button.
\newmucmd{guibutton}{textbf}

% Markup name of a ui button.
\newcommand{\button}[1]{\guibutton{#1}}

% Markup name of a gui text item.
\newmucmd{guitext}{textit}

% Markup name of a gui label.
\newmucmd{guilabel}{textbf}

% GUI variable
\newmucmd{guivar}{texttt}

% Scirun port
\newmucmd{srport}{texttt}

% Sockets port
\newcommand{\port}[1]{\texttt{#1}}

% Socket
\newcommand{\socket}[2]{\texttt{#1:#2}}

% Variable
\newcommand{\variable}[1]{\texttt{#1}}

% Data type
\newcommand{\datatype}[1]{\texttt{#1}}

% Function
\newcommand{\function}[1]{\texttt{#1}}

% Markup an inline code fragment.
\newcommand{\icode}[1]{\texttt{#1}}

% Markup a module name.
\newcommand{\module}[1]{\texttt{#1}}

% Markup a sr package name.
\newcommand{\package}[1]{\texttt{#1}}

% Markup a sr category name.
\newcommand{\category}[1]{\texttt{#1}}

% Env variable markup.
\newmucmd{envvar}{texttt}

% Markup for the title of a book or article or whatever.
\newcommand{\etitle}[1]{\textit{#1}}

% Markup name of a latex section.  First arg is section name.  Second arg
% is section's label.  When processed with latex, the section
% name is emphasized.  When processed with latex2html section name is made
% into a link to section.
\newcommand{\secname}[2]{\latexhtml{\emph{#1}}{\htmlref{#1}{#2}}}

% Section ref command.  Use of this command
% in place of \ref to reference sections (subsections etc.) will
% create a more web friendly version of your document.
% The command's first argument is the link text (used only on the html
% page) and the second argument is the section's label
\newcommand{\secref}[2]{\hyperref[ref]{\emph{#1}}{Section~}{}{#2}}
\newcommand{\chref}[2]{\hyperref[ref]{\emph{#1}}{Chapter~}{}{#2}}

% An alternative sec ref command.  When processed by latex, this
% command generates text like this: #1 (Section~\ref{#2}).  When
% processed by latex2html, this command generates a normal link using
% #1 as the link text.
\newcommand{\secrefalt}[2]{\latexhtml{\etitle{#1} (Section~\ref{#2})}{\htmlref{#1}{#2}}}

% Create a link to a doc in the tree.  Create a relative link if processed by
% latex2html and a footnoted http url to the sci web site when processed by
% latex.   Note that the macro \treetop must be defined before using
% this macro.
\newcommand{\scidoclink}[2]{\htmladdnormallinkfoot{#1}{\latexhtml{\scisoftware/doc}{\treetop}/#2}}

% A latex command
\newcommand{\latexcommand}[1]{\texttt{\textbackslash#1}}

% A latex package
\newcommand{\latexpackage}[1]{\textit{#1}}

% Use these in place of missing content.
\newcommand{\missing}[1]{\emph{#1 - Coming Soon.}}

% Use this to make note of incomplete content.
\newcommand{\incomplete}{\emph{More Comming Soon.}}

% Mark up a mail address
\newmucmd{mailto}{texttt}

% Mark up an xml attribute
\newmucmd{xmlattrname}{texttt}
\newmucmd{xmlattrvalue}{texttt}

% Alternate item commands inside a description environment
\newcommand{\descitem}[1]{\item[#1]\latex{\mbox{}\\}}
\newcommand{\menudesc}[1]{\item[\menu{#1}]\latex{\mbox{}\\}}
\newcommand{\menuitemdesc}[1]{\item[\menuitem{#1}]\latex{\mbox{}\\}}
\newcommand{\buttondesc}[1]{\item[\button{#1}]\latex{\mbox{}\\}}

% Sci Urls ========================================= 

% www style urls.
\newcommand{\scisoftware}{http://software.sci.utah.edu}
\newcommand{\scisoftwareurl}{\scisoftware}
\newcommand{\sciurl}{http://www.sci.utah.edu}
\newcommand{\scidocurl}{\scisoftware{}/doc}
\newcommand{\scidocurlplus}[1]{\scisoftware{}/doc/#1}
\newcommand{\bugsurl}{\scisoftware{}/bugzilla}
\newcommand{\scisoftwarearchiveurl}{\scisoftware{}/archive\_entry.html}
