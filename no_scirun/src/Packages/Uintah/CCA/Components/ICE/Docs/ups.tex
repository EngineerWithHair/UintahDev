\documentclass[fleqn,10pt]{article}
\usepackage{amsmath}
\usepackage{verbatim}
\usepackage[pdftex]{graphicx}
\pdfcompresslevel=9
\DeclareGraphicsExtensions{{.png},{.pdf},{.jpg},{jpeg}}
\graphicspath{ {png/} {jpg/} {pdf/} }

\begin{document}
%__________________________________
%  RESET THE COUNTERS
\setcounter{equation}{0}
\setcounter{figure}{0}
\setcounter{section}{0}


\section{ICE ups parameters}
\subsection{BoundaryConditions}
\small
\begin{verbatim}
<Grid>
  <BoundaryConditions>
    <Face side = "x-">
      <BCType id = "0"   label = "Pressure"     var = "Neumann">
                            <value> 0. </value>
      </BCType>
      <BCType id = "all" label = "Velocity"     var = "Neumann">
                            <value> [0.,0.,0.] </value>
      </BCType>
      <BCType id = "all" label = "Temperature"  var = "Neumann">
                            <value> 0.0 </value>
      </BCType>
      <BCType id = "all" label = "Density"      var = "Neumann">
                            <value> 0.0 </value>
      </BCType>
      <BCType id = "all" label = "SpecificVol"  var = "Neumann">
                            <value> 0.0  </value>
      </BCType>
    </Face>
      .
      [other faces]
      .
  </BoundaryConditions>
</Grid>
\end{verbatim}
\normalfont
%__________________________________
\subsection{Timestep control and advection}
\small
\begin{verbatim}
<CFD>
  <cfl>0.4</cfl>
  <ICE>
    <advection type = "SecondOrder"/>        
  </ICE>      
</CFD>
\end{verbatim}
\normalfont

%__________________________________
\subsection{AMR related parameters}
\small
\begin{verbatim}
<AMR>
  <ICE>
    <orderOfInterpolation>  1         </orderOfInterpolation>
    <do_Refluxing>          false     </do_Refluxing>
    <Refinement_Criteria_Thresholds>
      <Variable name = "vol_frac_CC"  value = "3" matl = "all" />
    </Refinement_Criteria_Thresholds>
  </ICE>
</AMR
\end{verbatim}
\normalfont

%______________________________________________________________________
%  XML tags
\subsection{XML tag description}


\begin{center}
Boundary Condition Variables
\begin{tabular}{l p{7cm} p{7cm}}
\small{XML tag} & \small{Options}& \small{Description}\\
\hline
\hline
id          &  [0,1,2...all]                                   &   material index.\\
label       & [Temperature, Pressure, Density, SpecificVol]    &   primiative variable name\\
var         & [Neumann, Dirichlet]                             &   type of boundary condition to apply \\
value       & [ double or vector]                              &   "value" to be used when applying a BC\\
\hline
\end{tabular}
\end{center}

%__________________________________
\begin {center}
\begin{tabular}{lllp{8cm}}
\\
\small{XML tag} & \small{Type} & \small{Dimensions} & \small{Description}\\
\hline
\hline
cfl                   & double &               &    Courant Number.\\
gravity               & Vector & $[L/t^2]$     &    gravitational acceleration, $\vec{g}$.\\
\\
\underline{\small{global material properties}} & & &\\
dynamic\_viscosity    & double & $[M/Lt]$      &    viscosity, $\mu$.\\
thermal\_conditucivity& double & $[ML/t^3T]$   &    thermal conductivity, $k$\\
specific\_heat        & double & $ [L^2/t^2 T]$ &   $c_p$\\
gamma                 & double &               &    ratio of specific heats, $\gamma$.\\
\\
\underline{\small{geometry object related}} & & &\\
res                   & vector &               &    resolution used for defining geometry objects.\\
velocity              & vector & $[L/t]$       &    initial velocity, $vec{u}$.\\
density               & double & $[M/L^3]$     &    initial density, $\rho$.\\
temperature           & double & $[T]$         &    initial temperature, $T$.\\
pressure              & double &               &    Not used. \\
\\
\underline{\small{AMR Parameters}} & & & \\
orderOfInterpolation  & integer &              &    Order of interpolation at the coarse/fine interfaces. \\
do\_Refluxing         & boolean &              &    on/off switch for correcting the flux of mass, momentum, and energy at the
                                                    course/fine interfaces.\\
\hline
\end{tabular}
\end{center}

\end{document}
