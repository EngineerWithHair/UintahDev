%\documentclass[onetabnum,onefignum,final]{siamltex}
\documentclass[11pt,oneside]{article}
\usepackage[dvips]{graphicx}
%\usepackage{epsf,here,rotate,makeidx}

%%%%% My macros - get rid of them later if needed.

\newcommand{\single }{\renewcommand{\baselinestretch}{1} \large \normalsize}
\newcommand{\hdouble}{\renewcommand{\baselinestretch}{1.5} \large \normalsize}
\newcommand{\double }{\renewcommand{\baselinestretch}{1.66} \large \normalsize}

\input amssym.def
\input amssym

\def \n {{\mbox{ }}\\}
\def \l {\lambda}
\def \L {\Lambda}
\def \be {\begin{equation}}
\def \ee {\end{equation}}
\def \loc {{\mathrm{local}}}
\def \e {\varepsilon}
\def \w {\omega}
\def \W {\Omega}
\def \dW {\partial \Omega}
\def \bW {\bar{\W}}
\def \sgn {{\mathrm{sgn}}}
\def \bsigma {\bar{\sigma}}
\def \bw {\bar{\w}}
\def \bN {\bar{N}}
\def \k {\kappa}
\def \opt {{\mathrm{opt}}}
\def \te {\tilde{\e}}
\def \a { \alpha }
\def \xa {x_{\a}}
\def \g {\gamma}
\def \gsc {GS$_c$ }
\def \gsct {GS$_c$}
\def \crelax {CR${(\nu_1,\nu_2)}$ }
\def \Cc {C^{(c)}}
\def \mod {{\mbox{\rm mod }}}
\def \t { \theta }
\def \et {e^{i \t \a}}
\def \etm {e^{-i \t \a}}
\def \pc {\frac{2\pi}{c}}
\def \pcv {(2 \pi)/c}
\def \tc#1 {\t^{c,(#1)} }
\def \tctwo#1 {\t^{c_2,(#1)} }
\def \Ac#1 {A_{#1}}
\def \ec {\eta}
\def \xic {\xi}
\def \ux {\underline{x}}
\def \uy {\underline{y}}
\def \T {\Theta}
\def \B {{\cal B}}
\def \V {{\cal V}}
\def \F {{\cal F}}
\def \H {{\cal H}}
\def \K {{\cal K}}
\def \PB {{\cal P}}
\def \qed    {$\blacksquare$}
\def \bmu {\overline{\mu}}
\def \bM {\overline{M}}
\def \mod {{\mbox{\rm mod }}}
\def \lcm {{\mbox{\rm lcm}}}
\def \sa {\Sigma_{\a}}
\def \ip#1#2 {\langle #1,#2 \rangle}
\def \vec#1#2     { \left( \begin{array}{c} #1 \\ #2 \end{array} \right) }
\def \mat#1#2#3#4 {\left(\begin{array}{cc} #1 & #2\\#3 & #4\end{array}\right) }
\def \D {\Delta}
\def \px {\partial_x}
\def \py {\partial_y}
\def \pz {\partial_z}
\def \Dh {\Delta^h}
\def \pxh {\partial_x^h}
\def \pyh {\partial_y^h}
\def \pzh {\partial_z^h}
\def \AA {\hat{A}}
\def \LL {\hat{L}}
\def \MM {\hat{M}}
\def \LM {\LL \MM}
\def \eee {\hat{e}}
\def \ff {\hat{f}}
\def \uu {\hat{u}}
\def \cfl {\mathrm{CFL}}

%%%%%%%%%%%%% for siam : comment the following lines
\newtheorem{theorem}{Theorem}
\newtheorem{lemma}{Lemma}
\newtheorem{definition}{Definition}
\newtheorem{corollary}{Corollary}
\newtheorem{remark}{Remark}
\newcommand\keywordsname{Key words}
\newcommand\AMSname{AMS subject classifications}
\newenvironment{@abssec}[1]{%
     \if@twocolumn
       \section*{#1}%
     \else
       \vspace{.05in}\footnotesize
       \parindent .2in
         {\upshape\bfseries #1. }\ignorespaces
     \fi}
     {\if@twocolumn\else\par\vspace{.1in}\fi}
\newenvironment{keywords}{\begin{@abssec}{\keywordsname}}{\end{@abssec}}
\newenvironment{AMS}{\begin{@abssec}{\AMSname}}{\end{@abssec}}
%%%%%%%%%%%%% siam: comment up to here

%\def \britem {\begin{romannum}}
%\def \eritem {\end{romannum}}

\def \bn {\begin{enumerate}}
\def \en {\end{enumerate}}
\def \bi {\begin{itemize}}
\def \ei {\end{itemize}}

\title{Shocktube Problem - Status Report 4 \\ ICE Algorithm Description}
\author{Oren E.~Livne \footnotemark[1]}

\begin{document}

%\def \myspace {\double}
\def \myspace {\single}
\myspace

\maketitle

\renewcommand{\thefootnote}{\fnsymbol{footnote}}
\footnotetext[1]{SCI Institute, 50 South Central Campus Dr., Room 3490,
University of Utah, Salt Lake City,
UT 84112.Phone: +1-801-581-4772. Fax: +1-801-585-6513.
Email address: {\tt livne@sci.utah.edu}}
\renewcommand{\thefootnote}{\arabic{footnote}}

\begin{abstract}
We describe the main ICE advection algorithm for scalar advection problems
(advection and Burgers), and then for the compressible Euler system of
equations (Sod's Shocktube problem), without boundary conditions. We explain
each step in this original Kashiwa ICE algorithm \cite{kashiwa2}, 
and its relation of both the Davis
scheme \cite{davis} and the algorithm currently implemented in Uintah.
\end{abstract}

\begin{keywords}
Advection, ICE, Shocktube problem, Euler equations, limiters, Davis scheme.
\end{keywords}

%\begin{AMS}
%42A16, 42A38, 65B99, 65G99, 65N06, 65N12, 65N55.
%\end{AMS}

\pagestyle{myheadings}
\thispagestyle{plain}
\markboth{}{{\small OREN E.~LIVNE: SHOCKTUBE STATUS REPORT 4}}

\newpage
\tableofcontents

%################### BUCKY 2002 ICE ALGORITHM FOR SCALAR ###################

\newpage
\section{ICE Algorithm for Scalar Advection}
\label{BuckyScalar}

We describe and implement Bucky's ICE algorithm for scalar conservation laws,
see \cite[pp.~27--29]{kashiwa2}.
This scheme turns to be almost equivalent to Davis' scheme \cite{davis},
however, it produces better results than those reported in \cite{report_s3}
for the Davis scheme. In \S \ref{ScalarEq} we define the equation.

\subsection{The Equations}
\label{ScalarEq}
We cosnider a general one dimensional scalar advection equation,
\be
q_t + f(q)_x = q_t + (f'(q)) q_x = 0,
\label{ScalarFlux}
\ee
defined over an infinite domain, or for $x \in [0,1]$ with periodic boundary
conditions. The convective velocity is denoted by $u = u(x) = f(q(x))$.

\subsection{Discretization}
We use a finite volume discretization on a uniform grid with meshsize $\D x$.
The discrete $\{q_j\}_j$ are defined at cell centers $x_j = j \D x$;
the numerical fluxes $\{f_{j+\frac12}\}_j$ are defined at
face centers, $x_{j+\frac12} = (j+\frac12) \D x$; the fluxing velocities
$u^*_{j+\frac12}$ are defined at face centers. See Fig.~\ref{ScalarGrid}.
\begin{figure}[htbp]
   \centering
   \includegraphics[width=4.8in]{ScalarGrid.eps}
   \caption{Cell-centered discretization of (\ref{ScalarFlux}). The discrete
   state variables $q_j$ are defined at cell centers $j$ (``x'').
   Cell faces $j+\frac12$ are marked by ``$|$'', where $f$ and $u$ are
   defined.}
   \label{ScalarGrid}
\end{figure}

\subsection{Advection Scheme}
The ICE algorithm timestep for computing $q^{(n+1)}$ at time
$t_{n+1}$ at all cell centers $j$ from $q=q^n$ at all cell centers
consists of the following steps:

\bn

%%%%%%%%%%%%%%%%% STEP 1: Compute speeds %%%%%%%%%%%%%%%%%%%

\item \underline{Compute the numerical flux}:
compute the fluxing velocity $u^*_{j+\frac12}$ at all faces using an averaged
Jacobian from neighboring cell centers Jacobians.
\be
   u^*_{j+\frac12} = 0.5 \left( f'\left(q_j\right) + f'(q_{j+1})\right)
   \qquad \forall j.
   \label{ScalarUStar}
\ee

%%%%%%%%%%%%%%%%% STEP 2: Compute the numerical flux %%%%%%%%%%%%%%%%%%%

\item \underline{Compute the numerical flux}:
compute the gradient-limited fluxes at cell faces $j+\frac12$:
\begin{eqnarray}
   r^+_j & = & (q_{j+2} - q_{j+1})/(q_{j+1} - q_j) \\
   r^-_j & = & (q_j - q_{j-1})/(q_{j+1} - q_j) \\
   \phi_j & = & 2 - \max \left\{0,\min\left\{1,2 r^+_j\right\}\right\} -
                \max\left\{0,\min\left\{1,2 r^-_j\right\}\right\} \\
   \label{ScalarDelTStar}
   \D^* t_j & = & 0.5\left( (1-\phi_j) \D t +
                  \phi_j \D x/u^*_{j+\frac12} \right)\\
   f_{j+\frac12} & = & u^*_{j+\frac12} \left( 0.5 (q_j+q_{j+1})  - 
   \left(u^*_{j+\frac12} \D^* t_j / \D x \right) (q_{j+1} - q_j) \right)
   \label{ScalarICEFlux}
\end{eqnarray}

%%%%%%%%%%%%%%%%% STEP 3: Advect & Advance %%%%%%%%%%%%%%%%%%%

\item \underline{Advect and advance in time}:
\be
   q^{n+1}_j = q^n_j - 
   \frac{\D t}{\D x} \left(f_{j+\frac12} - f_{j-\frac12}\right)
   \qquad \forall j.
   \label{ScalarAdvect}
\ee

%%%%%%%%%%%%%%%%% STEP 4: Set delT %%%%%%%%%%%%%%%%%%%

\item \underline{Compute $\D t$}: the global $\D t$ should satisfy
$|u^*_{j+\frac12}| \D t/\D x \leq 1$ for all $j$ \cite[p.~29]{kashiwa2},
hence we choose
\be
   \D t = \min_j\left\{ \frac{\cfl \cdot \D x}{|u^*_{j+\frac12}| + \varepsilon}\right\},
   \label{ScalarDelT}
\ee
where $\e = 10^{-30}$ is a small number, and $0 < \cfl < 1$ is a prescribed
desired Courant number.
\en

\subsection{Numerical Results}
We define two scalar model problems,
\bi
\item[(A)] \emph{Advection:} $f(q) = q u$, where $u = 1$ is a constant
speed propagation (wave propagates to the right).
\item[(B)] \emph{Burgers:} $f(q) = q^2/2$. Here the speed propagation is
$f'(q) = q$. A shock may develop; sonic points occur when $q=0$.
\ei
We test them for two initial conditions at $t=0$:
\bi
\item[(1)] \emph{Positive square wave:}
\be
\label{PositiveSquareWave}
q(x) := \cases{
0, & if $0 \leq x < 0.3$,\cr
1, & if $0.3 < x < 0.5$,\cr
0, & if $0.5 < x \leq 1$.
}
\ee
\item[(2)] \emph{Square wave:} identical to (1), except a shift by $0.5$
so that $q$ crosses zero at two points (initially at $x=0.3,0.5$).
\be
\label{SquareWave}
q(x) := \cases{
-0.5, & if $0 \leq x < 0.3$,\cr
0.5, & if $0.3 < x < 0.5$,\cr
0.5, & if $0.5 < x \leq 1$.
}
\ee
Sonic points occur only for the case (B2). The profile $q(x)$
after $100$ timesteps for each of the four cases (A1),(A2),(B1),(B2) is shown
in Fig.~\ref{ScalarResults1}.
\ei
We discretize in space using $N = 100$ points over the interval $[0,1]$,
hence $\D x = 0.01$. We use a CFL number of $0.2$.
\begin{figure}[htbp]
\begin{center}
\includegraphics[width=0.45\textwidth]
{advection_positivesquarewave_0.20_100.eps}
\includegraphics[width=0.45\textwidth]
{advection_squarewave_0.20_100.eps}
\end{center}
\vskip -0.5cm
\centerline{(A1) \hskip 5cm (A2)}
\vskip 1cm
\begin{center}
\includegraphics[width=0.45\textwidth]
{burgers_positivesquarewave_0.20_100.eps}
\includegraphics[width=0.45\textwidth]
{burgers_squarewave_0.20_100.eps}
\end{center}
\vskip -0.5cm
\centerline{(B1) \hskip 5cm (B2)}
\caption{Results of Kashiwa ICE algorithm for scalar equations, (A1)--(B2),
after $100$ timesteps with $100$ gridpoints and CFL of $0.2$.
Top line: constant advection; bottom line: Burger's
equation. Left column: positive square wave initial data; left column:
square wave initial data.}
\label{ScalarResults1}
\end{figure}

The ICE algorithm performs well for cases (A1),(A2) and (B1). It exhibits
poor results in (B2), due to the sonic points: in (\ref{ScalarUStar}) with
$f'(q_j)$, $f'(q_{j+1}) \approx 0$, $u^*_{j+\frac12}$ may be close to zero,
and when we divide by it in (\ref{ScalarDelTStar}), we may get absurd values
for $\D^* t$. A ``sonic point fix'' (adding artificial viscosity in these
cases) is required, as indicated in \cite[pp.~8--9]{davis}.


%################### CURRENT UINTAH ALGORITHM FOR EULER ###################

\newpage
\section{Uintah ICE Algorithm for Euler}
\label{UintahAlgorithm}

We describe the ICE algorithm currently implemented in Uintah for the 1D
compressible Euler equations.

\subsection{The Equations}
\label{EulerEq}
Because the Euler equations have many formulations, we specify the unknowns
and the conservation equations they satisfy. The unknowns consist of
a state vector $Q := (\rho,\rho u,\rho i)^T$ ($\rho$ is the density,
$u$ is the velocity, $i$ is the specific internal energy), and the pressure
$p$. To set boundary conditions, we first translate these variables into
$(\rho,u,T)$ and $p$, set B.C. for those, and translates back to the state
variables $Q$. This is a separate issue that will not be investigated here.
The equations are
\be
Q_t + C(Q)_x = -P(Q),
Q :=
\left[
\begin{array}{c}
\rho\\
\rho u\\
\rho i
\end{array}
\right],
C(Q) :=
\left[
\begin{array}{c}
u \rho\\
u \rho u\\
u \rho i
\end{array}
\right],
S(Q) :=
\left[
\begin{array}{c}
0\\
p_x\\
p u_x \\
\end{array}
\right].
\label{EulerFlux}
\ee
The Equation Of State (EOS) assumes an ideal gas model,
\be
  p = (\gamma-1) \rho e
  \label{EOS}
\ee
where $\gamma=1.4$. In (\ref{EulerFlux}), $C$ is the advection flux (convective
term) and $P$ represents sources due to pressure.
Notice that $P$ is not in flux form.

\subsection{Discretization}
We define all state vector at cell centers. Intermediate values inside
the timestep are calculated at the faces for $p$ and $u$. See
Fig.~\ref{Grid}.
\begin{figure}[htbp]
   \centering
   \includegraphics[width=4.8in]{Grid.eps}
   \caption{Cell-centered discretization of (\ref{EulerFlux}). The discrete
   state variables $Q_j$ are defined at cell centers $j$ (``x'').
   Cell faces $j+\frac12$ are marked by ``$|$''.}
   \label{Grid}
\end{figure}

We assume an infinite (or periodic) grid to avoid treating boundary conditions.

\subsection{Advection Scheme}
The ICE algorithm timestep for computing the quantities at time
$t_{n+1}$ (denoted $Q^{(n+1)}$) from quantities at time $n$ (denoted
$Q$, omitting the $n$-superscript) consists of the following steps:
\bn
%%%%%%%%%%%%%%%%% STEP 1 %%%%%%%%%%%%%%%%%%%

\item \underline{Compute pressure}:
compute $p_j$ from the EOS at all cell centers $j$,
\be
   p_j = (\gamma-1) \rho_j i_j, \qquad \forall j.
   \label{Equilibration Pressure}
\ee
Then compute the speed of sound,
\be
   c_j = \left(\sqrt{\frac{\partial p}{\partial \rho} +
   \frac{\partial p}{\partial i} \frac{p}{\rho^2}}\right)_j =
   \sqrt{(\gamma-1) \left(i_j + \frac{p_j}{\rho_j} \right)} , \qquad \forall j.
   \label{SpeedSound}
\ee

%%%%%%%%%%%%%%%%% STEP 2 %%%%%%%%%%%%%%%%%%%

\item \underline{Compute the face-centered velocities}:
we denote them by $u^*_{j+\frac12}$, at all faces $j+\frac12$. 
\be
   u^*_{j+\frac12} =
   \frac{\rho_j u_j + \rho_{j+1} u_{j+1}}{\rho_j + \rho_{j+1}} -
   \D t \frac12 \left( \frac{1}{\rho_j} + \frac{1}{\rho_{j+1}} \right)
   \frac{p_{j+1} - p_j}{\D x}, \qquad \forall j.
\ee

%%%%%%%%%%%%%%%%% STEP 3 %%%%%%%%%%%%%%%%%%%

\item \underline{Compute pressure correction and update pressure}.
Note that the correction is defined at cell centers.
\be
  (\D P)_j =
  -\D t c_j^2 \rho_j ADV(vol,u^*)_j, \qquad \forall j.
  \label{DelP}
\ee
\be
   p_j = p_j + (\D P)_j, \qquad \forall j.
   \label{UpdatePress}
\ee
Here $vol$ is the volume fraction of the material, defined at each cell center.
In this setting, we have one material, so $vol \equiv 1$.
The notation $ADV(a,u^*)$ is the discrete advection operator of the
function $a=a(x)$ using the velocities $u^*$ at the faces. It will be
further explained later on.

%%%%%%%%%%%%%%%%% STEP 4 %%%%%%%%%%%%%%%%%%%

\item \underline{Compute the face-centered pressure}:
denoted by $p^*_{j+\frac12}$, and computed at all faces $j+\frac12$ as
averages of the cell-centered pressure.
\be
   p^*_{j+\frac12} =
   \frac{\rho_{j+1} p_j + \rho_{j} p_{j+1}}{\rho_j + \rho_{j+1}}.
   \label{FacePressure}
\ee

%%%%%%%%%%%%%%%%% STEP 5 %%%%%%%%%%%%%%%%%%%

\item \underline{Compute Lagrangian quantities}: these values seem to be
the values of the cell-centered state variables, which are advected along
the characteristics of the system.
\begin{eqnarray}
   \rho^L_j     & = & \rho_j \\
   \label{Sources1}
   (\rho u)^L_j & = & \rho_j u_j - \D t \frac{p^*_{j+\frac12} -
               p^*_{j-\frac12}}{\D x} \\
   \label{Sources2}
   (\rho i)^L_j & = & \rho_j i_j - \D t p_j ADV(vol,u^*)_j
   \label{Sources3}
\end{eqnarray}
for all cell centers $j$.

%%%%%%%%%%%%%%%%% STEP 6 %%%%%%%%%%%%%%%%%%%

\item \underline{Advect and advance in time}: we pass back from the
Lagrangian coordinate system to the cell centers at time $n+1$.
\begin{eqnarray}
   \rho^{n+1}_j     & = & \rho^L_j - \D t ADV(\rho^L,u^*)_j \\
   (\rho u)^{n+1}_j & = & (\rho u)^L_j - \D t ADV((\rho u)^L,u^*)_j \\
   (\rho i)^{n+1}_j & = & (\rho i)^L_j - \D t ADV((\rho i)^L,u^*)_j. \\
   \label{Advection}
\end{eqnarray}
for all cell centers $j$.

%%%%%%%%%%%%%%%%% STEP 7: Set delT %%%%%%%%%%%%%%%%%%%

\item \underline{Compute $\D t$}: the next timestep is computed similarly
to (\ref{ScalarDelT}). Based on a user-specified Courant number $0 < \cfl < 1$,
\be
   \D t = \min_j\left\{ \frac{\cfl \cdot \D x}
   {|u^*_{j+\frac12} + c_j| + \varepsilon}\right\},
   \label{ICEDelT}
\ee
where $\e = 10^{-30}$ is a small number.
\en

\subsubsection{The $ADV$ operator}
This section is based on the Uintah code. It turns out that it contradicts
the description in \cite{kashiwa2}. As we will see, it is probable that
the paper is wrong and the code is correct. 

The advection operator $ADV(q,u^*)$ takes cell centered values $\{q_j\}_j$
of an advected quantity $q=q(x)$, and returns a correction defined at cell
centers. When this correction is multiplied by $\D t$ and subtracted,
we obtain $\{q^{(n+1)}_j\}_j$.
$ADV$ requires the face-centered velocities $\{u^*_{j+\frac12}\}_j$.
$ADV(q,u^*)_j$ can be interpreted as discrete divergence operator centered
at cell $j$.

\bi
\item \emph{Viewpoint 1: moving ``slabs''.} Geometrically, we can
approximate the amount of volume advected in and out every cell during
a $\D t$ time interval; the volumes depend on $\D x$, $\D t$ and the $u^*$'s,
but \emph{independent of $\{q_j\}_j$}. Then, we approximate the average of
$q$ on each slab to get the total in-flux and out-flux of $q$ to/from 
cell $j$. The resulting formula can thus be applied to all state variables
(substituted for $q$).

Thus, $ADV$ is given by
\be
ADV(q,u^*)_j := -\frac{1}{\D x \D t}
\left( q_{in} V_{in} - q_{out} V_{out} \right),
\label{InOut}
\ee
where $V_{in}$ is the slab volume advected into cell $j$, $q_{in}$ is
an average of $q$ over this in-slab, $V_{out}$ is the slab volume advected
outside cell $j$, and $q_{out}$ is the average of $q$ over the out-slab.

Fig.~\ref{Slabs} illustrates the case where $u^*_{j-\frac12}>0$,
$u^*_{j+\frac12}>0$. The in-slab is a volume of cell $j-1$ entering cell $j$,
and the out-slab is a volume of cell $j$ leaving cell $j$ and entering
cell $j+1$. Assuming that $u^*_{j+\frac12}$ is the average velocity at
the face $j+\frac12$ over the time period $[t_n,t_{n+1}]$, 
$V_{in,j} = \D t u^*_{j-\frac12}$ and $V_{ou,jt} = \D t u^*_{j-\frac12}$; 
if $u^*_{j+\frac12}$ is a velocity value at the face at $t_n+\frac12 \D t$,
the slab volumes are exact to $O(\D t^2)$. The average of $q$ over $V_{in}$
is approximated to $O(\D x^2)$ by the value $q^S_j$ of $q$ in the middle
of the slab.

\begin{figure}[htbp]
   \centering
   \includegraphics[width=4.8in]{Slabs.eps}
   \caption{The in-slab and out-slab of cell $j$ for the case of
   $u^*_{j-\frac12}>0$, $u^*_{j+\frac12}>0$.}
   \label{Slabs}
\end{figure}

In general, the in-flux is a summation of the in-flux through face
$j-\frac12$ and $j+\frac12$. If we define
\be
u^+_{j+\frac12} := \max\left\{ u^*_{j+\frac12}, 0 \right\}, \qquad
u^-_{j+\frac12} := \min\left\{ u^*_{j+\frac12}, 0 \right\}
\ee
then
\begin{eqnarray}
V_{in,j} &=& q^S_{j-1} \D t u^+_{j-\frac12} - q^S_{j+1} \D t u^-_{j+\frac12}\\
V_{out,j} &=& -q^S_{j} \D t u^-_{j-\frac12} + q^S_{j} \D t u^+_{j+\frac12}.\\
\end{eqnarray}
Substituting into (\ref{InOut}) gives
\begin{eqnarray}
\nonumber
ADV(q,u^*)_j & = & -\frac{1}{\D x} \left( 
                   q^S_{j-1} u^+_{j-\frac12} -
                   q^S_{j+1} u^-_{j+\frac12} + \right. \\
&&                 \left. q^S_{j} u^-_{j-\frac12} - 
                   q^S_{j} u^+_{j+\frac12} \right).
\label{AdvInOut}
\end{eqnarray}

\item \emph{Viewpoint 2: conservation form.} As in (\ref{ScalarAdvect}),
we can rearrange (\ref{AdvInOut}) in terms of face $j-\frac12$, $j+\frac12$
contributions rather than ``in'' and ``out'' contributions. Namely,
if
\be
ADV(q,u^*) = \frac{1}{\D x} \left(C^*_{j+\frac12} - C^*_{j-\frac12}\right),
\ee
where the numerical flux is
\be
C^*_{j+\frac12} := q_j^S u^+_{j+\frac12} + q^S_{j+1} u^-_{j+\frac12}.
\label{AdvFlux}
\ee
\ei

Note that when $q \equiv 1$, $C^*_{j+\frac12} =
u^+_{j+\frac12} + u^-_{j+\frac12} = u^*_{j+\frac12}$, and $ADV$ becomes
the discrete divergence operator.

The advantage of viewpoint 1 is that the same $ADV$ operator
(except for the limiter) applies to all state variables in the Euler equations,
as they are all advected with the same velocity $u^*$. The only quantities
that need to be computed for each new variable are the slab averages
$\{q^S_j\}_j$.

The advantage of viewpoint 2 is that it allows a direct relation to
schemes based on Riemann solvers. (\ref{AdvFlux}) is a special
case of a flux-vector splitting \cite[p.~83, (4.56)]{leveque2}.

We now describe the choice of $q^S_j$.
\bi
\item \emph{First order.} Here
\be
q^S_j := q_j.
\label{qslab1}
\ee
Thus (\ref{AdvFlux}) is
an upwind flux \cite[p.~75, (4.33)]{leveque2}. More generally, it can be
viewed as a Godunov method, where $q^S_j$ represents the value
solution $\hat{q}$ of the Riemann problem with piecewise constant initial data
$q_{j-1},q_j$ on the left and the right of the face $j+\frac12$, evaluated
at mid-time (i.e., $q^S_j = \hat{q}((j+\frac12) \D x, (n+\frac12) \D t)$).
\item \emph{Second order.} Here we use
\be
q^S_j := q_j + r_j \frac{q_{j+1} - q_{j-1}}{2 \D x},
\label{qslab2}
\ee
where $r_j := |\D x/2 - u^*_{j+\frac12} \D t/2|$ is the ``centroid'' vector,
that is, the vector pointing from the cell center to the slab center.
Following the derivation of \cite{kashiwa1} and \cite[\S 6.4]{leveque2},
we see that (\ref{AdvFlux}) is equivalent to a Godunov method based on
piecewise linear reconstruction of $q$. (\ref{qslab2}) is a second-order
approximation to the value of $q$ at the center of the slab.
\item \emph{Limited.} This combines the first and second order methods using
\be
q^S_j := q_j + r_j \phi_j \frac{q_{j+1} - q_{j-1}}{2 \D x},
\label{qslablimited}
\ee
$\phi_j$ is either a van-Leer gradient limiter, when advecting volume
fraction and density, or a compatible flux limiter \cite{kashiwa1}, for
all other advected quantities.
\ei

\subsection{Scheme Analysis}
We reformulate ICE in terms of the state variables $Q$ only. The target is
to find the operator transforming $Q^n$ into $Q^{n+1}$.
The ICE timestep can be rewritten as follows.
\bn
\item Advance pressure to time $n$ using the EOS ($p^n$).
\item Compute half-space, [locally] time advanced quantities $u^*$, $p^*$.
Let
\begin{eqnarray}
A(q)_{j+\frac12} & := & \frac12 \left( q_{j+1} + q_{j} \right) \\
D(q)_{j+\frac12}   & := & \frac12 \left( q_{j+1} - q_{j} \right);
\end{eqnarray}
for all $j$; then in vector form,
\begin{eqnarray}
  u^* & = & \frac{A(\rho u)}{A(\rho)} - \D t A\left(\frac{1}{\rho}\right) D(p^n) \\
  \label{uStar}
  \D p & = & -\D t c^2 \rho \left[ D\left(\frac{A(\rho u)}{A(\rho)}\right) -
  \D t D\left( A \left(\frac{1}{\rho}\right) D(p^n) \right) \right] \\
  \label{Dp}
  p^{n+1} & = & p^n - \D t c^2 \rho D\left( \frac{A(\rho u)}{A(\rho)} \right) +
  \D t^2 c^2 \rho D\left( A \left(\frac{1}{\rho}\right) D(p^n) \right) \\
  \label{PCC}
  p^* & = & \frac{1}{A\left(\frac{1}{\rho}\right)} \left[ A\left(\frac{p^{n}}{\rho} \right)
  - \D t c^2 D\left( \frac{A(\rho u)}{A(\rho)} \right) +
  \D t^2 c^2 D\left( A \left(\frac{1}{\rho}\right) D(p^n) \right) \right].
  \label{pStar}
\end{eqnarray}

\item \emph{Lagrangian phase:} Advance in time based on the sources,
\be
  Q^L = Q^n - \D t LAG(Q^n,u^*,p^*)
  \label{Lagrangian}
\ee
\item \emph{Eulerian phase:} advect forward in time. The advection operator
is based on the already-computed Lagrangian values.
\be
  Q^{n+1} = Q^L - \D t ADV(Q^L,u^*)
  \label{Eulerian}
\ee
\en
The Lagrangian operator advances $Q$ using the equation $Q_t = -S(Q)$
(without advection). Discretizing $S$ (see (\ref{EulerFlux})) at cell
centers (using differencing of the face-centered $u^*,p^*$, we define
\be
  LAG(Q)_j := 
  \left[
  \begin{array}{c}
  0\\
  \frac{p^*_{j+\frac12} - p^*_{j-\frac12}}{\D x} \\
  p^{n+1}_j \left( u^*_{j+\frac12} - u^*_{j-\frac12} \right)_j
  \end{array}
  \right].
\ee
Note that $LAG$ contains two pressures: face-centered $p^*$ and cell-centered
$p^{n+1}$. Both are time-advanced. In ICE we are not really conforming to
Bucky's page-$29$ scheme for $p^*$, rather, we average time-$n+1$ pressure
values (equivalent to $\D^*_p t = \D t$ and taking the average over the entire
right-hand-side for the $p^*$ equation on page 29).

\subsection{Numerical Results}
Our initial data is the shock tube piecewise constant data depicted
in Fig.~\ref{ShockTubeInitialData}.
\begin{figure}[tbp]
   \centering
   \includegraphics[width=4.8in]{iceInitialData.eps}
   \caption{Initial data for the compressible Euler equations (shock tube
   problem).}
   \label{ShockTubeInitialData}
\end{figure}

We tested the ICE algorithm using $\cfl = 0.45$ and
an initial $\D t = 10^{-6}$ for the first timestep (subsequent timesteps
are computed using (\ref{ICEDelT})). We discretize in space using $N = 100$
points over the interval $[0,1]$, hence $\D x = 0.01$.
The state variable profiles after $100$ timesteps are depicted in
Fig.~\ref{ICEResultLimiter}. We tested the same algorithm with first order
advection (\ref{qslab1}). The results are depicted in Fig.~\ref{ICEResultFO}.
\begin{figure}[htbp]
\begin{center}
\includegraphics[width=0.8\textwidth]{iceResult1Limiter.eps}
\end{center}
%\vskip -0.5cm
\begin{center}
\includegraphics[width=0.8\textwidth]{iceResult2Limiter.eps}
\end{center}
%\vskip -0.5cm
\caption{Results of Kashiwa ICE algorithm for the shock tube problem
with a limiter,
after $100$ timesteps with $100$ gridpoints and CFL of $0.45$. The top
two lines show $\rho,u_1\equiv u, T$ and $p$ vs. $x$. The last line
shows the internal variables $\D P$ and the speed of sound $c$ vs. $x$.}
\label{ICEResultLimiter}
\end{figure}
\begin{figure}[htbp]
\begin{center}
\includegraphics[width=0.8\textwidth]{iceResult1FO.eps}
\end{center}
%\vskip -0.5cm
\begin{center}
\includegraphics[width=0.8\textwidth]{iceResult2FO.eps}
\end{center}
%\vskip -0.5cm
\caption{The same as Fig.~\ref{ICEResultLimiter}, for first order advection
in $ADV$.}
\label{ICEResultFO}
\end{figure}

Note that
\begin{itemize}
\item All state variables have oscillations near the shocks, for first
order advection. I cannot explain this result - shouldn't the
first order advection operator $ADV$ be monotone (even though it's a system
of equations, not a scalar conservation law)?
\item First order advection is much more diffusive, which is to be expected.
\item The oscillation appear also for the second order limited case,
thus we might be led to thinking that the limiter is not working properly.
But even if it would reduce the advection order to first near shocks, as
we already saw, oscillations still occur. So the limiter might not be
the problem, although we do not know for sure before we resolve the
first order oscillations issue.
\end{itemize}

%################### MY CONSERVATIVE VERSION OF THE UINTAH ALGORITHM FOR EULER ###################

\newpage
\section{Conservative ICE Algorithm for Euler}
\label{ConservativeICE}

\subsection{The Equations}
\label{ConservativeEulerEq}
The difference between the algorithm in this section and the ICE algorithm
in \S \ref{UintahAlgorithm} is the formulation of the energy equation. We use
the conservative form of the Euler system, replacing $i$ (specific internal
energy) by $e$ (specific energy), where $e = i + \frac12 u^2$. We do not
deal here with the implications of this change on the boundary conditions.
Working in this form allows us to compare ICE to the very similar Zha-Bilgen
flux vector splitting described in \cite{zb}.

The equations are
\be
Q_t + F(Q)_x = 0,
\qquad
Q :=
\left[
\begin{array}{c}
\rho\\
\rho u\\
\rho e
\end{array}
\right],
\qquad
F(Q) :=
\left[
\begin{array}{c}
u \rho\\
\rho u^2 + p\\
(\rho e + p) u
\end{array}
\right],
\label{ConservativeEulerFlux}
\ee
The Equation Of State (EOS) assumes an ideal gas model,
\be
  p = (\gamma-1) \rho \left( e - \frac12 u^2 \right)
  \label{EOS2}
\ee
where $\gamma=1.4$. The eigenvalues of the Jacobian $\partial F/\partial Q$
has three real eigenvalues, $u, u+c, u-c$, where $c = \sqrt{\gamma p/\rho}$
is the speed of sound. We split $F$ as in \cite[eq.~(7)]{zb},
\be
  F = C + P, \qquad 
  C := 
  u
  \left[
  \begin{array}{c}
  \rho\\
  \rho u\\
  \rho e
  \end{array}
  \right],
  \qquad
  P := 
  \left[
  \begin{array}{c}
  0\\
  p\\
  p u
  \end{array}
  \right].
  \label{FluxSplit}
\ee
In (\ref{FluxSplit}), $F$ is total flux, $C$ is the convective flux and
$P$ is the pressure term, now in flux form (compare with (\ref{EulerFlux})).
The eigenvalues of $C$ are $u,u,u$ and $P$'s eigenvalues are $0,c,-c$.

\subsection{Discretization}
$Q$ and $p$ are defined at cell centers. We will again use $u^*$ and $p^*$ at
the face centers. We assume an infinite (or periodic) grid to avoid treating
boundary conditions.

\subsection{Advection Scheme}
In this case we can write ICE in discrete conservative form. ICE's operator
splitting consists of the stages
\begin{eqnarray}
  p^{n} & = & p\left(Q)\right) \quad {\mbox{(Evaluate EOS)}} \\
  \label{EvalEOS}
  && {\mbox{Compute $u^*,p^{n+1},p^*$}} \\
  Q^L & = & Q^n - \D t LAG(Q^n,u^*,p^*) \quad {\mbox{(Lagrangian phase)}} \\
  \label{LAG}
  Q^{n+1} & = & Q^L - \D t ADV(Q^L,u^*) \quad {\mbox{(Eulerian phase)}}
  \label{ADV}
\end{eqnarray}
$u^*,p^{n+1},p^*$ are computed as in the Uintah ICE algorithm.
The Lagrangian phase is now cast in terms of $(\rho,\rho u,\rho e)$ rather
than $(\rho,\rho u,\rho i)$. Instead of (\ref{Sources1})--(\ref{Sources3})
we have
\begin{eqnarray}
   \rho^L_j     & = & \rho_j \\
   \label{LAG1}
   (\rho u)^L_j & = & \rho_j u_j - \D t \frac{p^*_{j+\frac12} -
               p^*_{j-\frac12}}{\D x} \\
   \label{LAG2}
   (\rho e)^L_j & = & \rho_j e_j - \D t ADV(p^{(n+1)},u^*)_j.
   \label{LAG3}
\end{eqnarray}
Thus, the Lagrangian phase now has the numerical flux form
\begin{eqnarray}
  LAG(Q,u^*) & = &
  \frac{1}{\D x} \left(P^*_{j+\frac12} - P^*_{j-\frac12}\right), \\
  \label{LAGFluxForm}
  P^*_{j+\frac12} & := &
  \left[
  \begin{array}{c}
  0\\
  p^*_{j+\frac12} \\
  u^+_{j+\frac12} p^{n+1,S} + u^-_{j+\frac12} p^{n+1,S}
  \end{array}
  \right].
  \label{PStar}
\end{eqnarray}
The convective numerical flux is given as before (see (\ref{AdvFlux})) by
\begin{eqnarray}
  ADV(Q,u^*) & = &
  \frac{1}{\D x} \left(C^*_{j+\frac12} - C^*_{j-\frac12}\right), \\
  \label{ADVFluxForm}
  C^*_{j+\frac12} & := & u^+_{j+\frac12} Q_j^S + u^-_{j+\frac12} Q^S_{j+1}.
  \label{CStar}
\end{eqnarray}
Thus, the total numerical flux of the scheme is
\begin{eqnarray}
\nonumber
F^*_{j+\frac12} & = & C^*_{j+\frac12} + P^*_{j+\frac12} \\
\nonumber
  & = &
  u^+_{j+\frac12}
  \left[
  \begin{array}{c}
  \rho_j^S \\
  (\rho u)_j^S \\
  (\rho e)_j^S
  \end{array}
  \right]
  +
  u^-_{j+\frac12}
  \left[
  \begin{array}{c}
  \rho_{j+1}^S \\
  (\rho u)_{j+1}^S \\
  (\rho e)_{j+1}^S
  \end{array}
  \right] \\
  && +
  \left[
  \begin{array}{c}
  0\\
  p^*_{j+\frac12} \\
  u^+_{j+\frac12} p^{n+1,S} + u^-_{j+\frac12} p^{n+1,S}
  \end{array}
  \right].
\end{eqnarray}

\subsection{Numerical Results}
Our initial data is the shock tube piecewise constant data depicted
in Fig.~\ref{ShockTubeInitialData}.
We tested the ICE algorithm using $\cfl = 0.45$ and
an initial $\D t = 10^{-6}$ for the first timestep (subsequent timesteps
are computed using (\ref{ICEDelT})). We discretize in space using $N = 100$
points over the interval $[0,1]$, hence $\D x = 0.01$.
The state variable profiles after $100$ timesteps are depicted in
Fig.~\ref{ICEResultLimiter}. We tested first order advection only.
The results are depicted in Fig.~\ref{ICEConservativeResultFO}.
These are initial results only. The same wiggles occur in here as in Uintah
ICE. However, the velocity profile seems to be wrong, unlike the Uintah
results.
\begin{figure}[htbp]
\begin{center}
\includegraphics[width=0.8\textwidth]{iceConservativeResult1FO.eps}
\end{center}
%\vskip -0.5cm
\begin{center}
\includegraphics[width=0.8\textwidth]{iceConservativeResult2FO.eps}
\end{center}
%\vskip -0.5cm
\caption{The same as Fig.~\ref{ICEResultLimiter}, for first order advection
in $ADV$.}
\label{ICEConservativeResultFO}
\end{figure}

%################### SUMMARY AND QUESTIONS ###################

\newpage
\section{Concluding Remarks and Questions for Bucky}
\begin{itemize}
\item The scalar ICE algorithm works, except near sonic points, where we
have to add more diffusion. This action item can be treated separately
of the rest of the ``machinery''.
\item First order advection in ICE produces oscillations. Can we show that
it must be monotone? If not, what is the point of using a limiter to reduce
the advection order from second to first in non-smooth regions, if we cannot
anyway obtain a monotone profile using first order?
\item The precise role of $u^*$ and $p^*$ from \cite[p.~29]{kashiwa2} in
the ICE advection scheme is still not part of the
description of this report and should be added as soon as possible as it
might be the key to the questions above.
\item Should $p^*$ be based on old pressures (time $n$) or new pressure
averages (time $n+1$)? Bucky has an old pressure average when $\D^* p_t = 0$
pn page $29$, and again uses $p^L = p$ for explicit timestepping on pages
$41-42$, which are averaged to give $p^*$.
\item Can the entire ICE algorithm be written in conservation form (i.e.,
a grand numerical flux combining both the Lagrangian and Eulerian phases)?
\end{itemize}

\bibliographystyle{alpha}
\bibliography{reports}        % guide.bib is the name of our database

\end{document}
