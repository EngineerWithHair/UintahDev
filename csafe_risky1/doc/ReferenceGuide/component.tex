                                %
                                % component.tex
                                %
                                % Documentation for component specification xml documents.  Also see
                                % component.dtd. 
                                %
                                % * tjd *
                                %

\documentclass{article}
\usepackage{html}
\parindent 0in
\setlength{\parskip}{\medskipamount}

                                % Custom markup commands for this document.
\newcommand{\mytilde}{\symbol{'176}}
\newcommand{\ab}[1]{\latexhtml{$<$#1$>$}{<#1>}}
\newcommand{\la}{\latexhtml{$<$}{<}}
\newcommand{\ra}{\latexhtml{$>$}{>}}
\newcommand{\acronym}[1]{#1}
\newcommand{\dfn}[1]{\emph{#1}}
\newcommand{\filename}[1]{\texttt{#1}}
\newcommand{\xml}{\acronym{XML}}
\newcommand{\Xml}{\xml}
\newcommand{\psgml}{\acronym{PSGML}}
\newcommand{\Psgml}{\psgml}
\newcommand{\dtd}{\acronym{DTD}}
\newcommand{\gui}{\acronym{GUI}}
\newcommand{\pse}{\acronym{PSE}}
\newcommand{\Pse}{\pse}
\newcommand{\sci}{\acronym{SCI}}
\newcommand{\Sci}{\sci}
\newcommand{\cvs}{\acronym{CVS}}
\newcommand{\Cvs}{\cvs}
\newcommand{\emacs}{[x]emacs}
\newcommand{\Emacs}{[X]Emacs}
\newcommand{\sectitleref}[1]{\emph{#1}}
\newcommand{\element}[1]{\ab{\texttt{#1}}}
\newcommand{\elementitem}[1]{\item[\textit{\ab{#1}}]}
\newcommand{\keyboard}[1]{\texttt{#1}}
\newcommand{\screen}[1]{\texttt{#1}}
\newcommand{\menuitem}[1]{\textit{\textbf{#1}}}
\newcommand{\sciurl}{http://www.sci.utah.edu}
\newcommand{\psgmlurl}{http://www.lysator.liu.se/projects/about\_psgml.html}
\newcommand{\Note}[1]{\emph{Note: #1}}
\newcommand{\icode}[1]{\texttt{#1}}

                                % Section title commands (What a pain!)
\newcommand{\SECintro}{Introduction}
\newcommand{\SUBSECwhySpec}{Why a Component Specification?}
\newcommand{\SUBSECwhyXml}{Why \xml{?}}
\newcommand{\SUBSECwhatDtd}{What's a \dtd?}
\newcommand{\SECcontent}{Structure and Logical Content}
\newcommand{\SUBSECrules}{Rules}
\newcommand{\SUBSECstructContent}{Structure and Content}
\newcommand{\SUBSECdescElement}{A Description of the \ab{description} Element}
\newcommand{\SUBSUBSECcommonUsage}{Common Usage}
\newcommand{\SUBSUBSECdescElementIntro}{Introduction}
\newcommand{\SUBSUBSECreference}{Reference}
\newcommand{\SECexample}{Example}
\newcommand{\SECediting}{Editing the Spec}
\newcommand{\SUBSECgettingSources}{PSGML Mode - Do You Have It?}
\newcommand{\SUBSECdotEmacs}{PSGML Mode - .emacs}
\newcommand{\SUBSECgettingStarted}{PSGML Mode - Getting Started}
\newcommand{\SUBSECvalidation}{Validation}
\newcommand{\SECtools}{Tools}
\newcommand{\SUBSECmakeDevHtmlDoc}{Making a Developer Document (HTML)}
\newcommand{\SUBSECmakeUserHtmlDoc}{Making a User Document (HTML)}
                                % End of section title commands.

\title{Component \xml Spec Guide} 
\author{\htmladdnormallinkfoot{Ted Dustman}{mailto:dustman@cvrti.utah.edu}}

\begin{document}
\maketitle \pagenumbering{roman}
\label{toc}
\tableofcontents
\pagebreak \pagenumbering{arabic}

\section{\htmlref{\SECintro}{toc}}
\label{\SECintro}

\subsection{\htmlref{\SUBSECwhySpec}{toc}}
\label{\SUBSECwhySpec}

The component specification serves as documentation and data.  

It documents the use, design, implementation, and testing of a component.
Various forms of content (e.g. HTML) is automatically generated from
the component specification.

It serves as data for applications and databases.  For example:

\begin{itemize}
\item The collection of all component specifications will be part of a
  searchable database.
  
\item The component specification will help drive the component code generation
  process.

\item The component specification may drive an automated testing process.
\end{itemize}

Component specifications are written in a special markup language.  This
language (call it the component specification markup language if you
like) is in turn formulated using the \dfn{eXtensible Markup Language}
(\acronym{XML}).

\Note{Components that are not fully ``spec''ed will not be incorporated into official
  releases of the \pse\ software.}

\subsection{\SUBSECwhyXml}
\label{\SUBSECwhyXml}

An \xml\ formulation of the component specification is used to achieve
the following goals:

\begin{description}
\item[Data centric viewpoint] We want the component specification to represent
  a set of data that can be used in several contexts.

\item[Uniformity] We want all components to be specified the same way.

\item[Validation] We want to validate (check for correctness and
  completeness) component specifications.
\end{description}

The component specification markup language is similar in concept to HTML but
has a different (and much smaller) set of tags and a simpler and more
regular set of document composition rules.
Section~\hyperref{\sectitleref{\SECcontent}}{}{}{\SECcontent} describes
the component specification markup language in detail.

Note that in the \xml\ world tags are used to delimit (or identify)
\dfn{elements}.  Every start tag \emph{must} have a corresponding end tag.
Elements are the basic building blocks of an \xml\ document.

\subsection{\SUBSECwhatDtd}
\label{\SUBSECwhatDtd}

A \dfn{Document Type Definition} (\dtd) defines a set of elements (and
element attributes), and the way they must be used in order to produce a
valid \xml\ document.

The component specification's \dtd\ is contained in the file
\filename{component.dtd}. 

The component specification \dtd\ can (and should) be used to validate
component specifications.
Section~\hyperref{\sectitleref{\SUBSECvalidation}}{}{}{\SUBSECvalidation}
explains how you can validate your component specification using the
component specification \dtd.

The component specification \dtd\ can help you write a valid component
specification when used with a \acronym{DTD} aware editor.  \Emacs\ is one
such editor.  See
section~\hyperref{\sectitleref{\SECediting}}{}{}{\SECediting} for information
on using \emacs\ to write component specifications.

\section{\SECcontent}
\label{\SECcontent}

\subsection{\SUBSECrules}
\label{\SUBSECrules}

When composing a component specification a few simple composition rules
must be followed:

\begin{itemize}
\item A component specification document starts with 2 lines of preamble text:

\begin{verbatim}
<?xml version="1.0" encoding="UTF-8" ?>
<!DOCTYPE component SYSTEM "component.dtd">
\end{verbatim}
  
\item The preamble is followed by a set of nested elements starting with
  the \element{component} element.

\item Every element must be delimited with a
  start-tag and an end-tag.  

\item Tags are case sensitive and all tags must be written in lower case.
  
\item A few elements require an \dfn{attribute}.  Attributes are specified
  in an element's start tag.  An attribute looks like this:
  \verb+attribute_name="attribute_value"+.  The attribute's value must be
  enclosed in double quotes. Attributes are documented below along side
  their corresponding element.
  
\item Note that the literal form of the characters \la, \ra, and \& may not
  be used.  Instead you must type \keyboard{\&lt;}, \keyboard{\&gt;}, and
  \keyboard{\&amp;} respectively.

\end{itemize}

The set of valid elements are described next.

\subsection{\SUBSECstructContent}
\label{\SUBSECstructContent}

Below, the component specification's \xml\ structure and corresponding logical
content are described.  Each element's start tag is followed by a
description of its purpose and content, followed by its nested elements,
and finally by its end tag.  Indentation is used to show nesting
relationships among elements.

\begin{description}
  \elementitem{component name=''component-name''} Starts the component specification.  All other
  elements are nested within this one.  This element requires the
  \icode{name} attribute.  This attribute names your component.

  \begin{description}
    \elementitem{overview} Starts the overview section.  Contains only
    nested elements.

    \begin{description}
      \elementitem{authors}  List of component's authors.  Contains 1 or more
      \element{author} elements.

      \begin{description}
        \elementitem{author} Name of 1 author.
        \elementitem{/author}
      \end{description}

      \elementitem{/authors}

      \elementitem{summary} Short (1 or 2 sentance) summary of component's function.
      \elementitem{/summary}
      
      \elementitem{description} Comprehensive description of component's
      function.  The \element{description} element is a mini-documentation
      environment.  See
      section~\hyperref{\sectitleref{\SUBSECdescElement}}{}{}{\SUBSECdescElement}
      for details on the use of the \element{description} element.
      \elementitem{/description}

      \elementitem{examplesr} The name of an \filename{.sr} file that
      demonstrates the component's use.
      \elementitem{/examplesr}

    \end{description}
    \elementitem{/overview} 
    
    \elementitem{io} Component's inputs and outputs.  Contains 2 elements,
    \element{inputs} followed by \element{outputs}.

    \begin{description}
      \elementitem{inputs} Describes the components's inputs. Inputs may
      come from a port, a file, or a device.  Zero or more \element{port}
      elements are used to describe inputs from ports.  Zero or more
      \element{file} elements are used to describe inputs from files.  Zero
      or more \element{device} elements are used to describe inputs from
      devices.  There must be at least 1 \element{port} element or 1
      \element{file} element or 1 \element{device} element.

      \begin{description}
        \elementitem{port} States that input data comes from a port. 
        Subelements describe the data further.
        
        \begin{description}
          \elementitem{description} High level description of the data 
          accepted by the port, e.g. ``These are potential data from body 
          surface of a human.''
          \elementitem{/description}
          
          \elementitem{datatype} The name of a PSE datatype, i.e. the 
          datatype that is being transmitted on this port.
          \elementitem{/datatype}
          
          \elementitem{componentname} The name of an upstream component
          that is commonly used to send data to this component on this
          port.  This element may occur multiple times. 
          \elementitem{/componentname}
        \end{description}
        
        \elementitem{/port}
        
        \elementitem{file} States that input comes from a file.
        Subelements describe the data further.
        
        \begin{description}
          \elementitem{description} High level description of the data 
          provided by the file, e.g. ``These are potential data from body 
          surface of a human.''
          \elementitem{/description}
          
          \elementitem{datatype} The name of a \pse\ datatype, i.e. the 
          datatype being read from the file.  \Note{Only \pse\ datatypes
            may be read from files - no custom formats allowed!}
          \elementitem{/datatype}
          
        \end{description}
        
        \elementitem{/file}

        \elementitem{device} States that input comes from a device.
        Subelements describe the data further.
        
        \begin{description}
          \elementitem{description} High level description of the data 
          provided by the device.
          \elementitem{/description}
        \end{description}
        
        \elementitem{/device}
      \end{description}

      \elementitem{/inputs}
      
      \elementitem{outputs} Describes the component's outputs.  Outputs may
      go to a port or to a file or to a device.  Use 0 or more
      \element{port} elements to describe outputs to ports, 0 or more
      \element{file} elements to describe outputs to files, and 0 or more
      \element{device} elements to describe outputs to devices.

      \begin{description}
        \elementitem{port} States that output is sent a port. Subelements describe the data further.
        
        \begin{description}
          \elementitem{description} A high level description of the data
          sent to the port, e.g. ``These are potential data from the body
          surface of a human.''

          \elementitem{/description}
          
          \elementitem{datatype} The name of a \pse\ datatype, i.e. the 
          datatype that is being transmitted on this port.
          \elementitem{/datatype}
          
          \elementitem{componentname} The name of a downstream component
          that is commonly connected to this output port.  This
          element may occur multiple times. 
          \elementitem{/componentname}
        \end{description}
        
        \elementitem{/port}

        \elementitem{file}  States that output is sent to a file.  It's 
        subelements describe the outputs.
        
        \begin{description}
          \elementitem{description} A high level description of the data 
          sent to the file, e.g. ``These are potential data from body 
          surface of a human.''
          \elementitem{/description}
          
          \elementitem{datatype}  The name of a PSE datatype, i.e. the 
          datatype being written to the file.
          \elementitem{/datatype}
          
        \end{description}
        
        \elementitem{/file}

        \elementitem{device} States that output goes to a device.
        Subelements describe the output further.
        
        \begin{description}
          \elementitem{description} High level description of the data
          sent to the device.
          \elementitem{/description}
        \end{description}
        
        \elementitem{/device}
    \end{description}
    
    \elementitem{/outputs}
  \end{description}

  \elementitem{/io}
  
  \elementitem{gui} States that the component supports a \gui.
  Subelements describe the \gui.
  
  \begin{description}
    \elementitem{description} Describes the purpose of the \gui, e.g.
    ``The \gui\ allows you to steer the simulation.''
    \elementitem{/description}

    \elementitem{parameter} Declares 1 \gui\  parameter.  A parameter is a
    \gui\  item that can be modified by the user.  Subelements describe the
    parameter.  Use 1 parameter element for each item in the \gui.
    
    \begin{description}
      \elementitem{label}Name of parameter as it appears in \gui.
      \elementitem{/label}
      \elementitem{description}Describes the purpose of the parameter and
      how it can be used to control the component's behavior.
      \elementitem{/description}
    \end{description}
    \elementitem{/parameter}
    \elementitem{img}Name of a file that contains a picture of the \gui.
    \elementitem{/img}
  \end{description}

  \elementitem{/gui}%

  \elementitem{testing} Starts the testing section.  Subelements specify
  1 or more testing plans.

  \begin{description}
    \elementitem{plan} Declares 1 testing plan.  Subelements specify the
    testing plan's details.
    \begin{description}
      \elementitem{description}  \emph{Need your help here Marty}
      \elementitem{/description}
      \elementitem{steps}\emph{Need your help here Marty}

      \begin{description}
        \elementitem{step}\emph{Need your help here Marty}
        \elementitem{/step}
      \end{description}

      \elementitem{/steps}
    \end{description}

    \elementitem{/plan}
  \end{description}

  \elementitem{/testing}
\end{description}

\elementitem{/component}
\end{description}

\subsection{\SUBSECdescElement}
\label{\SUBSECdescElement}


\subsubsection{\SUBSUBSECdescElementIntro}
\label{\SUBSUBSECdescElementIntro}

The \element{description} element contains text for human consumption.
It allows you to organize your text into paragraphs, lists, and
other structured content.

Note that the literal form of the characters \la, \ra, and \& may not be
used (this is an \xml\ thing).  Instead you must type \keyboard{\&lt;},
\keyboard{\&gt;}, and \keyboard{\&amp;} respectively.

\subsubsection{\SUBSUBSECcommonUsage}
\label{\SUBSUBSECcommonUsage}

A \element{description} might contain only 1 or 2 sentences:

\begin{verbatim}
<description><p>My component is great.  It's so easy to use that no further
information is needed.</p></description>
\end{verbatim}

Note that even just 1 sentence must be enclosed in a paragraph.

A \element{description} may contain only a few paragraphs:

\begin{verbatim}
<description>
  <p>My component is great.  It's so easy to use that no further
     information is needed.</p>

  <p>I lied.  What follows is a description of my component.</p>
   .
   .
   .
</description>
\end{verbatim}

You may refer to other material using the \element{cite}, \element{rlink},
or \element{slink} elements:

\begin{verbatim}
<description>
My component does xyz.  It uses the xyz algorithm described in the paper
<cite>The XYZ algorithm</cite> by Xavier Yilmez Zideco.  A less formal
discussion of the xyz algorithm is
 <slink path=ncrr/publications/xyz.html>available online</slink>. 
</description>
\end{verbatim}

The \element{cite} element marks text as the title of a paper or
book. 

The \element{slink} and \element{rlink} elements allow you to reference
online material.  

\element{Slink} links to material on the \acronym{SCI}
web site (\sciurl).  It requires a path 
relative to \sciurl.  The path in the previous example resolves to
\sciurl{}/ncrr/publications/xyz.html.

\element{Rlink} links to material in the \acronym{CVS} tree
(\element{rlink} actually links to material on the documentation web site
but it is more easily explained as a path into the cvs tree).  It requires
a path relative to the location of your component specification document.  For
example, the path \keyboard{./stuff.html} resolves to the same directory
where your component specification document lives.

A more general link element is not available because the documentation
bureaucrats want some control over the material to which you link.

The \element{developer} element encloses material that is of interest to
developers:

\begin{verbatim}
<developer>
  <p>Some smart developer ought to improve this piece of rubbish.</p>
</developer>
\end{verbatim}

\subsubsection{\SUBSUBSECreference}
\label{\SUBSUBSECreference}

Below, the description element's \xml\ structure and corresponding logical
content are described.  Each element's start tag is followed by a
description of its purpose and content, followed by its nested elements,
and finally by its end tag.  Basic nesting relationships among elements are
shown by indentation.

\Note{Additional features may be added to the \element{description}
  element in the future.}

\begin{description}
  \elementitem{description} Starts a description.  A description consists
  of any combination of paragraphs (\element{p}), lists
  (\element{orderedlist}, \element{unorderedlist}, and \element{desclist}),
  admonitions (\element{note}, \element{tip}, or \element{warning}), and developer
  notes (\element{developer}).

  \begin{description}
    \elementitem{p} Starts a paragraph.  A paragraph consists of character
    data,  phrases (\element{term}, \element{keyword}, \element{keyboard},
    \element{cite}, and \element{acronym}) and links (\element{slink} and
    \element{rlink}). 
    \begin{description}
      \elementitem{term}
      \elementitem{/term}
      \elementitem{keyword}
      \elementitem{/keyword}
      \elementitem{keyboard}
      \elementitem{/keyboard}
      \elementitem{cite}
      \elementitem{/cite}
      \elementitem{acronym}
      \elementitem{/acronym}
      \elementitem{slink} Defines a \sci\ web site relative link.  It
      requires 1 attribute which is the path of the target relative to
      \sciurl.  Text bracketed by the start and end tags will be presented
      as a link in online documents.  The text may include phrase elements.
      \elementitem{/slink}%
      \elementitem{rlink} Defines a \pse\ \cvs\ relative link
      (\element{rlink} actually links to material on the documentation web
      site but it is more easily explained as a path into the cvs tree).  It
      requires a path relative to the location of your component
      specification document.  Text bracketed by the start and end tags will
      be presented as a link in online documents.  The bracketed text may
      include phrase elements.
      \elementitem{/rlink}%
    \end{description}
    \elementitem{/p}%
    \elementitem{orderedlist}  An ordered (numbered) list.  Contains 1 or
    more \element{listitem} elements.
    \begin{description}
      \elementitem{listitem}  An item in an ordered list.  A
      \element{listitem} may contain paragraphs and nested lists.
      \elementitem{/listitem}%
    \end{description}
    \elementitem{/orderedlist}%
    \elementitem{unorderedlist} An unordered list.  Contains 1 or
    more \element{listitem} elements.
    \begin{description}
      \elementitem{listitem}  An item in an unordered list.  A
      \element{listitem} may contain paragraphs and nested lists.
      \elementitem{/listitem}%
    \end{description}
    \elementitem{/unorderedlist}%
    \elementitem{desclist} A description list.  Contains a list of terms or
    short phrases
    and their definitions or descriptions.  Consists of 1 or more \element{desclistitem}
    elements. 
    \begin{description}
      \elementitem{desclistitem} Contains 1 \element{desclistterm} and 1
      \element{desclistdef}. 
      \begin{description}
        \elementitem{desclistterm} A word or short phrase (not enclosed in
        \element{p} element).
        \elementitem{/desclistterm}
        \elementitem{desclistdef} Definition or description of the
        corresponding term.  May consist of a mix of paragraph elements and
        list elements.
        \elementitem{/desclistdef}%
      \end{description}
      \elementitem{/desclistitem}%
    \end{description}
    \elementitem{/desclist}%
    \elementitem{note}  Calls attention to a piece of information.  Aimed
    towards users of the component.  Presumably the information within a
    \element{note} element will be rendered in a special
    way.  A note consists of paragraphs and lists only.
    \elementitem{/note}%
    \elementitem{tip} Contains an especially helpful piece of information.
    Aimed towards users of the component.  Presumably the information
    within a \element{tip} element will be rendered in a special way.  A
    tip consists of paragraphs and lists only.
    \elementitem{/tip}%
    \elementitem{warning} Contains a warning (e.g., ``Don't press the red
    button!''). Aimed towards users of the component.  Presumably the
    information within a \element{warning} element will be rendered in a
    special way.  A warning consists of paragraphs and lists only.
    \elementitem{/warning}%

    \elementitem{developer}  Starts a developer section.  Material
    here should be aimed towards fellow programmers.  The
    \element{developer} element acts like the \element{description} element
    except that \element{developer} elements may not be nested.
    \elementitem{/developer}%

  \end{description}

  \elementitem{/description} 
\end{description}


\section{\SECexample}
\label{\SECexample}
\begin{verbatim}
\end{verbatim}


\section{\SECediting}
\label{\SECediting}

An ordinary text editor may be used to create the component specification
document's content.  However, it is easy to get lost in the noise of the
\xml\ syntax.  Therefore, it is best to use an \xml{}/\dtd\ aware editor.
This type of editor will help you construct a valid component specification
document.

\Emacs\ is one such editor.  It supports an editing environment called
\xml\ mode (which is really a derivative of \psgml\ mode).  \Xml\ mode
highlights \xml\ syntax, indents nested elements and their content, and
automatically inserts elements and attributes based on the position of the
insertion point.  It is still possible to create invalid documents using
\emacs\ \xml\ mode.

The following sections describe the use of \emacs\ \xml\ mode.

\subsection{\SUBSECgettingSources}
\label{\SUBSECgettingSources}

Recent versions of \emacs\ come with \xml\ mode installed.  You can check
if yours does by typing \keyboard{M-x xml-mode}.  You \emph{don't} have
\xml\ mode if you get the message
\screen{[No match]} in return.

UUCS people may use the version of \psgml\ installed under
\keyboard{/usr/local/contrib/mcole/psgml}.  The following bit of lisp code
must be added to your \filename{.emacs} file:

\begin{verbatim}
(setq load-path (append '("/usr/local/contrib/mcole/psgml") load-path))
\end{verbatim}

Non-UUCS folks who don't have \xml\ mode may get it from \psgmlurl.

\subsection{\SUBSECdotEmacs}
\label{\SUBSECdotEmacs}

The following lisp code should be inserted into your \filename{.emacs}
file:

\begin{verbatim}
; Tell emacs to use sgml/xml mode for the following file types.
(setq auto-mode-alist
      (append
        '(("\\.sgm" . sgml-mode)
          ("\\.sgml" . sgml-mode)
          ("\\.xml" . xml-mode))
       auto-mode-alist))
(autoload 'sgml-mode "psgml" "Major mode to edit SGML files." t)
(autoload 'xml-mode "psgml" "Major mode to edit XML files." t) 

; Customize sgml/xml-mode default settings.
(add-hook 'sgml-mode-hook (lambda () (setq sgml-indent-data t)))

; Create some faces for use with sgml/xml mode.
; Change colors to suite your fancy.
(make-face 'sgml-start-tag-face) 
(set-face-foreground 'sgml-start-tag-face "slategrey") 
(make-face 'sgml-end-tag-face) 
(set-face-foreground 'sgml-end-tag-face "slategrey") 
(make-face 'sgml-entity-face) 
(set-face-foreground 'sgml-entity-face "Red") 
(make-face 'sgml-doctype-face) 
(set-face-foreground 'sgml-doctype-face "firebrick") 
(make-face 'sgml-comment-face) 
(set-face-foreground 'sgml-comment-face "blue") 

; Use faces defined above.
(setq sgml-set-face t)
(setq sgml-markup-faces 
      '((comment   . sgml-comment-face) 
        (start-tag . sgml-start-tag-face) 
        (end-tag   . sgml-end-tag-face) 
        (doctype   . sgml-doctype-face) 
        (entity    . sgml-entity-face))) 


\end{verbatim}

\subsection{\SUBSECgettingStarted}
\label{\SUBSECgettingStarted}

To start a new component specification with \emacs\ do this:

\begin{enumerate}
\item Create a file and insert the following 2 lines:
\begin{verbatim}
<?xml version="1.0" encoding="UTF-8" ?>
<!DOCTYPE component SYSTEM "component.dtd">
\end{verbatim}
\item Insert the top level \element{component}
  element using \emacs\ popup menu (shift-button1 for emacs and button3 for
  xemacs):  Position the cursor in the 3rd line of the file and select the item named
  \menuitem{component} from the popup menu.  After doing this you will be prompted for the \element{component} element's
  \screen{name} attribute.  Type in the name of your component and press
  return.  \Emacs\ will then automatically insert a number of other required elements.
\item Insert content into these elements.  For example, you may insert your
  name between the start and end tags of the \element{author} element.
\item Some elements, like the \element{description} element, are composed of
  subelements.  To add subelements, position the cursor between the start
  and end tags of the element and then insert an element using \emacs\ 
  popup menu.  The popup menu will list elements that are valid at the
  insertion point.
\item The \element{gui} element is not automatically inserted by \emacs\ 
  because a gui is optional.  Most components will support a gui though so
  you will probably need to insert the \element{gui} element.  Do this by
  positioning the cursor after the end tag of the \element{io} element.
  Then select the \element{gui} element from \emacs\ popup menu.  \Emacs\ 
  will insert the \element{gui} and its required subelements.
\item Other elements that require subelements are the \element{io} element's
  \element{inputs}, \element{outputs}, and others.
\item A few elements (namely the \element{slink} and \element{rlink}
  elements take attributes.  If these elements are added via \emacs\ popup
  menu then \emacs\ will prompt you for the element's attribute values.  If
  these element's are added by hand then you may add the attributes via the
  popup menu: position the cursor anywhere within the element's start tag
  and select the desired attribute from the popup menu's list of
  attributes.  \Emacs\ will prompt you for the value of the attribute and
  insert it in the element's start tag.
\item Complete the component specification by adding content, elements, and
  attributes.
\end{enumerate}



\subsection{\SUBSECvalidation}
\label{\SUBSECvalidation}

http://www.stg.brown.edu/service/xmlvalid/

\section{\SECtools}
\label{\SECtools}


\subsection{\SUBSECmakeDevHtmlDoc}
\label{\SUBSECmakeDevHtmlDoc}


\subsection{\SUBSECmakeUserHtmlDoc}
\label{\SUBSECmakeUserHtmlDoc}


\end{document}
