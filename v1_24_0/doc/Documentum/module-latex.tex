% -*-latex-*-
%
%  For more information, please see: http://software.sci.utah.edu
% 
%  The MIT License
% 
%  Copyright (c) 2004 Scientific Computing and Imaging Institute,
%  University of Utah.
% 
%  License for the specific language governing rights and limitations under
%  Permission is hereby granted, free of charge, to any person obtaining a
%  copy of this software and associated documentation files (the "Software"),
%  to deal in the Software without restriction, including without limitation
%  the rights to use, copy, modify, merge, publish, distribute, sublicense,
%  and/or sell copies of the Software, and to permit persons to whom the
%  Software is furnished to do so, subject to the following conditions:
% 
%  The above copyright notice and this permission notice shall be included
%  in all copies or substantial portions of the Software.
% 
%  THE SOFTWARE IS PROVIDED "AS IS", WITHOUT WARRANTY OF ANY KIND, EXPRESS
%  OR IMPLIED, INCLUDING BUT NOT LIMITED TO THE WARRANTIES OF MERCHANTABILITY,
%  FITNESS FOR A PARTICULAR PURPOSE AND NONINFRINGEMENT. IN NO EVENT SHALL
%  THE AUTHORS OR COPYRIGHT HOLDERS BE LIABLE FOR ANY CLAIM, DAMAGES OR OTHER
%  LIABILITY, WHETHER IN AN ACTION OF CONTRACT, TORT OR OTHERWISE, ARISING
%  FROM, OUT OF OR IN CONNECTION WITH THE SOFTWARE OR THE USE OR OTHER
%  DEALINGS IN THE SOFTWARE.
%


\documentclass[11pt]{article}
\usepackage[]{fancyhdr}
%\begin{latexonly}
\usepackage[]{../Utilities/TeX/scirun-doc}
%\end{latexonly}
\usepackage[]{graphicx}
\usepackage[]{html}
\begin{htmlonly}
  % -*-latex-*-
%
%  The contents of this file are subject to the University of Utah Public
%  License (the "License"); you may not use this file except in compliance
%  with the License.
%
%  Software distributed under the License is distributed on an "AS IS"
%  basis, WITHOUT WARRANTY OF ANY KIND, either express or implied. See the
%  License for the specific language governing rights and limitations under
%  the License.
%
%  The Original Source Code is SCIRun, released March 12, 2001.
%
%  The Original Source Code was developed by the University of Utah.
%  Portions created by UNIVERSITY are Copyright (C) 2001, 1994
%  University of Utah. All Rights Reserved.
%

%
% Latex package to be used by scirun documents.  Markup commands exist
% in two files.  This one and scirun-doc.tex.  This file contains
% commands that are processed by both latex and by the builtin
% capabilities of latex2html.  Scirun-doc.sty contains commands that are
% either completely ignored by latex2html or are processed by functions in
% the latex2html extension file scirun-doc.perl.
%
% Documents should include the
% following latex commands:
%
% \usepackage{scirun-doc}
% % -*-latex-*-
%
%  The contents of this file are subject to the University of Utah Public
%  License (the "License"); you may not use this file except in compliance
%  with the License.
%
%  Software distributed under the License is distributed on an "AS IS"
%  basis, WITHOUT WARRANTY OF ANY KIND, either express or implied. See the
%  License for the specific language governing rights and limitations under
%  the License.
%
%  The Original Source Code is SCIRun, released March 12, 2001.
%
%  The Original Source Code was developed by the University of Utah.
%  Portions created by UNIVERSITY are Copyright (C) 2001, 1994
%  University of Utah. All Rights Reserved.
%

%
% Latex package to be used by scirun documents.  Markup commands exist
% in two files.  This one and scirun-doc.tex.  This file contains
% commands that are processed by both latex and by the builtin
% capabilities of latex2html.  Scirun-doc.sty contains commands that are
% either completely ignored by latex2html or are processed by functions in
% the latex2html extension file scirun-doc.perl.
%
% Documents should include the
% following latex commands:
%
% \usepackage{scirun-doc}
% % -*-latex-*-
%
%  The contents of this file are subject to the University of Utah Public
%  License (the "License"); you may not use this file except in compliance
%  with the License.
%
%  Software distributed under the License is distributed on an "AS IS"
%  basis, WITHOUT WARRANTY OF ANY KIND, either express or implied. See the
%  License for the specific language governing rights and limitations under
%  the License.
%
%  The Original Source Code is SCIRun, released March 12, 2001.
%
%  The Original Source Code was developed by the University of Utah.
%  Portions created by UNIVERSITY are Copyright (C) 2001, 1994
%  University of Utah. All Rights Reserved.
%

%
% Latex package to be used by scirun documents.  Markup commands exist
% in two files.  This one and scirun-doc.tex.  This file contains
% commands that are processed by both latex and by the builtin
% capabilities of latex2html.  Scirun-doc.sty contains commands that are
% either completely ignored by latex2html or are processed by functions in
% the latex2html extension file scirun-doc.perl.
%
% Documents should include the
% following latex commands:
%
% \usepackage{scirun-doc}
% \input{scirun-doc.tex}
%

% (Mostly) Short cuts  ==============================
\newcommand{\SCI}{{\em SCI}}
\newcommand{\sci}{\SCI}
\newcommand{\scii}{SCI Institute}
\newcommand{\BIOPSE}{\textbf{BioPSE}}
\newcommand{\biopse}{\BIOPSE}
\newcommand{\SR}{\textbf{SCIRun}}
\newcommand{\sr}{\SR}
\newcommand{\eg}{{\em e.g.,}}
\newcommand{\ie}{{\em i.e.,}}
\newcommand{\etc}{{\em etc.}}
\newcommand{\etal}{{\em et al.}}
\newcommand{\degrees}{{$^{\circ}$}}
\newcommand{\splitline}{\begin{center}\rule{\columnwidth}{.7mm}\end{center}}
\newcommand{\X}[1]{#1\index{#1}}
\newcommand{\srig}{\sr{} Installation Guide}
\newcommand{\srug}{\sr{} User's Guide}
\newcommand{\srdg}{\sr{} Developer's Guide}

% Literal ~ character
\newcommand{\ltilde}{\textasciitilde}

% Encloses its argument between angle brackets.
\newcommand{\ab}[1]{\textless{}#1\textgreater}

% Inserts a left angle bracket.
\newcommand{\la}{\textless}

% Inserts a right angle bracket.
\newcommand{\ra}{\textgreater}


% Mark up commands =================================

% Url
%\newcommand{\url}[1]{#1}

% ip address
\newcommand{\ipaddr}[1]{#1}
\newcommand{\localhost}{\ipaddr{127.0.0.1}}

% Predefined acronyms
\newcommand{\gui}{\acronym{GUI}}
\newcommand{\tcl}{\acronym{TCL}}
\newcommand{\xml}{\acronym{XML}}
\newcommand{\pse}{\acronym{PSE}}

% Markup the first time use of term that may be unfamiliar to the
% reader. 
\newmucmd{dfn}{emph}

% In the next command #1 is the term and #2 is the shortcut or acronym that
% will be used in the rest of the document. 
\newcommand{\dfna}[2]{\emph{#1} (\acronym{#2})}

% Markup a file name.
\newmucmd{filename}{texttt}

% Directory name markup.
\newmucmd{directory}{texttt}

% Markup text which is the name of a command.
\newmucmd{command}{texttt} 

% Command option.
\newmucmd{option}{texttt}

% Markup text typed at the keyboard.
\newmucmd{keyboard}{texttt} 

% Parameterized text - marks up text that is to be
% substituted for by the reader.
\newmucmd{ptext}{textit}        % Obsolete-use replaceable instead.
\newmucmd{replaceable}{textit}

% Markup text the user might see on his screen.
\newmucmd{screen}{texttt}

% Markup the name of a GUI menu.
\newmucmd{guimenu}{textbf}

% Markup the name of a GUI menu.
\newcommand{\menu}[1]{\guimenu{#1}}

% Markup a gui menu item name.
\newmucmd{guimenuitem}{textbf}

% Markup a menu item name.
\newcommand{\menuitem}[1]{\guimenuitem{#1}}

% Markup name of a ui button.
\newmucmd{guibutton}{textbf}

% Markup name of a ui button.
\newcommand{\button}[1]{\guibutton{#1}}

% Markup name of a gui text item.
\newmucmd{guitext}{textit}

% Markup name of a gui label.
\newmucmd{guilabel}{textbf}

% GUI variable
\newmucmd{guivar}{texttt}

% Scirun port
\newmucmd{srport}{texttt}

% Sockets port
\newcommand{\port}[1]{\texttt{#1}}

% Socket
\newcommand{\socket}[2]{\texttt{#1:#2}}

% Variable
\newcommand{\variable}[1]{\texttt{#1}}

% Data type
\newcommand{\datatype}[1]{\texttt{#1}}

% Function
\newcommand{\function}[1]{\texttt{#1}}

% Markup an inline code fragment.
\newcommand{\icode}[1]{\texttt{#1}}

% Markup a module name.
\newcommand{\module}[1]{\texttt{#1}}

% Markup a sr package name.
\newcommand{\package}[1]{\texttt{#1}}

% Markup a sr category name.
\newcommand{\category}[1]{\texttt{#1}}

% Env variable markup.
\newmucmd{envvar}{texttt}

% Markup for the title of a book or article or whatever.
\newcommand{\etitle}[1]{\texttt{#1}}

% Markup name of a latex section.  First arg is section name.  Second arg
% is section's label.  When processed with latex, the section
% name is emphasized.  When processed with latex2html section name is made
% into a link to section.
\newcommand{\secname}[2]{\latexhtml{\emph{#1}}{\htmlref{#1}{#2}}}

% Section ref command.  Use of this command
% in place of \ref to reference sections (subsections etc.) will
% create a more web friendly version of your document.
% The command's first argument is the link text (used only on the html
% page) and the second argument is the section's label
\newcommand{\secref}[2]{\hyperref[ref]{\emph{#1}}{Section~}{}{#2}}
\newcommand{\chref}[2]{\hyperref[ref]{\emph{#1}}{Chapter~}{}{#2}}

% A latex command
\newcommand{\latexcommand}[1]{\texttt{\textbackslash#1}}

% A latex package
\newcommand{\latexpackage}[1]{\textit{#1}}

% Use these in place of missing content.
\newcommand{\missing}[1]{\emph{#1 - Coming Soon.}}

% Use this to make note of incomplete content.
\newcommand{\incomplete}{\emph{More Comming Soon.}}

% Mark up a mail address
\newmucmd{mailto}{texttt}

% Mark up an xml attribute
\newmucmd{xmlattrname}{texttt}

% Sci Urls ========================================= 

% www style urls.
\newcommand{\scisoftware}{http://software.sci.utah.edu}
\newcommand{\scisoftwareurl}{\scisoftware}
\newcommand{\sciurl}{http://www.sci.utah.edu}
\newcommand{\scidocurl}{\scisoftware{}/doc}
\newcommand{\scidocurlplus}[1]{\scisoftware{}/doc/#1}
\newcommand{\bugsurl}{\scisoftware{}/bugzilla}
\newcommand{\scisoftwarearchiveurl}{\scisoftware{}/archive\_entry.html}

%

% (Mostly) Short cuts  ==============================
\newcommand{\SCI}{{\em SCI}}
\newcommand{\sci}{\SCI}
\newcommand{\scii}{SCI Institute}
\newcommand{\BIOPSE}{\textbf{BioPSE}}
\newcommand{\biopse}{\BIOPSE}
\newcommand{\SR}{\textbf{SCIRun}}
\newcommand{\sr}{\SR}
\newcommand{\eg}{{\em e.g.,}}
\newcommand{\ie}{{\em i.e.,}}
\newcommand{\etc}{{\em etc.}}
\newcommand{\etal}{{\em et al.}}
\newcommand{\degrees}{{$^{\circ}$}}
\newcommand{\splitline}{\begin{center}\rule{\columnwidth}{.7mm}\end{center}}
\newcommand{\X}[1]{#1\index{#1}}
\newcommand{\srig}{\sr{} Installation Guide}
\newcommand{\srug}{\sr{} User's Guide}
\newcommand{\srdg}{\sr{} Developer's Guide}

% Literal ~ character
\newcommand{\ltilde}{\textasciitilde}

% Encloses its argument between angle brackets.
\newcommand{\ab}[1]{\textless{}#1\textgreater}

% Inserts a left angle bracket.
\newcommand{\la}{\textless}

% Inserts a right angle bracket.
\newcommand{\ra}{\textgreater}


% Mark up commands =================================

% Url
%\newcommand{\url}[1]{#1}

% ip address
\newcommand{\ipaddr}[1]{#1}
\newcommand{\localhost}{\ipaddr{127.0.0.1}}

% Predefined acronyms
\newcommand{\gui}{\acronym{GUI}}
\newcommand{\tcl}{\acronym{TCL}}
\newcommand{\xml}{\acronym{XML}}
\newcommand{\pse}{\acronym{PSE}}

% Markup the first time use of term that may be unfamiliar to the
% reader. 
\newmucmd{dfn}{emph}

% In the next command #1 is the term and #2 is the shortcut or acronym that
% will be used in the rest of the document. 
\newcommand{\dfna}[2]{\emph{#1} (\acronym{#2})}

% Markup a file name.
\newmucmd{filename}{texttt}

% Directory name markup.
\newmucmd{directory}{texttt}

% Markup text which is the name of a command.
\newmucmd{command}{texttt} 

% Command option.
\newmucmd{option}{texttt}

% Markup text typed at the keyboard.
\newmucmd{keyboard}{texttt} 

% Parameterized text - marks up text that is to be
% substituted for by the reader.
\newmucmd{ptext}{textit}        % Obsolete-use replaceable instead.
\newmucmd{replaceable}{textit}

% Markup text the user might see on his screen.
\newmucmd{screen}{texttt}

% Markup the name of a GUI menu.
\newmucmd{guimenu}{textbf}

% Markup the name of a GUI menu.
\newcommand{\menu}[1]{\guimenu{#1}}

% Markup a gui menu item name.
\newmucmd{guimenuitem}{textbf}

% Markup a menu item name.
\newcommand{\menuitem}[1]{\guimenuitem{#1}}

% Markup name of a ui button.
\newmucmd{guibutton}{textbf}

% Markup name of a ui button.
\newcommand{\button}[1]{\guibutton{#1}}

% Markup name of a gui text item.
\newmucmd{guitext}{textit}

% Markup name of a gui label.
\newmucmd{guilabel}{textbf}

% GUI variable
\newmucmd{guivar}{texttt}

% Scirun port
\newmucmd{srport}{texttt}

% Sockets port
\newcommand{\port}[1]{\texttt{#1}}

% Socket
\newcommand{\socket}[2]{\texttt{#1:#2}}

% Variable
\newcommand{\variable}[1]{\texttt{#1}}

% Data type
\newcommand{\datatype}[1]{\texttt{#1}}

% Function
\newcommand{\function}[1]{\texttt{#1}}

% Markup an inline code fragment.
\newcommand{\icode}[1]{\texttt{#1}}

% Markup a module name.
\newcommand{\module}[1]{\texttt{#1}}

% Markup a sr package name.
\newcommand{\package}[1]{\texttt{#1}}

% Markup a sr category name.
\newcommand{\category}[1]{\texttt{#1}}

% Env variable markup.
\newmucmd{envvar}{texttt}

% Markup for the title of a book or article or whatever.
\newcommand{\etitle}[1]{\texttt{#1}}

% Markup name of a latex section.  First arg is section name.  Second arg
% is section's label.  When processed with latex, the section
% name is emphasized.  When processed with latex2html section name is made
% into a link to section.
\newcommand{\secname}[2]{\latexhtml{\emph{#1}}{\htmlref{#1}{#2}}}

% Section ref command.  Use of this command
% in place of \ref to reference sections (subsections etc.) will
% create a more web friendly version of your document.
% The command's first argument is the link text (used only on the html
% page) and the second argument is the section's label
\newcommand{\secref}[2]{\hyperref[ref]{\emph{#1}}{Section~}{}{#2}}
\newcommand{\chref}[2]{\hyperref[ref]{\emph{#1}}{Chapter~}{}{#2}}

% A latex command
\newcommand{\latexcommand}[1]{\texttt{\textbackslash#1}}

% A latex package
\newcommand{\latexpackage}[1]{\textit{#1}}

% Use these in place of missing content.
\newcommand{\missing}[1]{\emph{#1 - Coming Soon.}}

% Use this to make note of incomplete content.
\newcommand{\incomplete}{\emph{More Comming Soon.}}

% Mark up a mail address
\newmucmd{mailto}{texttt}

% Mark up an xml attribute
\newmucmd{xmlattrname}{texttt}

% Sci Urls ========================================= 

% www style urls.
\newcommand{\scisoftware}{http://software.sci.utah.edu}
\newcommand{\scisoftwareurl}{\scisoftware}
\newcommand{\sciurl}{http://www.sci.utah.edu}
\newcommand{\scidocurl}{\scisoftware{}/doc}
\newcommand{\scidocurlplus}[1]{\scisoftware{}/doc/#1}
\newcommand{\bugsurl}{\scisoftware{}/bugzilla}
\newcommand{\scisoftwarearchiveurl}{\scisoftware{}/archive\_entry.html}

%

% (Mostly) Short cuts  ==============================
\newcommand{\SCI}{{\em SCI}}
\newcommand{\sci}{\SCI}
\newcommand{\scii}{SCI Institute}
\newcommand{\BIOPSE}{\textbf{BioPSE}}
\newcommand{\biopse}{\BIOPSE}
\newcommand{\SR}{\textbf{SCIRun}}
\newcommand{\sr}{\SR}
\newcommand{\eg}{{\em e.g.,}}
\newcommand{\ie}{{\em i.e.,}}
\newcommand{\etc}{{\em etc.}}
\newcommand{\etal}{{\em et al.}}
\newcommand{\degrees}{{$^{\circ}$}}
\newcommand{\splitline}{\begin{center}\rule{\columnwidth}{.7mm}\end{center}}
\newcommand{\X}[1]{#1\index{#1}}
\newcommand{\srig}{\sr{} Installation Guide}
\newcommand{\srug}{\sr{} User's Guide}
\newcommand{\srdg}{\sr{} Developer's Guide}

% Literal ~ character
\newcommand{\ltilde}{\textasciitilde}

% Encloses its argument between angle brackets.
\newcommand{\ab}[1]{\textless{}#1\textgreater}

% Inserts a left angle bracket.
\newcommand{\la}{\textless}

% Inserts a right angle bracket.
\newcommand{\ra}{\textgreater}


% Mark up commands =================================

% Url
%\newcommand{\url}[1]{#1}

% ip address
\newcommand{\ipaddr}[1]{#1}
\newcommand{\localhost}{\ipaddr{127.0.0.1}}

% Predefined acronyms
\newcommand{\gui}{\acronym{GUI}}
\newcommand{\tcl}{\acronym{TCL}}
\newcommand{\xml}{\acronym{XML}}
\newcommand{\pse}{\acronym{PSE}}

% Markup the first time use of term that may be unfamiliar to the
% reader. 
\newmucmd{dfn}{emph}

% In the next command #1 is the term and #2 is the shortcut or acronym that
% will be used in the rest of the document. 
\newcommand{\dfna}[2]{\emph{#1} (\acronym{#2})}

% Markup a file name.
\newmucmd{filename}{texttt}

% Directory name markup.
\newmucmd{directory}{texttt}

% Markup text which is the name of a command.
\newmucmd{command}{texttt} 

% Command option.
\newmucmd{option}{texttt}

% Markup text typed at the keyboard.
\newmucmd{keyboard}{texttt} 

% Parameterized text - marks up text that is to be
% substituted for by the reader.
\newmucmd{ptext}{textit}        % Obsolete-use replaceable instead.
\newmucmd{replaceable}{textit}

% Markup text the user might see on his screen.
\newmucmd{screen}{texttt}

% Markup the name of a GUI menu.
\newmucmd{guimenu}{textbf}

% Markup the name of a GUI menu.
\newcommand{\menu}[1]{\guimenu{#1}}

% Markup a gui menu item name.
\newmucmd{guimenuitem}{textbf}

% Markup a menu item name.
\newcommand{\menuitem}[1]{\guimenuitem{#1}}

% Markup name of a ui button.
\newmucmd{guibutton}{textbf}

% Markup name of a ui button.
\newcommand{\button}[1]{\guibutton{#1}}

% Markup name of a gui text item.
\newmucmd{guitext}{textit}

% Markup name of a gui label.
\newmucmd{guilabel}{textbf}

% GUI variable
\newmucmd{guivar}{texttt}

% Scirun port
\newmucmd{srport}{texttt}

% Sockets port
\newcommand{\port}[1]{\texttt{#1}}

% Socket
\newcommand{\socket}[2]{\texttt{#1:#2}}

% Variable
\newcommand{\variable}[1]{\texttt{#1}}

% Data type
\newcommand{\datatype}[1]{\texttt{#1}}

% Function
\newcommand{\function}[1]{\texttt{#1}}

% Markup an inline code fragment.
\newcommand{\icode}[1]{\texttt{#1}}

% Markup a module name.
\newcommand{\module}[1]{\texttt{#1}}

% Markup a sr package name.
\newcommand{\package}[1]{\texttt{#1}}

% Markup a sr category name.
\newcommand{\category}[1]{\texttt{#1}}

% Env variable markup.
\newmucmd{envvar}{texttt}

% Markup for the title of a book or article or whatever.
\newcommand{\etitle}[1]{\texttt{#1}}

% Markup name of a latex section.  First arg is section name.  Second arg
% is section's label.  When processed with latex, the section
% name is emphasized.  When processed with latex2html section name is made
% into a link to section.
\newcommand{\secname}[2]{\latexhtml{\emph{#1}}{\htmlref{#1}{#2}}}

% Section ref command.  Use of this command
% in place of \ref to reference sections (subsections etc.) will
% create a more web friendly version of your document.
% The command's first argument is the link text (used only on the html
% page) and the second argument is the section's label
\newcommand{\secref}[2]{\hyperref[ref]{\emph{#1}}{Section~}{}{#2}}
\newcommand{\chref}[2]{\hyperref[ref]{\emph{#1}}{Chapter~}{}{#2}}

% A latex command
\newcommand{\latexcommand}[1]{\texttt{\textbackslash#1}}

% A latex package
\newcommand{\latexpackage}[1]{\textit{#1}}

% Use these in place of missing content.
\newcommand{\missing}[1]{\emph{#1 - Coming Soon.}}

% Use this to make note of incomplete content.
\newcommand{\incomplete}{\emph{More Comming Soon.}}

% Mark up a mail address
\newmucmd{mailto}{texttt}

% Mark up an xml attribute
\newmucmd{xmlattrname}{texttt}

% Sci Urls ========================================= 

% www style urls.
\newcommand{\scisoftware}{http://software.sci.utah.edu}
\newcommand{\scisoftwareurl}{\scisoftware}
\newcommand{\sciurl}{http://www.sci.utah.edu}
\newcommand{\scidocurl}{\scisoftware{}/doc}
\newcommand{\scidocurlplus}[1]{\scisoftware{}/doc/#1}
\newcommand{\bugsurl}{\scisoftware{}/bugzilla}
\newcommand{\scisoftwarearchiveurl}{\scisoftware{}/archive\_entry.html}

\end{htmlonly}
\newcommand{\latexenv}[1]{\texttt{#1}}
\newcommand{\texutils}{../../Utilities/TeX}
\begin{document}

\begin{htmlonly}
%
%  For more information, please see: http://software.sci.utah.edu
% 
%  The MIT License
% 
%  Copyright (c) 2004 Scientific Computing and Imaging Institute,
%  University of Utah.
% 
%  License for the specific language governing rights and limitations under
%  Permission is hereby granted, free of charge, to any person obtaining a
%  copy of this software and associated documentation files (the "Software"),
%  to deal in the Software without restriction, including without limitation
%  the rights to use, copy, modify, merge, publish, distribute, sublicense,
%  and/or sell copies of the Software, and to permit persons to whom the
%  Software is furnished to do so, subject to the following conditions:
% 
%  The above copyright notice and this permission notice shall be included
%  in all copies or substantial portions of the Software.
% 
%  THE SOFTWARE IS PROVIDED "AS IS", WITHOUT WARRANTY OF ANY KIND, EXPRESS
%  OR IMPLIED, INCLUDING BUT NOT LIMITED TO THE WARRANTIES OF MERCHANTABILITY,
%  FITNESS FOR A PARTICULAR PURPOSE AND NONINFRINGEMENT. IN NO EVENT SHALL
%  THE AUTHORS OR COPYRIGHT HOLDERS BE LIABLE FOR ANY CLAIM, DAMAGES OR OTHER
%  LIABILITY, WHETHER IN AN ACTION OF CONTRACT, TORT OR OTHERWISE, ARISING
%  FROM, OUT OF OR IN CONNECTION WITH THE SOFTWARE OR THE USE OR OTHER
%  DEALINGS IN THE SOFTWARE.
%


%
% Code to generate sr top banners.  To use this in your latex doc insert
% the following code just after the \begin{document} command:
%
% \begin{htmlonly}
% %
%  For more information, please see: http://software.sci.utah.edu
% 
%  The MIT License
% 
%  Copyright (c) 2004 Scientific Computing and Imaging Institute,
%  University of Utah.
% 
%  License for the specific language governing rights and limitations under
%  Permission is hereby granted, free of charge, to any person obtaining a
%  copy of this software and associated documentation files (the "Software"),
%  to deal in the Software without restriction, including without limitation
%  the rights to use, copy, modify, merge, publish, distribute, sublicense,
%  and/or sell copies of the Software, and to permit persons to whom the
%  Software is furnished to do so, subject to the following conditions:
% 
%  The above copyright notice and this permission notice shall be included
%  in all copies or substantial portions of the Software.
% 
%  THE SOFTWARE IS PROVIDED "AS IS", WITHOUT WARRANTY OF ANY KIND, EXPRESS
%  OR IMPLIED, INCLUDING BUT NOT LIMITED TO THE WARRANTIES OF MERCHANTABILITY,
%  FITNESS FOR A PARTICULAR PURPOSE AND NONINFRINGEMENT. IN NO EVENT SHALL
%  THE AUTHORS OR COPYRIGHT HOLDERS BE LIABLE FOR ANY CLAIM, DAMAGES OR OTHER
%  LIABILITY, WHETHER IN AN ACTION OF CONTRACT, TORT OR OTHERWISE, ARISING
%  FROM, OUT OF OR IN CONNECTION WITH THE SOFTWARE OR THE USE OR OTHER
%  DEALINGS IN THE SOFTWARE.
%


%
% Code to generate sr top banners.  To use this in your latex doc insert
% the following code just after the \begin{document} command:
%
% \begin{htmlonly}
% %
%  For more information, please see: http://software.sci.utah.edu
% 
%  The MIT License
% 
%  Copyright (c) 2004 Scientific Computing and Imaging Institute,
%  University of Utah.
% 
%  License for the specific language governing rights and limitations under
%  Permission is hereby granted, free of charge, to any person obtaining a
%  copy of this software and associated documentation files (the "Software"),
%  to deal in the Software without restriction, including without limitation
%  the rights to use, copy, modify, merge, publish, distribute, sublicense,
%  and/or sell copies of the Software, and to permit persons to whom the
%  Software is furnished to do so, subject to the following conditions:
% 
%  The above copyright notice and this permission notice shall be included
%  in all copies or substantial portions of the Software.
% 
%  THE SOFTWARE IS PROVIDED "AS IS", WITHOUT WARRANTY OF ANY KIND, EXPRESS
%  OR IMPLIED, INCLUDING BUT NOT LIMITED TO THE WARRANTIES OF MERCHANTABILITY,
%  FITNESS FOR A PARTICULAR PURPOSE AND NONINFRINGEMENT. IN NO EVENT SHALL
%  THE AUTHORS OR COPYRIGHT HOLDERS BE LIABLE FOR ANY CLAIM, DAMAGES OR OTHER
%  LIABILITY, WHETHER IN AN ACTION OF CONTRACT, TORT OR OTHERWISE, ARISING
%  FROM, OUT OF OR IN CONNECTION WITH THE SOFTWARE OR THE USE OR OTHER
%  DEALINGS IN THE SOFTWARE.
%


%
% Code to generate sr top banners.  To use this in your latex doc insert
% the following code just after the \begin{document} command:
%
% \begin{htmlonly}
% \include{../Utilities/TeX/banner_top.tex}
% \end{htmlonly}
%
% and adjust the path above as needed.
%
\begin{rawhtml}
<script type="text/javascript">
var treetop="";
var path = location.pathname;
if (path.charAt(path.length-1) == "/") {
  path += "bogus.html"
}
var base = path.substr(path.lastIndexOf("/")+1);
var roottag = "doc";
while (base != roottag && base != "") {
  treetop += "../";
  path = path.substr(0, path.lastIndexOf("/"));
  base = path.substr(path.lastIndexOf("/")+1);
}
var inDocTree = base == roottag;
if (inDocTree) {
  document.write('<script language="JavaScript" src="',treetop,'doc/Utilities/HTML/banner_top.js"><\/script>');
}
</script>
\end{rawhtml}

% \end{htmlonly}
%
% and adjust the path above as needed.
%
\begin{rawhtml}
<script type="text/javascript">
var treetop="";
var path = location.pathname;
if (path.charAt(path.length-1) == "/") {
  path += "bogus.html"
}
var base = path.substr(path.lastIndexOf("/")+1);
var roottag = "doc";
while (base != roottag && base != "") {
  treetop += "../";
  path = path.substr(0, path.lastIndexOf("/"));
  base = path.substr(path.lastIndexOf("/")+1);
}
var inDocTree = base == roottag;
if (inDocTree) {
  document.write('<script language="JavaScript" src="',treetop,'doc/Utilities/HTML/banner_top.js"><\/script>');
}
</script>
\end{rawhtml}

% \end{htmlonly}
%
% and adjust the path above as needed.
%
\begin{rawhtml}
<script type="text/javascript">
var treetop="";
var path = location.pathname;
if (path.charAt(path.length-1) == "/") {
  path += "bogus.html"
}
var base = path.substr(path.lastIndexOf("/")+1);
var roottag = "doc";
while (base != roottag && base != "") {
  treetop += "../";
  path = path.substr(0, path.lastIndexOf("/"));
  base = path.substr(path.lastIndexOf("/")+1);
}
var inDocTree = base == roottag;
if (inDocTree) {
  document.write('<script language="JavaScript" src="',treetop,'doc/Utilities/HTML/banner_top.js"><\/script>');
}
</script>
\end{rawhtml}

\end{htmlonly}

% The title page
\title{\LaTeX{} Module Descriptions (Explicated)}
\author{Ted Dustman}
\maketitle 
\newpage
\tableofcontents
\newpage


\section{Quick Start}

\begin{enumerate}
\item Create a directory for the module description in the cvs tree.  If the
  description is part of the \sr{} package, create a directory
  in \directory{src/Dataflow/TeX}.  If the description is part of another
  package, create a directory in
  \directory{src/Packages/\ab{package}/Dataflow/TeX} where \ab{package} is
  the name of the appropriate package.  Name the directory 
  after the module it describes.
\item Copy the template file,
  \htmladdnormallink{\filename{module-desc-template.tex}}{../module-desc-template.tex},
  to the directory just created.  Name the template file after its parent
  directory with an extension of \filename{.tex}.  The template file is
  now the module description document.
\item Insert the document's figures into the directory created above.  Provide
  figures in JPG or GIF formats.
\item Look at the file \htmladdnormallink{\filename{scirun-doc.sty}}
  {\texutils{}/scirun-doc.sty} and observe the markup
  commands recommended for use in the module description file.
\item Flesh out the document by following the instructions in the
  module description document (the copy of the template).
\item Notify the doc group of the new module description
  (scirun-doc@sci.utah.edu)
\item For additional information, refer to the following sections.
\end{enumerate}


\section{Document Name and Location}

The document lives in the \directory{src} side of the CVS tree.  If the document
describes a module in the \sr{} package, it  lives in a
sub-directory of the directory \directory{src/Dataflow/TeX}.  If the document
describes a module in another package,  it lives in a sub-directory
of the directory \directory{src/Packages/\ab{package}/Dataflow/Tex} where
\ab{package} is the name of the appropriate package.

The document's file should be named after the module it
describes with an extension of \filename{.tex}.  The document's
directory \emph{must} have the same name as the document itself sans the
\filename{.tex} extension. 

The tex document must consist of one file only.

Insert figures into the document's directory (not a sub-directory).

\section{Markup}

Do not use \LaTeX{} sectioning commands.  You are restricted to the
following sectioning commands: \latexcommand{ModuleRef},
\latexcommand{ModuleRefSummary}, \latexcommand{ModuleRefUse},
\latexcommand{ModuleRefDetails}, \latexcommand{ModuleRefNotes},
\latexcommand{ModuleRefCredits}, and \latexcommand{ModuleRefSubSection},
\latexcommand{ModuleRefSubSubSection}.  The latter two commands can be used
to add sub-sections and sub-sub-sections to the top-level sections of the
module description document.  

Markup the document using \LaTeX{} commands and commands in the file
\htmladdnormallink{\filename{scirun-doc.sty}}{\texutils{}/scirun-doc.sty}. 

Avoid presentation style markup commands such as \latexcommand{bf} or
\latexcommand{emph}.  Most  \LaTeX{} commands can be used.

You may invent your own markup commands as long as they are not
presentation oriented and do not conflict with commands in
\htmladdnormallink{\filename{scirun-doc.sty}}{\texutils{}/scirun-doc.sty}.

\section{Organization}

The \LaTeX{} version of a module description document consists of six parts: The
\latexcommand{ModuleRef} command, and the Summary, Use, Details, Notes, and
Credits sections.  The purpose of each section is described below:

\begin{description}
\item[The \latexcommand{ModuleRef} Command]\mbox{}
  
  The \latexcommand{ModuleRef} command is the first command in the
  module description file.  It specifies the module's
  name, and the category and package to which the module belongs:

\begin{verbatim}
\ModuleRef{\Module{module-name}}{\Category{category-name}}
    {\Package{package-name}}
\end{verbatim}
  
  Substitute appropriate values for ``module-name'',
  ``category-name'', and ``package-name''.

\item[Summary]
  \label{sec:summary} \mbox{}

  The Summary section is a brief summary of the module's purpose.  It
  should be no more than one paragraph.

  The \latexcommand{ModuleRefSummary} command starts the Summary section:

\begin{verbatim}
\ModuleRefSummary

My module performs x, y, and z.     
\end{verbatim}

\item[Use] \label{sec:use} \mbox{}
  
  The Use section describe's the module's use.  It describes the
  module's upstream and downstream connections and use of the module's
  \gui{}.  In \acronym{JPG} or  \acronym{GIF} formats, include at least one image  of the module's graphical user interface.  See \secref{Including Images}{sec:inclimg} for details on
  including images.
  
  Relate control of the module via its \gui{} to its theoretical
  under-pinnings (and perhaps other details) described in the
  Details section.
  
  Walk the user through an example, such as a small
  complete network that uses the module.

  The \latexcommand{ModuleRefUse} command starts the Use section:
  
\begin{verbatim}
\ModuleRefUse

My module is so intuitive it needs no documentation.
\end{verbatim}

\item[Details] \mbox{}
  \label{sec:details}
  
  The \latexcommand{ModuleRefDetails} command starts the Details section.

  Details not directly associated with the module's use belong in
  the Details section.
  
  Describe the module's algorithms, mathematical under-pinnings, or
  implementation details if appropriate.
  
  Relate the theory to the control of the module via its \gui{}
  described in your Use section.

\begin{verbatim}
\ModuleRefDetails

Details, details, to heck with the details!
\end{verbatim}

\item[Notes]\mbox{}
  \label{sec:notes}
  
  Special installation instructions (special configure options), bugs, limitations, and possible future work should be
  described in the Notes section.

  The \latexcommand{ModuleRefNotes} command starts the Notes section:

\begin{verbatim}
\ModuleRefNotes

Don't tell anyone, but this module just adds noise to its input.
\end{verbatim}

\item[Credits]\mbox{}
  \label{sec:credits}
  
  The \latexcommand{ModuleRefCredits} command starts the Credits section:

\begin{verbatim}
\ModuleRefCredits

Me!
\end{verbatim}
\end{description}

See the Quick Start instructions above and the notes in
\\
\htmladdnormallink{\filename{module-desc-template.tex}}{../module-desc-template.tex}.

\section{Including Images}
\label{sec:inclimg}

The \secname{Use}{sec:use} section of the document must include at
least one image of the module's graphical user interface. Create
images in  \acronym{JPEG} or \acronym{GIF} formats.  Note that EPS versions of the
figures should not be provided. These figures will generate automatically as needed.

Including images is a two step process.  You must first create an
\dfn{image command} for each image.   Then you must use each
image command at the approiate place in the text.


\subsection{Creating an Image Command}
\label{sec:crimgcmd}

Create an image command by  copying the
following template for each image.  Then
replace angle-bracketed text with appropriate values.  Below is the template:

%begin{latexonly}
\begin{verbatim}
%begin{latexonly}
  \newcommand{\<image-command-name>}%
  {\centerline{\includegraphics[<options>]
               {\ModuleRefFigName{<figure-name>.eps}}}}
%end{latexonly}
\begin{htmlonly}
  \newcommand{\<image-command-name>}{%
    \htmladdimg[<options>]{\ModuleRefFigName{<figure-name.jpg-or-gif>}}}
\end{htmlonly}
\end{verbatim}
%end{latexonly}

\begin{rawhtml}
  <PRE>
  &#37begin{latexonly}
  \newcommand{\&lt;image-command-name&gt;}
  {\centerline{\includegraphics[&lt;options&gt;]
               {\ModuleRefFigName{&lt;<figure-name>.eps&gt;}}}}
  &#37end{latexonly}
  &#92begin{htmlonly}
  \newcommand{\&lt;image-command-name&gt;}
  &nbsp&nbsp   {\htmladdimg[&lt;options&gt;]
                {\ModuleRefFigName{&lt;<figure-name.jpg-or-gif>&gt;}}}
  &#92end{htmlonly}
  </PRE>
\end{rawhtml}


The ``latexonly'' part, ``options'' includes a list of key-value pair options.
Typically, the options specify graphic width or height, scale, and bounding
box.  See \etitle{The Latex Graphics Companion} for more information.  Note
the \latexcommand{includegraphics} command from the
\latexpackage{graphicx} package is used. Also, do not delete the two
comment characters before the \verb|\%begin{latexonly}| and
\verb|\%end{latexonly}|). 

In the ``htmlonly'' part ``options'' there is a comma separated list of options
such as ``align'', ``width'', ``height'', and ``alt''.  See \etitle{The
  Latex Web Companion} for details.

Note that use of the \latexcommand{ModuleRefFigName} command above is
required.

For example:
%
%begin{latexonly}
\begin{verbatim}
%begin{latexonly}
  \newcommand{\testfig}{\centerline{\includegraphics[scale=.75]
                        {\ModuleRefFigName{testfig.eps}}}}
%end{latexonly}
\begin{htmlonly}
  \newcommand{\testfig}{\htmladdimg[align=top,alt="Test Figure"]
              {\ModuleRefFigName{testfig.jpg}}}
\end{htmlonly}
\end{verbatim}
%end{latexonly}


\begin{rawhtml}
  <PRE>
  &#37begin{latexonly}
  \newcommand{\testfig}{\centerline{\includegraphics[scale=.75]
              {\ModuleRefFigName{testfig.eps}}}}
  &#37end{latexonly}
  &#92begin{htmlonly}
  \newcommand{\testfig}{\htmladdimg[align=top,alt="Test Figure"]{\ModuleRefFigName{testfig.jpg}}}
  &#92end{htmlonly}
  </PRE>
\end{rawhtml}


\subsection{Using an Image Command}
\label{sec:useimgcmd}

Use image commands by enclosing each image
command within a \latexenv{figure} environment as follows:

%begin{latexonly}
\begin{verbatim}
\begin{figure}
  \begin{makeimage}
  \end{makeimage}
  \<image-command-name>
  \caption{\label{fig:<figure label>} Caption-text}
\end{figure}
\end{verbatim}
%end{latexonly}

\begin{rawhtml}
  <PRE>
  \begin{figure}
    &#92begin{makeimage}
    &#92end{makeimage}
    \&lt;image-command-name&gt;
    \caption{\label{fig:figure-label} Caption-text}
  \end{figure}
  </PRE>
\end{rawhtml}


Replace labels and names appropriately. For example:

%begin{latexonly}
\begin{verbatim}
\begin{figure}
  \begin{makeimage}
  \end{makeimage}
  \testfig
  \caption{\label{fig:testfig} This is a test figure.}
\end{figure}
\end{verbatim}
%end{latexonly}

\begin{rawhtml}
  <PRE>
  \begin{figure}
    &#92begin{makeimage}
    &#92end{makeimage}
    \testfig
    \caption{\label{fig:testfig} This is a test figure.}
  \end{figure}
  </PRE>
\end{rawhtml}

\section{Using \LaTeX{} Packages}

 The \latexcommand{usepackage} command can be used only if the package it is
including is part of the standard \LaTeX{} distribution.  

Commands from packages \latexpackage{graphicx} and
\latexpackage{html} can be used without including them with the
\latexcommand{usepackage} command.

\section{Creating Citations}

Cite references to other publications with the \latexcommand{cite} command.

Provide a ``bib'' file containing bibliographic entries
referenced in the document.


\begin{htmlonly}
%
%  For more information, please see: http://software.sci.utah.edu
% 
%  The MIT License
% 
%  Copyright (c) 2004 Scientific Computing and Imaging Institute,
%  University of Utah.
% 
%  License for the specific language governing rights and limitations under
%  Permission is hereby granted, free of charge, to any person obtaining a
%  copy of this software and associated documentation files (the "Software"),
%  to deal in the Software without restriction, including without limitation
%  the rights to use, copy, modify, merge, publish, distribute, sublicense,
%  and/or sell copies of the Software, and to permit persons to whom the
%  Software is furnished to do so, subject to the following conditions:
% 
%  The above copyright notice and this permission notice shall be included
%  in all copies or substantial portions of the Software.
% 
%  THE SOFTWARE IS PROVIDED "AS IS", WITHOUT WARRANTY OF ANY KIND, EXPRESS
%  OR IMPLIED, INCLUDING BUT NOT LIMITED TO THE WARRANTIES OF MERCHANTABILITY,
%  FITNESS FOR A PARTICULAR PURPOSE AND NONINFRINGEMENT. IN NO EVENT SHALL
%  THE AUTHORS OR COPYRIGHT HOLDERS BE LIABLE FOR ANY CLAIM, DAMAGES OR OTHER
%  LIABILITY, WHETHER IN AN ACTION OF CONTRACT, TORT OR OTHERWISE, ARISING
%  FROM, OUT OF OR IN CONNECTION WITH THE SOFTWARE OR THE USE OR OTHER
%  DEALINGS IN THE SOFTWARE.
%


%
% Code to generate sr bottom banners.  To use this in your latex doc insert
% the following code just before the \end{document} command:
%
% \begin{htmlonly}
% %
%  For more information, please see: http://software.sci.utah.edu
% 
%  The MIT License
% 
%  Copyright (c) 2004 Scientific Computing and Imaging Institute,
%  University of Utah.
% 
%  License for the specific language governing rights and limitations under
%  Permission is hereby granted, free of charge, to any person obtaining a
%  copy of this software and associated documentation files (the "Software"),
%  to deal in the Software without restriction, including without limitation
%  the rights to use, copy, modify, merge, publish, distribute, sublicense,
%  and/or sell copies of the Software, and to permit persons to whom the
%  Software is furnished to do so, subject to the following conditions:
% 
%  The above copyright notice and this permission notice shall be included
%  in all copies or substantial portions of the Software.
% 
%  THE SOFTWARE IS PROVIDED "AS IS", WITHOUT WARRANTY OF ANY KIND, EXPRESS
%  OR IMPLIED, INCLUDING BUT NOT LIMITED TO THE WARRANTIES OF MERCHANTABILITY,
%  FITNESS FOR A PARTICULAR PURPOSE AND NONINFRINGEMENT. IN NO EVENT SHALL
%  THE AUTHORS OR COPYRIGHT HOLDERS BE LIABLE FOR ANY CLAIM, DAMAGES OR OTHER
%  LIABILITY, WHETHER IN AN ACTION OF CONTRACT, TORT OR OTHERWISE, ARISING
%  FROM, OUT OF OR IN CONNECTION WITH THE SOFTWARE OR THE USE OR OTHER
%  DEALINGS IN THE SOFTWARE.
%


%
% Code to generate sr bottom banners.  To use this in your latex doc insert
% the following code just before the \end{document} command:
%
% \begin{htmlonly}
% %
%  For more information, please see: http://software.sci.utah.edu
% 
%  The MIT License
% 
%  Copyright (c) 2004 Scientific Computing and Imaging Institute,
%  University of Utah.
% 
%  License for the specific language governing rights and limitations under
%  Permission is hereby granted, free of charge, to any person obtaining a
%  copy of this software and associated documentation files (the "Software"),
%  to deal in the Software without restriction, including without limitation
%  the rights to use, copy, modify, merge, publish, distribute, sublicense,
%  and/or sell copies of the Software, and to permit persons to whom the
%  Software is furnished to do so, subject to the following conditions:
% 
%  The above copyright notice and this permission notice shall be included
%  in all copies or substantial portions of the Software.
% 
%  THE SOFTWARE IS PROVIDED "AS IS", WITHOUT WARRANTY OF ANY KIND, EXPRESS
%  OR IMPLIED, INCLUDING BUT NOT LIMITED TO THE WARRANTIES OF MERCHANTABILITY,
%  FITNESS FOR A PARTICULAR PURPOSE AND NONINFRINGEMENT. IN NO EVENT SHALL
%  THE AUTHORS OR COPYRIGHT HOLDERS BE LIABLE FOR ANY CLAIM, DAMAGES OR OTHER
%  LIABILITY, WHETHER IN AN ACTION OF CONTRACT, TORT OR OTHERWISE, ARISING
%  FROM, OUT OF OR IN CONNECTION WITH THE SOFTWARE OR THE USE OR OTHER
%  DEALINGS IN THE SOFTWARE.
%


%
% Code to generate sr bottom banners.  To use this in your latex doc insert
% the following code just before the \end{document} command:
%
% \begin{htmlonly}
% \include{../Utilities/TeX/banner_bottom.tex}
% \end{htmlonly}
%
% and adjust the path above as needed.
%

\begin{rawhtml}
<script type="text/javascript">
if (inDocTree) {
  document.write('<script type="text/javascript" src="',treetop,'doc/Utilities/HTML/banner_bottom.js"><\/script>');
}
</script>
\end{rawhtml}

% \end{htmlonly}
%
% and adjust the path above as needed.
%

\begin{rawhtml}
<script type="text/javascript">
if (inDocTree) {
  document.write('<script type="text/javascript" src="',treetop,'doc/Utilities/HTML/banner_bottom.js"><\/script>');
}
</script>
\end{rawhtml}

% \end{htmlonly}
%
% and adjust the path above as needed.
%

\begin{rawhtml}
<script type="text/javascript">
if (inDocTree) {
  document.write('<script type="text/javascript" src="',treetop,'doc/Utilities/HTML/banner_bottom.js"><\/script>');
}
</script>
\end{rawhtml}

\end{htmlonly}

\end{document}
