%
%  For more information, please see: http://software.sci.utah.edu
% 
%  The MIT License
% 
%  Copyright (c) 2004 Scientific Computing and Imaging Institute,
%  University of Utah.
% 
%  License for the specific language governing rights and limitations under
%  Permission is hereby granted, free of charge, to any person obtaining a
%  copy of this software and associated documentation files (the "Software"),
%  to deal in the Software without restriction, including without limitation
%  the rights to use, copy, modify, merge, publish, distribute, sublicense,
%  and/or sell copies of the Software, and to permit persons to whom the
%  Software is furnished to do so, subject to the following conditions:
% 
%  The above copyright notice and this permission notice shall be included
%  in all copies or substantial portions of the Software.
% 
%  THE SOFTWARE IS PROVIDED "AS IS", WITHOUT WARRANTY OF ANY KIND, EXPRESS
%  OR IMPLIED, INCLUDING BUT NOT LIMITED TO THE WARRANTIES OF MERCHANTABILITY,
%  FITNESS FOR A PARTICULAR PURPOSE AND NONINFRINGEMENT. IN NO EVENT SHALL
%  THE AUTHORS OR COPYRIGHT HOLDERS BE LIABLE FOR ANY CLAIM, DAMAGES OR OTHER
%  LIABILITY, WHETHER IN AN ACTION OF CONTRACT, TORT OR OTHERWISE, ARISING
%  FROM, OUT OF OR IN CONNECTION WITH THE SOFTWARE OR THE USE OR OTHER
%  DEALINGS IN THE SOFTWARE.
%

% module-template.tex ----
%
% Notes:
%
% o Rename this file as <Module>-<Category>-<Package>.tex where
%   bracketed strings are replaced by appropriate names.
%
% o Complete relevant sections below.
%
% o Use markup commands defined in scirun-doc.tex.
%   No presentation markup allowed (e.g. no \textbf allowed and even use of
%   \emph is discouraged).
%
% o To make your document web-friendly, use commands from html package. See
%   latex2html manual or the book ``The Latex Web Companion.''
%
% o Don't add new subsections, but freely subsubsection 
%   existing subsections.
%
% o Add local commands if desired (via \newcommand).
%
% o Only use packages distributed with Latex2e.
%
% o Cite references with the \cite command. Put citations in
%   a bib file and send along with this file.
%
% o Please include at least one image of tk GUI. Provide 
%   an EPS version and a web friendly version (e.g., jpeg or gif).
%
% o Including images:  Create a figure command for each figure using the
%   template below.  Copy the template (make a copy of the template
%   for each figure) and uncomment only the first '%' in each line.
%   Then replace bracketed text with appropriate values.  Here is the
%   template: 
% 
%%begin{latexonly}
%  \newcommand{\<figure command name>}%
%  {\centerline{\includegraphics[options]{<path to eps file>}}}
%%end{latexonly}
%\begin{htmlonly}
%  \newcommand{\<figure command name (same as above)>}{%
%  \htmladdimg[options]{<path to jpg file>}}
%\end{htmlonly}
%
%   Where in the ``latexonly'' part, ``options'' is a list of key-value pair
%   options.  Typically the options specify graphic width, height, and
%   bounding box.  See ``The Latex Graphics Companion'' for more
%   information.  Note that we are using ``\includegraphics'' command
%   from the ``graphicx'' package.
%
%   In the ``htmlonly'' part ``options'' is a list options such as ``align'',
%   ``width'', ``height'', and ``alt''.  See ``The Latex Web Companion'' for
%   details.
%   
%   Now to actually create figures enclose each figure command as follows:
%
%\begin{figure}
%  \begin{makeimage}
%  \end{makeimage}
%  \<figurecommand>
%  \caption{\label{fig:<figure label>} <Caption text>}
%\end{figure}
%
%   and replace bracket text appropriately.
%   

% Uncomment following 5 commented lines if you want to process this document
% with Latex. You must also uncomment the last line of the file.

% Module information: replace the XXXX's

\newcommand{\x}{\times}

\ModuleRef{\Module{Tikhonov}}{\Category{Inverse}}{\Package{BioPSE}}

% Summary of module's purpose. No more than one paragraph.

\ModuleRefSummary

The purpose of this module is to apply Tikhonov regularization to an
existing forward model, with flexible control of regularization type and
parameters.  Currently, the module solves the inverse problem at a single
time instant.  This module requires a forward model matrix, geometries for
the associated surfaces, and some remote boundary conditions, \ie{}, torso or
head surface potentials.

% How to use the module.  Describe connections upstream and downstream
% and how to use the GUI.  If desired, Walk the user through an example.
% Please include at least one image of tk GUI. Provide 
% an EPS version and a web-friendly version (e.g., jpeg or gif).
% See the note above that explains how to include images.

\ModuleRefUse

The \module{Tikhonov} module has three inputs, two outputs, and a user
interface (UI) to select the regularization parameter. 
%
\begin{itemize} 
  \item {\bf Inputs:} 
        \begin{enumerate}          
          \item Forward problem matrix, $\left[{\bf A}\right]$. This matrix
                could be created using \module{SetupBEMatrix} module, or
                created elsewhere and saved as SCIRun matrix format and
                loaded via \module{MatrixReader} module.
          \item Regularization matrix, $\left[{\bf R}\right]$: this matrix will
                contain the transform that constrains the regularized
                inverse solution. It is typically the output of the
                \module{AttributeTrf} module, but as in the Forward matrix,
                it can be created externally and loaded via
                \module{MatrixReader} module. If nothing is connected to
                this input, identity matrix is used as default.
          \item Neumann Boundary conditions, $\left[{\bf y}\right]$: Values of
                potential on the outer boundary of the geometry, typically
                the body surface or head surface potentials. This is the
                attributes part of a field data type. The
                \module{ManageFieldData} module can be used to extract the
                vector of attributes (\ie the potentials) from the original
                field data.
        \end{enumerate}         
      
  \item {\bf Outputs:} 
        \begin{enumerate}         
          \item Solution potentials, $\left[{\bf x}_{\lambda}\right]$. This
                is a vector of potentials without any geometry
                information. This vector can be combined with the
                appropriate geometry file using the
                \module{ManageFieldData} module to create a field data
                format.
          \item Augmented forward matrix $\left[{\bf A}_{reg}\right]$ where
                ${\bf A}_{reg} = \left( {\bf A}^{T}\;{\bf A} +
                \lambda^{2}\;{\bf R}^{T}\;{\bf R}\right)$ 
% (I am not sure what to do with this :-) )
        \end{enumerate}         
             
  \item {\bf UI:} UI of the \module{Tikhonov} module is used to select
                method to choose $\lambda$. Fig.[Tikh-UI] shows the UI with
                the L-curve option selected and the corresponding L-curve
                plotted. Three options have been implemented so far:
        \begin{enumerate}         
          \item Enter a single value: the user can type any value in the UI
                and the solution is implemented for that $\lambda$ value
                only. 
          \item Choose from a slider: the user can select a value by moving
                the slider. So far, the range of the slider and the
                increments are pre-defined inside the code. 
          \item Determine the value using an L-curve: L-curve method
                described in section~\ref{Sec:regpar} is used to obtain the
                best $\lambda$ value. Currently, the range of
                regularization parameters that is used to obtain the
                L-curve is fixed. 
        \end{enumerate}
      
\end{itemize}

An example SCIRun network to solve for the potentials on the heart surface
at a particular time instant is shown in Fig.[Tikh-net]. 

% Description of module's algorithms, mathematics, etc.

\ModuleRefDetails

\begin{itemize}
  \item The standard discretized formulation for the forward problem:
        \begin{equation}
          \label{Eqn:prob}
          {\bf y} = {\bf A \; x} + {\bf n}
        \end{equation}
      where ${\bf y}$ is an $M\x1$ vector of outer surface potentials (\eg,
      torso potentials), ${\bf x}$ is the $N\x1$ vector of inner surface
      potentials (\eg, epicardial potentials), ${\bf A}$ is the $M\x N$
      matrix representing the forward solution and ${\bf n}$ is measurement
      noise of the same dimension as ${\bf y}$ 
             
  \item The idea in Tikhonov regularization is to define the regularized
        solution 
        ${\bf x}_{\lambda}$ as the minimizer of the following weighted
        combination 
        of the residual norm and the side constraint:  
        \begin{equation}
          {\bf x}_{\lambda} = \mbox{arg}\min_{{\bf x}}\left\{ ||{\bf A\,x}-{\bf
          y}||_{2}^{2} + \lambda^{2}\;||{\bf R\;x})||_{2}^{2}\right\}
        \end{equation}
      
      and          
        \begin{equation}         
          {\bf x}_{\lambda} = \left( {\bf A}^{T}\;{\bf A} + \lambda^{2}\;{\bf
          R}^{T}\;{\bf R}\right)^{-1}\; {\bf A}^{T}\;{\bf y}
        \end{equation}
      %
      where the regularization parameter $\lambda$ controls the weight given to
      minimization of the residual norm. 
\end{itemize}         

\ModuleRefSubSection{Selecting regularization parameter, $\lambda$}
\label{Sec:regpar}

\begin{itemize}
  \item {\bf The L-Curve}
        Perhaps the most convenient graphical tool for analysis of discrete
        ill-posed problems is the so-called L-curve which is a plot - for
        all valid regularization parameters - of the (semi)norm $||{\bf
        R\;x_{reg}})||$ of the regularized solution versus the
        corresponding residual norm $||{\bf A\,x_{reg}}-{\bf y}||$. In this
        way, the L-curve clearly displays the compromise between
        minimization of these two quantities, which is the heart of any
        regularization method. For discrete ill-posed problems it turns out
        that the L-curve, when plotted in log-log scale, almost always has
        a characteristic L-shaped appearance with a distinct corner
        separating the vertical and the horizontal parts of the curve.
        
\end{itemize}
             
% Special notes.  These may be installation instructions, bugs, limitations,
% etc.

%\ModuleRefNotes

%Write notes here.

% Who is responsible? 

\ModuleRefCredits

Authors of the \module{Tikhonov} module are Yesim Serinagaoglu, Alireza
Ghodrati and Dana H. Brooks with assistance from Rob MacLeod and Dave
Weinstein.
