\documentclass[12pt]{article}
\usepackage[dvips]{graphicx}

%%%%% My macros - get rid of them later if needed.

\newcommand{\single }{\renewcommand{\baselinestretch}{1} \large \normalsize}
\newcommand{\hdouble}{\renewcommand{\baselinestretch}{1.5} \large \normalsize}
\newcommand{\double }{\renewcommand{\baselinestretch}{1.66} \large \normalsize}

\input amssym.def
\input amssym

\def \n {{\mbox{ }}\\}
\def \l {\lambda}
\def \L {\Lambda}
\def \be {\begin{equation}}
\def \ee {\end{equation}}
\def \loc {{\mathrm{local}}}
\def \e {\varepsilon}
\def \w {\omega}
\def \W {\Omega}
\def \dW {\partial \Omega}
\def \bW {\bar{\W}}
\def \sgn {{\mathrm{sgn}}}
\def \bsigma {\bar{\sigma}}
\def \bw {\bar{\w}}
\def \bN {\bar{N}}
\def \k {\kappa}
\def \opt {{\mathrm{opt}}}
\def \te {\tilde{\e}}
\def \a { \alpha }
\def \xa {x_{\a}}
\def \g {\gamma}
\def \gsc {GS$_c$ }
\def \gsct {GS$_c$}
\def \crelax {CR${(\nu_1,\nu_2)}$ }
\def \Cc {C^{(c)}}
\def \mod {{\mbox{\rm mod }}}
\def \t { \theta }
\def \et {e^{i \t \a}}
\def \etm {e^{-i \t \a}}
\def \pc {\frac{2\pi}{c}}
\def \pcv {(2 \pi)/c}
\def \tc#1 {\t^{c,(#1)} }
\def \tctwo#1 {\t^{c_2,(#1)} }
\def \Ac#1 {A_{#1}}
\def \ec {\eta}
\def \xic {\xi}
\def \ux {\underline{x}}
\def \uy {\underline{y}}
\def \T {\Theta}
\def \B {{\cal B}}
\def \V {{\cal V}}
\def \F {{\cal F}}
\def \H {{\cal H}}
\def \K {{\cal K}}
\def \PB {{\cal P}}
\def \qed    {$\blacksquare$}
\def \bmu {\overline{\mu}}
\def \bM {\overline{M}}
\def \mod {{\mbox{\rm mod }}}
\def \lcm {{\mbox{\rm lcm}}}
\def \sa {\Sigma_{\a}}
\def \ip#1#2 {\langle #1,#2 \rangle}
\def \vec#1#2     { \left( \begin{array}{c} #1 \\ #2 \end{array} \right) }
\def \mat#1#2#3#4 {\left(\begin{array}{cc} #1 & #2\\#3 & #4\end{array}\right) }
\def \D {\Delta}
\def \px {\partial_x}
\def \py {\partial_y}
\def \pz {\partial_z}
\def \Dh {\Delta^h}
\def \pxh {\partial_x^h}
\def \pyh {\partial_y^h}
\def \pzh {\partial_z^h}
\def \AA {\hat{A}}
\def \LL {\hat{L}}
\def \MM {\hat{M}}
\def \LM {\LL \MM}
\def \eee {\hat{e}}
\def \ff {\hat{f}}
\def \uu {\hat{u}}
\def \cfl {\mathrm{CFL}}

%%%%%%%%%%%%% for siam : comment the following lines
\newtheorem{theorem}{Theorem}
\newtheorem{lemma}{Lemma}
\newtheorem{definition}{Definition}
\newtheorem{corollary}{Corollary}
\newtheorem{remark}{Remark}
\newcommand\keywordsname{Key words}
\newcommand\AMSname{AMS subject classifications}
\newenvironment{@abssec}[1]{%
     \if@twocolumn
       \section*{#1}%
     \else
       \vspace{.05in}\footnotesize
       \parindent .2in
         {\upshape\bfseries #1. }\ignorespaces
     \fi}
     {\if@twocolumn\else\par\vspace{.1in}\fi}
\newenvironment{keywords}{\begin{@abssec}{\keywordsname}}{\end{@abssec}}
\newenvironment{AMS}{\begin{@abssec}{\AMSname}}{\end{@abssec}}
%%%%%%%%%%%%% siam: comment up to here

%\def \britem {\begin{romannum}}
%\def \eritem {\end{romannum}}

\def \bn {\begin{enumerate}}
\def \en {\end{enumerate}}
\def \bi {\begin{itemize}}
\def \ei {\end{itemize}}

\title{Finite Volume Discretization Scheme for \\
Poisson's Equation on AMR Grids}
\author{Oren E.~Livne \footnotemark[1]}

%%%%%%%%%%%%%%%%%%%%%%%%%%%%% BEGINNING OF DOCUMENT %%%%%%%%%%%%%%%%%%%%%%%

\begin{document}

%\def \myspace {\double}
\def \myspace {\single}
\myspace

\maketitle

\renewcommand{\thefootnote}{\fnsymbol{footnote}}
\footnotetext[1]{SCI Institute, 50 South Central Campus Dr., Room 3490,
University of Utah, Salt Lake City,
UT 84112.Phone: +1-801-581-4772. Fax: +1-801-585-6513.
Email address: {\tt livne@sci.utah.edu}}
\renewcommand{\thefootnote}{\arabic{footnote}}

\begin{abstract}
We introduce a symmetric discretization scheme for Poisson's equation on a
composite grid resulting from Adaptive Mesh Refinement (AMR) with structured
patches. The discretization is intended for use with the FAC
\cite{} or Algebraic Multigrid (AMG) solvers \cite{}. We analyze the
discretization error for a model problem with a smooth solution.
\end{abstract}

\begin{keywords}
Adaptive Mesh Refinement (AMR), flux matching, Fast Adaptive Composite
(FAC), multigrid, second-order elliptic Partial Differential Equations (PDEs).
\end{keywords}

%\begin{AMS}
%42A16, 42A38, 65B99, 65G99, 65N06, 65N12, 65N55.
%\end{AMS}

\pagestyle{myheadings}
\thispagestyle{plain}
\markboth{}{{\small OREN E.~LIVNE: SHOCKTUBE STATUS REPORT 5}}

%\newpage
%\tableofcontents

%%%%%%%%%%%%%%%%%%%%%%%%%%%%% BEGINNING OF ACTUAL TEXT %%%%%%%%%%%%%%%%%%%%%%%

\section{Model Problems}
We discuss a discretization of Poisson's equation
\begin{eqnarray}
\Delta U(x) = f(x) &&  x \in \Omega \subset {\Bbb R}^d \\
U(x) = g(x) &&  x \in \partial \Omega
\end{eqnarray}
in $d$ dimensions ($d \leq 3$ is mostly the case). The grid is a composite
grid of structured mesh refinements. We assume the AMR
hierarchy is given in terms of increasingly finer
levels $k = 1,\ldots,K$. Each level admits a uniform meshsize $h_k$. For
simplicity, we assume the same meshsize in all directions. Each level
is a union of patches $q = 1,\ldots,k_q$. A patch is a rectilinear uniform
grid covering the entire domain $\Omega$ or part of it. Patches are nested
so that their boundaries align with their parent patch's gridlines, and there
is at least one layer of parent coarse patch cells
outside the fine patch extents. Patches are thus strictly contained in the
previous level's patches, except near domain boundaries, where the extra
layer of cells is not required.

Our model problems for analysis of discretization error of the proposed scheme
are:
\bi
\item \emph{Problem A:} $d = 2$, $\Omega = [0,1]^2$.
$f,g$ chosen so that the exact solution is
\be
U(x,y) = \sin\left(\pi x\right) \sin\left(\pi y\right).
\ee
This problem does not require local mesh refinements to obtain optimal
work/accuracy relationship. In other words, it cannot benefit from locally
refining the grid. We check our discretization on this problem with one level
($K=1$), to verify $O(h^2)$ discretization error in $U$, where $h=h_1$.

\item \emph{Problem B:} $d = 2$, $\Omega = [0,1]^2$.
$f,g$ chosen so that the exact solution is
\be
U(x,y) = \sin(\pi x) \sin(\pi y) + V(x,y),
\ee
where $V(x,y)$ is locally supported in $[\frac14,\frac12]^2$ and sufficiently
smooth. Specifically, we choose
\be
V(x,y) = \sin\left(2 \pi (x-\frac14)\right)
\sin\left(2 \pi (y - \frac14)\right).
\ee
This problem does can benefit from locally refining the grid in the central
region $[\frac14,\frac12]^2$. Hence, we can test our discretization scheme
with two levels $h_1=h$ and $h_2=h/2$, for different $h$'s. Level $1$ has
one patch that extends over the entire domain; level $2$ has a patch that
extends over the central region only; see \S \ref{Results2}. We check again
that the discretization error in $U$ is $O(h^2)$.
\ei

\newpage
\input FullResults

\bibliographystyle{alpha}
\bibliography{reports}        % guide.bib is the name of our database

\end{document}
