\documentclass[10pt]{book}
\usepackage{amsmath}
\usepackage{amssymb}

\begin{document}

\title{Mixing \& Reaction Models}

\author{James C. Sutherland}

%\maketitle

\bibliographystyle{unsrt}

\chapter{Table Generation}
%%----------------------------------------------------------
\section{B-Splines}
\subsection{L-U Factorization}
%%----------------------------------------------------------
\section{HDF5 Databases}
%%----------------------------------------------------------
\section{Software API}
\subsection{Table Generation}
\subsection{B-Splines}
\subsubsection{1-D Splines}
\subsubsection{2-D Splines}
\subsubsection{Higher-Dimensional Splines}

%%==========================================================

\chapter{Reaction Models}
Given a single-phase reacting system with $N_s$ species, one must
specify $N_s \! +\! 1$ variables (\emph{e.g.}, $N_s \! -\! 1$ mass
fractions, temperature and pressure) to uniquely specify the entire
thermochemical state, $\boldsymbol{\phi}$, of the system
\cite{williams85,felder86,smith01}.  Reaction models parameterize
$\boldsymbol{\phi}$ by $N_\eta \ll N_s+1$ parameters called
\emph{reaction variables}, which we represent collectively as
$\boldsymbol{\eta}$.  A reaction model then provides a unique mapping
from $\boldsymbol{\eta}$ to $\boldsymbol{\phi}$, \emph{i.e.}, each
$\phi_i$ is represented by an $N_\eta$-dimensional surface in
$\boldsymbol{\eta}$-space.

We define a reaction model as a mapping between thermochemical state
variables and reaction variables,
\begin{equation}
  \label{eqn:rxnMdlDef}
  \boldsymbol{\phi} = \boldsymbol{\phi}(\boldsymbol{\eta}).
\end{equation}

%%----------------------------------------------------------
%% The Mixture Fraction
%%----------------------------------------------------------
\section{The Mixture Fraction}
As many of the results will use the mixture fraction as a reaction
variable, it is briefly defined here.  Elemental mass fractions,
$Z_\ell$, may be defined in terms of the species mass fractions,
$Y_i$, as
\begin{equation} \label{eqn:elemMassFrac}
  Z_\ell = \sum_{i=1}^{N_s} \frac{a_{\ell,i} W_\ell}{W_i}  Y_i,
\end{equation}
where $N_s$ is the number of species, $a_{\ell,i}$ is the number of
atoms of element $\ell$ in species $i$, $W_\ell$ is the molecular
weight of element $\ell$, and $W_i$ is the molecular weight of species
$i$.  The mixture fraction, $f$, may be written in terms of coupling
functions, $\beta$, as \cite{williams85}
\begin{equation}  \label{eqn:mixfrac_general}
  f = \frac{\beta-\beta_0}{\beta_1-\beta_0},
\end{equation}
where $\beta_1$ and $\beta_0$ are constants evaluated in the fuel and
oxidizer streams, respectively. The coupling function, $\beta$, is
defined in terms of the elemental mass fractions as
\begin{equation} \label{eqn:betas}
  \beta = \sum_{\ell=1}^{N_e} \gamma_\ell Z_\ell 
  = \sum_{\ell=1}^{N_e} \gamma_\ell \sum_{i=1}^{N_s} \frac{a_{\ell,i} W_\ell Y_i}{W_i},
\end{equation}
where $\gamma_\ell$ are weighting factors.  The $\gamma_\ell$ are not
unique; for this study Bilger's definition \cite{bilger90} of the
mixture fraction is adopted, for which
$\gamma_\mathrm{C}=2/W_\mathrm{C}$, $\gamma_\mathrm{H}=1/(2
W_\mathrm{H})$, $\gamma_O-1/W_\mathrm{O}$, $\gamma_\mathrm{N}=0$.  The
stoichiometric mixture fraction, $f_{st}$, is determined from
equation (\ref{eqn:mixfrac_general}) with $\beta=0$.  The dissipation rate is
defined as $\chi=2D \nabla f \cdot \nabla f$, with $D$ obtained from
$\mathrm{Le}=\lambda/(\rho c_p D)$.  In this study, the Lewis number
of the mixture fraction was assumed to be unity.

%%----------------------------------------------------------
%% Effects of Heat Loss
%%----------------------------------------------------------
\section{Effects of Heat Loss}
Heat loss is an important aspect of combustion modeling, and may occur
through radiation or conduction/convection.  We define the fractional
heat loss, $\gamma$ as
\begin{equation}
  \label{eqn:heatLoss}
  \gamma = \frac{h_a - h}{h_a - h_c}
         = \frac{h_a-h}{h_{a,s}}
\end{equation}

$h_{a,s} \equiv \sum_{i=1}^{N_s} Y_i h_{a,i}-h_i^o$

with
\begin{eqnarray}
  c_p &=& \sum_{i=1}^{N_s} Y_i c_{p,i},  \label{eqn:cp} \\
  h &=& \sum_{i=1}^{N_s} Y_i h_i,   \label{eqn:enthalpy} \\
  h_i &=& h_i^o + \int_{T_o}^T c_{p,i} \mathrm{d}T, \label{eqn:hi}
\end{eqnarray}
where $h_i$ is the enthalpy of species $i$, $h_i^o$ is the enthalpy of
of formation of species $i$ at temperature $T_o$, and $c_p$ is the
isobaric heat capacity.

%%----------------------------------------------------------
%% Burke-Schumann Chemistry
%%----------------------------------------------------------
\section{Burke-Schuman Model}
Burke-Schuman chemistry assumes infinitely fast, irreversible, and
complete reaction.  Thus, composition is a piece-wise linear function
of mixture fraction.  The enthalpy is determined from the composition
and equation \eqref{eqn:enthalpy}.  For the adiabatic case, all state
variables are unique functions of the mixture fraction,
$\boldsymbol{\phi} = \boldsymbol{\phi}(f)$.

\subsection{Heat Loss}

%%----------------------------------------------------------
%% Equilibrium
%%----------------------------------------------------------
\section{Equilibrium Model}
The chemical equilibrium model is valid when chemistry is infinitely
fast and reversible.  For an adiabatic two-stream mixing problem, the
thermochemical state is uniquely parameterized by the mixture
fraction, $\boldsymbol{\phi} = \boldsymbol{\phi}(f)$.

\subsection{Heat Loss}


%%==========================================================

\chapter{Mixing Models}
In a Reynolds-Avaraged Navier-Stokes (RANS) or Large-Eddy Simulation
(LES) calculation, mean or filtered state state variables are needed as
functions of the mean (or filtered) reaction variables.  This is
typically achieved through
\begin{equation}
  \bar{\phi}_i = \int_{\eta_{N_\eta}} \cdots \int_{\eta_1}
  \phi_i^\ast(\boldsymbol{\eta}) \; P(\eta_1 \cdots \eta_{N_\eta}) \;
  \mathrm{d} \eta_1 \cdots \mathrm{d} \eta_{N_\eta},
\end{equation}
where $\phi_i^\ast(\boldsymbol{\eta})$ is the reaction model, which
provides state variables, $\phi$, as unique functions of the set
of reaction variables, $\boldsymbol{\eta}$, and $P(\eta_1 \cdots
\eta_{N_\eta})$ is the joint probability density function (PDF) of all
reaction variables.  Clearly, in a RANS or LES computation, the error
in $\bar{\phi}$ has contributions from the reaction model as well as
the model used to approximate the joint PDF.  This study examines only
the error due to the reaction model, $\phi_i(\boldsymbol{\eta})$.

%%==========================================================

\chapter{User-Interface}
\section{The Parser}

\section{Line Commands}

%%---------------------------------------------------------------------
\bibliography{references}
%%---------------------------------------------------------------------
\end{document}