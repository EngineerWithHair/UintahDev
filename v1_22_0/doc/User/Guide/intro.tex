% -*-latex-*-
%
%  For more information, please see: http://software.sci.utah.edu
% 
%  The MIT License
% 
%  Copyright (c) 2004 Scientific Computing and Imaging Institute,
%  University of Utah.
% 
%  License for the specific language governing rights and limitations under
%  Permission is hereby granted, free of charge, to any person obtaining a
%  copy of this software and associated documentation files (the "Software"),
%  to deal in the Software without restriction, including without limitation
%  the rights to use, copy, modify, merge, publish, distribute, sublicense,
%  and/or sell copies of the Software, and to permit persons to whom the
%  Software is furnished to do so, subject to the following conditions:
% 
%  The above copyright notice and this permission notice shall be included
%  in all copies or substantial portions of the Software.
% 
%  THE SOFTWARE IS PROVIDED "AS IS", WITHOUT WARRANTY OF ANY KIND, EXPRESS
%  OR IMPLIED, INCLUDING BUT NOT LIMITED TO THE WARRANTIES OF MERCHANTABILITY,
%  FITNESS FOR A PARTICULAR PURPOSE AND NONINFRINGEMENT. IN NO EVENT SHALL
%  THE AUTHORS OR COPYRIGHT HOLDERS BE LIABLE FOR ANY CLAIM, DAMAGES OR OTHER
%  LIABILITY, WHETHER IN AN ACTION OF CONTRACT, TORT OR OTHERWISE, ARISING
%  FROM, OUT OF OR IN CONNECTION WITH THE SOFTWARE OR THE USE OR OTHER
%  DEALINGS IN THE SOFTWARE.
%


% intro.tex
%

\chapter{Introduction}
\label{ch:intro}

This is the \etitle{\srug}.  It describes the purpose and use of the
\sr{} problem solving environment (\pse).  This guide is for users who
are building and executing \dfn{networks} within the \sr{}
environment.

Users installing \sr{} should read the
\htmladdnormallinkfoot{\srig}{\latexhtml{\scisoftware/doc}{../../..}/Installation/Guide/index.html}.

Users of the BioTensor Power App should see the
\htmladdnormallinkfoot{BioTensor
  tutorial}{\latexhtml{\scisoftware/doc/User/Tutorials/BioTensor/BioTensor.html}{../../Tutorials/BioTensor/BioTensor.html}}.

Users of the BioFEM Power App should see the
\htmladdnormallinkfoot{BioFEM
  tutorial}{\latexhtml{\scisoftware/doc/User/Tutorials/BioFEM/BioFEM.html}{../../Tutorials/BioFEM/BioFEM.html}}.

%\section{Conventions}
%\label{sec:conventions}

%\missing{Discussion of typographic conventions}

\section{Road Map}
\label{sec:roadmap}

This document is organized into the following sections:

\begin{description}
  \descitem{\chref{Introduction}{ch:intro}} This introduction.
  
  \descitem{\chref{Concepts}{ch:concepts}} Introduces the concept of
  an integrated problem solving environment and describes how \SR{}
  embodies these ideas.
  
  \descitem{\chref{Packages}{ch:packages}} Gives an overview
  of the \sr{} and \biopse{} packages.
  
  \descitem{\chref{Starting \sr}{ch:startingup}} Outlines procedure
  for starting \sr{} and related information.

  \descitem{\chref{Working with Networks}{ch:workwithnets}}
  Discusses building, editing, and executing
  networks.

  \descitem{\chref{Visualization}{ch:viewer}}
  Describes the purpose and use of the \viewer{} (visualization) module.

  \descitem{\chref{Importing data into \sr{}}{ch:import_export}}
  Describes ways to import/export ``foreign'' data into/out of \SR{}.

  \descitem{\chref{Wrapping  ITK Filters in \sr{}
      Modules}{ch:itk_mods}}
  Describes the mechanism for wrapping ITK filters in \sr{} modules.
\end{description}

\section{Help}
\label{sec:help}

Help is available from the following sources.

\subsection{Documentation Distribution}

\sr{} documentation is distributed separately from its source code.
\sr{}'s documentation distribution can be downloaded from
\htmladdnormallinkfoot{\sr{}'s software download
  page}{\scisoftwarearchiveurl}.  See the 
\htmladdnormallinkfoot{\srig}{\latexhtml{\scisoftware/doc}{../../..}/Installation/Guide/index.html}
for instructions on downloading and installing the documentation
distribution.

After installing the documentation, visit the
\filename{index.html} file located in the distribution's top level
\directory{doc} directory (\ie{} \ab{top of documentation
  distribution}/doc/index.html).

\subsection{The Web}

This and other documents related to \sr{} can be found 
\htmladdnormallinkfoot{online}{\scidocurl{}}.

Visit the \htmladdnormallinkfoot{\sci}{\sciurl} web site for more
information related to \sr{} and the \scii{}.

\subsection{Mailing Lists}
\index{mailing lists}

The \sr{} \emph{users} mail list is a forum for discussing \sr{}
related issues.  Users are given the opportunity to subscribe to this
list when downloading \sr{} for the first time from the software
download page.  Users may also subscribe by sending mail to:

\mailto{Majordomo@sci.utah.edu}

with the following command in the body of the message:

\keyboard{subscribe scirun-users}

After subscribing,  questions can be sent to
\mailto{scirun-users@sci.utah.edu}.

The \sr{} \emph{developers} list is a forum for network and module
developers.  To subscribe send mail to:

\mailto{Majordomo@sci.utah.edu}

with the following command in the body of the message:

\keyboard{subscribe scirun-develop}

After subscribing, questions can be sent to
\mailto{scirun-develop@sci.utah.edu}.

\section{Reporting Bugs}
\label{sec:bugs}

Please report bugs!  To report a bug visit \sr{}'s
\htmladdnormallinkfoot{bug database}{\bugsurl} web page.

Reporting bugs to the bug database, rather than the mailing list,  ensures
bugs are fixed in a timely manner.

%%% Local Variables: 
%%% mode: latex
%%% TeX-master: "usersguide"
%%% End: 
