% -*-latex-*-
%
%  For more information, please see: http://software.sci.utah.edu
% 
%  The MIT License
% 
%  Copyright (c) 2004 Scientific Computing and Imaging Institute,
%  University of Utah.
% 
%  License for the specific language governing rights and limitations under
%  Permission is hereby granted, free of charge, to any person obtaining a
%  copy of this software and associated documentation files (the "Software"),
%  to deal in the Software without restriction, including without limitation
%  the rights to use, copy, modify, merge, publish, distribute, sublicense,
%  and/or sell copies of the Software, and to permit persons to whom the
%  Software is furnished to do so, subject to the following conditions:
% 
%  The above copyright notice and this permission notice shall be included
%  in all copies or substantial portions of the Software.
% 
%  THE SOFTWARE IS PROVIDED "AS IS", WITHOUT WARRANTY OF ANY KIND, EXPRESS
%  OR IMPLIED, INCLUDING BUT NOT LIMITED TO THE WARRANTIES OF MERCHANTABILITY,
%  FITNESS FOR A PARTICULAR PURPOSE AND NONINFRINGEMENT. IN NO EVENT SHALL
%  THE AUTHORS OR COPYRIGHT HOLDERS BE LIABLE FOR ANY CLAIM, DAMAGES OR OTHER
%  LIABILITY, WHETHER IN AN ACTION OF CONTRACT, TORT OR OTHERWISE, ARISING
%  FROM, OUT OF OR IN CONNECTION WITH THE SOFTWARE OR THE USE OR OTHER
%  DEALINGS IN THE SOFTWARE.
%


\documentclass[11pt]{article}
\usepackage[]{fancyhdr}
%\begin{latexonly}
\usepackage[]{../Utilities/TeX/scirun-doc}
%\end{latexonly}
\usepackage[]{graphicx}
\usepackage[]{html}
\begin{htmlonly}
  % -*-latex-*-
%
%  The contents of this file are subject to the University of Utah Public
%  License (the "License"); you may not use this file except in compliance
%  with the License.
%
%  Software distributed under the License is distributed on an "AS IS"
%  basis, WITHOUT WARRANTY OF ANY KIND, either express or implied. See the
%  License for the specific language governing rights and limitations under
%  the License.
%
%  The Original Source Code is SCIRun, released March 12, 2001.
%
%  The Original Source Code was developed by the University of Utah.
%  Portions created by UNIVERSITY are Copyright (C) 2001, 1994
%  University of Utah. All Rights Reserved.
%

%
% Latex package to be used by scirun documents.  Markup commands exist
% in two files.  This one and scirun-doc.tex.  This file contains
% commands that are processed by both latex and by the builtin
% capabilities of latex2html.  Scirun-doc.sty contains commands that are
% either completely ignored by latex2html or are processed by functions in
% the latex2html extension file scirun-doc.perl.
%
% Documents should include the
% following latex commands:
%
% \usepackage{scirun-doc}
% % -*-latex-*-
%
%  The contents of this file are subject to the University of Utah Public
%  License (the "License"); you may not use this file except in compliance
%  with the License.
%
%  Software distributed under the License is distributed on an "AS IS"
%  basis, WITHOUT WARRANTY OF ANY KIND, either express or implied. See the
%  License for the specific language governing rights and limitations under
%  the License.
%
%  The Original Source Code is SCIRun, released March 12, 2001.
%
%  The Original Source Code was developed by the University of Utah.
%  Portions created by UNIVERSITY are Copyright (C) 2001, 1994
%  University of Utah. All Rights Reserved.
%

%
% Latex package to be used by scirun documents.  Markup commands exist
% in two files.  This one and scirun-doc.tex.  This file contains
% commands that are processed by both latex and by the builtin
% capabilities of latex2html.  Scirun-doc.sty contains commands that are
% either completely ignored by latex2html or are processed by functions in
% the latex2html extension file scirun-doc.perl.
%
% Documents should include the
% following latex commands:
%
% \usepackage{scirun-doc}
% % -*-latex-*-
%
%  The contents of this file are subject to the University of Utah Public
%  License (the "License"); you may not use this file except in compliance
%  with the License.
%
%  Software distributed under the License is distributed on an "AS IS"
%  basis, WITHOUT WARRANTY OF ANY KIND, either express or implied. See the
%  License for the specific language governing rights and limitations under
%  the License.
%
%  The Original Source Code is SCIRun, released March 12, 2001.
%
%  The Original Source Code was developed by the University of Utah.
%  Portions created by UNIVERSITY are Copyright (C) 2001, 1994
%  University of Utah. All Rights Reserved.
%

%
% Latex package to be used by scirun documents.  Markup commands exist
% in two files.  This one and scirun-doc.tex.  This file contains
% commands that are processed by both latex and by the builtin
% capabilities of latex2html.  Scirun-doc.sty contains commands that are
% either completely ignored by latex2html or are processed by functions in
% the latex2html extension file scirun-doc.perl.
%
% Documents should include the
% following latex commands:
%
% \usepackage{scirun-doc}
% \input{scirun-doc.tex}
%

% (Mostly) Short cuts  ==============================
\newcommand{\SCI}{{\em SCI}}
\newcommand{\sci}{\SCI}
\newcommand{\scii}{SCI Institute}
\newcommand{\BIOPSE}{\textbf{BioPSE}}
\newcommand{\biopse}{\BIOPSE}
\newcommand{\SR}{\textbf{SCIRun}}
\newcommand{\sr}{\SR}
\newcommand{\eg}{{\em e.g.,}}
\newcommand{\ie}{{\em i.e.,}}
\newcommand{\etc}{{\em etc.}}
\newcommand{\etal}{{\em et al.}}
\newcommand{\degrees}{{$^{\circ}$}}
\newcommand{\splitline}{\begin{center}\rule{\columnwidth}{.7mm}\end{center}}
\newcommand{\X}[1]{#1\index{#1}}
\newcommand{\srig}{\sr{} Installation Guide}
\newcommand{\srug}{\sr{} User's Guide}
\newcommand{\srdg}{\sr{} Developer's Guide}

% Literal ~ character
\newcommand{\ltilde}{\textasciitilde}

% Encloses its argument between angle brackets.
\newcommand{\ab}[1]{\textless{}#1\textgreater}

% Inserts a left angle bracket.
\newcommand{\la}{\textless}

% Inserts a right angle bracket.
\newcommand{\ra}{\textgreater}


% Mark up commands =================================

% Url
%\newcommand{\url}[1]{#1}

% ip address
\newcommand{\ipaddr}[1]{#1}
\newcommand{\localhost}{\ipaddr{127.0.0.1}}

% Predefined acronyms
\newcommand{\gui}{\acronym{GUI}}
\newcommand{\tcl}{\acronym{TCL}}
\newcommand{\xml}{\acronym{XML}}
\newcommand{\pse}{\acronym{PSE}}

% Markup the first time use of term that may be unfamiliar to the
% reader. 
\newmucmd{dfn}{emph}

% In the next command #1 is the term and #2 is the shortcut or acronym that
% will be used in the rest of the document. 
\newcommand{\dfna}[2]{\emph{#1} (\acronym{#2})}

% Markup a file name.
\newmucmd{filename}{texttt}

% Directory name markup.
\newmucmd{directory}{texttt}

% Markup text which is the name of a command.
\newmucmd{command}{texttt} 

% Command option.
\newmucmd{option}{texttt}

% Markup text typed at the keyboard.
\newmucmd{keyboard}{texttt} 

% Parameterized text - marks up text that is to be
% substituted for by the reader.
\newmucmd{ptext}{textit}        % Obsolete-use replaceable instead.
\newmucmd{replaceable}{textit}

% Markup text the user might see on his screen.
\newmucmd{screen}{texttt}

% Markup the name of a GUI menu.
\newmucmd{guimenu}{textbf}

% Markup the name of a GUI menu.
\newcommand{\menu}[1]{\guimenu{#1}}

% Markup a gui menu item name.
\newmucmd{guimenuitem}{textbf}

% Markup a menu item name.
\newcommand{\menuitem}[1]{\guimenuitem{#1}}

% Markup name of a ui button.
\newmucmd{guibutton}{textbf}

% Markup name of a ui button.
\newcommand{\button}[1]{\guibutton{#1}}

% Markup name of a gui text item.
\newmucmd{guitext}{textit}

% Markup name of a gui label.
\newmucmd{guilabel}{textbf}

% GUI variable
\newmucmd{guivar}{texttt}

% Scirun port
\newmucmd{srport}{texttt}

% Sockets port
\newcommand{\port}[1]{\texttt{#1}}

% Socket
\newcommand{\socket}[2]{\texttt{#1:#2}}

% Variable
\newcommand{\variable}[1]{\texttt{#1}}

% Data type
\newcommand{\datatype}[1]{\texttt{#1}}

% Function
\newcommand{\function}[1]{\texttt{#1}}

% Markup an inline code fragment.
\newcommand{\icode}[1]{\texttt{#1}}

% Markup a module name.
\newcommand{\module}[1]{\texttt{#1}}

% Markup a sr package name.
\newcommand{\package}[1]{\texttt{#1}}

% Markup a sr category name.
\newcommand{\category}[1]{\texttt{#1}}

% Env variable markup.
\newmucmd{envvar}{texttt}

% Markup for the title of a book or article or whatever.
\newcommand{\etitle}[1]{\texttt{#1}}

% Markup name of a latex section.  First arg is section name.  Second arg
% is section's label.  When processed with latex, the section
% name is emphasized.  When processed with latex2html section name is made
% into a link to section.
\newcommand{\secname}[2]{\latexhtml{\emph{#1}}{\htmlref{#1}{#2}}}

% Section ref command.  Use of this command
% in place of \ref to reference sections (subsections etc.) will
% create a more web friendly version of your document.
% The command's first argument is the link text (used only on the html
% page) and the second argument is the section's label
\newcommand{\secref}[2]{\hyperref[ref]{\emph{#1}}{Section~}{}{#2}}
\newcommand{\chref}[2]{\hyperref[ref]{\emph{#1}}{Chapter~}{}{#2}}

% A latex command
\newcommand{\latexcommand}[1]{\texttt{\textbackslash#1}}

% A latex package
\newcommand{\latexpackage}[1]{\textit{#1}}

% Use these in place of missing content.
\newcommand{\missing}[1]{\emph{#1 - Coming Soon.}}

% Use this to make note of incomplete content.
\newcommand{\incomplete}{\emph{More Comming Soon.}}

% Mark up a mail address
\newmucmd{mailto}{texttt}

% Mark up an xml attribute
\newmucmd{xmlattrname}{texttt}

% Sci Urls ========================================= 

% www style urls.
\newcommand{\scisoftware}{http://software.sci.utah.edu}
\newcommand{\scisoftwareurl}{\scisoftware}
\newcommand{\sciurl}{http://www.sci.utah.edu}
\newcommand{\scidocurl}{\scisoftware{}/doc}
\newcommand{\scidocurlplus}[1]{\scisoftware{}/doc/#1}
\newcommand{\bugsurl}{\scisoftware{}/bugzilla}
\newcommand{\scisoftwarearchiveurl}{\scisoftware{}/archive\_entry.html}

%

% (Mostly) Short cuts  ==============================
\newcommand{\SCI}{{\em SCI}}
\newcommand{\sci}{\SCI}
\newcommand{\scii}{SCI Institute}
\newcommand{\BIOPSE}{\textbf{BioPSE}}
\newcommand{\biopse}{\BIOPSE}
\newcommand{\SR}{\textbf{SCIRun}}
\newcommand{\sr}{\SR}
\newcommand{\eg}{{\em e.g.,}}
\newcommand{\ie}{{\em i.e.,}}
\newcommand{\etc}{{\em etc.}}
\newcommand{\etal}{{\em et al.}}
\newcommand{\degrees}{{$^{\circ}$}}
\newcommand{\splitline}{\begin{center}\rule{\columnwidth}{.7mm}\end{center}}
\newcommand{\X}[1]{#1\index{#1}}
\newcommand{\srig}{\sr{} Installation Guide}
\newcommand{\srug}{\sr{} User's Guide}
\newcommand{\srdg}{\sr{} Developer's Guide}

% Literal ~ character
\newcommand{\ltilde}{\textasciitilde}

% Encloses its argument between angle brackets.
\newcommand{\ab}[1]{\textless{}#1\textgreater}

% Inserts a left angle bracket.
\newcommand{\la}{\textless}

% Inserts a right angle bracket.
\newcommand{\ra}{\textgreater}


% Mark up commands =================================

% Url
%\newcommand{\url}[1]{#1}

% ip address
\newcommand{\ipaddr}[1]{#1}
\newcommand{\localhost}{\ipaddr{127.0.0.1}}

% Predefined acronyms
\newcommand{\gui}{\acronym{GUI}}
\newcommand{\tcl}{\acronym{TCL}}
\newcommand{\xml}{\acronym{XML}}
\newcommand{\pse}{\acronym{PSE}}

% Markup the first time use of term that may be unfamiliar to the
% reader. 
\newmucmd{dfn}{emph}

% In the next command #1 is the term and #2 is the shortcut or acronym that
% will be used in the rest of the document. 
\newcommand{\dfna}[2]{\emph{#1} (\acronym{#2})}

% Markup a file name.
\newmucmd{filename}{texttt}

% Directory name markup.
\newmucmd{directory}{texttt}

% Markup text which is the name of a command.
\newmucmd{command}{texttt} 

% Command option.
\newmucmd{option}{texttt}

% Markup text typed at the keyboard.
\newmucmd{keyboard}{texttt} 

% Parameterized text - marks up text that is to be
% substituted for by the reader.
\newmucmd{ptext}{textit}        % Obsolete-use replaceable instead.
\newmucmd{replaceable}{textit}

% Markup text the user might see on his screen.
\newmucmd{screen}{texttt}

% Markup the name of a GUI menu.
\newmucmd{guimenu}{textbf}

% Markup the name of a GUI menu.
\newcommand{\menu}[1]{\guimenu{#1}}

% Markup a gui menu item name.
\newmucmd{guimenuitem}{textbf}

% Markup a menu item name.
\newcommand{\menuitem}[1]{\guimenuitem{#1}}

% Markup name of a ui button.
\newmucmd{guibutton}{textbf}

% Markup name of a ui button.
\newcommand{\button}[1]{\guibutton{#1}}

% Markup name of a gui text item.
\newmucmd{guitext}{textit}

% Markup name of a gui label.
\newmucmd{guilabel}{textbf}

% GUI variable
\newmucmd{guivar}{texttt}

% Scirun port
\newmucmd{srport}{texttt}

% Sockets port
\newcommand{\port}[1]{\texttt{#1}}

% Socket
\newcommand{\socket}[2]{\texttt{#1:#2}}

% Variable
\newcommand{\variable}[1]{\texttt{#1}}

% Data type
\newcommand{\datatype}[1]{\texttt{#1}}

% Function
\newcommand{\function}[1]{\texttt{#1}}

% Markup an inline code fragment.
\newcommand{\icode}[1]{\texttt{#1}}

% Markup a module name.
\newcommand{\module}[1]{\texttt{#1}}

% Markup a sr package name.
\newcommand{\package}[1]{\texttt{#1}}

% Markup a sr category name.
\newcommand{\category}[1]{\texttt{#1}}

% Env variable markup.
\newmucmd{envvar}{texttt}

% Markup for the title of a book or article or whatever.
\newcommand{\etitle}[1]{\texttt{#1}}

% Markup name of a latex section.  First arg is section name.  Second arg
% is section's label.  When processed with latex, the section
% name is emphasized.  When processed with latex2html section name is made
% into a link to section.
\newcommand{\secname}[2]{\latexhtml{\emph{#1}}{\htmlref{#1}{#2}}}

% Section ref command.  Use of this command
% in place of \ref to reference sections (subsections etc.) will
% create a more web friendly version of your document.
% The command's first argument is the link text (used only on the html
% page) and the second argument is the section's label
\newcommand{\secref}[2]{\hyperref[ref]{\emph{#1}}{Section~}{}{#2}}
\newcommand{\chref}[2]{\hyperref[ref]{\emph{#1}}{Chapter~}{}{#2}}

% A latex command
\newcommand{\latexcommand}[1]{\texttt{\textbackslash#1}}

% A latex package
\newcommand{\latexpackage}[1]{\textit{#1}}

% Use these in place of missing content.
\newcommand{\missing}[1]{\emph{#1 - Coming Soon.}}

% Use this to make note of incomplete content.
\newcommand{\incomplete}{\emph{More Comming Soon.}}

% Mark up a mail address
\newmucmd{mailto}{texttt}

% Mark up an xml attribute
\newmucmd{xmlattrname}{texttt}

% Sci Urls ========================================= 

% www style urls.
\newcommand{\scisoftware}{http://software.sci.utah.edu}
\newcommand{\scisoftwareurl}{\scisoftware}
\newcommand{\sciurl}{http://www.sci.utah.edu}
\newcommand{\scidocurl}{\scisoftware{}/doc}
\newcommand{\scidocurlplus}[1]{\scisoftware{}/doc/#1}
\newcommand{\bugsurl}{\scisoftware{}/bugzilla}
\newcommand{\scisoftwarearchiveurl}{\scisoftware{}/archive\_entry.html}

%

% (Mostly) Short cuts  ==============================
\newcommand{\SCI}{{\em SCI}}
\newcommand{\sci}{\SCI}
\newcommand{\scii}{SCI Institute}
\newcommand{\BIOPSE}{\textbf{BioPSE}}
\newcommand{\biopse}{\BIOPSE}
\newcommand{\SR}{\textbf{SCIRun}}
\newcommand{\sr}{\SR}
\newcommand{\eg}{{\em e.g.,}}
\newcommand{\ie}{{\em i.e.,}}
\newcommand{\etc}{{\em etc.}}
\newcommand{\etal}{{\em et al.}}
\newcommand{\degrees}{{$^{\circ}$}}
\newcommand{\splitline}{\begin{center}\rule{\columnwidth}{.7mm}\end{center}}
\newcommand{\X}[1]{#1\index{#1}}
\newcommand{\srig}{\sr{} Installation Guide}
\newcommand{\srug}{\sr{} User's Guide}
\newcommand{\srdg}{\sr{} Developer's Guide}

% Literal ~ character
\newcommand{\ltilde}{\textasciitilde}

% Encloses its argument between angle brackets.
\newcommand{\ab}[1]{\textless{}#1\textgreater}

% Inserts a left angle bracket.
\newcommand{\la}{\textless}

% Inserts a right angle bracket.
\newcommand{\ra}{\textgreater}


% Mark up commands =================================

% Url
%\newcommand{\url}[1]{#1}

% ip address
\newcommand{\ipaddr}[1]{#1}
\newcommand{\localhost}{\ipaddr{127.0.0.1}}

% Predefined acronyms
\newcommand{\gui}{\acronym{GUI}}
\newcommand{\tcl}{\acronym{TCL}}
\newcommand{\xml}{\acronym{XML}}
\newcommand{\pse}{\acronym{PSE}}

% Markup the first time use of term that may be unfamiliar to the
% reader. 
\newmucmd{dfn}{emph}

% In the next command #1 is the term and #2 is the shortcut or acronym that
% will be used in the rest of the document. 
\newcommand{\dfna}[2]{\emph{#1} (\acronym{#2})}

% Markup a file name.
\newmucmd{filename}{texttt}

% Directory name markup.
\newmucmd{directory}{texttt}

% Markup text which is the name of a command.
\newmucmd{command}{texttt} 

% Command option.
\newmucmd{option}{texttt}

% Markup text typed at the keyboard.
\newmucmd{keyboard}{texttt} 

% Parameterized text - marks up text that is to be
% substituted for by the reader.
\newmucmd{ptext}{textit}        % Obsolete-use replaceable instead.
\newmucmd{replaceable}{textit}

% Markup text the user might see on his screen.
\newmucmd{screen}{texttt}

% Markup the name of a GUI menu.
\newmucmd{guimenu}{textbf}

% Markup the name of a GUI menu.
\newcommand{\menu}[1]{\guimenu{#1}}

% Markup a gui menu item name.
\newmucmd{guimenuitem}{textbf}

% Markup a menu item name.
\newcommand{\menuitem}[1]{\guimenuitem{#1}}

% Markup name of a ui button.
\newmucmd{guibutton}{textbf}

% Markup name of a ui button.
\newcommand{\button}[1]{\guibutton{#1}}

% Markup name of a gui text item.
\newmucmd{guitext}{textit}

% Markup name of a gui label.
\newmucmd{guilabel}{textbf}

% GUI variable
\newmucmd{guivar}{texttt}

% Scirun port
\newmucmd{srport}{texttt}

% Sockets port
\newcommand{\port}[1]{\texttt{#1}}

% Socket
\newcommand{\socket}[2]{\texttt{#1:#2}}

% Variable
\newcommand{\variable}[1]{\texttt{#1}}

% Data type
\newcommand{\datatype}[1]{\texttt{#1}}

% Function
\newcommand{\function}[1]{\texttt{#1}}

% Markup an inline code fragment.
\newcommand{\icode}[1]{\texttt{#1}}

% Markup a module name.
\newcommand{\module}[1]{\texttt{#1}}

% Markup a sr package name.
\newcommand{\package}[1]{\texttt{#1}}

% Markup a sr category name.
\newcommand{\category}[1]{\texttt{#1}}

% Env variable markup.
\newmucmd{envvar}{texttt}

% Markup for the title of a book or article or whatever.
\newcommand{\etitle}[1]{\texttt{#1}}

% Markup name of a latex section.  First arg is section name.  Second arg
% is section's label.  When processed with latex, the section
% name is emphasized.  When processed with latex2html section name is made
% into a link to section.
\newcommand{\secname}[2]{\latexhtml{\emph{#1}}{\htmlref{#1}{#2}}}

% Section ref command.  Use of this command
% in place of \ref to reference sections (subsections etc.) will
% create a more web friendly version of your document.
% The command's first argument is the link text (used only on the html
% page) and the second argument is the section's label
\newcommand{\secref}[2]{\hyperref[ref]{\emph{#1}}{Section~}{}{#2}}
\newcommand{\chref}[2]{\hyperref[ref]{\emph{#1}}{Chapter~}{}{#2}}

% A latex command
\newcommand{\latexcommand}[1]{\texttt{\textbackslash#1}}

% A latex package
\newcommand{\latexpackage}[1]{\textit{#1}}

% Use these in place of missing content.
\newcommand{\missing}[1]{\emph{#1 - Coming Soon.}}

% Use this to make note of incomplete content.
\newcommand{\incomplete}{\emph{More Comming Soon.}}

% Mark up a mail address
\newmucmd{mailto}{texttt}

% Mark up an xml attribute
\newmucmd{xmlattrname}{texttt}

% Sci Urls ========================================= 

% www style urls.
\newcommand{\scisoftware}{http://software.sci.utah.edu}
\newcommand{\scisoftwareurl}{\scisoftware}
\newcommand{\sciurl}{http://www.sci.utah.edu}
\newcommand{\scidocurl}{\scisoftware{}/doc}
\newcommand{\scidocurlplus}[1]{\scisoftware{}/doc/#1}
\newcommand{\bugsurl}{\scisoftware{}/bugzilla}
\newcommand{\scisoftwarearchiveurl}{\scisoftware{}/archive\_entry.html}

\end{htmlonly}

\begin{document}

\begin{htmlonly}
%
%  For more information, please see: http://software.sci.utah.edu
% 
%  The MIT License
% 
%  Copyright (c) 2004 Scientific Computing and Imaging Institute,
%  University of Utah.
% 
%  License for the specific language governing rights and limitations under
%  Permission is hereby granted, free of charge, to any person obtaining a
%  copy of this software and associated documentation files (the "Software"),
%  to deal in the Software without restriction, including without limitation
%  the rights to use, copy, modify, merge, publish, distribute, sublicense,
%  and/or sell copies of the Software, and to permit persons to whom the
%  Software is furnished to do so, subject to the following conditions:
% 
%  The above copyright notice and this permission notice shall be included
%  in all copies or substantial portions of the Software.
% 
%  THE SOFTWARE IS PROVIDED "AS IS", WITHOUT WARRANTY OF ANY KIND, EXPRESS
%  OR IMPLIED, INCLUDING BUT NOT LIMITED TO THE WARRANTIES OF MERCHANTABILITY,
%  FITNESS FOR A PARTICULAR PURPOSE AND NONINFRINGEMENT. IN NO EVENT SHALL
%  THE AUTHORS OR COPYRIGHT HOLDERS BE LIABLE FOR ANY CLAIM, DAMAGES OR OTHER
%  LIABILITY, WHETHER IN AN ACTION OF CONTRACT, TORT OR OTHERWISE, ARISING
%  FROM, OUT OF OR IN CONNECTION WITH THE SOFTWARE OR THE USE OR OTHER
%  DEALINGS IN THE SOFTWARE.
%


%
% Code to generate sr top banners.  To use this in your latex doc insert
% the following code just after the \begin{document} command:
%
% \begin{htmlonly}
% %
%  For more information, please see: http://software.sci.utah.edu
% 
%  The MIT License
% 
%  Copyright (c) 2004 Scientific Computing and Imaging Institute,
%  University of Utah.
% 
%  License for the specific language governing rights and limitations under
%  Permission is hereby granted, free of charge, to any person obtaining a
%  copy of this software and associated documentation files (the "Software"),
%  to deal in the Software without restriction, including without limitation
%  the rights to use, copy, modify, merge, publish, distribute, sublicense,
%  and/or sell copies of the Software, and to permit persons to whom the
%  Software is furnished to do so, subject to the following conditions:
% 
%  The above copyright notice and this permission notice shall be included
%  in all copies or substantial portions of the Software.
% 
%  THE SOFTWARE IS PROVIDED "AS IS", WITHOUT WARRANTY OF ANY KIND, EXPRESS
%  OR IMPLIED, INCLUDING BUT NOT LIMITED TO THE WARRANTIES OF MERCHANTABILITY,
%  FITNESS FOR A PARTICULAR PURPOSE AND NONINFRINGEMENT. IN NO EVENT SHALL
%  THE AUTHORS OR COPYRIGHT HOLDERS BE LIABLE FOR ANY CLAIM, DAMAGES OR OTHER
%  LIABILITY, WHETHER IN AN ACTION OF CONTRACT, TORT OR OTHERWISE, ARISING
%  FROM, OUT OF OR IN CONNECTION WITH THE SOFTWARE OR THE USE OR OTHER
%  DEALINGS IN THE SOFTWARE.
%


%
% Code to generate sr top banners.  To use this in your latex doc insert
% the following code just after the \begin{document} command:
%
% \begin{htmlonly}
% %
%  For more information, please see: http://software.sci.utah.edu
% 
%  The MIT License
% 
%  Copyright (c) 2004 Scientific Computing and Imaging Institute,
%  University of Utah.
% 
%  License for the specific language governing rights and limitations under
%  Permission is hereby granted, free of charge, to any person obtaining a
%  copy of this software and associated documentation files (the "Software"),
%  to deal in the Software without restriction, including without limitation
%  the rights to use, copy, modify, merge, publish, distribute, sublicense,
%  and/or sell copies of the Software, and to permit persons to whom the
%  Software is furnished to do so, subject to the following conditions:
% 
%  The above copyright notice and this permission notice shall be included
%  in all copies or substantial portions of the Software.
% 
%  THE SOFTWARE IS PROVIDED "AS IS", WITHOUT WARRANTY OF ANY KIND, EXPRESS
%  OR IMPLIED, INCLUDING BUT NOT LIMITED TO THE WARRANTIES OF MERCHANTABILITY,
%  FITNESS FOR A PARTICULAR PURPOSE AND NONINFRINGEMENT. IN NO EVENT SHALL
%  THE AUTHORS OR COPYRIGHT HOLDERS BE LIABLE FOR ANY CLAIM, DAMAGES OR OTHER
%  LIABILITY, WHETHER IN AN ACTION OF CONTRACT, TORT OR OTHERWISE, ARISING
%  FROM, OUT OF OR IN CONNECTION WITH THE SOFTWARE OR THE USE OR OTHER
%  DEALINGS IN THE SOFTWARE.
%


%
% Code to generate sr top banners.  To use this in your latex doc insert
% the following code just after the \begin{document} command:
%
% \begin{htmlonly}
% \include{../Utilities/TeX/banner_top.tex}
% \end{htmlonly}
%
% and adjust the path above as needed.
%
\begin{rawhtml}
<script type="text/javascript">
var treetop="";
var path = location.pathname;
if (path.charAt(path.length-1) == "/") {
  path += "bogus.html"
}
var base = path.substr(path.lastIndexOf("/")+1);
var roottag = "doc";
while (base != roottag && base != "") {
  treetop += "../";
  path = path.substr(0, path.lastIndexOf("/"));
  base = path.substr(path.lastIndexOf("/")+1);
}
var inDocTree = base == roottag;
if (inDocTree) {
  document.write('<script language="JavaScript" src="',treetop,'doc/Utilities/HTML/banner_top.js"><\/script>');
}
</script>
\end{rawhtml}

% \end{htmlonly}
%
% and adjust the path above as needed.
%
\begin{rawhtml}
<script type="text/javascript">
var treetop="";
var path = location.pathname;
if (path.charAt(path.length-1) == "/") {
  path += "bogus.html"
}
var base = path.substr(path.lastIndexOf("/")+1);
var roottag = "doc";
while (base != roottag && base != "") {
  treetop += "../";
  path = path.substr(0, path.lastIndexOf("/"));
  base = path.substr(path.lastIndexOf("/")+1);
}
var inDocTree = base == roottag;
if (inDocTree) {
  document.write('<script language="JavaScript" src="',treetop,'doc/Utilities/HTML/banner_top.js"><\/script>');
}
</script>
\end{rawhtml}

% \end{htmlonly}
%
% and adjust the path above as needed.
%
\begin{rawhtml}
<script type="text/javascript">
var treetop="";
var path = location.pathname;
if (path.charAt(path.length-1) == "/") {
  path += "bogus.html"
}
var base = path.substr(path.lastIndexOf("/")+1);
var roottag = "doc";
while (base != roottag && base != "") {
  treetop += "../";
  path = path.substr(0, path.lastIndexOf("/"));
  base = path.substr(path.lastIndexOf("/")+1);
}
var inDocTree = base == roottag;
if (inDocTree) {
  document.write('<script language="JavaScript" src="',treetop,'doc/Utilities/HTML/banner_top.js"><\/script>');
}
</script>
\end{rawhtml}

\end{htmlonly}

% The title page
\title{XML Module Descriptions (Explicated)}
\author{Ted Dustman}
\maketitle 
\newpage
\tableofcontents
\newpage

% Custom markup commands for this section.
\newcommand{\Xml}{\xml}
\newcommand{\psgml}{\acronym{PSGML}}
\newcommand{\Psgml}{\psgml}
\newcommand{\dtd}{\acronym{DTD}}
\newcommand{\Pse}{\pse}
\newcommand{\Sci}{\sci}
\newcommand{\Sr}{\sr}
\newcommand{\cvs}{\acronym{CVS}}
\newcommand{\Cvs}{\cvs}
\newcommand{\emacs}{emacs}
\newcommand{\Emacs}{Emacs}
\newcommand{\sectitle}[1]{\emph{#1}}
\newcommand{\sechyperref}[1]{\hyperref{\sectitle{#1}}{}{}{#1}}
\newcommand{\element}[1]{\ab{\texttt{#1}}}
\newcommand{\attribute}[1]{\texttt{#1}}
\newcommand{\elementitem}[1]{\item[\texttt{\ab{#1}}]}
\newcommand{\psgmlurl}{http://www.lysator.liu.se/projects/about\_psgml.html}
\newcommand{\Note}[1]{\emph{Note: #1}}
\newcommand{\ytbd}{\emph{Yet to be done}}
\newcommand{\doctitle}[1]{\textit{#1}}
\newcommand{\xmlvalidurl}{http://www.cvrti.utah.edu/~dustman/srvalidate.html}

% Section title commands (I'm experimenting - bear with me.)
\newcommand{\SECintro}{Introduction}
\newcommand{\SUBSECwhySpec}{Purpose}
\newcommand{\SUBSECwhyXml}{Why XML?}
\newcommand{\SUBSECwhatDtd}{The DTD}
\newcommand{\SECcontent}{Rules, Structure, and Content}
\newcommand{\SUBSECrules}{Rules}
\newcommand{\SUBSECstructContent}{Structure and Content}
\newcommand{\SUBSECdescElement}{A Description of the ``description'' Element}
\newcommand{\SUBSUBSECcommonUsage}{Common Usage}
\newcommand{\SUBSUBSECdescElementIntro}{Overview}
\newcommand{\SUBSUBSECreference}{Reference}
\newcommand{\SECexample}{Example}
\newcommand{\SECediting}{Editing the Spec}
\newcommand{\SUBSECgettingSources}{XML Mode - Do You Have It?}
\newcommand{\SUBSECdotEmacs}{XML Mode - .emacs}
\newcommand{\SUBSECgettingStarted}{XML Mode - Getting Started}
\newcommand{\SECvalidation}{Validation}
%%%%%%%%%%%%%%%%%%%%%%%%%%%%%%%%%%%%%%%%%%%%%%%%%%%%%%%%%%%%%%%%%%%%%%

\section{\SECintro}
\label{\SECintro}

This section introduces the component specification document.

\subsection{\SUBSECwhySpec}
\label{\SUBSECwhySpec}

Writing the \xml{} component specification is one of the first steps in the
design and implementation of a component.

The \xml{} component specification documents the use, design,
implementation, and testing of a \sr{} component.  

It also serves as data for applications and databases.  For example:

\begin{itemize}
\item Documentation (HTML and pdf forms) for component users and
  developers are generated from the component specification.
  
\item Collection of all component specifications are part of a 
  search-able database.
  
\item Component specification helps drive the component code
  generation process.
  
\item Component specification may drive an automated testing process.
\end{itemize}

Component specifications are written in a special mark-up language.  This
component specification mark-up language is formulated using the \dfn{eXtensible Mark-up Language}
(\xml).

% \Note{Components that are not accompanied by a valid and complete
%   specification document will not be incorporated into official
%   releases of the \sr{} software.}

\subsection{\SUBSECwhyXml}
\label{\SUBSECwhyXml}

An \xml{} formulation of the component specification is used to achieve
the following goals:

\begin{description}
\item[Data centric viewpoint] To represent the component specification as a set of data to be used in several contexts.

\item[Uniformity] To specify all components the same way.
  
\item[Validation] To validate (check for correctness and
  completeness) component specifications.
\end{description}

The component specification mark-up language is similar in concept to HTML, but
has a different, much smaller set of tags, and a more simple, more
regular set of document composition rules.
Section~\sechyperref{\SECcontent} describes
the component specification mark-up language.

\subsection{\SUBSECwhatDtd}
\label{\SUBSECwhatDtd}

A \dfn{Document Type Definition} (\dtd) defines a set of elements (and
element attributes), and the way they must be used in order to produce a
valid \xml{} document.

The component specification \dtd{} can be found here (relative to the top
of the cvs tree):

\begin{verbatim}
doc/ReferenceGuide/component.dtd
\end{verbatim}

The component specification \dtd{} is used to validate component
specifications. See 
section~\sechyperref{\SECvalidation} for more information on validation

The component specification \dtd{} is used to create and edit component
specifications.  See
section~\sechyperref{\SECediting} for more
information.

\section{\SECcontent}
\label{\SECcontent}

This section describes the component specification mark-up language.

\subsection{\SUBSECrules}
\label{\SUBSECrules}

When writing a component specification, a few simple composition rules
must be followed:

\begin{itemize}
\item In the \xml{} world tags are used to delimit (or identify)
  \dfn{elements}.  Elements are the basic building blocks of an \xml\
  document and contain content (text) or nested elements or both.
  Every start tag \emph{must} have a corresponding end tag.
  
\item A component specification document starts with two lines of preamble
  text:

\begin{verbatim}
<?xml version="1.0" encoding="UTF-8" ?>
<!DOCTYPE component SYSTEM 
    "<top-of-cvs-tree>/doc/Utilities/XML/component.dtd}">
\end{verbatim}

  where \filename{<top-of-cvs-tree>} is the path to the top of the
  SCIRun \cvs{} tree. 
  
\item The preamble is followed by a set of nested elements starting with
  the \element{component} element.
  
\item Every element must be delimited with a start-tag and an end-tag.
  Tags are enclosed in angle brackets. 
  The literal forms of the characters \keyboard{\la} and
  \keyboard{\ra} can not be used in the elements' content.  Instead, type
  \keyboard{\&lt;} and \keyboard{\&gt;} respectively.

\item Tags are case sensitive. All tags must be lower case.
  
\item A few elements require an \dfn{attribute}.  Attributes are specified
  in an element's start tag.  An attribute looks like this:
  \icode{attribute\_name="attribute\_value"}.  The attribute's value
  must be enclosed in double quotes. Attributes are documented below
  along side their corresponding element.
  
\item The literal form of the character \& may not be used in element
  content.  Instead, type \keyboard{\&amp;} instead.

\end{itemize}

The set of valid elements are described next.

\subsection{\SUBSECstructContent}
\label{\SUBSECstructContent}

The component specification's \xml{} structure and corresponding
logical content are described below.  Each element's start tag is followed by a
description of its purpose and content, followed by its nested elements,
and finally by its end tag.  Indentation is used to show nesting
relationships among elements. Section~\sechyperref{\SUBSECdescElement} describes the complex \element{description} element that is used in several contexts.

\begin{description}
  \elementitem{component name="component-name" 
    category="category-name"} Starts the component specification.  All
  other elements are nested within this one.  This element requires the
  \icode{name} and \icode{category} attributes.  These
  attributes name the component and specify
  the category the component belongs to respectively.
  \begin{description}
    \elementitem{overview} Starts the overview section.  Contains only
    nested elements.
    \begin{description}
      \elementitem{authors}  List of component's authors.  Contains one or more
      \element{author} elements.
      \begin{description}
        \elementitem{author} Name of one author.
        \elementitem{/author}
      \end{description}
      \elementitem{/authors}
      \elementitem{summary} Short (one or two sentence) summary of component's
      function. \elementitem{/summary}
      \elementitem{description} Comprehensive description of component's
      function.  The \element{description} element is a mini-documentation
      environment.  See
      section~\sechyperref{\SUBSECdescElement}
      for details on the use of the \element{description} element.
      \elementitem{/description}
      \elementitem{examplesr} The name of an \filename{.sr} file that
      demonstrates the component's use.
      \elementitem{/examplesr}
    \end{description}
    \elementitem{/overview} 
    \elementitem{implementation} Lists the names of
    a component's source code files \emph{excluding} the component's
    primary source file (which is implied by the \element{component}
    element's \attribute{name} attribute).
    
    The need for an implementation section is rare - the source code for
    most components is contained entirely within a component's primary 
    source code file.
    
    The subelements \element{ccfile}, \element{cfile}, and \element{ffile}
    list the names of c++, c, and fortran source code files respectively.
    These elements may occur multiple times and in any order.
    Note that file name extensions are omitted.
    \begin{description}
      \elementitem{ccfile} Names one c++ file,
      e.g. \element{ccfile}foo\element{/ccfile}. 
      \elementitem{/ccfile}
      \elementitem{cfile} Names one c file.
      \elementitem{/cfile}
      \elementitem{ffile} Names one fortran file.
      \elementitem{/ffile}
    \end{description}
    \elementitem{/implementation}
    \elementitem{io} Component's inputs and
    outputs.  Contains two elements, \element{inputs} followed by
    \element{outputs}.
    \begin{description}
      \elementitem{inputs lastportdynamic="(yes | no)"} Describes the
      component's inputs. Inputs may 
      come from a port, a file, or a device.  Zero or more \element{port}
      elements are used to describe inputs from ports.  Zero or more
      \element{file} elements are used to describe inputs from files.  Zero
      or more \element{device} elements are used to describe inputs from
      devices.  There must be at least one \element{port} element or one
      \element{file} element or one \element{device} element.

      If the component's last port is dynamic then the
      \attribute{lastportdynamic} attribute should be set to \keyboard{yes}
      (\attribute{lastportdynamic="yes"}).  If the component's last port
      is not dynamic, then omit this attribute or set its value to
      \keyboard{no}.

      \begin{description}
        \elementitem{port} States that input data comes from a port. 
        Sub-elements describe the data further.
        \begin{description}
          \elementitem{name} The port's name.
          \elementitem{/name}
          \elementitem{description} High level description of the data 
          accepted by the port, e.g. ``These are potential data from body 
          surface of a human.''
          \elementitem{/description}
          \elementitem{datatype} The name of a PSE data type, i.e. the 
          data type that is being transmitted on this port.
          \elementitem{/datatype}
          \elementitem{componentname} The name of an upstream component
          that is commonly used to send data to this component on this
          port.  This element may occur multiple times. 
          \elementitem{/componentname}
        \end{description}
        \elementitem{/port}
        \elementitem{file} States that input comes from a file.
        Sub-elements describe the data further.
        \begin{description}
          \elementitem{description} High level description of the data 
          provided by the file, e.g. ``These are potential data from body 
          surface of a human.''
          \elementitem{/description}
          \elementitem{datatype} The name of a \sr{} data type, i.e. the 
          data type being read from the file.  \Note{Only \sr{} data types
            may be read from files - no custom formats allowed!}
          \elementitem{/datatype}
        \end{description}
        \elementitem{/file}
        \elementitem{device} States that input comes from a device.
        Sub-elements describe the data further.
        \begin{description}
          \elementitem{devicename} Device's name.
          \elementitem{/devicename}
          \elementitem{description} High level description of the data 
          provided by the device.
          \elementitem{/description}
        \end{description}
        \elementitem{/device}
      \end{description}
      \elementitem{/inputs}
      \elementitem{outputs} Describes the component's outputs.  Outputs may
      go to a port, to a file, or to a device.  Use 0 or more
      \element{port} elements to describe outputs to ports, 0 or more
      \element{file} elements to describe outputs to files, and 0 or more
      \element{device} elements to describe outputs to devices.
      \begin{description}
        \elementitem{port} States that output is sent a port. Sub-elements describe the data further.
        \begin{description}
          \elementitem{description} A high level description of the data
          sent to the port, e.g. ``These are potential data from the body
          surface of a human.''
          \elementitem{/description}
          \elementitem{datatype} The name of a \sr{} data type, i.e. the 
          data type that is being transmitted on this port.
          \elementitem{/datatype}
          \elementitem{componentname} The name of a downstream component
          that is commonly connected to this output port.  This
          element may occur multiple times. 
          \elementitem{/componentname}
        \end{description}
        \elementitem{/port}
        \elementitem{file}  States that output is sent to a file.  It's 
        sub-elements describe the outputs.
        \begin{description}
          \elementitem{description} A high level description of the data 
          sent to the file, e.g. ``These are potential data from body 
          surface of a human.''
          \elementitem{/description}
          \elementitem{datatype}  The name of a PSE data type, i.e. the 
          data type being written to the file.
          \elementitem{/datatype}
        \end{description}
        \elementitem{/file}
        \elementitem{device} States that output goes to a device.
        Sub-elements describe the output further.
        \begin{description}
          \elementitem{devicename} Device's name.
          \elementitem{/devicename}
          \elementitem{description} High level description of the data
          sent to the device.
          \elementitem{/description}
        \end{description}
        \elementitem{/device}
      \end{description}
      \elementitem{/outputs}
    \end{description}
    \elementitem{/io}
    \elementitem{gui} States that the component supports a \gui.
    Sub-elements describe the \gui.
    \begin{description}
      \elementitem{description} Describes the purpose of the \gui, e.g.
      ``The \gui{} allows you to steer the simulation.''
      \elementitem{/description}
      \elementitem{parameter} Declares one \gui{}  parameter.  A parameter is a
      \gui{}  item that can be modified by the user.  Sub-elements describe the
      parameter.  Use one parameter element for each item in the \gui.
      \begin{description}
        \elementitem{widget} The type name of parameter's tcl widget.
        \elementitem{/widget}
        \elementitem{label}Name of parameter as it appears in \gui.
        \elementitem{/label}
        \elementitem{description}Describes the purpose of the parameter and
        how it can be used to control the component's behavior.
        \elementitem{/description}
      \end{description}
      \elementitem{/parameter}
      \elementitem{img}Name of a file that contains a picture of the \gui.
      \elementitem{/img}
    \end{description}
    \elementitem{/gui}%
    \elementitem{testing} Starts the testing section.  Sub-elements specify
    1oneor more testing plans.
    \begin{description}
      \elementitem{plan} Declares one testing plan.  Sub-elements specify the
      testing plan's details.
      \begin{description}
        \elementitem{description}  High level description of the testing
        plan.  The \element{description} element is followed by one or more
        \element{step} elements.
        \elementitem{/description}
        \elementitem{step} Detailed description of one step in the testing
        procedure.   The content model of the \element{step} element is the
        same as the that of the \element{description} element (i.e., see
        section~\sechyperref{\SUBSECdescElement} for details).
        \elementitem{/step}
      \end{description}
      \elementitem{/plan}
    \end{description}
    \elementitem{/testing}
  \end{description}
  \elementitem{/component}
\end{description}

\subsection{\SUBSECdescElement}
\label{\SUBSECdescElement}

This section describes the \element{description} element.

\paragraph{\SUBSUBSECdescElementIntro}
\label{\SUBSUBSECdescElementIntro}

The \element{description} element contains text for human consumption.
It allows you to organize your text into paragraphs, lists, and
other structured content.

Note that the literal forms of the characters \la, \ra, and \& may not be
used in element content.  Instead, type \keyboard{\&lt;},
\keyboard{\&gt;}, and \keyboard{\&amp;} respectively.

\paragraph{\SUBSUBSECcommonUsage}
\label{\SUBSUBSECcommonUsage}

A \element{description} might contain only one or two sentences:

\begin{verbatim}
<description>
<p>My component is great.  It's so easy to use that no further
information is needed. </p>
</description>
\end{verbatim}

Note that just one sentence must be enclosed in a paragraph element.

A \element{description} may contain only a few paragraphs:

\begin{verbatim}
<description>
<p>My component is great.  It's so easy to use that no further
information is needed.</p>

<p>I lied.  What follows is a description of my component.</p>
.
.
.
</description>
\end{verbatim}

Refer to other material using the \element{cite}, \element{rlink},
or \element{slink} elements:

\begin{verbatim}
<description>
<p>
My component does xyz.  It uses the xyz algorithm described in the paper
<cite>The XYZ Algorithm</cite> by Xavier Yilmez Zideco.  A less formal
discussion of the xyz algorithm is
<slink path=ncrr/publications/xyz.html>available online</slink>. 
</p>
</description>
\end{verbatim}

The \element{cite} element marks text as the title of a paper or
book, but is not itself a hyper-link.

The \element{slink} and \element{rlink} elements allow you to reference
online material.

\element{Slink} links to material on the \sci{} 
web site (\sciurl).  \element{Slink} requires a path 
relative to \sciurl.  The path in the previous example resolves to
\sciurl{}/ncrr/publications/xyz.html.

\element{Rlink} links to material in the \acronym{CVS} tree
\element{rlink} links to material on the documentation web site,
but it is more easily explained as a path into the cvs tree.  \element{Rlink} requires
a path relative to the location of the component specification document.  For
example, the path \keyboard{./stuff.html} resolves to the same directory
where the component specification document lives.

Documentation bureaucrats want some control over the material, so a more general link is not available.

The \element{developer} element encloses material that is of interest only to
developers:

\begin{verbatim}
<description>
.
.
.
<developer>
<p>Some smart developer ought to improve this rubbish.</p>
</developer>
.
.
.
<description>
\end{verbatim}

\paragraph{\SUBSUBSECreference}
\label{\SUBSUBSECreference}

The description element's \xml{} structure and corresponding logical
content are described below.  Each element's start tag is followed by a
description of its purpose and content, followed by its nested elements,
and finally by its end tag.  Basic nesting relationships among elements are
shown by indentation.

\Note{In the future, additional features may be added to the \element{description.}


\begin{description}
  \elementitem{description} Starts a description.  A description consists
  of any combination of paragraphs (\element{p}), lists
  (\element{orderedlist}, \element{unorderedlist}, and \element{desclist}),
  admonitions (\element{note}, \element{tip}, or \element{warning}), and developer
  notes (\element{developer}).
  \begin{description}
    \elementitem{p} Starts a paragraph.  A paragraph consists of character
    data,  phrases (\element{term}, \element{keyword}, \element{keyboard},
    \element{cite}, and \element{acronym}), links (\element{slink} and
    \element{rlink}), and inline latex source (see \element{latex}).
    \begin{description}
      \elementitem{term} Encloses a \dfn{term}.  A term is a word or short
      phrase introduced to your readers for the first time.
      \elementitem{/term}
      \elementitem{keyword} Encloses a \dfn{keyword}.
      \elementitem{/keyword}
      \elementitem{keyboard} Encloses material that is typed at a keyboard.
      \elementitem{/keyboard}
      \elementitem{cite} Encloses a reference to a paper or other
      publication.  Is not itself a hyper-link, but it can be used within the
      \element{slink} and \element{rlink} elements.
      \elementitem{/cite}
      \elementitem{acronym} Encloses an acronym (e.g.
      \element{acronym}XML\element{/acronym}).
      \elementitem{/acronym}
      \elementitem{slink} Defines a \sci{} web site relative link.  It
      requires one attribute that is the path of the target relative to
      \sciurl.  Text bracketed by the start and end tags will be presented
      as a link in online documents.  The text may include phrase elements.
      \elementitem{/slink}%
      \elementitem{rlink} Defines a \sr{} \cvs{} relative link.
      \element{rlink} links to material on the documentation web
      site, but  is more easily explained as a path into the cvs tree.  \element{Rlink}
      requires a path relative to the location of the component
      specification document.  Text bracketed by the start and end tags are
      presented as a link in online documents.  The bracketed text may
      include phrase elements.
      \elementitem{/rlink}%
      \elementitem{latex} Encloses a short bit of latex code such as an equation.
      \elementitem{/latex}%
    \end{description}
    \elementitem{/p}%
    \elementitem{orderedlist}  An ordered (numbered) list.  Contains one or
    more \element{listitem} elements.
    \begin{description}
      \elementitem{listitem}  An item in an ordered list.  A
      \element{listitem} may contain paragraphs and nested lists.
      \elementitem{/listitem}%
    \end{description}
    \elementitem{/orderedlist}%
    \elementitem{unorderedlist} An unordered list.  Contains one or
    more \element{listitem} elements.
    \begin{description}
      \elementitem{listitem}  An item in an unordered list.  A
      \element{listitem} may contain paragraphs and nested lists.
      \elementitem{/listitem}%
    \end{description}
    \elementitem{/unorderedlist}%
    \elementitem{desclist} A description list.  Contains a list of terms or
    short phrases
    and their definitions or descriptions.  Consists of one or more \element{desclistitem}
    elements. 
    \begin{description}
      \elementitem{desclistitem} Contains one \element{desclistterm} and one
      \element{desclistdef}. 
      \begin{description}
        \elementitem{desclistterm} A word or short phrase (not enclosed in
        \element{p} element).
        \elementitem{/desclistterm}
        \elementitem{desclistdef} Definition or description of the
        corresponding term.  May consist of a mix of paragraph elements and
        list elements.
        \elementitem{/desclistdef}%
      \end{description}
      \elementitem{/desclistitem}%
    \end{description}
    \elementitem{/desclist}%
    \elementitem{note}  Calls attention to a piece of information.  Aimed
    towards users of the component.  Presumably the information within a
    \element{note} element is rendered in a unique
    way.  A note consists of paragraphs and lists only.
    \elementitem{/note}%
    \elementitem{tip} Contains an especially helpful piece of information.
    Aimed towards users of the component.  Presumably, the information
    within a \element{tip} element is rendered in a unique way.  A
    tip consists of paragraphs and lists only.
    \elementitem{/tip}%
    \elementitem{warning} Contains a warning (e.g., ``Don't press the red
    button!''). Aimed towards users of the component.  Presumably, the
    information within a \element{warning} element is rendered in a
    unique way.  A warning consists of paragraphs and lists only.
    \elementitem{/warning}%
    \elementitem{developer}  Starts a developer section.  Material
    here should be aimed towards fellow programmers.  The
    \element{developer} element acts similar to the \element{description} element, 
    \element{developer} elements can not be nested.
    \elementitem{/developer}%
    \elementitem{latex} Encloses a block of latex code.  Note that this is
    the block level version of the \element{latex} element -- the inline
    version, which can only be used inside the \element{p} element 
    (discussed above).
    \elementitem{/latex}%
  \end{description}
  \elementitem{/description} 
\end{description}


\section{\SECediting}
\label{\SECediting}

An ordinary text editor can be used to create the component specification
document's content.  However, it is easy to get lost in the noise of the
\xml{} syntax.  It is best to use an \xml{}/\dtd{} aware editor.
This type of editor will help construct a valid component specification
document.

\Emacs{} is one such editor.  It supports an editing environment called
\xml{} mode, a derivative of \psgml{} mode.  \Xml{} mode
highlights \xml{} syntax, indents nested elements and their content, and
automatically inserts elements and attributes based on the position of the
insertion point.  It is still possible to create invalid documents using
\emacs{} \xml{} mode.
Use of \emacs{} \xml{} mode is described in the following sections.

\subsection{\SUBSECgettingSources}
\label{\SUBSECgettingSources}

Recent versions of \emacs{} come with \xml{} mode installed.  To check
if you have recent versions of \emacs{},  type \keyboard{M-x xml-mode}.  You \emph{don't} have
\xml{} mode if you get the message
\screen{[No match]} in return.

 \xml{} mode can be obtained
\htmladdnormallinkfoot{online}{\psgmlurl}.

\subsection{\SUBSECdotEmacs}
\label{\SUBSECdotEmacs}

The following lisp code should be inserted into your \filename{.emacs}
file:

\begin{verbatim}
; Tell emacs to use sgml/xml mode for the following file types.
(setq auto-mode-alist
(append
'(("\\.xml" . xml-mode))
auto-mode-alist))
(autoload 'sgml-mode "psgml" "Major mode to edit SGML files." t)
(autoload 'xml-mode "psgml" "Major mode to edit XML files." t) 

; Customize sgml/xml-mode default settings.
(add-hook 'sgml-mode-hook (lambda () (setq sgml-indent-data t)))

; Create some faces for use with sgml/xml mode.
; Change colors to suite your fancy.
(make-face 'sgml-start-tag-face) 
(set-face-foreground 'sgml-start-tag-face "MediumSeaGreen") 
(make-face 'sgml-end-tag-face) 
(set-face-foreground 'sgml-end-tag-face "SeaGreen") 
(make-face 'sgml-entity-face) 
(set-face-foreground 'sgml-entity-face "Red") 
(make-face 'sgml-doctype-face) 
(set-face-foreground 'sgml-doctype-face "firebrick") 
(make-face 'sgml-comment-face) 
(set-face-foreground 'sgml-comment-face "blue") 

; Use faces defined above.
(setq sgml-set-face t)
(setq sgml-markup-faces 
'((comment   . sgml-comment-face) 
(start-tag . sgml-start-tag-face) 
(end-tag   . sgml-end-tag-face) 
(doctype   . sgml-doctype-face) 
(entity    . sgml-entity-face))) 
\end{verbatim}

\subsection{\SUBSECgettingStarted}
\label{\SUBSECgettingStarted}

To start a new component specification or to edit an existing one with
\emacs{}:

\begin{enumerate}
\item Use the \keyboard{C-x C-f} (\keyboard{find-file}) command to create a
  new file or to open an existing file.  The file's name must have the
  suffix \filename{.xml}.
\item Insert the preamble:
\begin{verbatim}
<?xml version="1.0" encoding="UTF-8" ?>
<!DOCTYPE component SYSTEM 
"<top-of-cvs-tree>/doc/Utilities/XML/component.dtd">
\end{verbatim}
  Replace \screen{<top-of-cvs-tree>} with the path to the top of the
  \sr{} \cvs{} tree.
\item Insert the following \xml{} comment at the bottom of the file:
\begin{verbatim}
<!-- Keep this comment at the end of the file
Local variables:
mode: xml
sgml-default-dtd-file:"<top-of-cvs-tree>/doc/Utilities/XML/component.ced"
sgml-omittag:nil
sgml-shorttag:nil
End:
-->
\end{verbatim}
  
\item Insert the top level \element{component} element using \emacs{}
  pop-up menu (shift-button1 or shift-button3 for emacs and button3 for
  xemacs): Position the cursor after the \verb+<!DOCTYPE ...>+ declaration
  and select the item named \menuitem{component} from the pop-up menu.
  \Emacs{} will prompt for the \element{component}
  element's \attribute{name} and \attribute{category} attributes. After
  typing in each of the attributes, \Emacs{} will  insert a number of
  other required elements.
\item Insert content into these elements.  For example, insert your
  name between the start and end tags of the \element{author} element.
  Refer to section~\sechyperref{\SECcontent} for information about
  the content of any elements.
\item The \element{gui} element is not automatically inserted by \emacs{} 
  because a gui is optional.  Most components will support a gui, 
  so the \element{gui} will need to be inserted.
  element.  Do this by positioning the cursor after the end tag of
  the \element{io} element.  Then select the \element{gui} element
  from \emacs{} pop-up menu.  \Emacs{} will insert the \element{gui}
  and its required sub-elements.
\item Complete the component specification by adding content, elements, and
  attributes.
\end{enumerate}

Note:

\begin{itemize}
\item Most elements (like the \element{description} element) are composed of
  sub-elements.  To add sub-elements, position the cursor between the
  start and end tags of the element, then insert an element using
  \emacs{} pop-up menu.  The pop-up menu will list elements that are
  valid at the insertion point.
\item A few elements, namely \element{slink} and \element{rlink}
  elements, take attributes.  If these elements are added via \emacs{}
  pop-up menu, \emacs{} will prompt for the element's
  attribute values.  If these element's are added manually,
  add the attributes via the pop-up menu: position the cursor
  anywhere within the element's start tag and select the desired
  attribute from the pop-up menu's list of attributes.  \Emacs{} will
  prompt for the value of the attribute and insert it in the
  element's start tag.
\end{itemize}

\section{\SECvalidation}
\label{\SECvalidation}

Validate the component specification document to ensure that it
is complete and correct.  Do this before committing it to CVS otherwise you
may break \sr{} for yourself and others.  Online \htmladdnormallinkfoot{Validation Tool for SCIRun XML Files}{\xmlvalidurl} can be used.

\begin{htmlonly}
%
%  For more information, please see: http://software.sci.utah.edu
% 
%  The MIT License
% 
%  Copyright (c) 2004 Scientific Computing and Imaging Institute,
%  University of Utah.
% 
%  License for the specific language governing rights and limitations under
%  Permission is hereby granted, free of charge, to any person obtaining a
%  copy of this software and associated documentation files (the "Software"),
%  to deal in the Software without restriction, including without limitation
%  the rights to use, copy, modify, merge, publish, distribute, sublicense,
%  and/or sell copies of the Software, and to permit persons to whom the
%  Software is furnished to do so, subject to the following conditions:
% 
%  The above copyright notice and this permission notice shall be included
%  in all copies or substantial portions of the Software.
% 
%  THE SOFTWARE IS PROVIDED "AS IS", WITHOUT WARRANTY OF ANY KIND, EXPRESS
%  OR IMPLIED, INCLUDING BUT NOT LIMITED TO THE WARRANTIES OF MERCHANTABILITY,
%  FITNESS FOR A PARTICULAR PURPOSE AND NONINFRINGEMENT. IN NO EVENT SHALL
%  THE AUTHORS OR COPYRIGHT HOLDERS BE LIABLE FOR ANY CLAIM, DAMAGES OR OTHER
%  LIABILITY, WHETHER IN AN ACTION OF CONTRACT, TORT OR OTHERWISE, ARISING
%  FROM, OUT OF OR IN CONNECTION WITH THE SOFTWARE OR THE USE OR OTHER
%  DEALINGS IN THE SOFTWARE.
%


%
% Code to generate sr bottom banners.  To use this in your latex doc insert
% the following code just before the \end{document} command:
%
% \begin{htmlonly}
% %
%  For more information, please see: http://software.sci.utah.edu
% 
%  The MIT License
% 
%  Copyright (c) 2004 Scientific Computing and Imaging Institute,
%  University of Utah.
% 
%  License for the specific language governing rights and limitations under
%  Permission is hereby granted, free of charge, to any person obtaining a
%  copy of this software and associated documentation files (the "Software"),
%  to deal in the Software without restriction, including without limitation
%  the rights to use, copy, modify, merge, publish, distribute, sublicense,
%  and/or sell copies of the Software, and to permit persons to whom the
%  Software is furnished to do so, subject to the following conditions:
% 
%  The above copyright notice and this permission notice shall be included
%  in all copies or substantial portions of the Software.
% 
%  THE SOFTWARE IS PROVIDED "AS IS", WITHOUT WARRANTY OF ANY KIND, EXPRESS
%  OR IMPLIED, INCLUDING BUT NOT LIMITED TO THE WARRANTIES OF MERCHANTABILITY,
%  FITNESS FOR A PARTICULAR PURPOSE AND NONINFRINGEMENT. IN NO EVENT SHALL
%  THE AUTHORS OR COPYRIGHT HOLDERS BE LIABLE FOR ANY CLAIM, DAMAGES OR OTHER
%  LIABILITY, WHETHER IN AN ACTION OF CONTRACT, TORT OR OTHERWISE, ARISING
%  FROM, OUT OF OR IN CONNECTION WITH THE SOFTWARE OR THE USE OR OTHER
%  DEALINGS IN THE SOFTWARE.
%


%
% Code to generate sr bottom banners.  To use this in your latex doc insert
% the following code just before the \end{document} command:
%
% \begin{htmlonly}
% %
%  For more information, please see: http://software.sci.utah.edu
% 
%  The MIT License
% 
%  Copyright (c) 2004 Scientific Computing and Imaging Institute,
%  University of Utah.
% 
%  License for the specific language governing rights and limitations under
%  Permission is hereby granted, free of charge, to any person obtaining a
%  copy of this software and associated documentation files (the "Software"),
%  to deal in the Software without restriction, including without limitation
%  the rights to use, copy, modify, merge, publish, distribute, sublicense,
%  and/or sell copies of the Software, and to permit persons to whom the
%  Software is furnished to do so, subject to the following conditions:
% 
%  The above copyright notice and this permission notice shall be included
%  in all copies or substantial portions of the Software.
% 
%  THE SOFTWARE IS PROVIDED "AS IS", WITHOUT WARRANTY OF ANY KIND, EXPRESS
%  OR IMPLIED, INCLUDING BUT NOT LIMITED TO THE WARRANTIES OF MERCHANTABILITY,
%  FITNESS FOR A PARTICULAR PURPOSE AND NONINFRINGEMENT. IN NO EVENT SHALL
%  THE AUTHORS OR COPYRIGHT HOLDERS BE LIABLE FOR ANY CLAIM, DAMAGES OR OTHER
%  LIABILITY, WHETHER IN AN ACTION OF CONTRACT, TORT OR OTHERWISE, ARISING
%  FROM, OUT OF OR IN CONNECTION WITH THE SOFTWARE OR THE USE OR OTHER
%  DEALINGS IN THE SOFTWARE.
%


%
% Code to generate sr bottom banners.  To use this in your latex doc insert
% the following code just before the \end{document} command:
%
% \begin{htmlonly}
% \include{../Utilities/TeX/banner_bottom.tex}
% \end{htmlonly}
%
% and adjust the path above as needed.
%

\begin{rawhtml}
<script type="text/javascript">
if (inDocTree) {
  document.write('<script type="text/javascript" src="',treetop,'doc/Utilities/HTML/banner_bottom.js"><\/script>');
}
</script>
\end{rawhtml}

% \end{htmlonly}
%
% and adjust the path above as needed.
%

\begin{rawhtml}
<script type="text/javascript">
if (inDocTree) {
  document.write('<script type="text/javascript" src="',treetop,'doc/Utilities/HTML/banner_bottom.js"><\/script>');
}
</script>
\end{rawhtml}

% \end{htmlonly}
%
% and adjust the path above as needed.
%

\begin{rawhtml}
<script type="text/javascript">
if (inDocTree) {
  document.write('<script type="text/javascript" src="',treetop,'doc/Utilities/HTML/banner_bottom.js"><\/script>');
}
</script>
\end{rawhtml}

\end{htmlonly}

\end{document}
