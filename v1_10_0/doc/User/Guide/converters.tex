% -*-latex-*-
%
%  The contents of this file are subject to the University of Utah Public
%  License (the "License"); you may not use this file except in compliance
%  with the License.
%
%  Software distributed under the License is distributed on an "AS IS"
%  basis, WITHOUT WARRANTY OF ANY KIND, either express or implied. See the
%  License for the specific language governing rights and limitations under
%  the License.
%
%  The Original Source Code is SCIRun, released March 12, 2001.
%
%  The Original Source Code was developed by the University of Utah.
%  Portions created by UNIVERSITY are Copyright (C) 2001, 1994
%  University of Utah. All Rights Reserved.
%

\section{Importing and Exporting \sr{} Data}
\label{sec:import_export} 
\index{importing}
\index{exporting}
\index{converter}

\sr{} does not import or export ``foreign'' data directly. A set of
command line utilities, called \dfn{converters}, import and export
data.  The import utilities convert foriegn data to \sr{} data file
objects.  The export utilities convert \sr{} data file objects to
foreign data. The converters import/export the following \sr{} data
types: \datatype{Field}, \datatype{Matrix}, and \datatype{ColorMap}.
Foreign data is text based.  Its formats are described later in this
section.

The following sections discuss \sr{}'s data objects, import/export
``theory'', text format of foreign data, and use of converters.


\subsection{Persistent Data Objects}
\label{sec:sr_data_object}

\sr{} stores its data to disk as \dfn{Persistent Objects}.  A
persistant data object is a data ``snapshot''---the data and its state
are saved and the data object can be read at a later time.

\sr{}'s persistent objects are built atop XDR, a library that
abstracts away the architecture specific facets of data input and
output (e.g. endianness, pointer size, etc).

\note{\sr{}'s data files should not be manually edited.  Editing
  these files manually may result in data corruption.}

The converters can read and write \sr{}'s persistent data objects.

\subsubsection{Fields}

A \sr{} \datatype{Field} consists of a \datatype{Mesh} and a set of
data values and data value locations.  Data values can be located at
the nodes, edges, faces, or cells of the mesh.  A \datatype{Field} can
also consist of a \datatype{Mesh} with no associated data.

Note that the components of a field are exported separately.  Mesh
node locations are exported to a file, mesh connectivity data are
export to another file, and mesh data values are exported to a third
file.


\subsubsection{Meshes}

A \sr{} mesh consists of 

\sr{} meshes are \dfn{unstructured}, \dfn{structured}, or
\dfn{regular}.  

Node locations and connectivities for unstructured meshs are specified
explicitly.  The unstructured mesh types are:
\datatype{PointCloudMesh}, \datatype{CurveMesh},
\datatype{TriSurfMesh}, \datatype{QuadSurfMesh},
\datatype{TetVolMesh}, \datatype{HexVolMesh}.

For structured meshes, node locations are specified explicitly and
connectivities are known implicitly.  The structured meshes are
\datatype{StructCurveMesh}, \datatype{StructQuadSurfMesh}, and
\datatype{StructHexVolMesh}.

Node locations and connectivities are known implicitly for regular
meshes.  The regular mesh types are \datatype{ScanlineMesh},
\datatype{ImageMesh}, \datatype{LatVolMesh}.


\sr{} converters do not support fields containing regular meshes.
Fields with regular meshes can be imported/exported using the
\package{Teem} package.  
See the \htmladdnormallinkfoot{\package{Teem} file format documentation}{http://www.cs.utah.edu/~gk/teem/nrrd/format.html} and
the \htmladdnormallinkfoot{\package{Teem} module descriptions}{\htmlurl{\latexhtml{\scisoftware{}/doc}{../../..}/Developer/Modules/Teem.html}}
for information.

One other mesh type, \datatype{MaskedLatVolMesh} is used to mask out
invalid regions (nodes or elements) of a mesh.

See also \secref{Field Types}{sec:field-types}.

\subsubsection{Data}

\sr{} data types that can be associated with a Mesh are:
scalar (e.g. \datatype{char}, \datatype{int}, \datatype{double}),
\datatype{Vectors}, and \datatype{Tensors}.


\subsubsection{Matrices}

\sr{} supports three matrix types: \datatype{ColumnMatrix},
\datatype{DenseMatrix}, and \datatype{SparseRowMatrix}.  

\sr{}'s converters support all three types.

Matrix types can also be transferred between \sr{} and Matlab using
\sr{}'s \package{MatlabInterface} package. See the
\package{MatlabInterface} package \htmladdnormallinkfoot{module
  descriptions}{\htmlurl{\latexhtml{\scisoftware{}/src}{../../..}/Packages/MatlabInterface/Dataflow/TeX/Matlab/Matlab/Matlab.html}}.


\subsubsection{Color Maps}

Import and export of color-maps is supported by the converters.

\subsection{Exporting a Field}
\label{sec:export_field}

Field data objects cannot be exported directly by the converters.  A
field must be first split into a mesh part (still a field object
though) and a data part (a matrix object).

After a field is split, its mesh and data parts can be exported using
converters.  The mesh converter produces two files, a node location
data file and a node connectivity data file. The data converter
produces a file containing node data.

A field is split in \sr{} by constructing a simple network using
modules \datatype{FieldReader}, \datatype{ManageFieldData},
and \datatype{MatrixWriter}.  

A \datatype{FieldReader} module reads the field to be split.
\datatype{FieldReader}'s output port is connected to
\datatype{ManageFieldData}'s input port (\datatype{ManageFieldData}'s
matrix input port is unconnected).  \datatype{ManageFieldData}'s
matrix output port is connected to \datatype{MatrixWriter}'s input
port.  The network saves the field's data as a matrix object.

Converters are used to export mesh data from the field object and the
field's data from the matrix object.

\subsection{Importing a Field}
\label{sec:import_field}

To import a field, two converters and a small \sr{} network are needed.

The first converter produces an incomplete field object (mesh data
only) from node location and node connectivity files.  The second
converter produces a matrix data object from a node data file.

A complete field (mesh and data) is assembled in \sr{} by
constructing a simple network using modules \datatype{FieldReader},
\datatype{MatrixReader},  \datatype{ManageFieldData}, and
\datatype{FieldWriter}.

A \datatype{FieldReader} module reads the incomplete field object.
\datatype{FieldReader}'s output port is connected to
\datatype{ManageFieldData}'s field input port.  A
\datatype{MatrixReader} is used to read the matrix object.
\datatype{MatrixReader}'s output port is connected to
\datatype{ManageFieldData}'s matrix input port.
\datatype{ManageFieldData}'s field output port is connected to
\datatype{FieldWriter}'s input port (\datatype{ManageFieldData}'s
matrix output port is unconnected).  \datatype{FieldWriter} will write
the complete field object to a file.

\subsection{Node Location File Format}
\label{sec:node_loc_fmt}

Text-based node location files are called ``pts'' files (the file
extension is \filename{.pts}).  The format is as follows:

\begin{verbatim}
N
x0 y0 z0
x1 y1 z1
.
.
.
xN yN zN
\end{verbatim}

\verb|N| is the number of nodes in the file.  It can be ommitted.
If it is, the \option{-noPtsHeader} option must be given to the
converter command.  Node coordinates are given on the remaining
lines. X, y, and z values are separated by white space.

\subsection{Node Connectivity File Format}
\label{sec:node_conn_fmt}

Two types of text-based node connectivity files exist.  The first
type, called ``tri'' files contain triangle data.  A ``tri'' file
looks like this:

\begin{verbatim}
N
i0 j0 k0
i1 j1 k0
.
.
.
iN jN kN
\end{verbatim}

\verb|N| is optional.  It specifies the number of triangles in the
file.  If \verb|N| is omitted, option \option{-noTrisCount} must
be passed to the (import) converter command.  Each of the remaining lines
specify node indicies of each triangle.

``Tet'' connectivity files contain tetrahedral elements.  They look
like this:

\begin{verbatim}
N
i0 j0 k0 l0
i1 j1 k0 l0
.
.
.
iN jN kN lN
\end{verbatim}

\verb|N| is optional.  It specifies the number of tetrahedral elements in the
file.  If \verb|N| is omitted then option \option{-noTetsCount} must
be passed to the (import) converter command.  Each of the remaining lines
specifiy node indicies of each tetrahedron.

Node indicies in ``tri'' and ``tet'' files can be ``zero'' or ``one
based.''  Use the converter option \option{-oneBasedIndexing} if the
smallest \verb|i|, \verb|j|, \verb|k|, or \verb|l| index is 1.


\subsection{Edge File Format}




\subsection{Matrix File Format}
\label{sec:matrix_fmt}

There are three text-based matrix file formats, one each for
\datatype{ColumnMatrix}, \datatype{DenseMatrix}, and
\datatype{SparseRowMatrix} data objects.


\subsubsection{Column Matrix}

The  column matrix file format is:

\begin{verbatim}
N
v0 
v1
.
.
.
vN
\end{verbatim}

The converters \command{ColumnMatrixToText} and \command{TextToColumnMatrix} 
convert column matrices to and from text based forms.

\verb|N| is optional.  It specifies the number of matrix rows.  If not
specified the \option{-noHeader} option must be passed to
\command{TextToColumnMatrix}.  Following \verb|N| is a white space
separated list of data values.


\subsubsection{Dense Matrix}

The dense matrix file format is:

\begin{verbatim}
N M
v0
v1
.
.
.
vNxM
\end{verbatim}

The programs \command{DenseMatrixToText} and \command{TextToDenseMatrix} 
convert dense matrices to and from text based forms.

\verb|N| and \verb|M| are optional.  \verb|N|, \verb|M| are the number
of matrix rows and columns respectively .  If \verb|N| and \verb|M|
are not specified the \option{-noHeader nrows ncols} option must be
passed to \command{TextToDenseMatrix}.  Following \verb|N| and
\verb|M| is a list of data values given in row major order.


\subsubsection{Sparse Row Matrix}

The sparse row matrix file format is:

\begin{verbatim}
NR NC NE
r0 c0 v0
r1 c1 v1
.
.
.
rNE cNE vNE
\end{verbatim}

The programs \command{SparseRowMatrixToText} and
\command{TextToSparseRowMatrix} convert sparse row matrices to and from
text based forms.

\verb|NR|, \verb|NC|, and \verb|NE| are optional.  If present, they
specify the number of rows, number of columns, and number of matrix
entries respectively.  If omitted, option \option{-noHeader nrows
  ncols nentries} must be passed to \command{TextToSparseRowMatrix}.
Matrix entries must have ascending row indices. Column indices must be
in ascending order for rows with multiple entries.

\subsection{Colormap File Format}
\label{sec:colormap_fmt}

The color map file format is:

\begin{verbatim}
N
r1 g1 b1 a1
r2 g2 b2 a2
.
.
.
rN gN bN aN
\end{verbatim}

The programs \command{ColorMapToText} and \command{TextToColorMap}
convert color maps to/from their text format.

\verb|N| is optional.  It specifies the number of color map entries in
the file.  If not specified pass option \option{-noHeader} to
\command{TextToColorMap}  Following \verb|N| is a list of color map
entries, one per line.  Each entry consists of values for red, green,
blue, and alpha.  All entries range from 0.0 to 1.0 and specify 
evenly spaced entries in the color map table.


\subsection{Field Converters}

CurveFieldToText
TextToCurveField

HexVolFieldToText
TextToHexVolField

PointCloudFieldToText
TextToPointCloudField

QuadSurfFieldToText
TextToQuadSurfField

StructCurveFieldToText
TextToStructCurveField

StructHexVolFieldToText
TextToStructHexVolField

StructQuadSurfFieldToText
TextToStructQuadSurfField

TetVolFieldToText
TextToTetVolField

TriSurfFieldToText
TextToTriSurfField

\subsection{Matrix Converters}

ColumnMatrixToText
TextToColumnMatrix

DenseMatrixToText
TextToDenseMatrix

SparseRowMatrixToText
TextToSparseRowMatrix


\subsection{ColorMap Converters}

ColorMapToText
TextToColorMap

\subsection{Examples}
\label{sec:converter_ex}

The \directory{SCIRunData/convert-examples} directory
contains examples of data that have been converted to/from text and
\sr{} formats.  See the \filename{SCIRunData/convert-examples/README}
for details.
