% -*-latex-*-
%
%  The contents of this file are subject to the University of Utah Public
%  License (the "License"); you may not use this file except in compliance
%  with the License.
%
%  Software distributed under the License is distributed on an "AS IS"
%  basis, WITHOUT WARRANTY OF ANY KIND, either express or implied. See the
%  License for the specific language governing rights and limitations under
%  the License.
%
%  The Original Source Code is SCIRun, released March 12, 2001.
%
%  The Original Source Code was developed by the University of Utah.
%  Portions created by UNIVERSITY are Copyright (C) 2001, 1994
%  University of Utah. All Rights Reserved.
%

\section{Importing and Exporting \sr{} Data}
\label{sec:import_export} 
\index{importing}
\index{exporting}
\index{converter}

\sr{} does not import or export ``foreign'' data directly. A set of
command line utilities, called \dfn{converters}, import and export
data.  The import utilities convert foriegn data to \sr{} data file
objects.  The export utilities convert \sr{} data file objects to
foreign data. The converters import/export the following \sr{} data
types: \datatype{Field}, \datatype{Matrix}, and \datatype{ColorMap}.
Foreign data is text based.

The following sections discuss \sr{} data objects, format of foreign
data files, field export/import ``theory'', and the use of converters.


\subsection{\sr{} Data Objects}
\label{sec:sr_data_object}

\sr{} stores its data to disk as \dfn{persistent objects}.  A
persistant data object is a data ``snapshot''---data and its state
are saved and can be read at a later time.

\sr{}'s persistent objects can be saved in binary form or text form.
\sr{}'s binary objects are built atop XDR, a library that abstracts
away the architecture specific facets of data input/output (e.g.
endianness, pointer size, etc).

\note{\sr{}'s data files should not be manually edited.  Editing
  these files manually may result in data corruption.}

The converters read and write \sr{}'s persistent data objects.

\subsubsection{Fields}

A \sr{} \datatype{Field} consists of a \datatype{Mesh} and a set of
data values.  Data values can be located at the nodes, edges, faces,
or cells of the mesh.  A \datatype{Field} can consist of a
\datatype{Mesh} component only (no associated data).

The components of a field are exported separately.  Mesh node
locations are exported to a file, mesh connectivity data are export to
another file, and field data values are exported to a third file.


\subsubsection{Meshes}

\sr{} meshes are classified as \dfn{unstructured}, \dfn{structured}, or
\dfn{regular}.  In general, a mesh consists of nodes (data value location
data) and node connectivity data.

Node locations and connectivities for unstructured meshs are specified
explicitly.  The unstructured mesh types are:
\datatype{PointCloudMesh}, \datatype{CurveMesh},
\datatype{TriSurfMesh}, \datatype{QuadSurfMesh},
\datatype{TetVolMesh}, \datatype{HexVolMesh}.

Node locations are specified explicitly and connectivities are known
implicitly for structured meshes.  The structured meshes are
\datatype{StructCurveMesh}, \datatype{StructQuadSurfMesh}, and
\datatype{StructHexVolMesh}.

Node locations and connectivities are known implicitly for regular
meshes.  The regular mesh types are \datatype{ScanlineMesh},
\datatype{ImageMesh}, \datatype{LatVolMesh}.

\sr{} converters do not support fields containing regular meshes.
Fields with regular meshes can be imported/exported using the
\package{Teem} package.  See the \htmladdnormallinkfoot{\package{Teem} file format documentation}{http://www.cs.utah.edu/~gk/teem/nrrd/format.html} and
the \htmladdnormallinkfoot{\package{Teem} module
  descriptions}{\htmlurl{\latexhtml{\scisoftware{}/doc}{../../..}/Developer/Modules/Teem.html}}
for information.


\subsubsection{Field Data}

C++ data types \datatype{char}, \datatype{int}, \datatype{double}, and
\sr{} data types \datatype{Vector} and \datatype{Tensor} can be
associated with a field.

\subsubsection{Matrices}

\sr{} supports three matrix types: \datatype{ColumnMatrix},
\datatype{DenseMatrix}, and \datatype{SparseRowMatrix}. \sr{}'s
converters support all three types.

Matrix types can also be transferred between \sr{} and Matlab using
\sr{}'s \package{MatlabInterface} package. See the
\package{MatlabInterface} package \htmladdnormallinkfoot{module
  descriptions}{\htmlurl{\latexhtml{\scisoftware{}/src}{../../..}/Packages/MatlabInterface/Dataflow/TeX/Matlab/Matlab/Matlab.html}}.


\subsubsection{Color Maps}

Import and export of color-maps is supported by the converters.

\subsection{Exporting a Field}
\label{sec:export_field}

Field data objects cannot be exported directly by the converters.  A
field must be first split into a mesh part, a field object with no data,
and a data part as a matrix object.

After a field is split, its mesh and data parts can be exported using
field and matrix converters.  The field converters produce two files,
a node location data file (called a ``pts'' file) and a node
connectivity data file (``tri'', ``tet'', ``quad'', or ``hex'' files).
The matrix converter produces a file of data values (``txt'' files).

A field is split in \sr{} by constructing a simple network using
modules \datatype{FieldReader}, \datatype{ManageFieldData},
and \datatype{MatrixWriter}.  

A \datatype{FieldReader} module reads the field to be split.
\datatype{FieldReader}'s output port is connected to
\datatype{ManageFieldData}'s input port (\datatype{ManageFieldData}'s
matrix input port is unconnected).  \datatype{ManageFieldData}'s
matrix output port is connected to \datatype{MatrixWriter}'s input
port.  The network saves the field's data as a matrix object.

Converters are used to export mesh data from the field object and the
field's data from the matrix object.

\subsection{Importing a Field}
\label{sec:import_field}

To import a field, two converters and a small \sr{} network are needed.

A field converter produces an incomplete \sr{} field object (mesh data
only) from node location and node connectivity files.  A matrix
converter produces a \sr{} matrix data object from a node data file.

A complete field (mesh and data) is assembled in \sr{} by
constructing a simple network using modules \datatype{FieldReader},
\datatype{MatrixReader},  \datatype{ManageFieldData}, and
\datatype{FieldWriter}.

A \datatype{FieldReader} module reads the incomplete field object.
\datatype{FieldReader}'s output port is connected to
\datatype{ManageFieldData}'s field input port.  A
\datatype{MatrixReader} is used to read the matrix object.
\datatype{MatrixReader}'s output port is connected to
\datatype{ManageFieldData}'s matrix input port.
\datatype{ManageFieldData}'s field output port is connected to
\datatype{FieldWriter}'s input port (\datatype{ManageFieldData}'s
matrix output port is unconnected).  \datatype{FieldWriter} will write
the complete field object to a file.

\subsection{Node Location File Format}
\label{sec:node_loc_fmt}

Text-based node location files are called ``pts'' files (the file
extension is \filename{.pts}).  The format is as follows:

\begin{verbatim}
N
x0 y0 z0
x1 y1 z1
.
.
.
xN yN zN
\end{verbatim}

\verb|N| is the number of nodes in the file.  It can be ommitted.  If
it is, the \option{-noPtsHeader} option must be given to the (import)
converter command.  Each of the remaining lines specify the
coordinates of one node.

\subsection{Connectivity File Format}
\label{sec:node_conn_fmt}

Five types of node connectivity files exist.  They are all text
files.  The five file formats define trianglular, tetrahedral, edge,
hexahedral, and quadrilateral elements.

\subsubsection{Triangle (tri) File}

Triangle connectivity files define triangles.  A ``tri'' file has the
following format:

\begin{verbatim}
N
i0 j0 k0
i1 j1 k0
.
.
.
iN jN kN
\end{verbatim}

\verb|N| is optional.  It specifies the number of triangles in the
file.  If \verb|N| is omitted, option \option{-noTrisCount} must
be passed to the (import) converter command.  Each of the remaining lines
specify node indicies of one triangle.  Node indicies are assumed to be
zero-based unless  option \option{-oneBasedIndexing} is passed to
\command{TextToTriSurfField}.


\subsubsection{Tetrahedral (tet) File}
Tetrahedral connectivity files contain tetrahedral elements.  ``Tet''
files are formatted as follows:

\begin{verbatim}
N
i0 j0 k0 l0
i1 j1 k0 l0
.
.
.
iN jN kN lN
\end{verbatim}

\verb|N| is optional.  It specifies the number of tetrahedral elements in the
file.  If \verb|N| is omitted then option \option{-noTetsCount} must
be passed to the (import) converter command.  Each of the remaining lines
specifiy node indicies of one tetrahedron.

Node indicies in ``tri'' and ``tet'' files can be ``zero'' or ``one
based.''  Node indicies are assumed to be
zero-based unless  option \option{-oneBasedIndexing} is passed to
\command{TextToTetVolField}.  

\subsubsection{Edge File}

``Edge'' files are created by the converter \command{CurveFieldToText}
and read by \command{TextToCurveField}.  Edge files have a
\filename{.edge} extension.  Edge files have the following format:

\begin{verbatim}
N
i0 j0
i1 j1
.
.
.
iN jN
\end{verbatim}

\verb|N| is optional.  If present, \verb|N| specifies the number of
edges defined in the file.  If \verb|N| is omitted then option
{-noEdgesCount} must be passed to (import) converter
\command{TextToCurveField}.  Each of the remaining lines specify node
indicies of one edge.  Node indicies are assumed to be
zero-based unless  option \option{-oneBasedIndexing} is passed to
\command{TextToHexVolField}.

\subsubsection{Quadrilateral (quad) File}

``Quad'' files are created by the converter \command{QuadSurfFieldToText}
and read by \command{TextToQuadSurfField}.  Quad files have a
\filename{.quad} extension.  Quad files have the following format:

\begin{verbatim}
N
i0 j0 k0 l0
i1 j1 k1 l1
.
.
.
iN jN kN lN
\end{verbatim}

\verb|N| is optional.  If present, \verb|N| specifies the number of
quadrilateral elements defined in the file.  If \verb|N| is omitted then
option {-noQuadesCount} must be passed to (import) converter
\command{TextToCurveField}.  Each of the remaining lines specify node
indicies of one quadrilateral element.  Node indicies are assumed to be
zero-based unless  option \option{-oneBasedIndexing} is passed to
\command{TextToQuadSurfField}.



\subsection{Hexahedral (hex) File}

``Hex'' files are created by the converter \command{HexVolFieldToText}
and read by \command{TextToHexVolField}.  Hex files have a
\filename{.hex} extension.  Hex files have the following format:

\begin{verbatim}
N
i0 j0 k0 l0 m0 n0 o0 p0
i1 j1 k1 l1 m1 n1 o1 p1
.
.
.
iN jN kN lN mN nN oN pN
\end{verbatim}

\verb|N| is optional.  If present, \verb|N| specifies the number of
hexahedral elements defined in the file.  If \verb|N| is omitted then
option {-noHexesCount} must be passed to (import) converter
\command{TextToCurveField}.  Each of the remaining lines specify node
indicies of one hexahedral element.  Node indicies are assumed to be
zero-based unless  option \option{-oneBasedIndexing} is passed to
\command{TextToHexVolField}.

%% Needs work!
\subsection{Structured Meshes}

Structured meshes are imported/exported by the converters as node
location (pts) files.  Node locations are specified explicitly in a
pts file. Connectivities are implicit because node locations are
stored in \dfn{scanline} order.

For example, in a structured hexahedral mesh, the list of nodes comprising
element \(e_{i,j,k}\) is \(\{n_{i,j,k}, n_{i,j,k+1}, n_{i,j+1,k+1}, n_{i,j+1,k}, n_{i+1,j,k}, n_{i+1,j,k+1}, n_{i+1,j+1,k+1}, n_{i+1,j+1,k}\}\)

\subsection{Matrix File Format}
\label{sec:matrix_fmt}

There are three text-based matrix file formats, one each for
\datatype{ColumnMatrix}, \datatype{DenseMatrix}, and
\datatype{SparseRowMatrix} data objects.


\subsubsection{Column Matrix}

The  column matrix file format is:

\begin{verbatim}
N
v0 
v1
.
.
.
vN
\end{verbatim}

The converters \command{ColumnMatrixToText} and \command{TextToColumnMatrix} 
convert column matrices to and from text based forms.

\verb|N| is optional.  It specifies the number of matrix rows.  If not
specified the \option{-noHeader} option must be passed to
\command{TextToColumnMatrix}.  Following \verb|N| is a white space
separated list of data values.


\subsubsection{Dense Matrix}

The dense matrix file format is:

\begin{verbatim}
N M
v0
v1
.
.
.
vNxM
\end{verbatim}

The programs \command{DenseMatrixToText} and \command{TextToDenseMatrix} 
convert dense matrices to and from text based forms.

\verb|N| and \verb|M| are optional.  \verb|N|, \verb|M| are the number
of matrix rows and columns respectively .  If \verb|N| and \verb|M|
are not specified the \option{-noHeader nrows ncols} option must be
passed to \command{TextToDenseMatrix}.  Following \verb|N| and
\verb|M| is a list of data values given in row major order.


\subsubsection{Sparse Row Matrix}

The sparse row matrix file format is:

\begin{verbatim}
NR NC NE
r0 c0 v0
r1 c1 v1
.
.
.
rNE cNE vNE
\end{verbatim}

The programs \command{SparseRowMatrixToText} and
\command{TextToSparseRowMatrix} convert sparse row matrices to and from
text based forms.

\verb|NR|, \verb|NC|, and \verb|NE| are optional.  If present, they
specify the number of rows, number of columns, and number of matrix
entries respectively.  If omitted, option \option{-noHeader nrows
  ncols nentries} must be passed to \command{TextToSparseRowMatrix}.
Matrix entries must have ascending row indices. Column indices must be
in ascending order for rows with multiple entries.

\subsection{Colormap File Format}
\label{sec:colormap_fmt}

The color map file format is:

\begin{verbatim}
N
r1 g1 b1 a1
r2 g2 b2 a2
.
.
.
rN gN bN aN
\end{verbatim}

The programs \command{ColorMapToText} and \command{TextToColorMap}
convert color maps to/from their text format.

\verb|N| is optional.  It specifies the number of color map entries in
the file.  If not specified pass option \option{-noHeader} to
\command{TextToColorMap}  Following \verb|N| is a list of color map
entries, one per line.  Each entry consists of values for red, green,
blue, and alpha.  All entries range from 0.0 to 1.0 and specify 
evenly spaced entries in the color map table.


\subsection{Field Converter Synopses}

In the synopses that follow, \ptext{field} is the name of a \sr{}
field file, \ptext{pts} is the name of a node location (pts) file,
\ptext{edges}, \ptext{hexes}, \ptext{quads}, \ptext{tets}, and
\ptext{tris} are the names of node connectivity files.

\begin{verbatim}
CurveFieldToText field pts edges [-noPtsCount] [-noEdgesCount] [-oneBasedIndexing]

TextToCurveField pts edges field [-noPtsCount] [-noEdgesCount] [-oneBasedIndexing]

HexVolFieldToText field pts hexes [-noPtsCount] [-noHexesCount] [-oneBasedIndexing]

TextToHexVolField pts hexes field [-noPtsCount] [-noHexesCount] [-oneBasedIndexing] [-binOutput]

PointCloudFieldToText field pts [-noPtsCount]

TextToPointCloudField pts field [-noPtsCount]

QuadSurfFieldToText field pts quads [-noPtsCount] [-noQuadsCount] [-oneBasedIndexing]

TextToQuadSurfField pts quads field [-noPtsCount] [-noQuadsCount] [-oneBasedIndexing]

StructCurveFieldToText field pts [-noHeader]

TextToStructCurveField pts edges field [-noPtsCount] [-noEdgesCount] [-oneBasedIndexing]

StructHexVolFieldToText field pts [-noHeader]

TextToStructHexVolField pts field [-noHeader ni nj nk]

StructQuadSurfFieldToText field pts [-noHeader]

TextToStructQuadSurfField pts field [-noHeader ni nj]

TetVolFieldToText field pts tets [-noPtsCount] [-noTetsCount] [-oneBasedIndexing]
TextToTetVolField pts tets field [-noPtsCount] [-noTetsCount] [-oneBasedIndexing]

TriSurfFieldToText field pts tris [-noPtsCount] [-noTrisCount] [-oneBasedIndexing]
TextToTriSurfField pts tris field [-noPtsCount] [-noTrisCount] [-oneBasedIndexing]
\end{verbatim}


\subsection{Matrix Converter Synopses}

In the synopses that follow, \ptext{matrix-in} and \ptext{matrix-out}
are the name of either \sr{} matrix files or the names of matrix text
files, depending on context.

\begin{verbatim}
ColumnMatrixToText matrix-in matrix-out [-noHeader]

TextToColumnMatrix matrix-in matrix-out [-noHeader]

DenseMatrixToText matrix-in matrix-out [-noHeader]

TextToDenseMatrix matrix-in matrix-out [-noHeader nrows ncols]

SparseRowMatrixToText matrix-in matrix-out [-noHeader]

TextToSparseRowMatrix matrix-in matrix-out [-noHeader nrows ncols nnz]
\end{verbatim}

\subsection{ColorMap Converter Synopses}

\begin{verbatim}
In the synopses that follow, \ptext{matrix-in} is either the name of a \sr{}
matrix file or the name of a matrix text file, depending on context.

ColorMapToText  colormap-in colormap-out [-noHeader]

TextToColorMap  colormap-in colormap-out [-noHeader]
\end{verbatim}

\subsection{Examples}
\label{sec:converter_ex}

The directory \directory{SCIRunData/convert-examples} 
contains examples of data that have been converted to/from text and
\sr{} formats.  See the \filename{SCIRunData/convert-examples/README}
for details.
