% -*-latex-*-
%
%  The contents of this file are subject to the University of Utah Public
%  License (the "License"); you may not use this file except in compliance
%  with the License.
%
%  Software distributed under the License is distributed on an "AS IS"
%  basis, WITHOUT WARRANTY OF ANY KIND, either express or implied. See the
%  License for the specific language governing rights and limitations under
%  the License.
%
%  The Original Source Code is SCIRun, released March 12, 2001.
%
%  The Original Source Code was developed by the University of Utah.
%  Portions created by UNIVERSITY are Copyright (C) 2001, 1994
%  University of Utah. All Rights Reserved.
%

%
% Recommened markup for latex docs.
%
                                %
% (Mostly) Short cuts  ==============================
\newcommand{\SCI}{\emph{SCI}}
\newcommand{\sci}{\SCI}
\newcommand{\scii}{SCI Institute}
\newcommand{\PSE}{\emph{BioPSE}}
\newcommand{\pse}{\PSE}
\newcommand{\SR}{{\em SCIRun}}
\newcommand{\sr}{\SR}
\newcommand{\eg}{{\em e.g.,}}
\newcommand{\ie}{{\em i.e.,}}
\newcommand{\etc}{{\em etc.}}
\newcommand{\etal}{{\em et al.}}
\newcommand{\degrees}{{$^{\circ}$}}
\newcommand{\splitline}{\begin{center}\rule{\columnwidth}{.7mm}\end{center}}
\newcommand{\X}[1]{#1\index{#1}}
\newcommand{\rob}{Rob MacLeod (macleod@cvrti.utah.edu)}
\newcommand{\ted}{Ted Dustman (dustman@cvrti.utah.edu)}
\newcommand{\blythe}{Blythe D Nobleman (blythe@cs.utah.edu)}
\newcommand{\srig}{\sr{} Installation Guide}
\newcommand{\srug}{\sr{} User's Guide}
\newcommand{\srdg}{\sr{} Developer's Guide}

% Literal ~ character
\newcommand{\ltilde}{\textasciitilde}

% Encloses its argument between angle brackets.
\newcommand{\ab}[1]{\latexhtml{\textless{}#1\textgreater}{<#1>}}

% Inserts a left angle bracket.
\newcommand{\la}{\latexhtml{\textless}{<}}

% Inserts a right angle bracket.
\newcommand{\ra}{\latexhtml{\textgreater}{>}}

% Mark up commands =================================

% Url
%\newcommand{\url}[1]{#1}

% ip address
\newcommand{\ipaddr}[1]{#1}
\newcommand{\localhost}{\ipaddr{127.0.0.1}}

% Acronym
\newcommand{\acronym}[1]{#1}

% Predefined acronyms
\newcommand{\gui}{\acronym{GUI}}
\newcommand{\tcl}{\acronym{TCL}}
\newcommand{\xml}{\acronym{XML}}

% Markup the first time use of term that may be unfamiliar to the
% reader. 
\newcommand{\dfn}[1]{\emph{#1}}

% In the next command #1 is the term and #2 is the shortcut or acronym that
% will be used in the rest of the document. 
\newcommand{\dfna}[2]{\emph{#1} (#2)}

% Directory name markup.
\newcommand{\directory}[1]{\texttt{#1}}

% Markup a file name.
\newcommand{\filename}[1]{\texttt{#1}}

% Markup text which is the name of a command.
\newcommand{\command}[1]{\texttt{#1}} 

% Command option.
\newcommand{\option}[1]{\texttt{#1}}

% Markup text typed at the keyboard.
\newcommand{\keyboard}[1]{\texttt{#1}} 

% Parameterized text - marks up text that is to be
% substituted for by the reader.
\newcommand{\ptext}[1]{\textit{\ab{#1}}}

% Markup text the user might see on his screen.
\newcommand{\screen}[1]{\texttt{#1}}

% Markup the name of a GUI menu.
\newcommand{\guimenu}[1]{\textbf{#1}}

% Markup the name of a GUI menu.
\newcommand{\menu}[1]{\guimenu{#1}}

% Markup a gui menu item name.
\newcommand{\guimenuitem}[1]{\textbf{#1}}

% Markup a menu item name.
\newcommand{\menuitem}[1]{\guimenuitem{#1}}

% Markup name of a ui button.
\newcommand{\guibutton}[1]{\textbf{#1}}

% Markup name of a ui button.
\newcommand{\button}[1]{\guibutton{#1}}

% Markup name of a gui text item.
\newcommand{\guitext}[1]{\textit{#1}}

% Markup name of a gui label.
\newcommand{\guilabel}[1]{\textbf{#1}}

% GUI variable
\newcommand{\guivar}[1]{\texttt{#1}}

% Scirun port
\newcommand{\srport}[1]{\texttt{#1}}

% Sockets port
\newcommand{\port}[1]{\texttt{#1}}

% Socket
\newcommand{\socket}[2]{\texttt{#1:#2}}

% Variable
\newcommand{\variable}[1]{\texttt{#1}}

% Data type
\newcommand{\datatype}[1]{\texttt{#1}}

% Markup an inline code fragment.
\newcommand{\icode}[1]{\texttt{#1}}

% Markup a module name.
\newcommand{\module}[1]{\texttt{#1}}

% Markup a sr package name.
\newcommand{\package}[1]{\texttt{#1}}

% Markup a sr category name.
\newcommand{\category}[1]{\texttt{#1}}

% Call attention to some useful bit of information.
\newcommand{\note}[1]{\emph{Note: #1}} 

% Call attention to a warning.
\newcommand{\warning}[1]{\emph{Warning: #1}} 

% Call attention to a tip.
\newcommand{\tip}[1]{\emph{Tip: #1}} 

% Env variable markup.
\newcommand{\envvar}[1]{\texttt{#1}}

% Markup for the title of a book or article or whatever.
\newcommand{\etitle}[1]{\texttt{#1}}

% Markup name of a latex section.  First arg is section name.  Second arg
% is section's label.  When processed with latex, the section
% name is emphasized.  When processed with latex2html section name is made
% into a link to section.
\newcommand{\secname}[2]{\latexhtml{\emph{#1}}{\htmlref{#1}{#2}}}

% Section ref command.  Use of this command
% in place of \ref to reference sections (subsections etc.) will
% create a more web friendly version of your document.
% The command's first argument is the link text (used only on the html
% page) and the second argument is the section's label
\newcommand{\secref}[2]{\hyperref[ref]{\emph{#1}}{Section~}{}{#2}}

% A latex command
\newcommand{\latexcommand}[1]{\texttt{\textbackslash#1}}

% A latex package
\newcommand{\latexpackage}[1]{\textit{#1}}

% Use these in place of missing content.
\newcommand{\missing}[1]{\emph{#1 - Coming Soon.}}

% Use this to make note of incomplete content.
\newcommand{\incomplete}{\emph{More Comming Soon.}}

% Mark up a mail address
\newcommand{\mailto}[1]{\texttt{#1}}

% Sci Urls =========================================

% www style urls.
\newcommand{\scisoftwareurl}{\url{http://software.sci.utah.edu}}
\newcommand{\sciurl}{\url{http://www.sci.utah.edu}}
\newcommand{\scidocurl}{\scisoftwareurl}
\newcommand{\bugsurl}{\scisoftwareurl/bugzilla}
