%
% component.tex
%
% Documentation for component specification xml documents.  Also see
% component.dtd. 
%
% tjd is responsible for this atrocity.
%

\documentclass{article}
\usepackage{html}
\parindent 0in
\setlength{\parskip}{\medskipamount}

% Custom markup commands for this document.
\newcommand{\mytilde}{\symbol{'176}}
\newcommand{\ab}[1]{\latexhtml{$<$#1$>$}{<#1>}}
\newcommand{\la}{\latexhtml{$<$}{<}}
\newcommand{\ra}{\latexhtml{$>$}{>}}
\newcommand{\acronym}[1]{#1}
\newcommand{\dfn}[1]{\emph{#1}}
\newcommand{\filename}[1]{\texttt{#1}}
\newcommand{\xml}{\acronym{XML}}
\newcommand{\Xml}{\xml}
\newcommand{\psgml}{\acronym{PSGML}}
\newcommand{\Psgml}{\psgml}
\newcommand{\dtd}{\acronym{DTD}}
\newcommand{\gui}{\acronym{GUI}}
\newcommand{\pse}{\acronym{PSE}}
\newcommand{\Pse}{\pse}
\newcommand{\sci}{\acronym{SCI}}
\newcommand{\Sci}{\sci}
\newcommand{\sr}{\acronym{SCIRun}}
\newcommand{\Sr}{\sr}
\newcommand{\cvs}{\acronym{CVS}}
\newcommand{\Cvs}{\cvs}
\newcommand{\emacs}{[x]emacs}
\newcommand{\Emacs}{[X]Emacs}
\newcommand{\sectitle}[1]{\emph{#1}}
\newcommand{\sechyperref}[1]{\hyperref{\sectitle{#1}}{}{}{#1}}
\newcommand{\element}[1]{\ab{\texttt{#1}}}
\newcommand{\attribute}[1]{\texttt{#1}}
\newcommand{\elementitem}[1]{\item[\texttt{\ab{#1}}]}
\newcommand{\keyboard}[1]{\texttt{#1}}
\newcommand{\screen}[1]{\texttt{#1}}
\newcommand{\menuitem}[1]{\textbf{#1}}
\newcommand{\menu}[1]{\textbf{#1}}
\newcommand{\sciurl}{http://www.sci.utah.edu}
\newcommand{\psgmlurl}{http://www.lysator.liu.se/projects/about\_psgml.html}
\newcommand{\Note}[1]{\emph{Note: #1}}
\newcommand{\icode}[1]{\texttt{#1}}
\newcommand{\xmlvalidurl}{http://www.stg.brown.edu/service/xmlvalid/}
\newcommand{\dtdurl}{http://www.sci.utah.edu/scirun\_docs/doc/ReferenceGuide/component.dtd}
\newcommand{\thisdocurl}{http://www.sci.utah.edu/scirun\_docs/doc/ReferenceGuide/component.html}
\newcommand{\rgpath}{<top-of-cvs-tree>/doc/ReferenceGuide}
\newcommand{\dtdcvspath}{\rgpath{}/component.dtd}
\newcommand{\componentexpath}{\rgpath{}/component-ex.xml}
\newcommand{\ytbd}{\emph{Yet to be done}}
\newcommand{\specprocurl}{http://www.sci.utah.edu/scirun\_docs/doc/ReferenceGuide/specProcedure.html}
\newcommand{\doctitle}[1]{\textit{#1}}

% Section title commands (I'm experimenting - bear with me.)
\newcommand{\SECintro}{Introduction}
\newcommand{\SUBSECwhySpec}{Purpose}
\newcommand{\SUBSECwhyXml}{Why XML?}
\newcommand{\SUBSECwhatDtd}{The DTD}
\newcommand{\SECcontent}{Rules, Structure, and Content}
\newcommand{\SUBSECrules}{Rules}
\newcommand{\SUBSECstructContent}{Structure and Content}
\newcommand{\SUBSECdescElement}{A Description of the \ab{description} Element}
\newcommand{\SUBSUBSECcommonUsage}{Common Usage}
\newcommand{\SUBSUBSECdescElementIntro}{Introduction}
\newcommand{\SUBSUBSECreference}{Reference}
\newcommand{\SECexample}{Example}
\newcommand{\SECediting}{Editing the Spec}
\newcommand{\SUBSECgettingSources}{XML Mode - Do You Have It?}
\newcommand{\SUBSECdotEmacs}{XML Mode - .emacs}
\newcommand{\SUBSECgettingStarted}{XML Mode - Getting Started}
\newcommand{\SECvalidation}{Validation}
\newcommand{\SECtools}{Tools}
\newcommand{\SUBSECmakeHtmlDoc}{Generating HTML Documents}
\newcommand{\SUBSECmakePdfDoc}{Generating Pdf Documents}
\newcommand{\SUBSECmoduleMaker}{Generating a New Component}
% End of section title commands.

\title{\xml\ Component Specification Guide} 
\author{\htmladdnormallinkfoot{Ted Dustman}{mailto:dustman@cvrti.utah.edu}}

\begin{document}
\maketitle \pagenumbering{roman}
\label{toc}
\tableofcontents
\pagebreak \pagenumbering{arabic}

\section{\SECintro}
\label{\SECintro}

This section introduces the component specification document.  You should
have already read \htmladdnormallinkfoot{\doctitle{Module Design/Implementation Procedure}}{\specprocurl}

\begin{latexonly}
  An online version of this document is available\footnote{\thisdocurl}.
\end{latexonly}
\begin{htmlonly}
  A postscript verison of this document can be found, relative to the top
  of your \sr\ \cvs\ tree, in \filename{doc/ReferenceGuide/component.ps}.
\end{htmlonly}

\subsection{\SUBSECwhySpec}
\label{\SUBSECwhySpec}

Writing the \xml\ component specification is one of the first steps in the
design and implementation of a component.

The \xml\ component specification documents the use, design,
implementation, and testing of a \sr\ component.  

It also serves as data for applications and databases.  For example:

\begin{itemize}
\item Documentation (html and pdf forms) for component users and developers
  will be generated from the component specification.

\item The collection of all component specifications will be part of a
  search-able database.
  
\item The component specification will help drive the component code generation
  process.

\item The component specification may drive an automated testing process.
\end{itemize}

Component specifications are written in a special mark-up language.  This
language (call it the component specification mark-up language if you
like) is in turn formulated using the \dfn{eXtensible Mark-up Language}
(\xml).

\Note{Components that are not accompanied by a valid and complete
  specification document will not be incorporated into official
  releases of the \sr\ software.}

\subsection{\SUBSECwhyXml}
\label{\SUBSECwhyXml}

An \xml\ formulation of the component specification is used to achieve
the following goals:

\begin{description}
\item[Data centric viewpoint] We want the component specification to represent
  a set of data that can be used in several contexts.

\item[Uniformity] We want all components to be specified the same way.

\item[Validation] We want to validate (check for correctness and
  completeness) component specifications.
\end{description}

The component specification mark-up language is similar in concept to HTML but
has a different (and much smaller) set of tags and a simpler and more
regular set of document composition rules.
Section~\sechyperref{\SECcontent} describes
the component specification mark-up language in detail.

\subsection{\SUBSECwhatDtd}
\label{\SUBSECwhatDtd}

A \dfn{Document Type Definition} (\dtd) defines a set of elements (and
element attributes), and the way they must be used in order to produce a
valid \xml\ document.

The component specification \dtd\ can be found here (relative to the top
of your cvs tree):

\begin{verbatim}
doc/ReferenceGuide/component.dtd
\end{verbatim}

The component specification \dtd\ is used to validate component
specifications.  More on validation can be found in
section~\sechyperref{\SECvalidation}.

The component specification \dtd\ is used to create and edit component
specifications.  See
section~\sechyperref{\SECediting} for more
information.

\section{\SECcontent}
\label{\SECcontent}

This section describes the component specification mark-up language in
detail.

\subsection{\SUBSECrules}
\label{\SUBSECrules}

When writing a component specification a few simple composition rules
must be followed:

\begin{itemize}
\item In the \xml\ world tags are used to delimit (or identify)
  \dfn{elements}.  Elements are the basic building blocks of an \xml\ 
  document and contain content (text) or nested elements or both.  Every
  start tag \emph{must} have a corresponding end tag.
  
\item A component specification document starts with 2 lines of preamble
  text:

\begin{verbatim}
<?xml version="1.0" encoding="UTF-8" ?>
<!DOCTYPE component SYSTEM 
    "<top-of-cvs-tree>/doc/ReferenceGuide/component.dtd}">
\end{verbatim}

where \filename{<top-of-cvs-tree>} is the path to the top of your SCIRun \cvs\
tree. 
  
\item The preamble is followed by a set of nested elements starting with
  the \element{component} element.
  
\item Every element must be delimited with a start-tag and an end-tag.
  Tags are enclosed in angle brackets.  This means you may not use the
  literal forms of the characters \keyboard{\la} and \keyboard{\ra} in your
  elements' content.  Instead you must type \keyboard{\&lt;} and
  \keyboard{\&gt;} respectively.

\item Tags are case sensitive and all tags must be lower case.
  
\item A few elements require an \dfn{attribute}.  Attributes are specified
  in an element's start tag.  An attribute looks like this:
  \icode{attribute\_name="attribute\_value"}.  The attribute's value must be
  enclosed in double quotes. Attributes are documented below along side
  their corresponding element.
  
\item The literal form of the character \& may not be used in element
  content.  You must type \keyboard{\&amp;} instead.

\end{itemize}

The set of valid elements are described next.

\subsection{\SUBSECstructContent}
\label{\SUBSECstructContent}

Below, the component specification's \xml\ structure and corresponding
logical content are described.  Each element's start tag is followed by a
description of its purpose and content, followed by its nested elements,
and finally by its end tag.  Indentation is used to show nesting
relationships among elements.  The rather complex \element{description}
element, which is used in several contexts, is not described in detail here
-- section~\sechyperref{\SUBSECdescElement} describes it fully.

\begin{description}
  \elementitem{component name="component-name" 
    category="category-name"} Starts the component specification.  All
  other elements are nested within this one.  This element requires the
  \icode{name} and \icode{category} attributes.  These
  attributes name your component and specify
  the category the component belongs to respectively.
  \begin{description}
    \elementitem{overview} Starts the overview section.  Contains only
    nested elements.
    \begin{description}
      \elementitem{authors}  List of component's authors.  Contains 1 or more
      \element{author} elements.
      \begin{description}
        \elementitem{author} Name of 1 author.
        \elementitem{/author}
      \end{description}
      \elementitem{/authors}
      \elementitem{summary} Short (1 or 2 sentence) summary of component's function.
      \elementitem{/summary}
      \elementitem{description} Comprehensive description of component's
      function.  The \element{description} element is a mini-documentation
      environment.  See
      section~\sechyperref{\SUBSECdescElement}
      for details on the use of the \element{description} element.
      \elementitem{/description}
      \elementitem{examplesr} The name of an \filename{.sr} file that
      demonstrates the component's use.
      \elementitem{/examplesr}
    \end{description}
    \elementitem{/overview} 
    \elementitem{implementation} Lists the names of
    a component's source code files \emph{excluding} the component's
    primary source file (which is implied by the \element{component}
    element's \attribute{name} attribute).

    The need for an implementation section is rare - the source code for
    most components is contained entirely within a component's primary
    source code file.

    The subelements \element{ccfile}, \element{cfile}, and \element{ffile}
    list the names of c++, c, and fortran source code files respectively.
    These elements may occur multiple times and in any order.
    Note that file name extensions are omitted.
    \begin{description}
      \elementitem{ccfile} Names 1 c++ file, e.g. \element{ccfile}foo\element{/ccfile}.
      \elementitem{/ccfile}
      \elementitem{cfile} Names 1 c file.
      \elementitem{/cfile}
      \elementitem{ffile} Names 1 fortran file.
      \elementitem{/ffile}
    \end{description}
    \elementitem{/implementation}
    \elementitem{io} Component's inputs and
    outputs.  Contains 2 elements, \element{inputs} followed by
    \element{outputs}.
    \begin{description}
      \elementitem{inputs lastportdynamic="(yes | no)"} Describes the component's inputs. Inputs may
      come from a port, a file, or a device.  Zero or more \element{port}
      elements are used to describe inputs from ports.  Zero or more
      \element{file} elements are used to describe inputs from files.  Zero
      or more \element{device} elements are used to describe inputs from
      devices.  There must be at least 1 \element{port} element or 1
      \element{file} element or 1 \element{device} element.

      If your component's last port is dynamic then the
      \attribute{lastportdynamic} attribute should be set to \keyboard{yes}
      (\attribute{lastportdynamic="yes"}).  If your component's last port
        is not dynamic then you may omit this attribute or set its value to
        \keyboard{no}.
      \begin{description}
        \elementitem{port} States that input data comes from a port. 
        Sub-elements describe the data further.
        \begin{description}
          \elementitem{name} The port's name.
          \elementitem{/name}
          \elementitem{description} High level description of the data 
          accepted by the port, e.g. ``These are potential data from body 
          surface of a human.''
          \elementitem{/description}
          \elementitem{datatype} The name of a PSE data type, i.e. the 
          data type that is being transmitted on this port.
          \elementitem{/datatype}
          \elementitem{componentname} The name of an upstream component
          that is commonly used to send data to this component on this
          port.  This element may occur multiple times. 
          \elementitem{/componentname}
        \end{description}
        \elementitem{/port}
        \elementitem{file} States that input comes from a file.
        Sub-elements describe the data further.
        \begin{description}
          \elementitem{description} High level description of the data 
          provided by the file, e.g. ``These are potential data from body 
          surface of a human.''
          \elementitem{/description}
          \elementitem{datatype} The name of a \sr\ data type, i.e. the 
          data type being read from the file.  \Note{Only \sr\ data types
            may be read from files - no custom formats allowed!}
          \elementitem{/datatype}
        \end{description}
        \elementitem{/file}
        \elementitem{device} States that input comes from a device.
        Sub-elements describe the data further.
        \begin{description}
          \elementitem{devicename} Device's name.
          \elementitem{/devicename}
          \elementitem{description} High level description of the data 
          provided by the device.
          \elementitem{/description}
        \end{description}
        \elementitem{/device}
      \end{description}
      \elementitem{/inputs}
      \elementitem{outputs} Describes the component's outputs.  Outputs may
      go to a port or to a file or to a device.  Use 0 or more
      \element{port} elements to describe outputs to ports, 0 or more
      \element{file} elements to describe outputs to files, and 0 or more
      \element{device} elements to describe outputs to devices.
      \begin{description}
        \elementitem{port} States that output is sent a port. Sub-elements describe the data further.
        \begin{description}
          \elementitem{description} A high level description of the data
          sent to the port, e.g. ``These are potential data from the body
          surface of a human.''
          \elementitem{/description}
          \elementitem{datatype} The name of a \sr\ data type, i.e. the 
          data type that is being transmitted on this port.
          \elementitem{/datatype}
          \elementitem{componentname} The name of a downstream component
          that is commonly connected to this output port.  This
          element may occur multiple times. 
          \elementitem{/componentname}
        \end{description}
        \elementitem{/port}
        \elementitem{file}  States that output is sent to a file.  It's 
        sub-elements describe the outputs.
        \begin{description}
          \elementitem{description} A high level description of the data 
          sent to the file, e.g. ``These are potential data from body 
          surface of a human.''
          \elementitem{/description}
          \elementitem{datatype}  The name of a PSE data type, i.e. the 
          data type being written to the file.
          \elementitem{/datatype}
        \end{description}
        \elementitem{/file}
        \elementitem{device} States that output goes to a device.
        Sub-elements describe the output further.
        \begin{description}
          \elementitem{devicename} Device's name.
          \elementitem{/devicename}
          \elementitem{description} High level description of the data
          sent to the device.
          \elementitem{/description}
        \end{description}
        \elementitem{/device}
    \end{description}
    \elementitem{/outputs}
  \end{description}
  \elementitem{/io}
  \elementitem{gui} States that the component supports a \gui.
  Sub-elements describe the \gui.
  \begin{description}
    \elementitem{description} Describes the purpose of the \gui, e.g.
    ``The \gui\ allows you to steer the simulation.''
    \elementitem{/description}
    \elementitem{parameter} Declares 1 \gui\  parameter.  A parameter is a
    \gui\  item that can be modified by the user.  Sub-elements describe the
    parameter.  Use 1 parameter element for each item in the \gui.
    \begin{description}
      \elementitem{widget} The type name of parameter's tcl widget.
      \elementitem{/widget}
      \elementitem{label}Name of parameter as it appears in \gui.
      \elementitem{/label}
      \elementitem{description}Describes the purpose of the parameter and
      how it can be used to control the component's behavior.
      \elementitem{/description}
    \end{description}
    \elementitem{/parameter}
    \elementitem{img}Name of a file that contains a picture of the \gui.
    \elementitem{/img}
  \end{description}
  \elementitem{/gui}%
  \elementitem{testing} Starts the testing section.  Sub-elements specify
  1 or more testing plans.
  \begin{description}
    \elementitem{plan} Declares 1 testing plan.  Sub-elements specify the
    testing plan's details.
    \begin{description}
      \elementitem{description}  High level description of the testing
      plan.  The \element{description} element is followed by 1 or more
      \element{step} elements.
      \elementitem{/description}
      \elementitem{step} Detailed description of 1 step in the testing
      procedure.   The content model of the \element{step} element is the
      same as the that of the \element{description} element (i.e., see
      section~\sechyperref{\SUBSECdescElement} for details).
      \elementitem{/step}
    \end{description}
    \elementitem{/plan}
  \end{description}
  \elementitem{/testing}
\end{description}
\elementitem{/component}
\end{description}

\subsection{\SUBSECdescElement}
\label{\SUBSECdescElement}

This section describes the \element{description} element in detail.

\subsubsection{\SUBSUBSECdescElementIntro}
\label{\SUBSUBSECdescElementIntro}

The \element{description} element contains text for human consumption.
It allows you to organize your text into paragraphs, lists, and
other structured content.

Note that the literal forms of the characters \la, \ra, and \& may not be
used in element content.  Instead you must type \keyboard{\&lt;},
\keyboard{\&gt;}, and \keyboard{\&amp;} respectively.

\subsubsection{\SUBSUBSECcommonUsage}
\label{\SUBSUBSECcommonUsage}

A \element{description} might contain only 1 or 2 sentences:

\begin{verbatim}
<description>
  <p>My component is great.  It's so easy to use that no further
    information is needed. </p>
</description>
\end{verbatim}

Note that even just 1 sentence must be enclosed in a paragraph element.

A \element{description} may contain only a few paragraphs:

\begin{verbatim}
<description>
  <p>My component is great.  It's so easy to use that no further
     information is needed.</p>

  <p>I lied.  What follows is a description of my component.</p>
   .
   .
   .
</description>
\end{verbatim}

You may refer to other material using the \element{cite}, \element{rlink},
or \element{slink} elements:

\begin{verbatim}
<description>
  <p>
    My component does xyz.  It uses the xyz algorithm described in the paper
    <cite>The XYZ Algorithm</cite> by Xavier Yilmez Zideco.  A less formal
    discussion of the xyz algorithm is
    <slink path=ncrr/publications/xyz.html>available online</slink>. 
  </p>
</description>
\end{verbatim}

The \element{cite} element marks text as the title of a paper or
book (but is not itself a hyper-link).

The \element{slink} and \element{rlink} elements allow you to reference
online material.

\element{Slink} links to material on the \sci\ 
web site (\sciurl).  It requires a path 
relative to \sciurl.  The path in the previous example resolves to
\sciurl{}/ncrr/publications/xyz.html.

\element{Rlink} links to material in the \acronym{CVS} tree
(\element{rlink} actually links to material on the documentation web site
but it is more easily explained as a path into the cvs tree).  It requires
a path relative to the location of your component specification document.  For
example, the path \keyboard{./stuff.html} resolves to the same directory
where your component specification document lives.

A more general link element is not available because the documentation
bureaucrats want some control over the material to which you link.

The \element{developer} element encloses material that is of interest only to
developers:

\begin{verbatim}
<description>
  .
  .
  .
  <developer>
    <p>Some smart developer ought to improve this rubbish.</p>
  </developer>
  .
  .
  .
<description>
\end{verbatim}

\subsubsection{\SUBSUBSECreference}
\label{\SUBSUBSECreference}

Below, the description element's \xml\ structure and corresponding logical
content are described.  Each element's start tag is followed by a
description of its purpose and content, followed by its nested elements,
and finally by its end tag.  Basic nesting relationships among elements are
shown by indentation.

\Note{Additional features may be added to the \element{description}
  element in the future.}

\begin{description}
  \elementitem{description} Starts a description.  A description consists
  of any combination of paragraphs (\element{p}), lists
  (\element{orderedlist}, \element{unorderedlist}, and \element{desclist}),
  admonitions (\element{note}, \element{tip}, or \element{warning}), and developer
  notes (\element{developer}).
  \begin{description}
    \elementitem{p} Starts a paragraph.  A paragraph consists of character
    data,  phrases (\element{term}, \element{keyword}, \element{keyboard},
    \element{cite}, and \element{acronym}) and links (\element{slink} and
    \element{rlink}). 
    \begin{description}
      \elementitem{term} Encloses a \dfn{term}.  A term is a word or short
      phrase you are introducing to your readers for the first time.
      \elementitem{/term}
      \elementitem{keyword} Encloses a \dfn{keyword}.
      \elementitem{/keyword}
      \elementitem{keyboard} Encloses material that is typed at a keyboard.
      \elementitem{/keyboard}
      \elementitem{cite} Encloses a reference to a paper or other
      publication.  Is not itself a hyper-link but it may be used within the
      \element{slink} and \element{rlink} elements.
      \elementitem{/cite}
      \elementitem{acronym} Encloses an acronym (e.g.
        \element{acronym}XML\element{/acronym}).
      \elementitem{/acronym}
      \elementitem{slink} Defines a \sci\ web site relative link.  It
      requires 1 attribute which is the path of the target relative to
      \sciurl.  Text bracketed by the start and end tags will be presented
      as a link in online documents.  The text may include phrase elements.
      \elementitem{/slink}%
      \elementitem{rlink} Defines a \sr\ \cvs\ relative link
      (\element{rlink} actually links to material on the documentation web
      site but it is more easily explained as a path into the cvs tree).  It
      requires a path relative to the location of your component
      specification document.  Text bracketed by the start and end tags will
      be presented as a link in online documents.  The bracketed text may
      include phrase elements.
      \elementitem{/rlink}%
    \end{description}
    \elementitem{/p}%
    \elementitem{orderedlist}  An ordered (numbered) list.  Contains 1 or
    more \element{listitem} elements.
    \begin{description}
      \elementitem{listitem}  An item in an ordered list.  A
      \element{listitem} may contain paragraphs and nested lists.
      \elementitem{/listitem}%
    \end{description}
    \elementitem{/orderedlist}%
    \elementitem{unorderedlist} An unordered list.  Contains 1 or
    more \element{listitem} elements.
    \begin{description}
      \elementitem{listitem}  An item in an unordered list.  A
      \element{listitem} may contain paragraphs and nested lists.
      \elementitem{/listitem}%
    \end{description}
    \elementitem{/unorderedlist}%
    \elementitem{desclist} A description list.  Contains a list of terms or
    short phrases
    and their definitions or descriptions.  Consists of 1 or more \element{desclistitem}
    elements. 
    \begin{description}
      \elementitem{desclistitem} Contains 1 \element{desclistterm} and 1
      \element{desclistdef}. 
      \begin{description}
        \elementitem{desclistterm} A word or short phrase (not enclosed in
        \element{p} element).
        \elementitem{/desclistterm}
        \elementitem{desclistdef} Definition or description of the
        corresponding term.  May consist of a mix of paragraph elements and
        list elements.
        \elementitem{/desclistdef}%
      \end{description}
      \elementitem{/desclistitem}%
    \end{description}
    \elementitem{/desclist}%
    \elementitem{note}  Calls attention to a piece of information.  Aimed
    towards users of the component.  Presumably the information within a
    \element{note} element will be rendered in a special
    way.  A note consists of paragraphs and lists only.
    \elementitem{/note}%
    \elementitem{tip} Contains an especially helpful piece of information.
    Aimed towards users of the component.  Presumably the information
    within a \element{tip} element will be rendered in a special way.  A
    tip consists of paragraphs and lists only.
    \elementitem{/tip}%
    \elementitem{warning} Contains a warning (e.g., ``Don't press the red
    button!''). Aimed towards users of the component.  Presumably the
    information within a \element{warning} element will be rendered in a
    special way.  A warning consists of paragraphs and lists only.
    \elementitem{/warning}%
    \elementitem{developer}  Starts a developer section.  Material
    here should be aimed towards fellow programmers.  The
    \element{developer} element acts like the \element{description} element
    except that \element{developer} elements may not be nested.
    \elementitem{/developer}%
  \end{description}
  \elementitem{/description} 
\end{description}


\section{\SECexample}
\label{\SECexample}

An example component specification can be found here (relative to the top
of your cvs tree):

\begin{verbatim}
doc/ReferenceGuide/component-ex.xml
\end{verbatim}

\section{\SECediting}
\label{\SECediting}

An ordinary text editor may be used to create the component specification
document's content.  However, it is easy to get lost in the noise of the
\xml\ syntax.  Therefore, it is best to use an \xml{}/\dtd\ aware editor.
This type of editor will help you construct a valid component specification
document.

\Emacs\ is one such editor.  It supports an editing environment called
\xml\ mode (which is really a derivative of \psgml\ mode).  \Xml\ mode
highlights \xml\ syntax, indents nested elements and their content, and
automatically inserts elements and attributes based on the position of the
insertion point.  It is still possible to create invalid documents using
\emacs\ \xml\ mode.

The following sections describe the use of \emacs\ \xml\ mode.

\subsection{\SUBSECgettingSources}
\label{\SUBSECgettingSources}

Recent versions of \emacs\ come with \xml\ mode installed.  You can check
if yours does by typing \keyboard{M-x xml-mode}.  You \emph{don't} have
\xml\ mode if you get the message
\screen{[No match]} in return.

If you are a University of Utah Computer Science person then you may use
the \xml\ mode installed under \filename{/usr/local/contrib/mcole/psgml}.
You must add the following bit of lisp code to your \filename{.emacs} file:

\begin{verbatim}
(setq load-path (append '("/usr/local/contrib/mcole/psgml") load-path))
\end{verbatim}

If you are not a UUCS person and you don't have \xml\ mode you may get it
\htmladdnormallinkfoot{online}{\psgmlurl}.

\subsection{\SUBSECdotEmacs}
\label{\SUBSECdotEmacs}

The following lisp code should be inserted into your \filename{.emacs}
file:

\begin{verbatim}
; Tell emacs to use sgml/xml mode for the following file types.
(setq auto-mode-alist
      (append
        '(("\\.sgm" . sgml-mode)
          ("\\.sgml" . sgml-mode)
          ("\\.xml" . xml-mode))
       auto-mode-alist))
(autoload 'sgml-mode "psgml" "Major mode to edit SGML files." t)
(autoload 'xml-mode "psgml" "Major mode to edit XML files." t) 

; Customize sgml/xml-mode default settings.
(add-hook 'sgml-mode-hook (lambda () (setq sgml-indent-data t)))

; Create some faces for use with sgml/xml mode.
; Change colors to suite your fancy.
(make-face 'sgml-start-tag-face) 
(set-face-foreground 'sgml-start-tag-face "MediumSeaGreen") 
(make-face 'sgml-end-tag-face) 
(set-face-foreground 'sgml-end-tag-face "SeaGreen") 
(make-face 'sgml-entity-face) 
(set-face-foreground 'sgml-entity-face "Red") 
(make-face 'sgml-doctype-face) 
(set-face-foreground 'sgml-doctype-face "firebrick") 
(make-face 'sgml-comment-face) 
(set-face-foreground 'sgml-comment-face "blue") 

; Use faces defined above.
(setq sgml-set-face t)
(setq sgml-markup-faces 
      '((comment   . sgml-comment-face) 
        (start-tag . sgml-start-tag-face) 
        (end-tag   . sgml-end-tag-face) 
        (doctype   . sgml-doctype-face) 
        (entity    . sgml-entity-face))) 
\end{verbatim}

\subsection{\SUBSECgettingStarted}
\label{\SUBSECgettingStarted}

To start a new component specification with \emacs\ do this:

\begin{enumerate}
\item Use the \keyboard{C-x C-f} (\keyboard{find-file}) command to create a
  new file.  The file's name must have the suffix \filename{.xml}.
\item Insert the preamble:
\begin{verbatim}
<?xml version="1.0" encoding="UTF-8" ?>
<!DOCTYPE component SYSTEM 
    "<top-of-cvs-tree>/doc/ReferenceGuide/component.dtd">
\end{verbatim}
  Replace \screen{<top-of-cvs-tree>} with the path to the top of your \sr\
  \cvs\ tree.
\item Select the \menuitem{Parse DTD} item from the \menu{DTD} menu
  which can be found in the window's main menu bar, or type \keyboard{C-c
    C-p}.
\item Insert the top level \element{component} element using \emacs\ pop-up
  menu (shift-button1 or shift-button3 for emacs and button3 for xemacs):
  Position the cursor at the end of the file and select the item named
  \menuitem{component} from the pop-up menu.  After doing this you will be
  prompted for the \element{component} element's \attribute{name} and
  \attribute{category} attributes.  attribute.  After typing in each of the
  attributes \Emacs\ will then insert a number of other required elements.
\item Insert content into these elements.  For example, you may insert your
  name between the start and end tags of the \element{author} element.
  Refer to section~\sechyperref{\SECcontent} if you have question about the
  content of any elements.
\item The \element{gui} element is not automatically inserted by \emacs\ 
  because a gui is optional.  Most components will support a gui though so
  you will probably need to insert the \element{gui} element.  Do this by
  positioning the cursor after the end tag of the \element{io} element.
  Then select the \element{gui} element from \emacs\ pop-up menu.  \Emacs\ 
  will insert the \element{gui} and its required sub-elements.
\item Complete the component specification by adding content, elements, and
  attributes.
\end{enumerate}

Note:

\begin{itemize}
\item Most elements (like the \element{description} element) are composed of
  sub-elements.  To add sub-elements, position the cursor between the start
  and end tags of the element and then insert an element using \emacs\ 
  pop-up menu.  The pop-up menu will list elements that are valid at the
  insertion point.
\item A few elements (namely \element{slink} and \element{rlink})
  elements take attributes.  If these elements are added via \emacs\ pop-up
  menu then \emacs\ will prompt you for the element's attribute values.  If
  these element's are added manually then you may add the attributes via the
  pop-up menu: position the cursor anywhere within the element's start tag
  and select the desired attribute from the pop-up menu's list of
  attributes.  \Emacs\ will prompt you for the value of the attribute and
  insert it in the element's start tag.
\end{itemize}


\section{\SECvalidation}
\label{\SECvalidation}

You must validate your component specification document to ensure that it
is complete and correct.

A validation tool can be found
\htmladdnormallinkfoot{online}{\xmlvalidurl}.

To use this tool you will need to change the second line of your document's
preamble section to the following:

\begin{verbatim}
<!DOCTYPE component SYSTEM 
  "http://www.sci.utah.edu/scirun_docs/doc/ReferenceGuide/component.dtd">
\end{verbatim}

This change will break emacs' \xml\ mode though so be sure to revert
to the original before using emacs.

\section{\SECtools}
\label{\SECtools}

This section describes tools that do useful things with the component
specification document.

\subsection{\SUBSECmoduleMaker}
\label{\SUBSECmoduleMaker}

\ytbd

\subsection{\SUBSECmakeHtmlDoc}
\label{\SUBSECmakeHtmlDoc}

\ytbd

\subsection{\SUBSECmakePdfDoc}
\label{\SUBSECmakePdfDoc}

\ytbd

\end{document}
