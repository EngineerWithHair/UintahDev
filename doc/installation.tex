\section{Downloading the Software} \label{Sec:download}

The Uintah problem solving environment provides a framework for solving PDEs on structured grids

Uintah can be obtained either from a tarball (http://www.uintah.utah.edu) or by using svn to download the latest source from the following:

\begin{Verbatim}[fontsize=\footnotesize]
svn co https://code.sci.utah.edu/svn/SCIRun/trunk Uintah
\end{Verbatim}

The above command checks out the Uintah source tree and installs it
into a directory called Uintah in the users home directory.

The Thirdparty library can similarly be obtained via:
\begin{Verbatim}[fontsize=\footnotesize]
svn co https://code.sci.utah.edu/svn/Thirdparty/1.25.4 Thirdparty
\end{Verbatim}

The Thirdparty library source code is downloaded into a directory
called Thirdparty.

\subsection{Installing Thirdparty, SCIRun, and Uintah}

\subsubsection{Thirdparty Install}


Please read the README.txt found in \tt~/Thirdparty.\normalfont

Thirdparty should be installed in \tt~/Thirdparty.\normalfont
As root, create this directory:

\begin{Verbatim}[fontsize=\footnotesize]

     \# mkdir /usr/local/Thirdparty

\end{Verbatim}

Change to the Thirdparty directory you checked out, i.e. \tt cd ~/Thirdparty.\normalfont

After reading the README.txt file type the follow as the root user:

\textbf{32bit OS:}

\begin{Verbatim}[fontsize=\footnotesize]

      \# ./install.sh /usr/local/Thirdparty/ 32

\end{Verbatim}

\textbf{64bit OS:}


\begin{Verbatim}[fontsize=\footnotesize]

      \# ./install.sh /usr/local/Thirdparty/ 64

\end{Verbatim}


\subsubsection{Configuring Uintah}

cd to \tt ~/SCIRun \normalfont and create the following directories: dbg and opt

cd to dbg and type the following to configure for a debug build:

\begin{Verbatim}[fontsize=\footnotesize]
../src/configure --enable-debug --enable-sci-malloc 
--enable-package=Uintah 
--with-thirdparty=/usr/local/Thirdparty/1.25.5/Linux/gcc-4.3.1-2-32bit/
\end{Verbatim}

Then build the software by typing \texttt{make} at the command
line. Once the debug build has finished which can take roughly an hour
on a single processor Pentium IV computer, cd to the opt/ and type the
following to configure for an optimized build:

\begin{Verbatim}[fontsize=\footnotesize]
./src/configure '--enable-optimze=-march=pentium4 -msse -msse2 
-mfpmath=sse -03' --disable-sci-malloc --enable-assertion-level=0 
--enable-package=Uintah 
--with-thirdparty=/usr/local/Thirdparty/1.25.4/Linux/gcc-4.3.1-2-32bit/
\end{Verbatim}

Then build the software by typing \tt make \normalfont at the command line.

\subsection{Installing Visualization Software}

Visualization of Uintah data is currently possible using any of three
software packages.  These are SCIRun, VisIt and Manta.  Of these, SCIRun is
no longer supported, although legacy versions will continue to work.  The
VisIt package from LLNL is general purpose visualization software that offers
all of the usual capabilities for rendering scientific data.  It is still
developed and maintained by LLNL staff, and its interface to Uintah data is
supported by the Uintah team.  Manta offers volume rendering and particle
visualizaton based on parallel (shared memory) ray tracing techniques.
While the capabilities of Manta are more limited, it is a fast way to
interactively interrogate reasonably large datasets, provided the user has
access to a reasonable shared memory resource, (e.g. an 8 core desktop system).

\subsubsection{Installing VisIt}

\subsubsection{Installing Manta}


