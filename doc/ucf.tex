\section{Computational Framework} \label{Sec:UCF}

%__________________________________
\subsection{Mechanics of Running sus}
 - Explain how to run on multiple processors\\
 - what do the different command line options mean\\
 - how to restart an uda\\
 
%__________________________________
\subsection{Time Related Variables}
Uintah components are timestepping codes.  As such, one of the first entries
in each input file describes the timestepping parameters.  An input file
segment is given below that encompasses all of the possible parameters.
Most are self-explanatory, and not all are required,
(e.g \tt <max\_Timestep>,
<max\_delt\_increase>, <end\_on\_max\_time\_exactly> \normalfont and
\tt <delt\_init> \normalfont are all optional).
\tt <timestep\_multiplier> \normalfont serves
as a CFL number, that is, a number, usually less than 1.0, that is used to
moderate the timestep automatically calculated by the individual components. 

\begin{Verbatim}[fontsize=\footnotesize]
<Time>
    <maxTime>            1.0         </maxTime>
    <initTime>           0.0         </initTime>
    <delt_min>           0.0         </delt_min>
    <delt_max>           1.0         </delt_max>
    <delt_init>          1.0e-9      </delt_init>
    <max_delt_increase>  2.0         </max_delt_increase>
    <timestep_multiplier>1.0         </timestep_multiplier>
    <max_Timestep>       100         </max_Timestep>
    <end_on_max_time_exactly>true    </end_on_max_time_exactly>
</Time>

\end{Verbatim}
%
%__________________________________
\subsection{Data Archiver}
- variable labels\\
- checkpointing \\
- different options for specifying the output frequency\\
%
%__________________________________
\subsection{Geometry objects}
- different objects available and how to specify them\\
- what is res \\
- operators, union, difference\\
- example of combining several geom\_objects\\
%
%__________________________________
\subsection{Grid specification}
Explain how a grid is specified and what these tags mean

\begin{Verbatim}[fontsize=\footnotesize]
<Level>
    <Box label="1">
       <lower>        [0,0,0]          </lower>
       <upper>        [5,5,5]          </upper>
       <extraCells>   [1,1,1]          </extraCells>
       <patches>      [1,1,1]          </patches>
    </Box>
    <spacing>         [0.5,0.5,0.5]    </spacing>
</Level>
 \end{Verbatim}
%
%__________________________________
\subsection{Adapative Mesh Refinement}
- need to discuss the input options for the different regridders.\\
- How is a cell flagged as needing to be refined
%
%__________________________________
\subsection{load Balancer}
- to be filled in

%__________________________________
\subsection{uda}
- discuss the directory structure of an uda
- discuss how to modify an input parameter before you restart an uda.

%__________________________________
\subsection{Visualization tools}
- VisIT\\
- manta?

%__________________________________
\subsection{Tools}
- puda, lineextract, timeextract, compare\_uda\\

%__________________________________
\subsection{Code}
-explain the basic directory structure of src
\begin{Verbatim}[fontsize=\footnotesize]
|-- CCA
|-- Components
|   |-- Angio
|   |-- Arches
|   |-- DataArchiver
|   |-- Examples
|   |-- ICE
|   |-- LoadBalancers
|   |-- MPM
|   |-- MPMArches
|   |-- MPMICE
|   |-- Models
|   |-- OnTheFlyAnalysis
|   |-- Parent
|   |-- PatchCombiner
|   |-- ProblemSpecification
|   |-- Regridder
|   |-- Schedulers
|   |-- SimulationController
|   |-- Solvers
|   |-- SpatialOps
|   `-- SwitchingCriteria
`-- Ports
|-- Core
|-- R_Tester
|-- StandAlone
|-- Teem
|-- VisIt
|-- build_scripts
|-- include
|-- on-the-fly-libs
|-- orderAccuracy
|-- osx
|-- scripts
|-- tau
|-- testprograms
`-- tools
\end{Verbatim}
