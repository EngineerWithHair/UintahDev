\section{Using Uintah} \label{Sec:UCF}

Several executable programs have been developed using the Uintah
Computational Framework (ucf).  The primary code is called sus which
stands for the Standalone Uintah Simulation.  The code was originally
developed to solve a complex fluid structure problem involving a
container filled with an explosive set in a fire.  The code would
model the fire and the subsequent heat transfer to the container
followed by the resultant container deformation and ultimate rupture
due to the ignition and burning of the explosive material all running
on thousands of processors requiring thousands of hours of computer
time and hundreds of gigabytes of data storage.  Although this code
was developed originally to solve this complicated multi-physics
problem, the general nature of the algorithms and the framework have
afforded researchers to use the code to investigate a wide range of
problems.  The framework is general purpose enough to allow for the
implementation of This code leverages the task based parallelism
inherent in the ucf to implement several time stepping algorithms for
structural mechanics, fluid dynamics, and fluid structure
interactions.  What follows is a description of using sus within the
realm of structural mechanics, fluid mechanics and structure-fluid
interactions.



%__________________________________
\subsection{Mechanics of Running sus}
 - Explain how to run on multiple processors\\
 - what do the different command line options mean\\
 - how to restart an uda\\

For single processor simulations, the sus excutable (Standalone Uintah Simulation) is run from the command line prompt like this:

\begin{Verbatim}
  
  sus input.ups

\end{Verbatim}

where the input.ups is an xml formatted input file.  The Uintah
software release contains numerous example input files located in the
src/StandAlone/inputs directory.

For multiprocessor runs, the general format is to use mpirun to launch
the jobs.  Depending on the environment, batch scheduler, launch
scripts, etc, mpirun mayy or may not be used.  However, the general
format would be something like the following:

\begin{Verbatim}

  mpirun -np num_processors sus -mpi input.ups

\end{Verbatim}

The num_processors is the number of processors that will be used.  The
input file must contain a patch layout that has at least the same
number (or greater) of patches as processors specified by a number
following the -np option shown above.

In addition, the -mpi is optional but often times necessary if the mpi
environment is not automatically detected from within the sus
executable.
 
%__________________________________
\subsection{Time Related Variables} \label{Sec:TimeRelatedVariables}
Uintah components are timestepping codes.  As such, one of the first entries
in each input file describes the timestepping parameters.  An input file
segment is given below that encompasses all of the possible parameters.
Most are self-explanatory, and not all are required,
(e.g \tt <max\_Timestep>,
<max\_delt\_increase>, <end\_on\_max\_time\_exactly> \normalfont and
\tt <delt\_init> \normalfont are all optional).
\tt <timestep\_multiplier> \normalfont serves
as a CFL number, that is, a number, usually less than 1.0, that is used to
moderate the timestep automatically calculated by the individual components. 

\begin{Verbatim}[fontsize=\footnotesize]
<Time>
    <maxTime>            1.0         </maxTime>
    <initTime>           0.0         </initTime>
    <delt_min>           0.0         </delt_min>
    <delt_max>           1.0         </delt_max>
    <delt_init>          1.0e-9      </delt_init>
    <max_delt_increase>  2.0         </max_delt_increase>
    <timestep_multiplier>1.0         </timestep_multiplier>
    <max_Timestep>       100         </max_Timestep>
    <end_on_max_time_exactly>true    </end_on_max_time_exactly>
</Time>

\end{Verbatim}
%
%__________________________________
\subsection{Data Archiver} \label{Sec:DataArchiver}
- variable labels\\
- checkpointing \\
- different options for specifying the output frequency\\
%
%__________________________________
\subsection{Geometry objects}
- different objects available and how to specify them\\
- what is res \\
- operators, union, difference\\
- example of combining several geom\_objects\\

%__________________________________
\subsection{Boundary conditions}
- describe how to have a jet in the floor of the domain.
%
%__________________________________
\subsection{Grid specification} \label{Sec:Grid}
Explain how a grid is specified and what these tags mean

\begin{Verbatim}[fontsize=\footnotesize]
<Level>
    <Box label="1">
       <lower>        [0,0,0]          </lower>
       <upper>        [5,5,5]          </upper>
       <extraCells>   [1,1,1]          </extraCells>
       <patches>      [1,1,1]          </patches>
    </Box>
    <spacing>         [0.5,0.5,0.5]    </spacing>
</Level>
 \end{Verbatim}
%
%__________________________________
\subsection{Adapative Mesh Refinement}
- need to discuss the input options for the different regridders.\\
- How is a cell flagged as needing to be refined
%
%__________________________________
\subsection{load Balancer}
- to be filled in

%__________________________________
\subsection{uda}
- discuss the directory structure of an uda
- discuss how to modify an input parameter before you restart an uda.

%__________________________________
\subsection{Visualization tools}
- VisIT\\
- manta?

%__________________________________
\subsection{Tools}
- puda, lineextract, timeextract, compare\_uda\\

%__________________________________
\subsubsection{plotting tools}
- plotStats\\
- plotRegridder \\
- plotCPU\_usage \\
- plotComponents

%__________________________________
\subsection{Code}
-explain the basic directory structure of src
\begin{Verbatim}[fontsize=\footnotesize]
|-- CCA
|-- Components
|   |-- Angio
|   |-- Arches
|   |-- DataArchiver
|   |-- Examples
|   |-- ICE
|   |-- LoadBalancers
|   |-- MPM
|   |-- MPMArches
|   |-- MPMICE
|   |-- Models
|   |-- OnTheFlyAnalysis
|   |-- Parent
|   |-- PatchCombiner
|   |-- ProblemSpecification
|   |-- Regridder
|   |-- Schedulers
|   |-- SimulationController
|   |-- Solvers
|   |-- SpatialOps
|   `-- SwitchingCriteria
`-- Ports
|-- Core
|-- R_Tester
|-- StandAlone
|-- Teem
|-- VisIt
|-- build_scripts
|-- include
|-- on-the-fly-libs
|-- orderAccuracy
|-- osx
|-- scripts
|-- tau
|-- testprograms
`-- tools
\end{Verbatim}
