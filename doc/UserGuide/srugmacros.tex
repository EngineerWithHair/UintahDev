% -*-latex-*-
% Document name: /u/srug/srugmacros.tex
% Creator: Rob MacLeod [macleod@cvrti.utah.edu]
% Last update: Tue Mar  6 14:33:05 2001 by Rob MacLeod
%    - created from things in the style file that latex2html could not
%      find.
% Creation Date: Wed Feb 21 23:50:38 2001
%%%%%%%%%%%%%%%%%%%%%%%%%%%%%%%%%%%%%%%%%%%%%%%%%%%%%%%%%%%%%%%%%%%%%%
% Markup and other useful text substitution commands.

\newcommand{\PSE}{\emph{BioPSE}}
\newcommand{\SR}{\emph{SCIRun}}
\newcommand{\eg}{{\em e.g.,}}
\newcommand{\ie}{{\em i.e.,}}
\newcommand{\etc}{{\em etc.,}}
\newcommand{\etal}{{\em et al.}}
\newcommand{\degrees}{{$^{\circ}$}}
\newcommand{\splitline}{\begin{center}\rule{\columnwidth}{.7mm}\end{center}}
\newcommand{\X}[1]{#1\index{#1}}
\newcommand{\version}{1.0}
\newcommand{\rob}{Rob MacLeod (macleod@cvrti.utah.edu)}
\newcommand{\ted}{Ted Dustman (dustman@cvrti.utah.edu)}
\newcommand{\srug}{BioPSE User's Guide}
\newcommand{\srdg}{BioPSE Developer's Guide}
\newcommand{\srig}{BioPSE Installation Guide}
\newcommand{\viewer}{\emph{Viewer}}


% Literal ~ character
\newcommand{\ltilde}{\symbol{'176}}

% Encloses its argument between angle brackets.
\newcommand{\ab}[1]{\latexhtml{$<$#1$>$}{<#1>}}

% Inserts a left angle bracket.
\newcommand{\la}{\latexhtml{$<$}{<}}

% Inserts a right angle bracket.
\newcommand{\ra}{\latexhtml{$>$}{>}}

% Markup an acronym
\newcommand{\acronym}[1]{\emph{#1}}

% Predefined acronyms
\newcommand{\gui}{\acronym{GUI}}
\newcommand{\pse}{\acronym{BioPSE}}
\newcommand{\Pse}{\pse}
\newcommand{\sci}{\acronym{SCI}}
\newcommand{\Sci}{\sci}
\newcommand{\sr}{\acronym{SCIRun}}
\newcommand{\Sr}{\sr}
\newcommand{\tcl}{\acronym{TCL}}

% Markup the first time use of term that may be unfamiliar to the
% reader. 
\newcommand{\dfn}[1]{\emph{#1}}
% In the next command #1 is the term and #2 is the shortcut or acronym that
% will be used in the rest of the document. 
\newcommand{\dfna}[2]{\emph{#1} (#2)}

% Directory name markup.
\newcommand{\directory}[1]{\texttt{#1}}

% Markup a file name.
\newcommand{\filename}[1]{\texttt{#1}}

% Markup text which is the name of a command.
\newcommand{\command}[1]{\texttt{#1}} 

% Markup text typed at the keyboard.
\newcommand{\keyboard}[1]{\texttt{#1}} 

\newcommand{\ptext}[1]{\textit{\ab{#1}}}

% Markup text the user might see on his screen.
\newcommand{\screen}[1]{\texttt{#1}}

% Markup the name of a menu.
\newcommand{\menu}[1]{\textbf{#1}}

% Markup a menu item name.
\newcommand{\menuitem}[1]{\textbf{#1}}

% Markup name of a ui button.
\newcommand{\button}[1]{\textbf{#1}}

% Markup an inline code fragment.
\newcommand{\icode}[1]{\texttt{#1}}

% Markup a module name.
\newcommand{\module}[1]{\texttt{#1}}

% Markup a sr package name.
\newcommand{\package}[1]{\texttt{#1}}

% Markup a sr category name.
\newcommand{\category}[1]{\texttt{#1}}

% Sci url
\newcommand{\sciurl}{http://www.sci.utah.edu}

% Call attention to some useful bit of information.
\newcommand{\note}[1]{\emph{Note: #1}} 

% Call attention to a warning.
\newcommand{\warning}[1]{\emph{Warning: #1}} 

% Call attention to a tip.
\newcommand{\tip}[1]{\emph{Tip: #1}} 

% Env variable markup.
\newcommand{\envvar}[1]{\texttt{#1}}

% Markup for the title of a book or article or whatever.
\newcommand{\etitle}[1]{\texttt{#1}}

% Section command.  This command should be
% used in place to reference sections (subsections etc.).
% The command's first argument is the link text (used only on the html
% page) and the second argument is the section's label
\newcommand{\secref}[2]{\hyperref[ref]{\emph{#1}}{Section~}{}{#2}}

% Use these in place of missing content.
\newcommand{\missing}[1]{\emph{#1 - Coming Soon.}}

% Use this to make note of incomplete content.
\newcommand{\incomplete}{\emph{More Comming Soon.}}

%%% Local Variables: 
%%% mode: latex
%%% TeX-master: "usersguide"
%%% End: 
