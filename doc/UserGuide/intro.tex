% intro.tex
%

\section{Introduction}
\label{sec:intro}

This is the \etitle{\srug}.  It describes the purpose and use of the \pse{}
problem solving environment.  It is for those users who will be building
and executing \dfn{networks} within the \pse{} environment.

Those who will be installing \pse{} should read the \etitle{\srig}.

Those wishing to develop modules should read the \etitle{\srdg}.


\subsection{Conventions}
\label{sec:conventions}

\missing{Discussion of typographic conventions}

\subsection{Roadmap}
\label{sec:roadmap}

This document is organized into the following main sections:

\begin{description}
  \item \secref{Introduction}{sec:intro} You are reading this section now.
  \item \secref{Concepts}{sec:concepts} Introduces the concept of an
        integrated problem solving environment generally and describes how
        \PSE{} embodies some of these ideas.
  \item \secref{Starting \sr}{sec:startingup}  Procedure for starting \pse{}
        and related information.
  \item \secref{Working with Networks}{sec:workwithnets} Discusses tasks
        involved in building, editing, and executing networks.
  \item \secref{Visualization with the \viewer{}}{sec:viewer} Describes the
        purpose and use of a visualization module called the \viewer{}.
        The \viewer{} is probably \pse{}'s most commonly used module.
  \item \secref{Packages}{sec:packages}  Describes the content of the \sr{}
        and \pse{} packages (which provide nearly all of \pse{}'s
        computational services).
\end{description}

\subsection{Getting Help}
\label{sec:help}
\missing{Help resources}

\subsubsection{In the Distribution}
\missing{}

\subsubsection{On the Web}
\missing{}

\subsubsection{From the Mailing Lists}
\missing{}

\subsection{Reporting Bugs}
\label{sec:bugs}
\missing{}



%%% Local Variables: 
%%% mode: latex
%%% TeX-master: "usersguide"
%%% End: 
