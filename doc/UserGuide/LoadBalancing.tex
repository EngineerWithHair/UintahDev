\section{Dynamic Load Balancing} \label{loadbalancer}

Uintah has a couple of load balancing options which may be useful for increasing performance by decreasing
the load imbalance.  The following describes the loadbalancer section of an input file and what effects
it has on the load balancer.  

If no load balancer is specified then a simple load balancing method which assigns an equal number of patches
to processors. This is not ideal in most cases and should be avoided.

\subsection{Input File Specs}
\begin{Verbatim}[fontsize=\footnotesize]
   <LoadBalancer type="DLB"> 
        <!-- DLB specific flags -->
        <costAlgorithm>ModelLS</costAlgorithm>
        <hasParticles>true</hasParticles>

        <!-- DLB/PLB flags -->
        <timestepInterval>25</timestepInterval>
        <gainThreshold>0.15</gainThreshold>
        <outputNthProc>1</outputNthProc>
        <doSpaceCurve>true</doSpaceCurve>

   </LoadBalancer>
\end{Verbatim}

There are two main load balancers used in Uintah.  The first is the DLB load balancer.
This is a robust load balancer that is good for many problems.  In addition,
this load balancer can utilize profiling in order to tune itself during the runtime
in order to achieve better results.  

To use this load balancer the user must specify the type as "DLB".  It is also suggested
that the user specify a costAlgorithm which can be "Model", "ModelLS", "Memory", or
"Kalman" with the default being "ModelLS".  If "hasParticles" is set to true then
these cost algorithms will take the number of particles into account when determining
the cost.

This algorithm first orders the patches linearly.  If doSpaceCurve is set to true
then this ordering is done according to a Hilbert Space-Filling curve, which will
likely provide better clusterings.  Once the patches are ordered linearly, costs
are assigned to each patch and the patches are distributed onto processors so that
the costs on each processor are even.  

The PLB load balancer is an alterantive to the DLB load balancer which is
likely more efficent for particle based calculations.  This load balancer
divides the patches into two sets (cell dominate and particle domintate), 
which is determined using the particleCost and cellCost parameters.
The particle dominate patches are then assigned to processors while trying
to equalize the number of particles on each processor.  Finally the 
cell dominate patches are assigned to patches in order to equalize the number
of cells while accounting for the number of cells already assigned during 
the particle assignment phase.  This method can also utilize a space-filling
curve.  

The following list describes other flags utilized by these load balancers:
\begin{itemize}
  \item timestepInterval - how many timesteps must pass before reevaluating the load balance.  
  \item gainThreshold - the predicted percent improvement that is required to reload balance.  
  \item outputNthProc - output data on only every Nth processor (experimental). 
\end{itemize}

