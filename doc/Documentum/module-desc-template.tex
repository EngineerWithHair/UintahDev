% module-template.tex --
%
% Notes:
%
% o Rename this file appropriately.
%
% o Complete relevant sections below.
%
% o Use markup commands defined in scirun-doc.sty.
%   Avoid presentation markup (e.g. \textbf or \emph).
%
% o To make your document web-friendly, use commands from html package. See
%   latex2html manual or the book "The Latex Web Companion."
%
% o Don't add new subsections, but freely subsubsection 
%   existing subsections.
%
% o Add local commands if desired (via \newcommand).  But first check for
%   equivalent markup in scirun-doc.sty.
%
% o Only use packages distributed with Latex2e.
%
% o Cite references with the \cite command. Put citations in
%   a bib file and send along with this file.
%
% o Please include at least one image of tk GUI. Provide 
%   an EPS version and a web friendly version (e.g., jpeg or gif).
%
% o Including images:  Create a figure command for each figure using the
%   template below.  Copy the template (make a copy of the template
%   for each figure) and uncomment only the first '%' in each line.
%   Then replace bracketed text with appropriate values.  Here is the
%   template: 
% 
%%begin{latexonly}
%  \newcommand{\<figure command name>}%
%  {\centerline{\includegraphics[options]{<path to eps file>}}}
%%end{latexonly}
%\begin{htmlonly}
%  \newcommand{\<figure command name (same as above)>}{%
%  \htmladdimg[options]{<path to jpg file>}}
%\end{htmlonly}
%
%   Where in the ``latexonly'' part, ``options'' is a list of key-value pair
%   options.  Typically the options specify graphic width, height, and
%   bounding box.  See ``The Latex Graphics Companion'' for more
%   information.  Note that we are using ``\includegraphics'' command
%   from the ``graphicx'' package.
%
%   In the ``htmlonly'' part ``options'' is a list options such as ``align'',
%   ``width'', ``height'', and ``alt''.  See ``The Latex Web Companion'' for
%   details.
%   
%   Now to actually create figures enclose each figure command as follows:
%
%\begin{figure}
%  \begin{makeimage}
%  \end{makeimage}
%  \<figurecommand>
%  \caption{\label{fig:<figure label>} <Caption text>}
%\end{figure}
%
%   and replace bracket text appropriately.
%   

% Module information: replace the XXXX's

\ModuleRef{\Module{XXXX}}{\Category{XXXX}}{\Package{XXXX}}

% Summary of module's purpose. No more than one paragraph.

\subsection{Summary}

Summary here.

% How to use the module.  Describe connections upstream and downstream
% and how to use the GUI.  If desired, Walk the user through an example.
% Please include at least one image of tk GUI. Provide 
% an EPS version and a web-friendly version (e.g., jpeg or gif).
% See the note above that explains how to include images.

\subsection{Use}

Describe use here.

% Description of module's algorithms, mathematics, etc.

\subsection{Detail}

Details here.

% Special notes.  These may be installation instructions, bugs, limitations,
% etc.

\subsection{Notes}

Write notes here.

% Who is responsible? 

\subsection{Credits}

Who wrote this.
