% -*-latex-*-
%
%  The contents of this file are subject to the University of Utah Public
%  License (the "License"); you may not use this file except in compliance
%  with the License.
%
%  Software distributed under the License is distributed on an "AS IS"
%  basis, WITHOUT WARRANTY OF ANY KIND, either express or implied. See the
%  License for the specific language governing rights and limitations under
%  the License.
%
%  The Original Source Code is SCIRun, released March 12, 2001.
%
%  The Original Source Code was developed by the University of Utah.
%  Portions created by UNIVERSITY are Copyright (C) 2001, 1994
%  University of Utah. All Rights Reserved.
%
% Document name: package.tex
%

\chapter{Packages}
\label{ch:packages}
\index{packages}

Packages are collections of modules organized by category. The \SR{} and 
packages are discussed below. 

\section{The \sr{} Package}
\label{sec:srpackage}
\index{SCIRun@\sr{}!package}


The \sr{} package contains these categories:

\begin{description}
  \descitem{\category{DataIO}} Modules for reading and writing \sr{}
  data types.  Use these to read \sr{} type data into a network and
  write data to disk from a network.  These modules only know about
  \sr{} type data.  Non-\sr{} (foreign) data may be converted to \sr{}
  type data using \sr{}'s converter programs (see \chref{Importing
    data into \sr{}}{ch:import_export} for more information).
  
  \descitem{\category{Fields}} Modules for creating, manipulating, and
  processing field data.
  
  \descitem{\category{Math}} Modules providing matrix operations and
  equation solving capabilities.
  
  \descitem{\category{Render}} Modules for visualizing data produced by
  other modules.  The main visualization tool at present
  is the Viewer, which is described in more detail in
  \chref{Visualization with the \viewer{}}{ch:viewer}.
\end{description}

\section{The BioPSE Package}
\label{sec:biopsepackage}
\index{SCIRun@\BIOPSE{}!package}


These modules provide  tools tailored to solve bioelectric field
problems.

\begin{description}
  \descitem{\category{DataIO}} Modules for reading and writing \sr{}
  data types.
  
  \descitem{\category{Forward Problems}} Modules for solving forward
  problems in the bioelectric field domain.
  
  \descitem{\category{Inverse Problems}} Modules for solving inverse
  problems.
  
  \descitem{\category{LeadField Calculations}} Modules that support
  the construction and analysis of fields derived from experimental or
  computed data gathered or computed at electrode sensor locations.
  
  \descitem{\category{Modeling}} Modules that support the
  construction of surface and volume bioelectric field models.
  
  \descitem{\category{Visualization}} Modules for visualizing
  bioelectric field data.
\end{description}

%%% Local Variables: 
%%% mode: latex
%%% TeX-master: "usersguide"
%%% End: 
