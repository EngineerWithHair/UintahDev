% -*-latex-*-
%
%  For more information, please see: http://software.sci.utah.edu
% 
%  The MIT License
% 
%  Copyright (c) 2004 Scientific Computing and Imaging Institute,
%  University of Utah.
% 
%  License for the specific language governing rights and limitations under
%  Permission is hereby granted, free of charge, to any person obtaining a
%  copy of this software and associated documentation files (the "Software"),
%  to deal in the Software without restriction, including without limitation
%  the rights to use, copy, modify, merge, publish, distribute, sublicense,
%  and/or sell copies of the Software, and to permit persons to whom the
%  Software is furnished to do so, subject to the following conditions:
% 
%  The above copyright notice and this permission notice shall be included
%  in all copies or substantial portions of the Software.
% 
%  THE SOFTWARE IS PROVIDED "AS IS", WITHOUT WARRANTY OF ANY KIND, EXPRESS
%  OR IMPLIED, INCLUDING BUT NOT LIMITED TO THE WARRANTIES OF MERCHANTABILITY,
%  FITNESS FOR A PARTICULAR PURPOSE AND NONINFRINGEMENT. IN NO EVENT SHALL
%  THE AUTHORS OR COPYRIGHT HOLDERS BE LIABLE FOR ANY CLAIM, DAMAGES OR OTHER
%  LIABILITY, WHETHER IN AN ACTION OF CONTRACT, TORT OR OTHERWISE, ARISING
%  FROM, OUT OF OR IN CONNECTION WITH THE SOFTWARE OR THE USE OR OTHER
%  DEALINGS IN THE SOFTWARE.
%

% Document name: package.tex
%

\chapter{Packages}
\label{ch:packages}
\index{packages}

Packages are collections of modules organized by category. \sr{}'s 
packages and their categories are listed below. 

\section{\sr{}}
\label{sec:srpackage}
\index{SCIRun@\sr{}!package}

The \sr{} package provides a base set of datatypes and algorithms.  It
is divided into the following categories:

\begin{itemize}
  \item DataIO
  \item FieldsCreate
  \item FieldsData
  \item FieldsGeometry
  \item FieldsOther
  \item Math
  \item Render
  \item Visualization
\end{itemize}

\section{BioPSE}
\label{sec:biopsepackage}
\index{SCIRun@\BIOPSE{}!package}


The BioPSE package provides modules to solve bioelectric field
problems.  It is divided into the following categories:

\begin{itemize}
\item DataIO
\item Forward
\item Inverse
\item LeadField
\item Modeling
\item Visualization
\end{itemize}

\section{Teem}
\label{sec:teempackage}

\scidoclink{Modules}{Developer/Modules/Teem\_bycat.html} of the Teem
package provide functionality of the \htmladdnormallinkfoot{Teem}
{http://teem.sourceforge.net/} project's \command{unu} and
\command{tend} command-line programs and corresponding NRRD and Ten
libraries.

The Teem project's NRRD library provides data types and functions for
representing and processing N-dimensional raster data.  The Ten
library provides types and functions for diffusion tensor processing,
analysis, and visualization.

Typically, multiple instances of commands \command{unu} and
\command{tend} are connected by pipes to solve a problem---the
data-flow network equivalent is available in \sr{} using the Teem
package.  In other words, the Teem package provides a
``visual'' programming equivalent to the \command{unu} and
\command{tend} command-line tools.

The Teem package is divided into the following categories:

\begin{itemize}
\item DataIO Modules read/write NRRD data from/to files and
  modules that transform data to/from the NRRD representation.
\item NrrdData Modules for displaying and setting NRRD property values
  and for performing miscellaneous operations on NRRD data.
\item Tend Modules equivalent to commands of the \command{tend}
  program.  For example, module \module{TendAnhist} implements the
  command \command{tend anhist}
\item Segmentation  algorithms operating on NRRD data.
\item Unu Modules equivalent to commands of the \command{unu} program.
  For example, module \module{UnuMinMax} implements the command
  \command{unu minmax}
\end{itemize}

See \scidoclink{Teem module
  documentation}{Developer/Modules/Teem\_bycat.html} for more
information.

\section{Matlabinterface}
\label{sec:matlabpackage}

Modules of the Matlabinterface package allow \sr{} to use the
facilities of Matlab via interactive commands and Matlab scripts.

Matlabinterface modules can be used simply, for example, to convert
data files between \sr{} and Matlab formats. Or, Matlabinterface
modules can be part of \sr{} networks that invoke Matlab scripts to
perform complex calculations.

See Matlabinterface \scidoclink{module
  documentation}{Developer/Modules/MatlabInterface.html} for more
information.

\section{Insight (ITK)}
\label{sec:insightpackage}

The \scidoclink{Insight}{Developer/Modules/Insight\_bycat.html}
package provides modules encapsulating \htmladdnormallinkfoot{ITK
  toolkit}{http://www.itk.org} filters.

\begin{itemize}
\item Converters
\item DataIO
\item Filters
\end{itemize}

\section{DataIO}
\label{sec:dataiopackage}

Package \scidoclink{DataIO}{Developer/Modules/DataIO\_bycat.html}
provides modules for reading data stored in non-SCIRun formats.
Formats HDF5 and MDSplus are currently supported. The modules convert
both geometry and data into NRRDS. The Teem package must also be
installed.

%%% Local Variables: 
%%% mode: latex
%%% TeX-master: "usersguide"
%%% End: 
