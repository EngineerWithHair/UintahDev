% -*-latex-*-
% Document name: package.tex
%
%%%%%%%%%%%%%%%%%%%%%%%%%%%%%%%%%%%%%%%%%%%%%%%%%%%%%%%%%%%%%%%%%%%%%%
\section{Packages}
\label{sec:packages}
\index{packages}

Packages are what provide the real utility of \SR{} from most users'
perspectives.  The are sets of modules that use the basic \SR{}
functionality to perform specific tasks such a solving bioelectric field
problems.  To remind yourself about the way packages fit into the picture,
see \secref{SCIRun versus BioPSE}{sec:srversuspse}.

\subsection{The \sr{} Package}
\label{sec:srpackage}
\index{SCIRun@\sr{}!package}

The \sr{} package contains these categories:

\begin{description}

\item[\category{DataIO}] Contains modules for reading and writing \sr{}
  type data.  Use these to read \sr{} type data into a network and write
  data to disk from a network.  These only know about \sr{} type data.
  Non-\sr{} (foreign) data may be converted to \sr{} type data using
  \sr{}'s converter programs.  See \secref{Importing data into
  \sr{}}{sec:import} for more information.

\item[\category{Fields}] \missing{}
\item[\category{Math}] \missing{}
\item[\category{Render}] \missing{}
\item[\category{Visualization}] the main visualization tool at present is
  the Viewer, which is described in more detail in \secref{Visualization
  with the \viewer{}}{sec:viewer}.
\end{description}

\subsection{The BioPSE Package}
\label{sec:biopsepackage}
\index{SCIRun@\PSE{}!package}


These modules provide you with tools tailored to solving bioelectric field
problems, which is the theme of the \htmladdnormallink{NCRR
center}{http://www.sci.utah.edu/ncrr} we have developed.


\begin{description}
\item[\category{Forward Problems}] \missing{}
\item[\category{Inverse Problems}] \missing{}
\item[\category{LeadField Calculations}] \missing{}
%\item[\category{Visualization}] \missing{}
\end{description}

%%% Local Variables: 
%%% mode: latex
%%% TeX-master: "usersguide"
%%% End: 
