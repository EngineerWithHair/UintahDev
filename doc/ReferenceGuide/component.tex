%
% modulespec.tex
%
% Documentation for module-spec.xml
%
% * tjd *
%

\documentclass{article}
\usepackage{html}
\parindent 0in
\setlength{\parskip}{\medskipamount}
\newcommand{\mytilde}{\symbol{'176}}
\newcommand{\ab}[1]{\latexhtml{$<$#1$>$}{<#1>}}
\newcommand{\la}{\latexhtml{$<$}{<}}
\newcommand{\ra}{\latexhtml{$>$}{>}}
\newcommand{\acronym}[1]{#1}
\newcommand{\dfn}[1]{\emph{#1}}
\newcommand{\filename}[1]{\texttt{#1}}
\newcommand{\xml}{\acronym{XML}}
\newcommand{\Xml}{\xml}
\newcommand{\psgml}{\acronym{PSGML}}
\newcommand{\Psgml}{\psgml}
\newcommand{\dtd}{\acronym{DTD}}
\newcommand{\gui}{\acronym{GUI}}
\newcommand{\emacs}{[x]emacs}
\newcommand{\Emacs}{[X]Emacs}
\newcommand{\sectitleref}[1]{\emph{#1}}
\newcommand{\element}[1]{\ab{\texttt{#1}}}
\newcommand{\elementitem}[1]{\item[\textit{\ab{#1}}]}
\newcommand{\keyboard}[1]{\texttt{#1}}
\newcommand{\screen}[1]{\texttt{#1}}
\newcommand{\sciurl}{http://www.cvrti.utah.edu}
\newcommand{\psgmlurl}{http://www.lysator.liu.se/projects/about\_psgml.html}

% Section title commands (What a pain!)
\newcommand{\SECintro}{Introduction}
\newcommand{\SUBSECwhySpec}{Why a Module Specification?}
\newcommand{\SUBSECwhyXml}{Why \xml{?}}
\newcommand{\SUBSECwhatDtd}{What's a \dtd?}
\newcommand{\SECcontent}{Structure and Logical Content}
\newcommand{\SUBSECpreamble}{Preamble}
\newcommand{\SUBSECstructContent}{Structure and Content}
\newcommand{\SUBSECdescElement}{A Description of the \ab{description} Element}
\newcommand{\SUBSUBSECcommonUsage}{Common Usage}
\newcommand{\SUBSUBSECdescElementIntro}{Introduction}
\newcommand{\SUBSUBSECreference}{Reference}
\newcommand{\SECexample}{Example}
\newcommand{\SECediting}{Editing the Spec}
\newcommand{\SUBSECgettingSources}{PSGML Mode - Do You Have It?}
\newcommand{\SUBSECdotEmacs}{PSGML Mode - .emacs}
\newcommand{\SUBSECgettingStarted}{PSGML Mode - Getting Started}
\newcommand{\SUBSECvalidation}{Validation}
\newcommand{\SECtools}{Tools}
\newcommand{\SUBSECmakeDevHtmlDoc}{Making a Developer Document (HTML)}
\newcommand{\SUBSECmakeUserHtmlDoc}{Making a User Document (HTML)}
% End of section title commands.

\title{Module \xml Spec Guide\\(module-spec.xml)} 
\author{\htmladdnormallinkfoot{Ted Dustman}{mailto:dustman@cvrti.utah.edu}}

\begin{document}
\maketitle \pagenumbering{roman}
\label{toc}
\tableofcontents
\pagebreak \pagenumbering{arabic}

\section{\htmlref{\SECintro}{toc}}
\label{\SECintro}

\subsection{\htmlref{\SUBSECwhySpec}{toc}}
\label{\SUBSECwhySpec}

The module specification serves as documentation and data.  

It documents the use, design, implementation, and testing of a module.
Various forms of content (e.g. HTML) will be automatically generated from
the module specification.

It serves as data for applications and databases.  For example:

\begin{itemize}
\item The collection of all module specifications will be part of a
  searchable database.
  
\item The module specification will help drive the module code generation
  process.

\item The module specification may drive an automated testing process.
\end{itemize}

Module specifications are written in a special markup language.  This
language (call it the module specification markup language if you
like) is in turn formulated using the \dfn{eXtensible Markup Language}
(\acronym{XML}).

\subsection{\SUBSECwhyXml}
\label{\SUBSECwhyXml}

An \xml\ formulation of the module specification is used to achieve
the following goals:

\begin{description}
\item[Data centric viewpoint] We want the module specification to represent
  a set of data that can be used in several contexts.

\item[Uniformity] We want all modules to be specified the same way.

\item[Validation] We want to validate (check for correctness and
  completeness) module specifications.
\end{description}

The module specification markup language is similar in concept to HTML but
has a different (and much smaller) set of tags and a simpler and more
regular set of document composition rules.
Section~\hyperref{\sectitleref{\SECcontent}}{}{}{\SECcontent} describes
the module specification markup language in detail.

Note that in the \xml\ world tags are used to delimit (or identify)
\dfn{elements}.  Every start tag \emph{must} have a corresponding end tag.
Elements are the basic building blocks of an \xml\ document.

\subsection{\SUBSECwhatDtd}
\label{\SUBSECwhatDtd}

A \dfn{Document Type Definition} (\dtd) defines a set of elements (and
element attributes), and the way they must be used in order to produce a
valid \xml document.

The module specification's \dtd\ is contained in the file
\filename{modulespec.dtd}. 

The module specification \dtd\ can be used to validate module
specifications.
Section~\hyperref{\sectitleref{\SUBSECvalidation}}{}{}{\SUBSECvalidation}
explains how you can validate your module specification using the module
specification \dtd.

The module specification \dtd\ can help you write a valid module
specification when used with a \acronym{DTD} aware editor.  \Emacs\ is one
such editor.  See
section~\hyperref{\sectitleref{\SECediting}}{}{}{\SECediting} for information
on using \emacs\ to write module specifications.

\section{\SECcontent}
\label{\SECcontent}


\subsection{\SUBSECpreamble}
\label{\SUBSECpreamble}

A module specification document starts with 2 lines of preamble text:

\begin{verbatim}
<?xml version=''1.0'' encoding=''UTF-8'' ?>
<!DOCTYPE module-spec SYSTEM "modulespec.dtd">
\end{verbatim}

A set of nested elements follows starting with the \element{modulespec}
element.  

Every element (with a few exceptions) must be delimited with a
start-tag and an end-tag.  

Tags are case sensitive.  All tags must be written in lower case.

The set of valid elements are described next.

\subsection{\SUBSECstructContent}
\label{\SUBSECstructContent}

Below, the module specification's \xml\ structure and corresponding logical
content are described.  Each element's start tag is followed by a
description of its purpose and content, followed by its nested elements,
and finally by its end tag.  Indentation is used to show nesting
relationships among elements.

\begin{description}
  \elementitem{modulespec} Starts the module specification.  All other
  elements are nested within this one.

  \begin{description}
    \elementitem{overview} Starts the overview section.  Contains only
    nested elements.

    \begin{description}
      \elementitem{module-name} The module's name.
      \elementitem{/module-name}

      \elementitem{summary} Short (1 or 2 sentance) summary of module's function.
      \elementitem{/summary}
      
      \elementitem{description} Comprehensive description of module's
      function.  See
      section~\hyperref{\sectitleref{\SUBSECdescElement}}{}{}{\SUBSECdescElement}
      for details on the use of the \element{description} element.
      \elementitem{/description}

      \elementitem{authors}  List of module's authors.  Contains 1 or more
      \element{author} elements.

      \begin{description}
        \elementitem{author} Name of 1 author.
        \elementitem{/author}
      \end{description}

      \elementitem{/authors}

      \elementitem{examplesr} The name of an \filename{.sr} file that
      demonstrates the module's use.
      \elementitem{/examplesr}
    \end{description}

    \elementitem{/overview} 
    
    \elementitem{io} Module's inputs and outputs.  Contains 2 elements,
    \element{inputs} followed by \element{outputs}.

    \begin{description}
      \elementitem{inputs} Describes the modules's inputs (the
      \element{noinputs/} element is used in place of the \element{inputs}
      element if the module accepts no inputs).
      Inputs may come from a port or a file.  Zero or more \element{port}
      elements are used to describe inputs from ports and 0 or more
      \element{file} elements are used to describe inputs from files.
      There must be at least 1 \element{port} element or 1 \element{file}
      element.

      \begin{description}
        \elementitem{port} States that input data comes from a port. 
        Subelements describe the data further.
        
        \begin{description}
          \elementitem{description} High level description of the data 
          accepted by the port, e.g. ``These are potential data from body 
          surface of a human.''
          \elementitem{/description}
          
          \elementitem{datatype} The name of a PSE datatype, i.e. the 
          datatype that is being transmitted on this port.
          \elementitem{/datatype}
          
          \elementitem{modulename} The name of an upstream module
          that is commonly used to send data to this module on this
          port.  This element may occur multiple times. 
          \elementitem{/modulename}
        \end{description}
        
        \elementitem{/port}
        
        \elementitem{file} States that input comes from a file.
        Subelements describe the data further.
        
        \begin{description}
          \elementitem{description} High level description of the data 
          provided by the file, e.g. ``These are potential data from body 
          surface of a human.''
          \elementitem{/description}
          
          \elementitem{datatype} The name of a PSE datatype, i.e. the 
          datatype being read from the file.  Use the
          \element{nodatatype} element if a PSE datatype is not being
          read from the file.
          \elementitem{/datatype}
          
          \elementitem{nodatatype} A description of the content and format of
          the file if the file does not contain a PSE datatype.  Use this
          element as you would the \element{description} element.  The
          \element{datatype} element should be used if a PSE datatype is being
          read from the file.  \elementitem{/nodatatype}

        \end{description}
        
        \elementitem{/file}
      \end{description}

      \elementitem{/inputs}
      
      \elementitem{outputs} Describes the module's outputs (the
      \element{nooutputs/} element is used in place of the \element{outputs}
      element if the module accepts no inputs).  Outputs may go to
      a port or to a file.  Use 0 or more \element{port} elements to
      describe outputs to ports and 0 or more \element{file} elements to
      describe outputs to files.

      \begin{description}
        \elementitem{port} States that output is sent a port. Subelements describe the data further.
        
        \begin{description}
          \elementitem{description} A high level description of the data 
          sent to the port, e.g. ``These are potential data from body 
          surface of a human.''
          \elementitem{/description}
          
          \elementitem{datatype} The name of a PSE datatype, i.e. the 
          datatype that is being transmitted on this port.
          \elementitem{/datatype}
          
          \elementitem{modulename} The name of a downstream module
          that is commonly connected to this output port.  This
          element may occur multiple times. 
          \elementitem{/modulename}
        \end{description}
        
        \elementitem{/port}

        \elementitem{file}  States that output is sent to a file.  It's 
        subelements give a description of the data and datatype information.
        
        \begin{description}
          \elementitem{description} A high level description of the data 
          sent to the file, e.g. ``These are potential data from body 
          surface of a human.''
          \elementitem{/description}
          
          \elementitem{datatype}  The name of a PSE datatype, i.e. the 
          datatype being written to the file.  Use the
          \element{nodatatype} element if a PSE datatype is not being
          written to the file.
          \elementitem{/datatype}
          
          \elementitem{nodatatype} A description of content and format of the
          file if the file does not contain a PSE datatype, Use this element as
          you would the \element{description} element.  The \element{datatype}
          element should be used if a PSE datatype is being written to the file.
          \elementitem{/nodatatype}

        \end{description}
        
        \elementitem{/file}
      \end{description}
      
      \elementitem{/outputs}
    \end{description}

    \elementitem{/io}

    \elementitem{gui} States that the module supports a \gui\  (the
    \element{nogui} element is used in place of the \element{gui} if the
    module does not support a \gui).  Subelements describe the \gui.
    
    \begin{description}
      \elementitem{description} Describes the purpose of the \gui, e.g. ``The
      \gui\  allows you to steer the simulation.''
      \elementitem{/description}

      \elementitem{parameter} Declares 1 \gui\  parameter.  A parameter is a
      \gui\  item that can be modified by the user.  Subelements describe the
      parameter.  Use 1 parameter element for each item in the \gui.
      
      \begin{description}
        \elementitem{label}Name of parameter as it appears in \gui.
        \elementitem{/label}
        \elementitem{description}Describes the purpose of the parameter and
        how it can be used to control the module's behavior.
        \elementitem{/description}
      \end{description}
      \elementitem{/parameter}
      \elementitem{img}Name of a file that contains a picture of the \gui.
      \elementitem{/img}
    \end{description}

    \elementitem{/gui}%

    \elementitem{testing} Starts the testing section.  Subelements specify
    1 or more testing plans.

    \begin{description}
      \elementitem{plan} Declares 1 testing plan.  Subelements specify the
      testing plan's details.
      \begin{description}
        \elementitem{description}  \emph{Need your help here Marty}
        \elementitem{/description}
        \elementitem{steps}\emph{Need your help here Marty}

        \begin{description}
          \elementitem{step}\emph{Need your help here Marty}
          \elementitem{/step}
        \end{description}

        \elementitem{/steps}
      \end{description}

    \elementitem{/plan}

    \end{description}

    \elementitem{/testing}
  \end{description}

  \elementitem{/module-spec}
\end{description}

\subsection{\SUBSECdescElement}
\label{\SUBSECdescElement}


\subsubsection{\SUBSUBSECdescElementIntro}
\label{\SUBSUBSECdescElementIntro}


The \element{description} element contains text for human consumption.
It allows you to organize your text into paragraphs, lists, and
other structured content.

Note that the literal form of the characters \la, \ra, and \& may not be
used (this is an \xml\ thing).  Instead you must type \keyboard{\&lt;},
\keyboard{\&gt;}, and \keyboard{\&amp;} respectively.

\subsubsection{\SUBSUBSECcommonUsage}
\label{\SUBSUBSECcommonUsage}

A \element{description} might contain only 1 or 2 sentences:

\begin{verbatim}
<description><p>My module is great.  It's so easy to use that no further
information is needed.</p></description>
\end{verbatim}

Note that even just 1 sentence must be enclosed in a paragraph.

A \element{description} may contain only a few paragraphs:

\begin{verbatim}
<description>
  <p>My module is great.  It's so easy to use that no further
     information is needed.</p>

  <p>I lied.  What follows is a description of my module.</p>
   .
   .
   .
</description>
\end{verbatim}

You may refer to other material using the \element{cite}, \element{rlink},
or \element{slink} elements:

\begin{verbatim}
<description>
My module does xyz.  It uses the xyz algorithm described in the paper
<cite>The XYZ algorithm</cite> by Xavier Yilmez Zideco.  A less formal
discussion of the xyz algorithm is
 <slink path=ncrr/publications/xyz.html>available online</slink>. 
</description>
\end{verbatim}

The \element{cite} element simply marks text as the title of a paper or
book. 

The \element{slink} and \element{rlink} elements allow you to reference
online material.  

\element{Slink} links to material on the \acronym{SCI}
web site (\sciurl).  It requires a path 
relative to \sciurl.  The path in the previous example resolves to
\sciurl{}/ncrr/publications/xyz.html.

\element{Rlink} links to material in the \acronym{CVS} tree
(\element{rlink} actually links to material on the documentation web site
but it is more easily explained as a path into the cvs tree).  It requires
a path relative to the location of your module specification document.  For
example, the path \keyboard{./stuff.html} resolves to the same directory
where your module specification document lives.

A more general link element is not available because the documentation
bureaucrats want some control over linked to material.

\subsubsection{\SUBSUBSECreference}
\label{\SUBSUBSECreference}

\section{\SECexample}
\label{\SECexample}
\begin{verbatim}
\end{verbatim}


\section{\SECediting}
\label{\SECediting}

An ordinary text editor may be used to create the
module specification document's content.  However, it is easy to get lost
in the noise of the \xml\ syntax.  Therefore, it is best to use
an \xml{}/\dtd\ aware editor.  This type of editor will help you construct
a valid module specification document.

\Emacs\ is one such editor.  It supports an editing environment called
\xml\ mode (which is really a derivative of \psgml\ mode).  \Xml\ mode
highlights \xml\ syntax, indents nested elements and their content, and
automatically inserts elements based on the position of the insertion point.
It is still possible to create invalid documents using \emacs\ 
\xml\ mode though.

The following sections describe the use of \emacs\ \xml\ mode.

\subsection{\SUBSECgettingSources}
\label{\SUBSECgettingSources}

Recent versions of \emacs\ come with \xml\ mode installed.  You can check
if your does by typing \keyboard{M-x xml-mode}.  You \emph{don't} have
\xml\ mode if you get the message
\screen{[No match]} in return.

UUCS people may use the version of \psgml\ installed under
\keyboard{/usr/local/contrib/mcole/psgml}.  You need to add the
following bit of lisp code to your \filename{.emacs} file:

\begin{verbatim}
(setq load-path (append '("/usr/local/contrib/mcole/psgml") load-path))
(setq auto-mode-alist  
   (append '(("\\.xml\$"  . xml-mode)) auto-mode-alist))   
\end{verbatim}

Non-UUCS folks who don't have \xml\ mode may get it from \psgmlurl.

\subsection{\SUBSECdotEmacs}
\label{\SUBSECdotEmacs}



\subsection{\SUBSECgettingStarted}
\label{\SUBSECgettingStarted}

\subsection{\SUBSECvalidation}
\label{\SUBSECvalidation}

http://www.stg.brown.edu/service/xmlvalid/

\section{\SECtools}
\label{\SECtools}


\subsection{\SUBSECmakeDevHtmlDoc}
\label{\SUBSECmakeDevHtmlDoc}


\subsection{\SUBSECmakeUserHtmlDoc}
\label{\SUBSECmakeUserHtmlDoc}


\end{document}
