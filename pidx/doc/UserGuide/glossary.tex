\chapter{Glossary} \label{Sec:Glossary}

\begin{itemize}

\item Data Warehouse (NewDW, OldDW, DW) - The \tt Data Warehouse
         \normalfont is an abstraction (and implementation vehicle) used in
         Uintah to provide data to simulation components (across
         distributed memory spaces as necessary).  OldDW refers to a
         DW from the previous time step.  NewDW refers to the DW for
         the current time step.  In practice, variables are usually
         pulled from the OldDW, updated, and placed in the NewDW.

\item Time step - Uintah is a time dependent code.  A time step refers
         to a unique point in simulation time.  The state of the
         simulation is updated one time step at a time.
         
\item Adaptive Mesh Refinement (AMR) - In brief, AMR allows spending
  less CPU time on ``inactive'' (less interesting) areas of the
  simulation, and spend more time computing where there are many
  particles reacting.  Resolution is low in the center where things
  are stable, but high at the edges.  This feature is in ICE, but not
  ARCHES.

\item CCA - Common Component Architecture.

\item CFD - Computational Fluid Dynamics modeling.

\item DistCC - Parallel, distributed compiler.

\item Doxygen - Doxygen (code documentation) web interface.

\item GhostCells (and Extra Cells)

\item Grid - The problem's physical domain.  The number of cells in the
  grid determine the resolution of the simulation.

\item Handle - Smart pointers.  Handles track the number of references
  to a given object, and when the number reaches zero, de-allocates
  the memory.

\item Level - Not a 'level' in 3d-space, but a level of recursion into an
  AMR grid.  ARCHES doesn't support AMR or nonuniform cells, and
  therefore doesn't need recursion, so it works on a single level '1'.

\item Material Point Method (MPM) - The main component for simulating
  structures (physcial objects) in the UCF.

\item Message Passing Interface (MPI) - Communication library used by many
  distributed software packages to communicate data between multiple
  processors.  Besides send'ing and recv'ing data, data
  reduction (UCF Reduction Variables) is supported. 
  \begin{itemize}
    \item OpenMPI
  \end{itemize}

\item Patch - A physical region of the grid assigned one to each
  processor.  The processor working on a patch will compute properties
  for each of the cells contained in the patch.  Think of this as a big
  cube that contains hundreds of little cubes.

\item Regression Tester (RT) - Runs nightly accuracy, memory, and
  completion tests on Uintah simulations.

%\item <table border=''1'' cellpadding=''5'' width=''80%''><tr><td
%                                %bgcolor=''#eeeeff''><b>Programmers
%                                %Note:</b> Programming examples and
%                                %notes will appear like
%                                %this.</td></tr></table>

\item SCIRun - A Problem Solving Environment (PSE) originally used to
  provide core software building blocks for Uintah as well as an
  extensive visualization package for viewing Uintah data archives.

\item SUS - Standalone Uintah Simulator.  This is the main executable
  program in the Uintah project.

\item SVN - Subversion code versioning system.

\item Uintah - The general name of the C-SAFE simulation code.
  Sometimes also refered to as the UCF.  The name comes from the
  Uintah mountain range in Utah.

\item Uintah Computational Framework (UCF) - The core software
  infrastructure for Uintah.
  \begin{itemize}
    \item Variables (CC, NC, FC) - Cell centered, Node centered, and
      Face centered (respectively) data structures used within the UCF.
  \end{itemize}

\item Uintah Data Archive (UDA) - The directory/file/data layout for
  storing Uintah simulation data.

\item Uintah Problem Specification (UPS (Section~\ref{Sec:UPS})) - An XML based file used to
  specify Uintah simulation properties.

\item Uintah Software Organization
  \begin{itemize}
    \item Visualization
    \item scinew - a wrapper for the C++ new() function that allows for memory tracking.
  \end{itemize}

\end{itemize}


% Appendicies
\newpage
% \section*{Appendicies}
% \subsection*{Appendix A. Bomb Units} \label{appendixBombUnits}
\appendix
\addappheadtotoc

\chapter{Bomb Units} \label{appendixBombUnits}

Following is a table of conversion factors from bomb units to mks units.  Bomb units are useful in small scale
simulations that occur very quickly, such as detonation and deformation. \\

\begin{tabular}{|c|c|}
\hline 
\bf{Measure} & \bf{Conversion Factor}     \\
\hline
mass	& $\mu g$               \\
length  & cm                    \\
time    & $\mu s$               \\ [1ex]
kinematic viscosity & $1\frac{\displaystyle cm^2}{\displaystyle \mu s} = 10^2 \frac{\displaystyle m^2}{\displaystyle s}$                          \\ [1ex]
velocity            & $1\frac{\displaystyle cm}{\displaystyle \mu s} = 10^4 \frac{\displaystyle m}{\displaystyle s}$                              \\ [2ex]
force               & $1\frac{\displaystyle \mu gcm}{\displaystyle \mu s^2} = 10 N$                                                               \\ [2ex]
pressure            & $\frac{\displaystyle 10 N}{\displaystyle cm^2} = 10^5 Pa$                                                                   \\ [1ex]
viscosity           & $10^5 Pa \mu s = 10^{-1} Pa s$                                                                                              \\ [1ex]
density             & $1\frac{\displaystyle \mu g}{\displaystyle cm^3} = 1\frac{\displaystyle g}{\displaystyle m^3}$                              \\ [1ex]
heat capacity       & $\frac{\displaystyle 10N cm}{\displaystyle \mu g K} = 10^8 \frac{\displaystyle J}{\displaystyle kg K}$                      \\ [2ex]
power               & $\frac{\displaystyle 10N cm}{\displaystyle \mu s} = 10^5 W$                                                                 \\ [1ex]
thermal conductivity& $\frac{\displaystyle 10^5 W}{\displaystyle cm K} = 10^7 \frac{\displaystyle W}{\displaystyle m K}$                          \\ [2ex]
surface energy      & $\frac{\displaystyle 10 N cm}{\displaystyle cm^2} = 10^3 \frac{\displaystyle J}{\displaystyle m^2}$                         \\ [2ex]
fracture toughness  & $\frac{\displaystyle 10 N}{\displaystyle cm^{\frac{3}{2}}} = 10^{-2} \frac{\displaystyle N}{\displaystyle m^{\frac{3}{2}}}$ \\ [2ex]
enthalpy            & $1\frac{\displaystyle J}{\displaystyle kg} = 10^{-8} \frac{\displaystyle cm^2}{\displaystyle \mu s^2}$                      \\ [2ex]
\hline
\end{tabular}


